Veamos un análisis tanto temporal como espacial de los diferentes enfoques analizados para resolver el problema:

\begin{longtable}{|c|p{5.5cm}|p{5.5cm}|}
	\hline
	& \textbf{Complejidad temporal} & \textbf{Complejidad espacial} \\
	\hline
	\textbf{Enfoque recursivo} &  La complejidad temporal de este
	enfoque recursivo es exponencial $O(2^W)$ ya que existen un casos de subproblemas superpuestos &  No se utiliza ningún espacio
	externo para almacenar valores
	aparte del espacio de la pila interna por tanto $O(1)$\\ \hline
	\textbf{Memorización} & En el peor de los casos se tendría que calcular todos los posibles estados que sería como recorrer todas las posiciones de la
	matriz por lo que la complejidad
	es $O(W\times n )$ donde $W$ es el peso máximo de la mochila y $n$ la cantidad de  diferentes de tipos objetos & Es evidente que el uso de una
	matriz para almacenar los cálculos de los diferentes estados hace
	que la complejidad espacial para
esta variante es $O(W\times n )$  \\ \hline
	\textbf{Programación Dinámica} & Es evidente que se calcula todos los posibles estados que sería como recorrer todas las posiciones de la
	matriz por lo que la complejidad
	es $O(W\times n )$ donde $W$ es el peso máximo de la mochila y $n$ la cantidad de  diferentes de tipos objetos  &  El uso de un arreglo para almacenar lo mejor para cada posible peso de $1$ a $W$ hace que esta complejidad sea $O(W)$ \\ \hline
	\textbf{Enfoque eficiente} & Esta optimización o enfoque mas eficiente hace que la complejidad sea igual $O( N + \min(wt[i], W) \times N)$ &  El uso de un arreglo para almacenar lo mejor para cada posible peso de $1$ a $W$ hace que esta complejidad sea $O(W)$ \\ \hline
\end{longtable} 