El algoritmo de Fleury es un algoritmo elegante pero ineficiente que data de 1883. Considere un grafo que se sabe que tiene todas las aristas en el mismo componente y como máximo dos vértices de grado impar. El algoritmo comienza en un vértice de grado impar o, si el grafo no tiene ninguno, comienza con un vértice elegido arbitrariamente. En cada paso, elige la siguiente arista en la ruta para que sea uno cuya eliminación no desconecte el grafo, a menos que no exista tal arista, en cuyo caso elige la arista restante que queda en el vértice actual. Luego se mueve al otro extremo de esa arista y elimina la arista. Al final del algoritmo no quedan aristas, y la secuencia a partir de la cual se eligieron las aristas forma un ciclo euleriano si el gráfico no tiene vértices de grado impar, o un camino euleriano si hay exactamente dos vértices de grado impar.

Mientras que el recorrido del gráfico en el algoritmo de Fleury es lineal en el número de aristas, es decir O($E$), también debemos tener en cuenta la complejidad de detectar puentes . Si vamos a volver a ejecutar el algoritmo de búsqueda de puentes de tiempo lineal de Tarjan después de eliminar cada arista, el algoritmo de Fleury tendrá una complejidad de tiempo de O( $E ^{2}$). Un algoritmo dinámico de búsqueda de puentes de Thorup permite mejorar esto O($E\cdot \log ^{3}E\cdot \log \log E$), pero sigue siendo significativamente más lento que los algoritmos alternativos.