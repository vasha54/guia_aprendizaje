\subsection{Punto 2D}
El punto en la geometría es uno de los entes fundamentales de la geometría, junto con la recta y el 
plano, pues son considerados conceptos primarios, es decir, que solo es posible describirlos en relación 
con otros elementos similares o parecidos. El punto carece de largo, espesor o grosor. Se suelen
describir apoyándose en los postulados característicos, que determinan las relaciones entre los entes 
geométricos fundamentales. El punto es la unidad más simple, irreductiblemente mínima, de la 
comunicación visual; es una figura geométrica sin dimensión, tampoco tiene longitud, área, volumen, ni 
otro ángulo dimensional. No es un objeto físico. Describe una posición en el plano, determinada respecto 
de un sistema de coordenadas preestablecidas. 

\subsection{Segmento}
En geometría, el segmento es un fragmento de la recta que está comprendido entre dos puntos, llamados puntos extremos o finales. Así, dado dos puntos A y B, se llama segmento AB a la intersección de la semirrecta de origen A que contiene al punto B con la semirrecta de origen B que contiene al punto A. Los puntos A y B son extremos del segmento y los puntos sobre la recta a la que pertenece el segmento. 