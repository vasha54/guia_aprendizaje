\subsection{C++}

\subsubsection{Función totiente de Euler}
\begin{lstlisting}[language=C++]
int phi(int n) {
   int result = n;
   for (int i = 2; i * i <= n; i++) {
      if (n % i == 0) {
         while (n % i == 0) n /= i;
         result -= result / i;
      }
   }
   if (n > 1) result -= result / n;
   return result;
}
\end{lstlisting}

\subsubsection{Encontrar el totiente del 1 al $n$ usando la propiedad de la suma del divisor}
\begin{lstlisting}[language=C++]
void phi_1_to_n(int n) {
   vector<int> phi(n + 1);
   for (int i = 0; i <= n; i++) phi[i] = i;
   for (int i = 2; i <= n; i++) {
      if(phi[i] == i) {
         for (int j = i; j <= n; j += i) phi[j] -= phi[j] / i;
      }
   }
}
\end{lstlisting}


\subsection{Java}

\subsubsection{Función totiente de Euler}
\begin{lstlisting}[language=Java]
public static int phi(int n) {
   int result = n;
   for (int i = 2; i * i <= n; i++) {
      if (n % i == 0) {
         while (n % i == 0) n /= i;
         result -= result / i;
      }
   }
   if (n > 1) result -= result / n;
   return result;
}
\end{lstlisting}

\subsubsection{Encontrar el totiente del 1 al $n$ usando la propiedad de la suma del divisor}
\begin{lstlisting}[language=Java]
public static int [] phi_1_to_n(int n) {
   int [] phi = new int [n + 1];
   for (int i = 0; i <= n; i++) phi[i] = i;
   for (int i = 2; i <= n; i++) {
      if(phi[i] == i) {
         for (int j = i; j <= n; j += i) phi[j] -= phi[j] / i;
      }
   }
   return phi;
}
\end{lstlisting}
