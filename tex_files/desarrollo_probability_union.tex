La unión $A \cup B$ significa \textbf{A o B suceden}. Por ejemplo, la unión de
$A=\{~2,~5\}$ y $B=\{~4,~5,~6\}$ es $A \cup B=\{~2,~4,~5,~6\}$. La probabilidad de 
la unión $A \cup B$ se calcula mediante la fórmula:
	
	$$P(A \cup B) = P(A)+P(B) - P(A \cap B)$$
	
Por ejemplo, al lanzar un dado, la unión de los eventos $A=\text{el resultado es par}$ y $B=\text{el resultado es inferior a 4}$ es $A \cup B=\text{el resultado es par o menor que 4}$   y su probabilidad es:

 $$P(A \cup B) = P(A)+P(B) - P(A \cap B) = \dfrac{1}{2} + \dfrac{1}{2} - \dfrac{1}{6} = \dfrac{5}{6}$$
 
Si los eventos $A$ y $B$ son disjuntos, es decir, $A \cap B$ está vacío, la probabilidad del evento $A \cup B$ es simplemente:

$$P(A \cup B) = P(A)+P(B)$$