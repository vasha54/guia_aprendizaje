\subsection{Sistema de numeración}
Un sistema de numeración es un conjunto de símbolos y reglas de generación que permiten construir todos los números válidos.  
\subsection{Sistema binario}
El sistema binario, también llamado sistema diádico en ciencias de la computación, es un sistema de numeración en el que los números son representados utilizando únicamente dos cifras: 0 (cero) y 1 (uno). Es uno de los sistemas que se utilizan en las computadoras, debido a que estas trabajan internamente con dos niveles de voltaje (0 apagado, 1 conectado), por lo cual su sistema de numeración natural es el sistema binario.
\subsection{Complemento a Dos}
El complemento a 2 es una forma de representar números negativos en el sistema binario. El complemento a dos de un número $N$ , expresado en el sistema binario con $n$ dígitos, se define como: 

$$C_2(N)=2^n-N$$

donde total de números positivos será $2^{n-1}-1$ y el de negativos $2^{n-1}$, siendo $n$ el número de bits. El $0$ contaría como positivo, ya que los positivos son los que empiezan por $0$ y los negativos los que empiezan por $1$. 