Estos operadores son propiamente de C++ y no están presentes en Java. 

\begin{tabular}{|c|p{2cm}|p{11cm}|}
	\hline
	\textbf{Operador}	& \textbf{Nombre} & \textbf{Descripción}  \\
	\hline
	. & Pertenecia directa & El operador de pertenencia (punto) se utiliza junto con el nombre de la estructura o unión, para
especificar un miembro de las mismas. Si tenemos una estructura cuyo nombre es nombre,
	y miembro es un miembro especificado por el patrón de la estructura, nombre.miembro
	identifica dicho miembro de la estructura. El operador de pertenencia puede utilizarse de la
	misma forma en uniones. \\
	\hline
	-> & Pertenecia indirecta & El operador de pertenencia indirecto: se usa con un puntero estructura o unión para identificar
	un miembro de las mismas. Supongo que ptrstr es un puntero a una estructura que contiene
	un miembro especificado en el patrón de estructura con el nombre miembro. En este caso identifica al miembro correspondiente de la estructura apuntada. El operador de pertenencia
	indirecto puede utilizarse de igual forma con uniones. \\
	\hline
	
\end{tabular}