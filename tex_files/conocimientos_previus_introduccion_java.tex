\paragraph{Sistema Operativo} Es el conjunto de programas de un sistema informático que gestiona los recursos de hardware y provee servicios a los programas de aplicación de software. Estos programas se ejecutan en modo privilegiado respecto de los restantes.

\paragraph{Lenguaje de Programación:} Es una herramienta
que nos permite comunicarnos e instruir a la computadora para que realice una tarea específica. Cada
lenguaje de programación posee una sintaxis y un léxico particular, es decir, forma de escribirse que
es diferente en cada uno por la forma que fue creado y por la forma que trabaja su compilador para
revisar, acomodar y reservar el mismo programa en memoria.

\paragraph{Compilador:} Es un programa informático que traduce un programa escrito en un lenguaje de programación a otro lenguaje de programación. Usualmente el segundo lenguaje es lenguaje de máquina, pero tambíen puede ser un código intermedio (bytecode), o simplemente texto.

\paragraph{Lenguaje de máquina:} Es el sistema de códigos directamente interpretable por un circuito microprogramable como el microprocesador de una computadora o microntrolador de un automata.

\paragraph{Bytecode:} Es un tipo de código de programa que se compila y ejecuta en sistemas informáticos llamados VM (máquinas virtuales, por sus siglas en inglés). Los programadores pueden usar el bytecode en su forma original en cualquier plataforma en la que opere la VM, por lo que el código es independiente de la plataforma. También es intermedio, ya que sus funciones están a medio camino entre el código fuente y el código máquina. El bytecode también se conoce como código portable o P-Code.

\paragraph{Entorno Integrado de Desarrollo (IDE):} Un entorno de desarrollo integrado, conocido también como IDE (\emph{Integrated Development
	Environment}) por sus siglas en inglés, es un programa informático compuesto por un conjunto de herramientas de programación que facilitan el desarrollo de aplicaciones. Un IDE
puede denominarse como un entorno de programación, esto significa que consiste en un
editor de código, un compilador, un depurador y un constructor de interfaz gráfica. Los
IDE proveen un marco de trabajo amigable para la mayoría de los lenguajes de programación