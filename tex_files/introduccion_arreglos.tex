En varios problemas debemos trabajar con varias informaciones de forma grupal y dichas informaciones tienen en común que comporten el mismo tipo de dato. Una variante bastante trivial es declarar un variable por cada información con que necesito trabajar. Por ejemplo si tengo el salario mensual de un trabajador durante un año con declarar 12 variables me sería suficiente para luego realizar un grupo de operaciones como el promedio salarial, el mínimo y máximo salario del trabajador. Estas operaciones aunque se pueden implementar no cabe duda que tienen su complejidad en cuanto a la implementación que puede ser un tanto tediosas. Bueno imaginemos que ahora tengamos que hacer ese mismo trabajo con un quinquenio o decada de trabajo del trabajador la complejidad de implementación aumentería casi que literal en cino o diez veces más. 

Vamos a ver como podemos resolver esto en programación con el uso de la estructura de datos más primitiva que nos brinda los lenguajes de programación: \textbf{arreglos}  