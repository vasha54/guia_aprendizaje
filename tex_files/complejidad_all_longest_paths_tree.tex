En  cuanto a la complejidad de los algoritmos analizamos en esta guía podemos mencionar que la primera variante utilizando programación dinámica con dos DFS hace que esta solución tenga una complejidad temporal de  O($2(V+E)$) donde $V$ es la cantidad de vértices del árbol y $E$ las aristas, pero reglas del cálculo de complejidades de algoritmo la constante $2$ se desprecia (en lo personal eso se ve muy bonito para las reglas pero para la programación competitiva es importante tener en cuenta para la cantidad de operaciones), otra sustitución dentro de la expresión es que la cantidad de aristas de un árbol es igual a la cantidad de vértices del árbol por tanto $E = V-1$ y sustituyendo en la expresión final nos queda O($2(2V-1)$).

En cuanto a la complejidad temporal de la segunda variante como se realiza 3 \emph{BFS} la misma es O($3(V+E)$) que haciendo un análisis similar al anterior y sustiyendo $E$ por su equivalente en función de $V$ nos queda O($3(2V-1)$). 

Viendo ya complejidades de ambas variantes algorítmicas es evidente que la basada en programación dinámica es un tanto mas rápida que la basada en \emph{BFS} aunque a favor de esta que su implementación es más facil de implementar y se basa en una idea mas asequible de entender.  