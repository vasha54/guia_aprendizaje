Las pruebas de primalidad tienen diversas aplicaciones en diferentes campos, entre las cuales se pueden mencionar:

\begin{enumerate}
	\item \textbf{Criptografía:} En criptografía, los números primos juegan un papel fundamental en la generación de claves seguras. Los algoritmos de encriptación y firma digital utilizan números primos en sus operaciones, por lo que es importante poder verificar si un número dado es primo. Las pruebas de primalidad se utilizan para garantizar la seguridad de los sistemas criptográficos.
	
	\item \textbf{Generación de números aleatorios:} En la generación de números aleatorios, es común utilizar números primos para asegurar la aleatoriedad y la calidad de los números generados. Las pruebas de primalidad permiten verificar si un número es primo antes de ser utilizado en algoritmos de generación de números aleatorios.
	
	\item \textbf{Algoritmos de factorización:} La factorización de números enteros en sus factores primos es un problema importante en matemáticas y computación. Las pruebas de primalidad se utilizan en algoritmos de factorización para determinar si un número es primo y así facilitar el proceso de descomposición en factores primos.
	
	\item \textbf{Teoría de números computacional:} En el campo de la teoría de números computacional, las pruebas de primalidad son fundamentales para investigar propiedades de los números primos, estudiar distribuciones y patrones, y resolver problemas matemáticos complejos relacionados con los números primos.
	
	\item \textbf{Optimización y búsqueda de números primos grandes:} En la búsqueda de números primos grandes, como los utilizados en registros de primos o en concursos matemáticos, las pruebas de primalidad son esenciales para verificar si un número dado es primo y contribuir al descubrimiento de nuevos números primos.
	
\end{enumerate}

Estas son solo algunas de las aplicaciones más comunes de las pruebas de primalidad. La teoría de números y las pruebas de primalidad tienen una amplia gama de aplicaciones en diversos campos, desde la criptografía hasta la investigación matemática. 
