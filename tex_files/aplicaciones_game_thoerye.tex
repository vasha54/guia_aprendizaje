La teoría de juegos tiene una amplia gama de aplicaciones, que incluyen psicología, biología evolutiva, guerra, política, economía y negocios. A pesar de sus muchos avances, la teoría de juegos sigue siendo una ciencia joven y en desarrollo.

Como método de matemática aplicada, la teoría de juegos se ha utilizado para estudiar una amplia variedad de comportamientos humanos y animales. Inicialmente se desarrolló en economía para comprender una gran colección de comportamientos económicos, incluidos los comportamientos de empresas, mercados y consumidores. El primer uso del análisis de teoría de juegos fue por Antoine Augustin Cournot en 1838 con su solución del duopolio de Cournot. El uso de la teoría de juegos en las ciencias sociales se ha expandido, y la teoría de juegos también se ha aplicado a los comportamientos políticos, sociológicos y psicológicos.

Aunque los naturalistas anteriores al siglo XX, como Charles Darwin, hicieron declaraciones de tipo teórico de juegos, el uso del análisis teórico de juegos en biología comenzó con los estudios de Ronald Fisher sobre el comportamiento animal durante la década de 1930. Este trabajo es anterior al nombre de «teoría de juegos», pero comparte muchas características importantes con este campo. Los desarrollos en economía fueron aplicados más tarde a la biología en gran parte por John Maynard Smith en su libro Evolution and the Theory of Games.

Además de usarse para describir, predecir y explicar el comportamiento, la teoría de juegos también se ha utilizado para desarrollar teorías del comportamiento ético o normativo y para prescribir dicho comportamiento. En economía y filosofía, los académicos han aplicado la teoría de juegos para ayudar a comprender el comportamiento bueno o apropiado. Los argumentos teóricos del juego de este tipo se pueden encontrar desde Platón. Una versión alternativa de la teoría de juegos, llamada teoría de juegos químicos, representa las elecciones del jugador como moléculas reactivas químicas metafóricas llamadas «knowlecules». La teoría del juego químico luego calcula los resultados como soluciones de equilibrio para un sistema de reacciones químicas.