La complejidad de un algoritmo es un estimado de tiempo de acuerdo a la entrada. La idea es representar la eficiencia del algoritmo como una función parametrizada donde los parámetros son el tamaño de las entradas del problema.

El tiempo de complejidad de un algoritmo se denota como O($\dots$) donde los tres puntos representan alguna función que usualmente utiliza variables para denominar cada una de los tamaños de las entradas. Por ejemplo $N$ donde dicha variable pueda representar la cantidad de elementos de un arreglo.

A la hora de medir el tiempo, siempre lo haremos en función del número de
operaciones elementales que realiza dicho algoritmo, entendiendo por operaciones
elementales (en adelante OE) aquellas que el ordenador realiza en tiempo acotado
por una constante. Así, consideraremos OE las operaciones aritméticas básicas,
asignaciones a variables de tipo predefinido por el compilador, los saltos (llamadas
a funciones y procedimientos, retorno desde ellos, etc.), las comparaciones lógicas
y el acceso a estructuras indexadas básicas, como son los vectores y matrices. Cada
una de ellas contabilizará como 1 OE.

Resumiendo, el tiempo de ejecución de un algoritmo va a ser una función que
mide el número de operaciones elementales que realiza el algoritmo para un
tamaño de entrada dado.

También es importante hacer notar que el comportamiento de un algoritmo
puede cambiar notablemente para diferentes entradas (por ejemplo, lo ordenados
que se encuentren ya los datos a ordenar). De hecho, para muchos programas el
tiempo de ejecución es en realidad una función de la entrada específica, y no sólo
del tamaño de ésta. Así suelen estudiarse tres casos para un mismo algoritmo:\emph{caso peor}, \emph{caso mejor} y \emph{caso medio}.

El caso mejor corresponde a la traza (secuencia de sentencias) del algoritmo que
 realiza menos instrucciones. Análogamente, el caso peor corresponde a la traza del
 algoritmo que realiza más instrucciones. Respecto al caso medio, corresponde a la
traza del algoritmo que realiza un número de instrucciones igual a la esperanza
matemática de la variable aleatoria definida por todas las posibles trazas del
algoritmo para un tamaño de la entrada dado, con las probabilidades de que éstas
ocurran para esa entrada.

\subsection{Reglas generales para el cálculo del número de OE}

La siguiente lista presenta un conjunto de reglas generales para el cálculo del
número de OE, siempre considerando el peor caso. Estas reglas definen el número
de OE de cada estructura básica del lenguaje, por lo que el número de OE de un
algoritmo puede hacerse por inducción sobre ellas.

\begin{enumerate}
	\item Vamos a considerar que el tiempo de una OE es, por definición, de orden 1. La
constante c que menciona el Principio de Invarianza dependerá de la
implementación particular, pero nosotros supondremos que vale 1.
	\item El tiempo de ejecución de una secuencia consecutiva de instrucciones se calcula
sumando los tiempos de ejecución de cada una de las instrucciones.
	\item El tiempo de ejecución de la sentencia  $\text{CASE C OF} v1:S1|v2:S2|\dots|vn:Sn\quad
	 \text{END};$ es $T = T(C) + \max\{T(S_1 ),T(S_2 ),\dots,T(S_n)\}$. Obsérvese que $T(C)$ incluye el
	tiempo de comparación con $v_1 , v_2 ,\dots, v_n$ 
	\item El tiempo de ejecución de la sentencia $\text{IF C THEN S1 ELSE S2 END;}$ es \\
	$T = T(C) + \max\{T(S 1 ),T(S 2 )\}$.
	\item El tiempo de ejecución de un bucle de sentencias $\text{WHILE C DO S END;}$ es
	$T = T(C) + (n^o \quad iteraciones)\times(T(S) + T(C))$. Obsérvese que tanto $T(C)$ como $T(S)$
pueden variar en cada iteración, y por tanto habrá que tenerlo en cuenta para su
cálculo.
	\item Para calcular el tiempo de ejecución del resto de sentencias iterativas (FOR,
	REPEAT, LOOP) basta expresarlas como un bucle WHILE.
	\item El tiempo de ejecución de una llamada a un procedimiento o función
	$F(P_1 , P_2 ,\dots, P_n )$ es 1 (por la llamada), más el tiempo de evaluación de los
	parámetros $P_1$ , $P_2$ ,$\dots$, $P_n$ , más el tiempo que tarde en ejecutarse $F$, esto es,
	$T = 1 + T(P_1 ) + T(P_2 ) + \dots + T(P_n ) + T(F)$. No contabilizamos la copia de los
argumentos a la pila de ejecución, salvo que se trate de estructuras complejas
(registros o vectores) que se pasan por valor. En este caso contabilizaremos
tantas OE como valores simples contenga la estructura. El paso de parámetros
por referencia, por tratarse simplemente de punteros, no contabiliza tampoco.
	\item El tiempo de ejecución de las llamadas a procedimientos recursivos va a dar
lugar a ecuaciones en recurrencia, que veremos posteriormente no en esta guía.
	\item También es necesario tener en cuenta, cuando el compilador las incorpore, las
optimizaciones del código y la forma de evaluación de las expresiones, que
pueden ocasionar \emph{cortocircuitos} o realizarse de forma \emph{perezosa} (lazy). En el
presente trabajo supondremos que no se realizan optimizaciones, que existe el
cortocircuito y que no existe evaluación perezosa.
\end{enumerate}

\subsection{Clases o tipos de complejidades}

A continuciación una lista de las complejidades mas comunes que nos podemos encontrar en el diseño de algoritmos:

\begin{itemize}
	\item O($1$): Complejidad \textbf{constante} que significa que el algoritmo no de depende del tamaño de la entrada. Por lo general esta complejidad se ve en algoritmos con fórmula que directamente calcula la respuesta.
	\item O($\log N$): Complejidad \textbf{logarítmica} que implica que el algoritmo en el próximo o iteración descarta la mitad del tamaño de la entrada del paso o iteracción actual. Ejemplo de esto es la búsqueda binaria.
	\item O($\sqrt{N}$): Una complejidad \textbf{raiz cuadrádica} es mas lenta que una O($\log N$) pero más rápida que O($N$). Una especial propiedad de la raíz cuadrada es que $\sqrt{N}=N/\sqrt{N}$, entonces la raíz de $\sqrt{N}$ miente, en algunos casos es la mitad del tamaño de la entrada.
	\item O($N$): Una complejidad \textbf{linear} en un algoritmo implica que se tuvo que recorrer todos los elementos de la entrada para dar la respuesta.
	\item O($N \log N$): Esta complejidad indica que el algoritmo ordeno la entrada porque es la complejidad de los algoritmos de ordenamiento eficientes. Otra posibilidad es que el algoritmo utiliza alguna estructura de datos que por cada operación tiene una complejidad de O($\log N$).
	\item O($N^2$): Complejidad \textbf{cuadrática} en un algoritmo a menudo indica que este presenta dos ciclo anidado como pudiera ser el caso de generar todos los pares de las entradas. 
	\item O($N^3$): Complejidad \textbf{cúbica} en un algoritmo a menudo indica que este presenta tres ciclo anidado como es el caso del algoritmo Floyd-Warshall
	\item O($2^N$): Esta complejidad indica a menudo que el algoritmo itera sobre todos los subconjuntos de los elementos de la entrada
	\item O($N!$): Esta complejidad indica a menudo que el algoritmo itera sobre todas las permutaciones  de los elementos de la entrada
\end{itemize}

Un algoritmo tiene complejidad polinomial si se expresa como O($n^k$) donde $k$ es una constante. Todas las complejidades vistas anteriormente son polinomiales excepto O($2^N$) y O($N!$). Lo que sucede es que en la práctica el valor de $k$ es usualemente pequeño lo que hace que esas complejidades polinomiales para un algoritmo sean eficientes.


