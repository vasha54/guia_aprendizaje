En nuestro contexto, Divide y Vencerás es una técnica de diseño de algoritmos
que consiste en resolver un problema a partir de la solución de subproblemas del
mismo tipo, pero de menor tamaño. Si los subproblemas son todavía relativamente
grandes se aplicará de nuevo esta técnica hasta alcanzar subproblemas lo
suficientemente pequeños para ser solucionados directamente. Ello naturalmente
sugiere el uso de la recursión en las implementaciones de estos algoritmos.

La resolución de un problema mediante esta técnica consta fundamentalmente
de los siguientes pasos:

\begin{enumerate}
	\item En primer lugar ha de plantearse el problema de forma que pueda ser
	descompuesto en $k$ subproblemas del mismo tipo, pero de menor tamaño. Es
	decir, si el tamaño de la entrada es $n$, hemos de conseguir dividir el problema en
	$k$ subproblemas (donde $1 \le  k \le n$) cada uno con una entrada de tamaño $n_k$ y
	donde $0 \le n_k \le n$. A esta tarea se le conoce como \emph{división}.
	\item En segundo lugar han de resolverse independientemente todos los
	subproblemas, bien directamente si son elementales o bien de forma recursiva.
	El hecho de que el tamaño de los subproblemas sea estrictamente menor que el
	tamaño original del problema nos garantiza la convergencia hacia los casos
	elementales, también denominados casos \emph{base}.
	\item Por último, \emph{combinar} las soluciones obtenidas en el paso anterior para construir
	la solución del problema original.
\end{enumerate}

Hemos de hacer unas apreciaciones en este esquema sobre el procedimiento
Divide, sobre el número $k$ que representa el número de subproblemas, y sobre el
tamaño de los subproblemas, ya que de todo ello va a depender la eficiencia del
algoritmo resultante.

En primer lugar, el número $k$ debe ser pequeño e independiente de una entrada
determinada. En el caso particular de los algoritmos Divide y Vencerás que
contienen sólo una llamada recursiva, es decir $k = 1$, hablaremos de algoritmos de
\emph{simplificación}. Tal es el caso del algoritmo recursivo que resuelve el cálculo del
factorial de un número, que sencillamente reduce el problema a otro subproblema
del mismo tipo de tamaño más pequeño. También son algoritmos de simplificación
el de búsqueda binaria en un vector o el que resuelve el problema del k-ésimo
elemento.

La ventaja de los algoritmos de simplificación es que consiguen reducir el
tamaño del problema en cada paso, por lo que sus tiempos de ejecución suelen ser
muy buenos (normalmente de orden logarítmico o lineal). Además pueden admitir
una mejora adicional, puesto que en ellos suele poder eliminarse fácilmente la
recursión mediante el uso de un bucle iterativo, lo que conlleva menores tiempos
de ejecución y menor complejidad espacial al no utilizar la pila de recursión,
aunque por contra, también en detrimento de la legibilidad del código resultante.

Por el hecho de usar un diseño recursivo, los algoritmos diseñados mediante la
técnica de Divide y Vencerás van a heredar las ventajas e inconvenientes que la
recursión plantea:

\begin{itemize}
	\item Por un lado el diseño que se obtiene suele ser simple, claro, robusto y elegante,
	lo que da lugar a una mayor legibilidad y facilidad de depuración y
	mantenimiento del código obtenido.
	
	\item Sin embargo, los diseños recursivos conllevan normalmente un mayor tiempo
	de ejecución que los iterativos, además de la complejidad espacial que puede
	representar el uso de la pila de recursión.
\end{itemize}

Desde un punto de vista de la eficiencia de los algoritmos Divide y Vencerás, es
muy importante conseguir que los subproblemas sean independientes, es decir, que
no exista solapamiento entre ellos. De lo contrario el tiempo de ejecución de estos
algoritmos será exponencial. Como ejemplo pensemos en el cálculo de la sucesión
de Fibonacci, el cual, a pesar de ajustarse al esquema general y de tener sólo dos
llamadas recursivas, tan sólo se puede considerar un algoritmo recursivo pero no
clasificarlo como diseño Divide y Vencerás. Esta técnica está concebida para
resolver problemas de manera eficiente y evidentemente este algoritmo, con tiempo
de ejecución exponencial, no lo es.

En cuanto a la eficiencia hay que tener en también en consideración un factor
importante durante el diseño del algoritmo: el número de subproblemas y su
tamaño, pues esto influye de forma notable en la complejidad del algoritmo
resultante.

Otra consideración importante a la hora de diseñar algoritmos Divide y
Vencerás es el reparto de la carga entre los subproblemas, puesto que es importante
que la división en subproblemas se haga de la forma más equilibrada posible. En
caso contrario nos podemos encontrar con \emph{anomalías de funcionamiento} como le
ocurre al algoritmo de ordenación \emph{Quicksort}. Éste es un representante claro de los
algoritmos Divide y Vencerás, y su caso peor aparece cuando existe un
desequilibrio total en los subproblemas al descomponer el vector original en dos
subvectores de tamaño $0$ y $n-1$. En este caso su
orden es O($n^2$), frente a la buena complejidad, O($n\log n$), que consigue cuando
descompone el vector en dos subvectores de igual tamaño.
También es interesante tener presente la dificultad y es esfuerzo requerido en
cada una de estas fases va a depender del planteamiento del algoritmo concreto.
Por ejemplo, los métodos de ordenación por Mezcla y Quicksort son dos
representantes claros de esta técnica pues ambos están diseñados siguiendo el
esquema presentado: dividir y combinar.