\subsection{Búsqueda completa}
La búsqueda completa es un método general que se puede utilizar para resolver casi cualquier
 problema de algoritmo. La idea es generar todas las posibles soluciones al problema utilizando la
 fuerza bruta, y luego seleccionar la mejor solución o contar el número de soluciones, dependiendo
del problema.

\subsection{Algoritmos golosos (\emph{Greedy})}
Un algoritmo goloso construye una solución al problema al tomar siempre una decisión que se ve mejor en este momento. Un algoritmo goloso nunca recupera sus opciones, pero construye directamente la solución final. Por esta razón, los algoritmos golosos suelen ser muy eficientes.

La dificultad para diseñar algoritmos golosos es encontrar una estrategia codiciosa que siempre produzca una solución óptima al problema. Las opciones localmente óptimas en un algoritmo goloso también deben ser globalmente óptimos. A menudo es difícil argumentar que funciona un algoritmo goloso.

\subsection{Función recursiva}
La recursividad es una técnica de programación que se utiliza para realizar una llamada a una
 función desde ella misma, de allí su nombre. Un algoritmo recursivo es un algoritmo que expresa la solución de un problema en términos
de una llamada a sí mismo. La llamada a sí mismo se conoce como llamada recursiva o recurrente.