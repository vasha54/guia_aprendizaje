\subsection{Número primo}
Un número primo es un número natural mayor que 1 que solamente es divisible por sí mismo y por 1, es decir, no tiene más divisores que esos dos números. Algunos ejemplos de números primos son 2, 3, 5, 7, 11, 13, 17, etc. Los números primos son fundamentales en matemáticas y tienen muchas propiedades interesantes.


\subsection{Pierre de Fermat}

Pierre de Fermat fue un matemático francés del siglo XVII conocido por sus contribuciones a la teoría de números, geometría analítica y cálculo. Una de las contribuciones más famosas de Fermat es el \emph{Teorema de Fermat}, que enunció en una nota al margen de su ejemplar de la obra de Diofanto. El teorema establece que no existen enteros positivos a, b, c y n mayores que 2 que satisfagan la ecuación $a^n + b^n = c^n$. Este teorema fue uno de los problemas abiertos más famosos en matemáticas y se conoció como el \emph{Último Teorema de Fermat}. Fue finalmente demostrado por Andrew Wiles en 1994 utilizando técnicas avanzadas de matemáticas modernas.

\subsection{Exponenciación binaria}
La exponenciación binaria es un método eficiente para calcular potencias de un número. Consiste en descomponer el exponente en su representación binaria y realizar operaciones con las potencias de 2.

Este método es eficiente porque reduce el número de multiplicaciones requeridas para calcular la potencia, ya que solo se realizan multiplicaciones cuando el bit correspondiente en la representación binaria del exponente es 1.


\subsection{Garry Miller}
Gary Miller es un matemático y científico de la computación estadounidense conocido por su trabajo en teoría de números y algoritmos. Es especialmente reconocido por sus contribuciones al campo de la criptografía y la computación numérica.

\subsection{Michael O. Rabin}
Michael O. Rabin es un matemático e informático israelí-estadounidense conocido por sus contribuciones a la teoría de la computación, la criptografía y la complejidad computacional. Es uno de los pioneros en el campo de la informática teórica y ha realizado importantes investigaciones en varios aspectos fundamentales de la ciencia de la computación.
