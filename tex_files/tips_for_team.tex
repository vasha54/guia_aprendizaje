\begin{itemize}
	\item Practica codificar (o escribir pseudocódigo) en un papel en blanco. Esto es útil cuando su
	compañero de equipo está usando la computadora. Cuando sea su turno de usar la computadora, puede
	luego simplemente escriba el código lo más rápido posible.
	\item La estrategia de "enviar e imprimir": si su código obtiene un veredicto AC, ignore la impresión. Si todavía no es AC, depure su código usando esa impresión (y deje que su compañero de equipo use la computadora por otro problema). Cuidado: La depuración sin la computadora no es una
	habilidad fácil de dominar. 
	\item Si su compañero de equipo está programando actualmente (y no tiene idea de otros problemas), entonces preparar datos de prueba  (y con suerte el código de su compañero de equipo pasa todos
	aquellos). Con dos miembros del equipo acordando la corrección (potencial) de un código, la
	probabilidad de tener penalidades disminuye o desaparece.
	\item Si sabe que su compañero de equipo es (significativamente) más fuerte en cierto tipo de problema que usted mismo y actualmente está leyendo un problema con ese tipo (especialmente en
	la etapa inicial del concurso), considere pasar el problema a su compañero de equipo en su lugar
	de insistir en resolverlo uno mismo.
	\item Practique la codificación de un algoritmo bastante largo/complicado como un par o incluso como un triple (con una presión de límite de tiempo de codificación) para la situación del final del concurso o competencia donde su equipo tiene como objetivo obtener +1 AC más en los últimos minutos.
	\item El factor X: hazte amigo de tus compañeros de equipo fuera de las sesiones de entrenamiento y los concursos.
\end{itemize}