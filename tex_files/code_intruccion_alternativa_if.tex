La implementación de esta estructura es similar tanto en C++ y Java asi que vamos a ver algunos ejemplos el uso de esta estructura.

\subsection{Java}
Si la calificación es igual o superior a 60 el estudiante esta aprobado
\begin{lstlisting}[language=Java]
if ( calificacionEstudiante >= 60 )
   System.out.println( "Aprobado" );	
\end{lstlisting}

Si la calificación es igual o superior a 60 el estudiante esta aprobado sino esta reprobado
\begin{lstlisting}[language=Java]
if ( calificacion >= 60 )
   System.out.println( "Aprobado" );
else
   System.out.println( "Reprobado" );
\end{lstlisting}

Acorde al valor de la calificación y en el rango que esta ubicado dicha calificación será la repuesta.
\begin{lstlisting}[language=Java]
if ( calificacionEstudiante >= 90 )
   System.out.println( "A" );
else if ( calificacionEstudiante >= 80 )
   System.out.println( "B" );
else if ( calificacionEstudiante >= 70 )
   System.out.println( "C" );
else if ( calificacionEstudiante >= 60 )
   System.out.println( "D" );
else
   System.out.println( "F" );
\end{lstlisting}



\subsection{C++}
Si la calificación es igual o superior a 60 el estudiante esta aprobado
\begin{lstlisting}[language=Java]
if ( calificacionEstudiante >= 60 )
   cout<<"Aprobado";	
\end{lstlisting}

Si la calificación es igual o superior a 60 el estudiante esta aprobado sino esta reprobado
\begin{lstlisting}[language=Java]
if ( calificacion >= 60 )
   cout<<"Aprobado";
else
   cout<<"Reprobado";
\end{lstlisting}

Acorde al valor de la calificación y en el rango que esta ubicado dicha calificación será la repuesta.
\begin{lstlisting}[language=Java]
if ( calificacionEstudiante >= 90 )
   cout<<"A";
else if ( calificacionEstudiante >= 80 )
   cout<<"B";
else if ( calificacionEstudiante >= 70 )
   cout<<"C";
else if ( calificacionEstudiante >= 60 )
   cout<<"D";
else
   cout<<"F" ;
\end{lstlisting}