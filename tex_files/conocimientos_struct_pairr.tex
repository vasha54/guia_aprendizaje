\subsection{Arreglo}

En programación, un arreglo (llamados en inglés array) es una zona de almacenamiento continuo, que contiene una serie de elementos del mismo tipo.

\subsection{Constructor}
Los constructores se conocen con frecuencia como las funciones miembro esenciales requeridas para inicializar estructuras y objetos de tipo clase. En C++, los constructores se usan para estructuras para crear objetos con un método especial, para evitar un comportamiento indefinido o no inicializado.

\subsection{Operador \emph{new}}
C++ y Java permite a los programadores controlar la asignación de memoria en un programa,
para cualquier tipo integrado o definido por el usuario. Esto se conoce como administración dinámica de memoria y se lleva a cabo mediante el operador new. Podemos usar el operador new
para asignar (reservar) en forma dinámica la cantidad exacta de memoria requerida para contener
cada nombre en tiempo de ejecución. La asignación dinámica de memoria de esta forma hace que
se cree un arreglo (o cualquier otro tipo integrado o definido por el usuario) en el almacenamiento
libre (algunas veces conocido como el heap o montón): una región de memoria asignada a cada
programa para almacenar los objetos que se asignan en forma dinámica.