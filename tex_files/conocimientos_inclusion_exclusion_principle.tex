\subsection{Diagrama de Venn}
Los diagramas de Venn son esquemas usados en la teoría de conjuntos, tema de interés en matemáticas, lógica de clases y razonamiento diagramático. Estos diagramas muestran colecciones (conjuntos) de cosas (elementos) por medio de líneas cerradas. La línea cerrada exterior abarca a todos los elementos bajo consideración, el conjunto universal U. Los diagramas de Venn tienen el nombre de su creador, John Venn, matemático y filósofo británico. Estudiante y más tarde profesor del Caius College de la Universidad de Cambridge, Venn desarrolló toda su producción intelectual en ese ámbito.


\subsection{Coeficientes binomiales}
En matemáticas, los coeficientes binomiales, números combinatorios o combinaciones son números estudiados en combinatoria que corresponden al número de formas en que se puede extraer subconjuntos a partir de un conjunto dado. Sin embargo, dependiendo del enfoque que tenga la exposición, se pueden usar otras definiciones equivalentes. 