Tenga en cuenta que si el número que desea factorizar es en realidad un número primo, la mayoría de los algoritmos, especialmente el algoritmo de factorización de Fermat, el p-1 de Pollard, el algoritmo rho de Pollard se ejecutarán muy lentamente. Por lo tanto, tiene sentido realizar una prueba de primalidad probabilística (o determinista rápida) antes de intentar factorizar el número.

El método de factorización de Fermat puede ser muy rápido, si la diferencia entre los dos factores $p$ y $q$ es pequeño. El algoritmo se ejecuta en $O(|p-q|)$ tiempo. Sin embargo, dado que es muy lento, una vez que los factores están muy separados, rara vez se usa en la práctica.

El método de Pollard's \emph{p-1} es un algoritmo probabilístico. Puede suceder que el algoritmo no encuentre un factor. la complejidad es $O(B \log B \log^2 n)$ por iteración.

El algoritmo de búsqueda de ciclos de Floyd se ejecuta (normalmente) en $O(\sqrt[4]{n} \log(n))$ tiempo.

El algoritmo de Brent también se ejecuta en tiempo lineal, pero suele ser más rápido que el algoritmo de Floyd, ya que utiliza menos evaluaciones de la función $f$.