\subsection{International Olympiad in Informatics (IOI)}

IOI se inició en 1989 (en Bulgaria) y ha existido anualmente desde entonces. La competencia del IOI consta (normalmente) de 2 horas de práctica, 4 sesiones y dos días de competencia, 5 horas
por sección. IOI es un concurso individual. Cada concurso (normalmente) consta de 3 tareas, normalmente
una tarea fácil (más), una mediana y una tarea difícil (más), que se dividen en subtareas
con varios puntos.

\subsection{International Collegiate Programming Contests (ICPC)}

ICPC se estableció en 1970, se originó en los EE. UU., Se extendió por todo el mundo a partir de la
1990 Desde 2000 (excepto 2003 y 2007), los ganadores suelen ser rusos (especialmente de 2012 a 2019) y universidades asiáticas.

La competencia ICPC tiene una duración de 5 horas. Cada equipo está formado por tres estudiantes universitarios. Cada equipo sólo cuenta con un ordenador. Solo las presentaciones que son Aceptadas (totalmente correcta) dará +1 punto al equipo. El equipo recibe una penalización por cada envío no aceptado (generalmente +20 minutos a su tiempo total). 

Los conjuntos de problemas del ICPC generalmente están diseñados de tal manera que todos los equipos resuelven al menos uno.
problema (para evitar desmoralizar totalmente a los recién llegados a la competencia, esto es lo que nos esforzamos
para ayudar a través de este libro), ningún equipo resuelve todos los problemas (para que el concurso sea interesante hasta que el
final de la quinta hora), y todos los problemas son solucionables por al menos un equipo (así se minimiza el
cantidad de problemas 'imposibles' que requieren mucho más de 5 horas para pensar y codificar
la solución correctamente, incluso para los equipos favoritos percibidos antes del concurso).