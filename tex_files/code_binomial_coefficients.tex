\subsection{C++}
\subsubsection{Fórmula analítica}
\begin{lstlisting}[language=C++]
int C(int n, int k) {
   int res = 1;
   for (int i = n - k + 1; i <= n; ++i) res *= i;
   for (int i = 2; i <= k; ++i) res /= i;
   return res;
}

int Cv2(int n, int k) {
   double res = 1;
   for (int i = 1; i <= k; ++i)
      res = res * (n - k + i) / i;
   return (int)(res + 0.01);
}	
\end{lstlisting}


\subsubsection{Triangulo de Pascal}
\begin{lstlisting}[language=C++]
const int maxn = 1000;
int pt[maxn + 1][maxn + 1];

void pascalTriangle(){
   pt[0][0] = 1;
   for (int n = 1; n <= maxn; ++n) {
      pt[n][0] = pt[n][n] = 1;
      for (int k = 1; k < n; ++k)
          pt[n][k] = pt[n - 1][k - 1] + pt[n - 1][k];
   }
}

int C(int n, int k) {
   return pt[n][k];
}

\end{lstlisting}

\subsubsection{Coeficiente binomial módulo primo grande}
\begin{lstlisting}[language=C++]
#define int long long
int factorial[MAXN+10],m;

//precalculamos todos los factoriales
void calculateFactorial(){
   factorial[0] = 1;
   for (int i = 1; i <= MAXN; i++) {
      factorial[i] = factorial[i - 1] * i % m;
   }
}

int inverse(int a) {
   return a <= 1 ? a : m - (int)(m/a) * inverse(m % a) % m;
}	

//Y luego podemos calcular el coeficiente binomial.
int binomial_coefficient(int n, int k) {
   return factorial[n] * inverse(factorial[k] * factorial[n - k] % m) % m;
}
	
\end{lstlisting}

Incluso podemos calcular el coeficiente binomial en de forma más eficiente tiempo si precalculamos los inversos de todos los factoriales usando el método regular para calcular el 
inverso, usando la congruencia $(x!)^{-1} \equiv ((x-1)!)^{-1} \cdot x^{-1}$.

\begin{lstlisting}[language=C++]
#define int long long
int m;
vector<int> factorial, inverse_factorial;

int gcd(int a, int b, int& x, int& y) {
   x = 1, y = 0;
   int x1 = 0, y1 = 1, a1 = a, b1 = b;
   while (b1) {
      int q = a1 / b1;
      tie(x, x1) = make_tuple(x1, x - q * x1); tie(y, y1) = make_tuple(y1, y - q * y1);
      tie(a1, b1) = make_tuple(b1, a1 - q * b1);
   }
   return a1;
}

vector<int> invs(const std::vector<int> &a, int m) {
   int n = a.size();
   if (n == 0) return {};
   vector<int> b(n);
   int v = 1;
   for (int i = 0; i != n; ++i) {
      b[i] = v; v = static_cast<int>(v) * a[i] % m;
   }
   int x, y;
   gcd(v, m, x, y);
   x = (x % m + m) % m;
   for (int i = n - 1; i >= 0; --i) {
      b[i] = static_cast<int>(x) * b[i] % m;
      x = static_cast<int>(x) * a[i] % m;
   }
   return b;
}

void calculateFactorial(){
   factorial.resize(MAX_N+5); factorial[0] = 1;
   for (int i = 1; i <= MAX_N; i++) {
      factorial[i] = factorial[i - 1] * i % m;
   }
   inverse_factorial = invs(factorial,m);
}

int binomial_coefficient(int n, int k) {
   return factorial[n]*inverse_factorial[k]%m*inverse_factorial[n-k]%m;
}
	
\end{lstlisting}

\subsection{Java}

\subsubsection{Fórmula analítica}
\begin{lstlisting}[language=Java]
public static int C(int n, int k) {
   int res = 1;
   for (int i = n - k + 1; i <= n; ++i) res *= i;
   for (int i = 2; i <= k; ++i) res /= i;
   return res;
}

public static int Cv2(int n, int k) {
   double res = 1;
   for (int i = 1; i <= k; ++i)
      res = res * (n - k + i) / i;
   return (int)(res + 0.01);
}	
\end{lstlisting}

\subsubsection{Triangulo de Pascal}
\begin{lstlisting}[language=Java]
public class Main {
   private final int MAX_N = 1000;
   private long [][] tp;
	
   public void pascalTriangle(){
      tp = new long [MAX_N+3][MAX_N+3];	
      tp[0][0] = 1;
      for (int n = 1; n <= MAX_N; ++n) {
          tp[n][0] = tp[n][n] = 1;
          for(int k=1;k<n;++k) tp[n][k] = tp[n-1][k-1] + tp[n-1][k];
      }
   }
	
   public long C(int n, int k) { return tp[n][k]; }
}
\end{lstlisting}

\subsubsection{Coeficiente binomial módulo primo grande}
\begin{lstlisting}[language=Java]
public class Main {
   private final int MAX_N = 1000010;
   private long [] factorial;
   private long m;
	
   //precalculamos todos los factoriales
   public void calculateFactorial(){
      factorial = new long [MAX_N+5];
      factorial[0] = 1;
      for (int i = 1; i <= MAX_N; i++) {
         factorial[i] = factorial[i - 1] * i % m;
      }
   }
	
   public long inverse(long a) {
      return a <= 1 ? a : m - (int)(m/a) * inverse(m % a) % m;
   }	
	
   //Y luego podemos calcular el coeficiente binomial.
   public long binomial_coefficient(int n, int k) {
      return factorial[n] * inverse(factorial[k] * factorial[n - k] % m) % m;
   }
}
\end{lstlisting}