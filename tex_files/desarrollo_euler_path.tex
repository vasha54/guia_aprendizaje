Antes de buscar el camino o ciclo euleriano debemos comprobar que existe el mismo para esto nos vamos apoyar el los siguientes teoremas:

\begin{itemize}
	\item \textbf{Teorema 1:} Sea G un grafo o multigrafo no dirigido. Entonces G tiene un ciclo de Euler si, y solo si, es conexo y todo vértice tiene grado par. Diremos que es un grafo Euleriano.
	\item \textbf{Teorema 2:} Sea G un grafo o multigrafo no dirigido. Entonces G tiene un camino de Euler si, y solo si, es conexo y tiene solo dos vértices de grado impar. Diremos que el grafo es semi-Euleriano.
\end{itemize}

Los resultados obtenidos para grafos no dirigidos pueden extenderse de inmediato para grafos dirigidos. En un grafo dirigido el grado o valencia de entrada de un vértice es el numero de lados incidentes hacia este y el grado de salida es el numero de lados que son incidentes desde este. 

\begin{itemize}
	\item \textbf{Teorema 4:} Un grafo dirigido tiene un ciclo de Euler si y solo si es conexo y el grado de entrada de cualquier vértice es igual a su salida.
	\item \textbf{Teorema 5:} Un grafo dirigido tiene un camino de Euler si y solo si es conexo y el grado de entrada de cualquier vértice es igual a su grado de salida con la posible excepción de solo dos vértices. Para estos dos vértices el grado de entrada de uno de ellos es mayor que su grado de salida y el grado de entrada del otro es menor que su grado de salida.
\end{itemize}

 

\subsection{Algoritmo de Fleury}
El algoritmo de Fleury es un algoritmo elegante pero ineficiente que data de 1883. Considere un grafo que se sabe que tiene todas las aristas en el mismo componente y como máximo dos vértices de grado impar. El algoritmo comienza en un vértice de grado impar o, si el grafo no tiene ninguno, comienza con un vértice elegido arbitrariamente. En cada paso, elige la siguiente arista en la ruta para que sea uno cuya eliminación no desconecte el grafo, a menos que no exista tal arista, en cuyo caso elige la arista restante que queda en el vértice actual. Luego se mueve al otro extremo de esa arista y elimina la arista. Al final del algoritmo no quedan aristas, y la secuencia a partir de la cual se eligieron las aristas forma un ciclo euleriano si el gráfico no tiene vértices de grado impar, o un camino euleriano si hay exactamente dos vértices de grado impar.

Mientras que el recorrido del gráfico en el algoritmo de Fleury es lineal en el número de aristas, es decir O($E$), también debemos tener en cuenta la complejidad de detectar puentes . Si vamos a volver a ejecutar el algoritmo de búsqueda de puentes de tiempo lineal de Tarjan después de eliminar cada arista, el algoritmo de Fleury tendrá una complejidad de tiempo de O( $E ^{2}$). Un algoritmo dinámico de búsqueda de puentes de Thorup permite mejorar esto O($E\cdot \log ^{3}E\cdot \log \log E$), pero sigue siendo significativamente más lento que los algoritmos alternativos.

\subsection{Algoritmo de Hierholzer}
El artículo de 1873 de Hierholzer proporciona un método diferente para encontrar ciclos de Euler que sea más eficiente que el algoritmo de Fleury:

\begin{itemize}
	\item Elija cualquier vértice de inicio $v$ , y siga un rastro de aristas desde ese vértice hasta volver a $v$ . No es posible atascarse en ningún vértice que no sea $v$ , porque el grado par de todos los vértices asegura que, cuando el camino entre en otro vértice $w$ , debe haber una arista sin usar que salga de $w$ . El recorrido así formado es un recorrido cerrado, pero puede que no cubra todos los vértices y aristas del grafo inicial.
	\item Siempre que exista un vértice $u$ que pertenezca al recorrido actual pero que tenga aristas adyacentes que no formen parte del recorrido, se inicia otro recorrido desde $u$ , siguiendo aristas no utilizadas hasta volver a $u$ , y se une el recorrido así formado al anterior recorrido.
	\item Dado que asumimos que el grafo original está conectado , repetir el paso anterior agotará todos las aristas del grafo.
\end{itemize}

Mediante el uso de una estructura de datos como una lista doblemente enlazada para mantener el conjunto de aristas no utilizadas que inciden en cada vértice, para mantener la lista de vértices en el recorrido actual que tienen aristas no utilizadas y para mantener el recorrido en sí, las operaciones individuales del algoritmo (encontrar bordes no utilizados que salen de cada vértice, encontrar un nuevo vértice de inicio para un recorrido y conectar dos recorridos que comparten un vértice) se puede realizar en tiempo constante cada uno, por lo que el algoritmo general toma un tiempo lineal, O($E$).

Este algoritmo también se puede implementar con un deque. Debido a que solo es posible atascarse cuando el deque representa un recorrido cerrado, se debe rotar el deque eliminando los bordes de la cola y agregándolos a la cabeza hasta que se despegue, y luego continuar hasta que se tengan en cuenta todos los bordes. Esto también lleva un tiempo lineal, ya que el número de rotaciones realizadas nunca es mayor que  (Intuitivamente, cualquier mala arista se mueve a la cabeza, mientras que las aristas nuevas se agregan a la cola)

Para encontrar el camino euleriano/ciclo euleriano, podemos usar la siguiente estrategia: encontramos todos los ciclos simples y los combinamos en uno: este será el ciclo euleriano. Si el grafo es tal que la ruta euleriana no es un ciclo, agregue la arista que falta, encuentre el ciclo euleriano y luego elimine la arista adicional.

Buscar todos los ciclos y combinarlos se puede hacer con un procedimiento recursivo simple:

\begin{lstlisting}[language=C++]
procedimiento FindEulerPath(V)
  1. iterar a traves de todas las aristas que salen del vertice V;
	eliminar esta arista del grafo,
	y llame a FindEulerPath desde el segundo extremo de esta arista;
  2. agregue el vertice V a la respuesta.
	
\end{lstlisting}

Pero podemos escribir el mismo algoritmo en la versión no recursiva:

\begin{lstlisting}[language=C++]
pila St;
ponga el vertice inicial en St;
hasta que St este vacio
  sea V el valor en la parte superior de St;
  si grado(V) = 0, entonces
    suma V a la respuesta;
    quitar V de la parte superior de St;
  sino
    encuentre cualquier arista que salga de V;
    eliminarla del grafo;
    ponga el segundo extremo de esta arista en St;
\end{lstlisting}

Es fácil comprobar la equivalencia de estas dos formas del algoritmo. Sin embargo, la segunda forma es obviamente más rápida y el código será mucho más eficiente.

Primero, el función verifica el grado de los vértices: si no hay vértices con un grado impar, entonces 
el grafo tiene un ciclo de Euler, si hay $2$ vértices con un grado impar, entonces en el grafo solo hay 
un camino de Euler (pero no un ciclo de Euler), si hay más de $2$ tales vértices, entonces en el grafo 
no hay ciclo de Euler o camino de Euler. Para encontrar el camino de Euler (no un ciclo), hagamos esto: 
si $V1$ y $V2$ son dos vértices de grado impar, luego solo agregue una arista $(V1,V2)$, en el grafo 
resultante encontramos el ciclo de Euler (obviamente existirá), y luego eliminamos la arista 
\emph{ficticia} $(V1, V2)$ de la respuesta Buscaremos el ciclo de Euler exactamente como se describe 
arriba (versión no recursiva), y al mismo tiempo al final de este algoritmo verificaremos si el grafo 
estaba conectado o no (si el grafo no estaba conectado, entonces al final del algoritmo algunas aristas 
permanecerán en el grafo, y en este caso necesitamos imprimir $-1$). Finalmente, la función tiene en 
cuenta que pueden existir vértices aislados en el grafo.
