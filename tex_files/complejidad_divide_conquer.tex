En definitiva, el diseño Divide y Vencerás produce algoritmos recursivos cuyo
tiempo de ejecución  se puede expresar
mediante una ecuación en recurrencia del tipo:

$$
T(n) = 
\begin{cases}
	cn^k &\text{si  } 1 \le n < b  \\ 
	aT(n/b) + cn^k   &\text{s1  } n \ge b  
\end{cases}
$$

donde $a$, $c$ y $k$ son números reales, $n$ y $b$ son números naturales, y donde $a>0$, $c > 0$,
$k \ge 0$ y $b\ge1$. El valor de a representa el número de subproblemas, $n/b$ es el tamaño
de cada uno de ellos, y la expresión cn k representa el coste de descomponer el
problema inicial en los a subproblemas y el de combinar las soluciones para
producir la solución del problema original, o bien el de resolver un problema
elemental. La solución a esta ecuación es:

$$
T(n) \in
\begin{cases}
	\theta(n^k) &\text{si  }  a < b^k  \\ 
	\theta(n^k \log n) &\text{si  }  a = b^k  \\ 
	\theta(n^{\log_b a}) &\text{si  }  a > b^k 
\end{cases}
$$

Las diferencias surgen de los distintos valores que pueden tomar $a$ y $b$, que en
definitiva determinan el número de subproblemas y su tamaño. Lo importante es
observar que en todos los casos la complejidad es de orden polinómico o
polilogarítmico pero nunca exponencial, frente a los algoritmos recursivos que
pueden alcanzar esta complejidad en muchos casos. Esto se
debe normalmente a la repetición de los cálculos que se produce al existir
solapamiento en los subproblemas en los que se descompone el problema original.

Para aquellos problemas en los que la solución haya de construirse a partir de
las soluciones de subproblemas entre los que se produzca necesariamente
solapamiento existe otra técnica de diseño más apropiada, y que permite eliminar el
problema de la complejidad exponencial debida a la repetición de cálculos.
Estamos hablando de la \emph{Programación Dinámica}.