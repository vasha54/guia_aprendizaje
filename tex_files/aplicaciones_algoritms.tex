Es una pérdida de tiempo; como dicen, ¿por qué escribir código dos veces?.

Eso podría ser correcto en el caso de problemas simples y directos. Sin embargo, a medida que aumentan la complejidad y el tamaño del problema, comienzan a darse cuenta de que la generación de la representación del algoritmo facilita mucho la escritura del código real. Le ayuda a darse cuenta de posibles problemas o fallas de diseño en el algoritmo antes de la etapa de desarrollo.

Por lo tanto, ahorra más tiempo y esfuerzo en corregir errores y evitar errores. Además, el las diferentes representaciones de algoritmos permitió a los programadores comunicarse de manera más eficiente con otras personas de diferentes orígenes, ya que ofrece la idea del algoritmo sin la complejidad de las restricciones de sintaxis.

Una representanción del algoritmo claro, conciso y directo puede marcar una gran diferencia en el camino que va desde la idea hasta la implementación, un camino tranquilo para el programador. Es una de las herramientas generales subestimadas por la comunidad de programación pero, desafiante, debe utilizarse más.