\subsection{C++}

\subsubsection{Implementación clásica}
\begin{lstlisting}[language=C++]
int n;
vector<bool> is_prime(n+1, true);
is_prime[0] = is_prime[1] = false;
for (int i = 2; i <= n; i++) {
   if (is_prime[i] && (long long)i * i <= n) {
      for (int j = i * i; j <= n; j += i)
      is_prime[j] = false;
   }
}
\end{lstlisting}

\subsubsection{Iterar hasta la raíz}
\begin{lstlisting}[language=C++]
int n;
vector<bool> is_prime(n+1, true);
is_prime[0] = is_prime[1] = false;
for (int i = 2; i * i <= n; i++) {
   if (is_prime[i]) {
      for (int j = i * i; j <= n; j += i)
         is_prime[j] = false;
   }
}
\end{lstlisting}

\subsubsection{Iterar por bloques}
\begin{lstlisting}[language=C++]
int count_primes(int n) {
   const int S = 10000;
   vector<int> primes;
   int nsqrt = sqrt(n);
   vector<char> is_prime(nsqrt + 2, true);
   for (int i = 2; i <= nsqrt; i++) {
      if (is_prime[i]) {
         primes.push_back(i);
         for (int j = i * i; j <= nsqrt; j += i)
         is_prime[j] = false;
      }
   }
   
   int result = 0;
   vector<char> block(S);
   for (int k = 0; k * S <= n; k++) {
      fill(block.begin(), block.end(), true);
      int start = k * S;
      for (int p : primes) {
         int start_idx = (start + p - 1) / p;
         int j = max(start_idx, p) * p - start;
         for (; j < S; j += p)
            block[j] = false;
      }
      if (k == 0) block[0] = block[1] = false;
      for (int i = 0; i < S && start + i <= n; i++) {
         if (block[i]) result++;
      }
   }
   return result;
}
\end{lstlisting}

\subsubsection{Encontrar los primos en un rango}
\begin{lstlisting}[language=C++]
vector<char> segmentedSieve(long long L, long long R) {
// generar todos los numeros primos hasta sqrt(R)
   long long lim = sqrt(R);
   vector<char> mark(lim + 1, false);
   vector<long long> primes;
   for (long long i = 2; i <= lim; ++i) {
      if (!mark[i]) {
         primes.emplace_back(i);
         for (long long j = i * i; j <= lim; j += i) mark[j] = true;
      }
   }
   vector<char> isPrime(R - L + 1, true);
   for (long long i : primes)
      for (long long j = max(i * i, (L + i - 1) / i * i); j <= R; j += i)
         isPrime[j - L] = false;
   if(L == 1)
      isPrime[0] = false;
   return isPrime;
}
\end{lstlisting}

También es posible que no generemos previamente todos los números primos:

\begin{lstlisting}[language=C++]
vector<char> segmentedSieveNoPreGen(long long L, long long R) {
   vector<char> isPrime(R - L + 1, true);
   long long lim = sqrt(R);
   for (long long i = 2; i <= lim; ++i)
      for (long long j = max(i * i, (L + i - 1) / i * i); j <= R; j += i)
	     isPrime[j - L] = false;
   if (L == 1) isPrime[0] = false;
   return isPrime;
}	
\end{lstlisting}

\subsection{Java}


\subsubsection{Implementación clásica}
\begin{lstlisting}[language=Java]
int n;
boolean [] is_prime = new boolean [n+1];
Arrays.fill(is_prime,true);
is_prime[0] = is_prime[1] = false;
for (int i = 2; i <= n; i++) {
   if (is_prime[i] && i * i <= n) {
      for (int j = i * i; j <= n; j += i)
         is_prime[j] = false;
   }
}
\end{lstlisting}

\subsubsection{Iterar hasta la raíz}
\begin{lstlisting}[language=Java]
int n;
boolean[] is_prime= new boolean[n+1];
Arrays.fill(is_prime,true);
is_prime[0] = is_prime[1] = false;
for (int i = 2; i * i <= n; i++) {
   if (is_prime[i]) {
      for (int j = i * i; j <= n; j += i)
         is_prime[j] = false;
   }
}	
\end{lstlisting}


\subsubsection{Iterar por bloques}
\begin{lstlisting}[language=Java]
public static int count_primes(int n) {
   final int S = 10000;
   List<Integer> primes =new ArrayList<Integer>();
   int nsqrt = (int)Math.sqrt(n);
   char [] is_prime=new char [nsqrt + 2];
   Arrays.fill(is_prime,'T');
   for (int i = 2; i <= nsqrt; i++) {
      if (is_prime[i]=='T') {
         primes.add(i);
         for (int j = i * i; j <= nsqrt; j += i)
            is_prime[j] = 'F';
      }
   }
   int result = 0;
   char [] block =new char [S];
   for (int k = 0; k * S <= n; k++) {
      Arrays.fill(block, 'T');
      int start = k * S;
      for (int p : primes) {
         int start_idx = (start + p - 1) / p;
         int j = Math.max(start_idx, p) * p - start;
         for (; j < S; j += p)
            block[j] = 'F';
      }
      if (k == 0) block[0] = block[1] = 'F';
         for (int i = 0; i < S && start + i <= n; i++) {
            if (block[i]=='T') result++;
      }
   }
   return result;
}
\end{lstlisting}

\subsubsection{Encontrar los primos en un rango}
\begin{lstlisting}[language=Java]
public static char [] segmentedSieve(int L, int R) {
   // generar todos los numeros primos hasta sqrt(R)
   int lim = (int)Math.sqrt(R);
   char[] mark= new char[lim + 1];
   Arrays.fill(mark,'F');
   ArrayList<Integer> primes = new ArrayList<Integer>();
   for (int i = 2; i <= lim; ++i) {
      if (mark[i]=='F') {
         primes.add(i);
         for (int j = i * i; j <= lim; j += i) mark[j] = 'T';
      }
   }
   char [] isPrime = new char[R - L + 1];
   Arrays.fill(isPrime,'T');
   for (int i : primes)
      for (int j = Math.max(i * i, (L + i - 1) / i * i); j <= R; j += i)
         isPrime[j - L] = 'F';
   if(L == 1) isPrime[0] = 'F';
   return isPrime;
}
\end{lstlisting}

También es posible que no generemos previamente todos los números primos:

\begin{lstlisting}[language=Java]
public static char[] segmentedSieveNoPreGen(int L, int R) {
   char [] isPrime=new char[R - L + 1];
   Arrays.fill(isPrime,'T');
   int lim = (int)Math.sqrt(R);
   for (int i = 2; i <= lim; ++i)
      for (int j = Math.max(i * i, (L + i - 1) / i * i); j <= R; j += i) 
         isPrime[j - L] = 'F';
   if (L == 1) isPrime[0] = 'F';
   return isPrime;
}
\end{lstlisting}

La implementación clásica primero marca todos los números excepto el cero y el uno como números primos potenciales, luego comienza el proceso de cribado de números compuestos. Para esto itera sobre todos los números de $2$ a $n$. Si el número actual $i$ es un número primo, marca todos los números que son múltiplos de $i$ como números compuestos, a partir de $i^2$. Esto ya es una optimización sobre una forma ingenua de implementarlo y está permitido ya que todos los números más pequeños que son múltiplos de $i$ necesario también tener un factor primo que sea menor que $i$, por lo que todos ellos ya fueron tamizados antes. Desde $i^2$ puede desbordar fácilmente el tipo \textbf{int}, la verificación adicional se realiza usando el tipo \textbf{long long} antes del segundo bucle anidado.