El cálculo de la distancia de edición, también conocida como distancia de Levenshtein, tiene diversas aplicaciones en diferentes campos. Algunas de estas aplicaciones incluyen:

\begin{enumerate}
	\item \textbf{Corrección ortográfica:} La distancia de Levenshtein se utiliza para sugerir correcciones para palabras mal escritas. Se compara la palabra ingresada con las palabras en un diccionario y se sugiere la palabra más cercana en términos de distancia de edición.
	\item \textbf{Búsqueda de similitud:} La distancia de edición se utiliza para encontrar la similitud entre dos cadenas de texto. Esto puede ser útil en la clasificación o agrupación de documentos basados en su similitud textual.
	\item \textbf{Algoritmos de autocompletar:} La distancia de Levenshtein se puede utilizar para implementar algoritmos de autocompletar en campos como los motores de búsqueda o los sistemas de entrada de texto predictivo en dispositivos móviles.
	\item \textbf{Comparación de secuencias genéticas:} La distancia de edición se utiliza en bioinformática para comparar secuencias genéticas y determinar su similitud o diferencia. Esto es útil en el estudio de la evolución genética y la identificación de relaciones entre diferentes especies.
	\item \textbf{Corrección gramatical:} La distancia de Levenshtein se puede utilizar para detectar errores gramaticales en oraciones y sugerir correcciones gramaticales adecuadas.
\end{enumerate}