\subsection{Biblioteca o librería}
En C++, se conoce como librerías (o bibliotecas) a cierto tipo de archivos que podemos importar o incluir en nuestro programa. Estos archivos contienen las especificaciones de diferentes funcionalidades ya construidas y utilizables que podremos agregar a nuestro programa, como por ejemplo leer del teclado o mostrar algo por pantalla entre muchas otras más. 

La declaración de librerías, tanto en C como en C++, se debe hacer al principio de todo nuestro código, antes de la declaración de cualquier función o línea de código, debemos indicarle al compilador que librerías usar, para el saber que términos estaran correctos en la escritura de nuestro código y cuáles no. La sintaxis es la siguiente: \#include <nombre de la librería>

\subsection{Macro}
Los macros son muy utilizados en C y C++. Estos básicamente son un alias que podemos incluir en nuestro código el cual, al momento de compilar, sera reemplazado por lo que hayamos definido. Para la declarar una macro se utiliza la directiva \#define seguido de la definición de la macro. Esto puede ser útil si tienes que escribir varias veres el mismo código y quieres ahorrar tiempo de tipeo, o bien, si quieres que sea mas legible.