\subsection{C++}

\subsubsection{El número de primos relativos en un intervalo dado}
\begin{lstlisting}[language=C++]
int solve (int n, int r) {
   vector<int> p;
   for (int i=2; i*i<=n; ++i)
      if (n % i == 0) {
         p.push_back (i);
	     while (n % i == 0) n /= i;
      }
   if (n > 1)
   p.push_back (n);

  int sum = 0;
  for (int msk=1; msk<(1<<p.size()); ++msk) {
     int mult = 1,bits = 0;
     for (int i=0; i<(int)p.size(); ++i)
        if (msk & (1<<i)) {
           ++bits; mult *= p[i];
        }
     int cur = r / mult;
     if (bits % 2 == 1) sum += cur;
     else sum -= cur;
  }
  return r - sum;
}
\end{lstlisting} 

\subsubsection{El número de tripletes armónicos}
\begin{lstlisting}[language=C++]
int n;
bool good[MAXN];
int deg[MAXN], cnt[MAXN];

long long solve() {
   memset (good, 1, sizeof good);
   memset (deg, 0, sizeof deg);
   memset (cnt, 0, sizeof cnt);
	
   long long ans_bad = 0;
   for (int i=2; i<=n; ++i) {
      if (good[i]) {
         if (deg[i] == 0)  deg[i] = 1;
         for (int j=1; i*j<=n; ++j) {
            if (j > 1 && deg[i] == 1)
               if (j % i == 0) good[i*j] = false;
               else ++deg[i*j];
               cnt[i*j] += (n / i) * (deg[i]%2==1 ? +1 : -1);
            }
       }
       ans_bad += (cnt[i] - 1) * 1ll * (n-1 - cnt[i]);
    }
	return (n-1)*1ll*(n-2)*(n-3)/6- ans_bad/2;
}
\end{lstlisting}


\subsection{Java}

\subsubsection{El número de primos relativos en un intervalo dado}
\begin{lstlisting}[language=Java]
public int solve(int n, int r) {
   ArrayList<Integer> p = new ArrayList<Integer>();
   for (int i = 2; i * i <= n; ++i)
      if (n % i == 0) {
         p.add(i);
         while (n % i == 0) n /= i;
      }
   if (n > 1) p.add(n);
   int sum = 0;
   for (int msk = 1; msk < (1 << p.size()); ++msk) {
      int mult = 1, bits = 0;
      for (int i = 0; i < (int) p.size(); ++i)
         if ((msk & (1 << i)) == (1 << i)) {
           ++bits;
           mult *= p.get(i);
	     }
      int cur = r / mult;
      if (bits % 2 == 1) sum += cur;
      else sum -= cur;
   }
   return r - sum;
}
\end{lstlisting} 

\subsubsection{El número de tripletes armónicos}
\begin{lstlisting}[language=Java]
public long solve(int n) {
   boolean [] good = new boolean[n+1];
   int [] deg =new int [n+1];
   int [] cnt =new int [n+1];
	
   Arrays.fill(good, true); Arrays.fill(deg, 0);
   Arrays.fill(cnt, 0);
	
   long ans_bad = 0;
   for (int i=2; i<=n; ++i) {
      if (good[i]) {
         if (deg[i] == 0)  deg[i] = 1;
         for (int j=1; i*j<=n; ++j) {
            if (j > 1 && deg[i] == 1)
              if (j % i == 0) good[i*j] = false;
              else ++deg[i*j];
            cnt[i*j] += (n / i) * (deg[i]%2==1 ? +1 : -1);
         }
       }
       ans_bad += (cnt[i] - 1) * 1L * (n-1 - cnt[i]);
    }
    return (n-1)*1L*(n-2)*(n-3)/6- ans_bad/2;
}
\end{lstlisting}