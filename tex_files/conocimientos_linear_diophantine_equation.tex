\subsection{Inverso modular}
El inverso modular es un concepto importante en teoría de números y criptografía que se relaciona con la aritmética modular. En términos simples, el inverso modular de un número $a$ módulo $n$ es otro número $b$ tal que $(a * b) \equiv 1 (\bmod n)$. En otras palabras, el inverso modular de $a$ módulo $n$ es el número $b$ que, al multiplicarlo por $a$ y tomar el residuo módulo $n$, da como resultado 1.

La existencia de un inverso modular para un número $a$ módulo $n$ está condicionada a que $a$ sea coprimo con $n$, es decir, que el máximo común divisor de $a$ y $n$ sea igual a $1$ ($gcd(a, n) = 1$). En caso de que $a$ y $n$ no sean coprimos, no existe un inverso modular para $a$ módulo $n$.



\subsection{Números coprimos}

Dos números se consideran coprimos si su máximo común divisor (gcd) es igual a 1. En otras palabras, dos números son coprimos si no tienen ningún factor primo en común, excepto el 1. Por ejemplo, los números 15 y 28 son coprimos porque su único factor primo común es 1.