Los coeficientes binomiales $ \binom n k $ son la cantidad de formas de seleccionar un conjunto de elementos de $ K $ de $ n $ elementos diferentes sin tener en cuenta el orden de acuerdo de estos elementos (es decir, el número de conjuntos desordenados). Por ejemplo, $ \binom 5 3   = 10$, porque el conjunto $\{1, 2, 3, 4, 5\}$ tiene $10$ subconjuntos de $3$ elementos:

$$\{1, 2, 3\}, \{1, 2, 4\}, \{1, 2, 5\}, \{1, 3, 4\}, \{1, 3, 5\}, \{1, 4, 5\}, \{2, 3, 4\}, \{2, 3, 5\}, \{2, 4, 5\}, \{3, 4, 5\} $$

De como calcularlos y su utilizacion en la solución de problemas de programación competitiva abordará la siguiente guía.