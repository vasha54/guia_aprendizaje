La implementación de la búsqueda binaria tiene dos variantes una recursiva y otra iterativa. En este caso vamos a presentar la iterativa por ser mas conveniente para lo que deseamos utilizar.

\subsection{C++}
\lstset{language=C++, breaklines=true, basicstyle=\footnotesize}
\begin{lstlisting}[language=C++]

#define MAX_N 1000001

int n, x;
int niz[MAX_N];


inline int b_search(int left, int right, int x){
   int i = left;
   int j = right;
   while (i < j){
      int mid = (i+j)/2;
      if (niz[mid] == x) return mid;
      if (niz[mid] < x) i = mid+1;
      else j = mid-1;
   }
   if (niz[i] == x) return i;
   return -1;
}
\end{lstlisting}

\subsection{Java}
\begin{lstlisting}[language=Java]
/* Busqueda de un elemento en un arreglo
* Retorna -1 si el elemento no esta si no la posicion del elemento*/

int binarySearch(int arr[], int x) {
   int l = 0, r = arr.length - 1;
   while (l <= r) {
      int m = l + (r - l) / 2;
      if (arr[m] == x) {
         return m;
      }
		
      if (arr[m] < x) {
         l = m + 1;
      } else {
         r = m - 1;
      }
   }
   return -1;
}

/*las clases Arrays y Colections poseen el metodo busqueda binaria 
* para buscar un elemento bien sea dentro de un arreglo o coleccion, 
* en ambos caso devuelve la posicion donde se encuentra el elemento 
* buscado. En caso de no encontrarse el valor devuelto es -1. El arreglo
* o colecion debe esta ordenado de forma ascendente  previamente */

\end{lstlisting}