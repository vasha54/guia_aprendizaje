\subsection{C++}
\begin{lstlisting}[language=C++]
int n; //cantidad de nodos del grafo
vector< vector<int> > adj; // lista de adyacencia para representar el grafo
vector<bool> used;
vector<int> comp;

void dfs(int v) {
   stack<int> st;
   st.push(v);
	
   while (!st.empty()) {
      int curr = st.top();
      st.pop();
      if (!used[curr]) {
         used[curr] = true;
         comp.push_back(curr);
         for(int i = adj[curr].size()-1;i>=0;i--){
            st.push(adj[curr][i]);
         }
      }
   }
}

void find_comps() {
   fill(used.begin(), used.end(), 0);
   for (int v = 0; v < n ; ++v) {
      if (!used[v]) {
         comp.clear();
         dfs(v);
         cout << "Los nodos de esta componente son:" ;
         for (int u : comp)
            cout << ' ' << u;
         cout << endl ;
      }
   }
}
\end{lstlisting}


La función más importante que se utiliza es \emph{find\_comps()} la que encuentra y muestra los componentes conectados en el grafo. El grafo se almacena en representación de lista de adyacencia, es decir, $adj[v]$ contiene una lista de vértices que tienen aristas desde el vértice $v$. El vector $comp$ contiene una lista de nodos en el componente conectado actual.




\subsection{Java}

\begin{lstlisting}[language=Java]
private class Graph {
  private int nodes;
  private List<List<Integer>> ady;
	
  public Graph(int n) {
     this.nodes = n;
     this.ady = new ArrayList<List<Integer>>();
		
     for (int i = 0; i < this.nodes; i++)
			this.ady.add(new ArrayList<Integer>());
  }
	
  public void addEdge(int a, int b) {
     this.ady.get(a).add(b); this.ady.get(b).add(a);
  }
	
  public void dfs(int v, boolean[] used, ArrayList<Integer> comp) {
     Stack<Integer> st = new Stack<Integer>();
     st.push(v);
		
     while (!st.empty()) {
        int curr = st.pop();
        if (!used[curr]) {
           used[curr] = true; comp.add(curr);
           for (int i = this.ady.get(curr).size() - 1; i >= 0; i--) 
              st.push(this.ady.get(curr).get(i));
        }
	 }
  }
	
  public void find_comps() {
     boolean[] used = new boolean[this.nodes];
     Arrays.fill(used, false);
     ArrayList<Integer> comp = new ArrayList<Integer>();
     for (int v = 0; v < this.nodes; ++v) {
        if (!used[v]) {
           comp.clear(); dfs(v, used, comp);
           System.out.printf("Los nodos de esta componente son:\n");
           for (Integer u : comp) System.out.printf(" %d", u);
           System.out.printf("\n");
        }
      }
   }
}
\end{lstlisting}	