Los operadores relacionales sirven para
 realizar comparaciones de igualdad,
 desigualdad y relación de menor o mayor.
El resultado de estos operadores es
siempre un valor boolean (true o false)
 según se cumpla o no la relación
considerada.

\begin{tabular}{|c|p{3cm}|p{9cm}|}
	\hline
	\textbf{Operador}	& \textbf{Utilización} &  \textbf{El resultado es verdadero (true)} \\
	\hline
	> & op1 > op2 & si op1 es mayor que op2 \\
	\hline
	>= & op1 >= op2 & si op1 es mayor o igual que op2 \\
	\hline
	< & op1 < op2 & si op1 es menor que op2 \\
	\hline
	<= & op1 <= op2 & si op1 es menor o igual que op2 \\
	\hline
	== & op1 == op2 & si op1 y op2 son iguales \\
	\hline
	!= & op1 != op2 & si op1 y op2 son diferentes \\
	\hline
\end{tabular}

Todos los operadores relacionales son operadores binarios (tienen dos operandos), y su
forma general es la siguiente:

\begin{lstlisting}[language=C++]
expresion1 op expresion2
\end{lstlisting}

donde \textbf{op} es uno de los operadores (==, <, >, <=, >=, !=). El funcionamiento de
estos operadores es el siguiente: se evalúan \textbf{expresion1} y \textbf{expresion2}, y se comparan los
valores resultantes. Si la condición representada por el operador relacional se cumple, el
resultado es verdadero (true,1); si la condición no se cumple, el resultado es falso (false,0).

\subsubsection{Prioridad de los Operadores Relacionales}
Todos los operadores relacionales tienen el mismo nivel de prioridad en su evaluación. En general, los operadores relacionales tienen menor prioridad que los aritméticos. 