La impletemación de esta estructura es similar en Java y C++ la única direrencia es que en Java permite que la variable que sea utilizada en switch sea de tipo string algo que no sucede en C++.

\subsection{Java}
\begin{lstlisting}[language=Java]
//determina cual calificacion se introdujo
switch ( calificacion ) //instruccion switch
{
   case 'A': //calificacion fue A mayuscula
   case 'a': //or a minuscula
      aCuenta++; // incrementa aCuenta
      break; // es necesario salir del switch
   case 'B': // calificacion fue B mayuscula
   case 'b': // o b minuscula
      bCuenta++; // incrementa bCuenta
      break; // sale del switch
   case 'C': //calificacion fue C mayuscula
   case 'c': // o c minuscula
      cCuenta++; // incrementa cCuenta
      break; // sale del switch
   case 'D': //calificacion fue D mayuscula
   case 'd': //o d minuscula
      dCuenta++; // incrementa dCuenta
      break; // sale del switch
   case 'F': // calificacion fue F mayuscula
   case 'f': // o f minuscula
      fCuenta++; // incrementa fCuenta
      break; // sale del switch
   case '\n': // ignora caracteres de nueva linea,
   case '\t': // tabuladores
   case ' ': // y espacios en la entrada
      break; // sale del switch
   default: // atrapa todos los demas caracteres
      cout << "Se introdujo una letra de calificacion incorrecta." << endl;
      break; // opcional; saldra del switch de todas formas
} // fin de switch	
\end{lstlisting}


\subsection{C++}
\begin{lstlisting}[language=C++]
//determina cual calificacion se introdujo
switch ( calificacion ) //instruccion switch
{
   case 'A': //calificacion fue A mayuscula
   case 'a': //or a minuscula
      aCuenta++; // incrementa aCuenta
      break; // es necesario salir del switch
   case 'B': // calificacion fue B mayuscula
   case 'b': // o b minuscula
      bCuenta++; // incrementa bCuenta
      break; // sale del switch
   case 'C': //calificacion fue C mayuscula
   case 'c': // o c minuscula
      cCuenta++; // incrementa cCuenta
      break; // sale del switch
   case 'D': //calificacion fue D mayuscula
   case 'd': //o d minuscula
      dCuenta++; // incrementa dCuenta
      break; // sale del switch
   case 'F': // calificacion fue F mayuscula
   case 'f': // o f minuscula
      fCuenta++; // incrementa fCuenta
      break; // sale del switch
   case '\n': // ignora caracteres de nueva linea,
   case '\t': // tabuladores
   case ' ': // y espacios en la entrada
      break; // sale del switch
   default: // atrapa todos los demas caracteres
      cout << "Se introdujo una letra de calificacion incorrecta." << endl;
      break; // opcional; saldra del switch de todas formas
} // fin de switch	
\end{lstlisting}

