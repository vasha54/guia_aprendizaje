\subsection{Árbol de búsqueda binario equilibrado}
Un árbol de búsqueda binario equilibrado, también conocido como árbol AVL (por las iniciales de los inventores Adelson-Velsky y Landis), es una estructura de datos en forma de árbol binario de búsqueda en la que se garantiza que la diferencia de alturas entre los subárboles izquierdo y derecho de cada nodo no sea mayor que 1. Esto asegura que el árbol esté equilibrado y mantiene su eficiencia en términos de tiempo de búsqueda, inserción y eliminación.


\subsection{Árboles rojo-negro}
Un árbol rojo-negro es otra estructura de datos en forma de árbol binario de búsqueda que se utiliza para mantener el equilibrio y garantizar un rendimiento eficiente en términos de tiempo de búsqueda, inserción y eliminación. Los árboles rojo-negro se caracterizan por tener nodos que son rojos o negros, y cumplen con ciertas reglas para mantener el equilibrio del árbol.

Las reglas fundamentales de un árbol rojo-negro son las siguientes:

\begin{enumerate}
	\item Cada nodo es rojo o negro.
	\item La raíz del árbol siempre es negra. 
	\item Todos los nodos hoja (nodos nulos) son negros. 
	\item Si un nodo es rojo, entonces sus hijos deben ser negros. 
	\item Para cualquier nodo dado, todos los caminos desde ese nodo hasta sus nodos hoja descendientes contienen el mismo número de nodos negros. 
\end{enumerate}

Estas reglas garantizan que la altura negra de cualquier camino desde la raíz hasta un nodo hoja sea la misma, lo que mantiene el equilibrio del árbol y asegura un rendimiento eficiente en las operaciones de búsqueda, inserción y eliminación.

Durante las operaciones de inserción y eliminación en un árbol rojo-negro, se aplican rotaciones y cambios de colores en los nodos para mantener las propiedades del árbol. Estas operaciones se realizan de manera similar a los árboles AVL, pero con reglas específicas para los colores de los nodos.


\subsection{PBDS}

PBDS es una abreviatura que puede referirse a \emph{Policy-Based Data Structures} en el contexto de la biblioteca de estructuras de datos estándar de C++ (STL). Las PBDS son una extensión de la STL que proporciona una forma flexible de definir estructuras de datos con políticas personalizadas para operaciones como inserción, eliminación, búsqueda, etc. Esto permite a los desarrolladores adaptar las estructuras de datos a sus necesidades específicas mediante la especificación de políticas en lugar de implementar estructuras de datos personalizadas desde cero.

En el caso de los árboles rojo-negro, la biblioteca PBDS en C++ ofrece una implementación eficiente y flexible de árboles rojo-negro que se pueden personalizar con políticas específicas según los requisitos del usuario. Esta característica hace que sea más fácil y conveniente trabajar con estructuras de datos complejas como los árboles rojo-negro en C++, ya que se pueden adaptar a diferentes escenarios y requisitos sin tener que reescribir toda la estructura desde cero.

