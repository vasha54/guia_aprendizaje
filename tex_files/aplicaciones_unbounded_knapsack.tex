Este es de los problemas de la programación dinámica, uno de los denomindados como clásicos. Aunque los problemas de tipo programación dinámica son muy popular con una alta frecuencia de aparición en concursos de programación
recientes, los problemas clásicos de programación dinámica en su forma pura (como el presentado en esta guía) por lo general ya no aparecen en los IOI o ICPC modernos. A pesar de esto es necesario su estudio ya
que nos permite entender la programación dinámica y como poder resolver aquellos problema de programación dinámica clasificados como no-clásicos e incluso nos permite desarrollar nuestras habilidades de programación
dinámica en el proceso.

El problema de la mochila sin límites tiene diversas aplicaciones en la vida real, entre las cuales se incluyen:

\begin{enumerate}
	\item \textbf{Gestión de inventario:} La solución de Unbounded Knapsack se puede utilizar para optimizar la gestión de inventario en empresas. Por ejemplo, si una tienda tiene varios productos con diferentes valores y cantidades disponibles, se puede utilizar este algoritmo para determinar la combinación óptima de productos que generen el mayor beneficio.
	
	\item \textbf{Planificación de proyectos:} En la gestión de proyectos, es común tener recursos limitados y tareas que requieren diferentes cantidades de esos recursos. \emph{Unbounded Knapsack} puede ayudar a determinar la asignación óptima de recursos a las tareas para maximizar el valor del proyecto.
	
	\item \textbf{Optimización de publicidad en línea:} En la publicidad en línea, los anunciantes a menudo tienen un presupuesto limitado y diferentes opciones de anuncios con diferentes tasas de conversión y costos. \emph{Unbounded Knapsack} se puede utilizar para seleccionar la combinación de anuncios que maximice el retorno de la inversión dentro del presupuesto asignado.
	
	\item \textbf{Selección de cartera de inversiones: }En el ámbito de las finanzas, los inversores a menudo necesitan seleccionar una cartera de inversiones que maximice el rendimiento esperado dado un conjunto de activos disponibles. \emph{Unbounded Knapsack} puede ayudar a determinar la combinación óptima de activos para maximizar el rendimiento esperado.
\end{enumerate}

Estas son solo algunas de las aplicaciones comunes de \emph{Unbounded Knapsack}, pero existen muchas otras áreas donde este problema de optimización puede ser utilizado para mejorar la eficiencia y la toma de decisiones.