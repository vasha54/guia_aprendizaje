En matemáticas, la aritmética modular (es una categoría especial de aritmética que utiliza solo enteros. En otras palabras, la aritmética modular es la aritmética de la congruencia. La aritmética modular a veces se conoce como aritmética de reloj, ya que uno de los usos más conocidos de la aritmética modular es el reloj de 12 horas, que tiene el período de tiempo dividido en dos mitades iguales.

En su libro \emph{Disquistiones Arithmeticae} publicado en 1801, Carl Friedrich Gauss introdujo el enfoque moderno de la aritmética modular. Según las matemáticas, la aritmética modular se considera la aritmética de cualquier imagen homomórfica no trivial del anillo de enteros. En aritmética modular, los números que se tratan son solo enteros y las operaciones que se usan son solo suma, resta, multiplicación y división. En aritmética modular, los números se envuelven o redondean al alcanzar un cierto valor, haciendo uso del módulo. En esta forma de aritmética, se consideran los restos. La aritmética modular generalmente se asocia con números primos. Dos números se consideran equivalentes, el resto de ambos números dividido por un número único es igual.

\subsection{Suma (a+b) \% m}
Probaremos que\textbf{ (A + B) mod C = (A mod C + B mod C) mod C}\\
\\
Debemos mostrar que LI=LD\\
\\
A partir del teorema del cociente y del residuo podemos escribir A y B como:\\
A = C * Q1 + R1 donde 0 $\leq$ R1 < C y Q1 son enteros. A mod C = R1\\
B = C * Q2 + R2 donde 0  $\leq$ R2 < C y Q2 son enteros. B mod C = R2\\
\\
(A + B) = C * (Q1 + Q2) + R1+R2\\
LI = (A + B) mod C\\
LI = (C * (Q1 + Q2) + R1+ R2) mod C\\
\\
Podemos eliminar los múltiplos de C cuando tomamos mod C.\\
LI = (R1 + R2) mod C\\
LD = (A mod C + B mod C) mod C\\
LD = (R1 + R2) mod C\\
LI=LD= (R1 + R2) mod C\\

\subsection{Multiplicación (a*b) \% m}

Probaremos que \textbf{(A * B) mod C = (A mod C * B mod C) mod C}\\
\\
Debemos mostrar que LI = LD\\
\\
A partir del teorema del cociente y del residuo podemos escribir A y B como:\\
A = C * C1 + R1 donde 0 $\leq$ R1 <  C y C1 es un integral. A módulo C = R1\\
B = C * C2 + R2 donde 0 $\leq$ R2 <  C y C2 es un integral. B módulo C = R2\\
\\
LI = (A * B) mod C\\
LI = ((C * C1 + R1 ) * (C * C2 + R2) ) mod C\\
LI = (C * C * C1 * C2 + C * C1 * R2 + C * C2 * R1 + R1 * R2 )  mod C\\
LI = (C * (C * C1 * C2 + C1 * R2 + C2 * R1)  + R1 * R2 )  mod C\\
\\
Podemos eliminar los múltiplos de C cuando tomamos el módulo C:\\
LI = (R1 * R2) mod C\\
\\
Ahora hagamos el LD\\
LD = (A mod C * B mod C) mod C\\
LI = (R1 * R2 ) mod C\\
\\
Por lo tanto, LD = LI\\
LI = LD = (R1 * R2 ) mod C\\
\subsection{Resta (a-b) \% m}

La desmostración de la operación de la resta es similar a la suma pero con un detalle importante el $b \mod m $ puede ser mayor que el $a \mod m$ por lo que quedaría un valor negativo lo cual no sería correcto pero haciendo un ajuste de corrimiento de una vuelta completa del anillo que se genera con los posibles restos de  $\mod m$ se resuelve el problema por lo que podemos concluir que :

 {\centering\textbf{ (A - B) mod C = (A mod C - B mod C + C) mod C}}
 
 Note que se le suma el valor de $C$ para provocar la vuelta al anillo.
 
\subsection{División (a/b) \% m}

La división la vamos a trabajar como una mulplicación con el siguiente proceso partiendo de lo siguiente:
\\
$$\frac{a}{b} \mod m$$

vamos a plantear la división como una multiplicación de la siguiente manera:
$$(a \times \frac{1}{b}) \mod m$$\\ 
$$(a \times b^{-1}) \mod m$$

Aplicando lo conocido de la multiplicación nos quedaría de la siguiente manera:
$$(a \mod m \times b^{-1} \mod m ) \mod m$$

Con el primer término no debe existir duda pero con el segundo término $b^{-1} \mod m$ nos queda ver como operar. Si estudiamos un poco de congruencia nos podemos percatar que podemos sustituir el término $b^{-1}$ por su inverso multiplicativo al cual le vamos a dedicar un espacio a continuación.
 
\subsubsection{Inverso Multiplicativo Modular } 
Un inverso multiplicativo modular de un entero $a$ es un entero $x$ tal que  $a\cdot x$ es congruente con $1$ modular algún módulo $m$. Para escribirlo 
de manera formal: queremos encontrar un número entero $x$ de modo que $$a 
\cdot x \equiv 1 \mod m.$$ También denotaremos $x$ simplemente con $a^{-1}$.


Debemos tener en cuenta que el inverso modular no siempre existe. Por ejemplo, deja $m=4$, $a=2$. Comprobando todos los valores posibles 
módulo $m$ debe quedar claro que no podemos encontrar $a^{-1}$ satisfaciendo la ecuación anterior. Se puede probar que el inverso modular 
existe si y solo si $a$ y $m$ son relativamente primos (es decir $\gcd(a, m) = 1$).

Presentamos dos métodos para encontrar el inverso modular en caso de que exista, y un método para encontrar el inverso modular para todos los números en tiempo lineal.

\subsubsection{Encontrar el inverso modular usando el algoritmo euclidiano extendido}

Considere la siguiente ecuación (con incógnita $x$ y $y$):

$$a \cdot x + m \cdot y = 1$$

Esta es una ecuación Diofántica Lineal en dos variables. Cuando $\gcd(a, m) = 1$, la ecuación tiene una solución que se puede encontrar utilizando el algoritmo euclidiano extendido . Tenga en cuenta que $\gcd(a, m) = 1$ es también la condición para que exista el inverso modular.

Ahora bien, si tomamos módulo $m$ de ambos lados, podemos deshacernos de $m \cdot y$, y la ecuación se convierte en:

$$a \cdot x \equiv 1 \mod m$$

Así, el inverso modular de $a$ es $x$.

\subsubsection{Encontrar el inverso modular usando exponenciación binaria}

Otro método para encontrar el inverso modular es usar el teorema de Euler, que establece que la siguiente congruencia es verdadera si $a$ y $m$ son relativamente primos:

$$a^{\phi (m)} \equiv 1 \mod m$$

$\phi$ es la función Totient de Euler . De nuevo, tenga en cuenta que $a$ y $m$ ser primo relativo también era la condición para que existiera el inverso modular.

Si $m$ es un número primo, esto se simplifica al pequeño teorema de Fermat :

$$a^{m - 1} \equiv 1 \mod m$$

Multiplica ambos lados de las ecuaciones anteriores por $a^{-1}$ , y obtenemos:

\begin{itemize}
	\item Para un módulo arbitrario (pero coprimo) $m$:$a^{\phi(m)-1}\equiv a ^{-1}\mod m$
	\item Para un módulo primo $m$:$a^{m-2} \equiv a^{-1}\mod m$
\end{itemize}

A partir de estos resultados, podemos encontrar fácilmente el inverso modular usando el algoritmo de exponenciación binaria , que funciona en $O(\log m)$ tiempo.

Aunque este método es más fácil de entender que el método descrito en el párrafo anterior, en el caso de que $m$ no es un número primo, necesitamos calcular la función phi de Euler, que implica la factorización de $m$, que puede ser muy difícil. Si la factorización prima de $m$ se conoce, entonces la complejidad de este método es $O(\log m)$.

\subsubsection{Encontrar el inverso modular usando la división euclidiana}

Dado que $m>i$ (o podemos modular para hacerlo más pequeño en 1 paso), según la división euclidiana

$$m = k \cdot i + r$$

dónde $k= \left\lfloor \frac{m}{i} \right\rfloor$ y $r=m \bmod i$, entonces


\begin{align*}
	& \implies & 0          & \equiv k \cdot i + r   & \mod m \\
	& \iff & r              & \equiv -k \cdot i      & \mod m \\
	& \iff & r \cdot i^{-1} & \equiv -k              & \mod m \\
	& \iff & i^{-1}         & \equiv -k \cdot r^{-1} & \mod m
\end{align*}


Una vez visto las formas de encontrar el inverso multiplicativo modular solo nos queda sustituir donde está $b^{-1}$ por su inverso multiplicativo al cual llamaremos $B$ y de esta forma la expresión $(a/b) \% m$ se sustituye por $(a mod m * B mod m) mod m$ donde $B$ es el inverso multiplicativo entre de $b$ y $m$ 

