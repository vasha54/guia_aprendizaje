Una matriz es un concepto matemático que corresponde a un arreglo bidimensional en programación. Por ejemplo:

$$
  A = \begin{bmatrix}
  6	& 13 & 7 & 4 \\
  7	& 0 & 8 & 2 \\
  9	& 5 & 4 & 18
\end{bmatrix}$$

es una matriz de tamaño $3 \times 4$, es decir, tiene $3$ filas y $4$ columnas. La notación $[i, j]$ se refiere al elemento en la fila $i$ y la columna $j$ en una matriz. Por ejemplo, en la anterior matriz, $A [2, 3] = 8$ y $A [3, 1] = 9$.

Un caso especial de matriz es un vector que es una matriz unidimensional de tamaño $n \times 1$. Por ejemplo,

$$
V = \begin{bmatrix}
	4 \\
	7 \\
	5
\end{bmatrix}$$

es un vector que contiene tres elementos. De las diferentes operaciones que se pueden llevar a cabo con este concepto matemático y su implicaciones en la programación competitiva abordaremos la siguiente guía.