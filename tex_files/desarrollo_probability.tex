Una probabilidad es un número real entre $0$ y $1$ que indica qué tan probable es un evento. Si es seguro que ocurrirá un evento, su probabilidad es $1$, y si es imposible, su probabilidad es $0$. La probabilidad de un evento se denota P($\dots$) donde los tres puntos describen el evento.

Por ejemplo, al lanzar un dado, el resultado es un número entero entre 1 y 6 y la probabilidad de cada resultado es 1/6. Por ejemplo, podemos calcular las siguientes probabilidades:

\begin{itemize}
	\item P(el resultado es 4) = 1/6
	\item P(el resultado no es 6) = 5/6
	\item P(el resultado es par) = 1/2
\end{itemize}

\subsection{Cálculo}

Para calcular la probabilidad de un evento, podemos usar combinatoria o simular el proceso que genera el evento. Como ejemplo, calculemos la probabilidad de sacar tres cartas del mismo valor de una baraja de cartas barajadas (por ejemplo $\spadesuit$8, $\clubsuit$8 y $\diamondsuit$8).

\subsubsection{Método 1}
Podemos calcular la probabilidad usando la fórmula.

$$\dfrac{\text{número de resultados deseados}}{\text{número total de resultados}}$$

En este problema, los resultados deseados son aquellos en los que el valor de
cada carta es el mismo. Hay $13\dfrac{4}{3}$ de esos resultados, porque hay 13 posibilidades por el valor de las cartas y $\dfrac{4}{3}$ formas de elegir $3$ palos entre $4$ palos posibles. 

Hay un total de $\dfrac{52}{3}$ resultados, porque elegimos 3 cartas de 52 cartas. Por tanto, la probabilidad del evento es

$$ \dfrac{13\dfrac{4}{3}}{\dfrac{52}{3}} = \dfrac{1}{425} $$  

\subsubsection{Método 2}
Otra forma de calcular la probabilidad es simular el proceso que genera el evento. En este ejemplo, robamos tres cartas, por lo que el proceso consta de tres pasos. Requerimos que cada paso del proceso sea exitoso.

Sin duda, sacar la primera carta es un éxito, porque no hay restricciones. El segundo paso tiene éxito con probabilidad 3/51, porque quedan 51 cartas. y 3 de ellas tienen el mismo valor que la primera carta. De manera similar, el tercero el paso tiene éxito con una probabilidad de 2/50.

La probabilidad de que todo el proceso tenga éxito es:

$$1 \times \dfrac{3}{51} \times \dfrac{2}{50} = \dfrac{1}{425} $$

\subsection{Eventos}
Un evento en la teoría de la probabilidad se puede representar como un conjunto.

$$A \subset X$$

donde X contiene todos los resultados posibles y A es un subconjunto de
resultados. Por ejemplo, al tirar un dado, los resultados son:

$$X = \{~1,~2,~3,~4,~5,~6\}$$

Ahora, por ejemplo, el evento \textbf{el resultado es par} corresponde al conjunto

$$A = \{~2,~4,~6\}$$

A cada resultado $x$ se le asigna una probabilidad $p(x)$. Entonces, la probabilidad $P(A)$
de un evento $A$ se puede calcular como una suma de probabilidades de resultados usando
la fórmula 

$$P(A) = \sum_{x \in A}^{} p(x)$$

Por ejemplo, al lanzar un dado, $p(x) = \dfrac{1}{6}$ para cada resultado $x$, por lo que la
probabilidad del evento \textbf{el resultado es par} es

$$p(2)+ p(4)+ p(6) = 
\dfrac{1}{2}$$

La probabilidad total de los resultados en $X$ debe ser $1$, es decir, $P(X)$ = $1$.

Dado que los eventos en la teoría de la probabilidad son conjuntos, podemos manipularlos usando operaciones de conjunto



\subsubsection{Complemento}
El complemento $\bar{A}$ significa \textbf{A no sucede}. Por ejemplo, cuando
al lanzar un dado, el complemento de $A = \{~2,~4,~6\}$ es $\bar{A} = \{~1,~3,~5\}$. La probabilidad del complemento $\bar{A}$ se calcula mediante la fórmula:

$$P(\bar{A})=1-P(A)$$

A veces, podemos resolver un problema fácilmente usando complementos resolviendo el
problema opuesto. Por ejemplo, la probabilidad de obtener al menos un seis al lanzar un dado diez veces es:

$$1 - \dfrac{5}{6}\times 10 $$

Aquí $\dfrac{5}{6}$ es la probabilidad de que el resultado de un solo lanzamiento no sea seis, y $\dfrac{5}{6}\times 10$ es la probabilidad de que ninguno de los diez lanzamientos sea seis. El complemento de esto es la respuesta al problema.

\subsubsection{Union}
La unión $A \cup B$ significa \textbf{A o B suceden}. Por ejemplo, la unión de
$A=\{~2,~5\}$ y $B=\{~4,~5,~6\}$ es $A \cup B=\{~2,~4,~5,~6\}$. La probabilidad de 
la unión $A \cup B$ se calcula mediante la fórmula:
	
	$$P(A \cup B) = P(A)+P(B) - P(A \cap B)$$
	
Por ejemplo, al lanzar un dado, la unión de los eventos $A=\text{el resultado es par}$ y $B=\text{el resultado es inferior a 4}$ es $A \cup B=\text{el resultado es par o menor que 4}$   y su probabilidad es:

 $$P(A \cup B) = P(A)+P(B) - P(A \cap B) = \dfrac{1}{2} + \dfrac{1}{2} - \dfrac{1}{6} = \dfrac{5}{6}$$
 
Si los eventos $A$ y $B$ son disjuntos, es decir, $A \cap B$ está vacío, la probabilidad del evento $A \cup B$ es simplemente:

$$P(A \cup B) = P(A)+P(B)$$

\subsubsection{La probabilidad condicional}
La probabilidad condicional 

$$P(A|B) = \dfrac{P(A \cap B)}{P(B)}$$

es la probabilidad de que $A$ suponga que $B$ sucede. Por lo tanto, al calcular la probabilidad de $A$, solo consideramos los resultados que también pertenecen a $B$.

Usando los conjuntos anteriores $P(A|B) = \dfrac{1}{3}$ porque los resultados de $B$ son $\{~1,~2,~3\}$ y uno de ellos es par. Esta es la probabilidad de un resultado par si sabemos que el resultado está entre $1 \dots 3$.


\textbf{Las reglas básicas de la probabilidad condicional:}

\begin{itemize}
	\item $P(A \cap B) = P(A)\cdot P(B|A)$, cuando $A$ y $B$ son independientes $P(A \cap B) = P(A)\cdot P(B)$
	\item $P(A \cap B \cap C) = P(A)\cdot P(B|A) \cdot P(C|A \cap B)$, y así sucesivamente.
	\item $P(A) = P(A|B_1) \cdot P(B_1) + \dots + P(A|B_k) \cdot P(B_k)$, cuando $B_1, \dots, B_k$ son disjuntos y exhaustivos.
\end{itemize}



\subsubsection{Intersección}
La intersección $A \cap B$ significa \textbf{A y B suceden}. Por ejemplo, la intersección de $A = \{~2,~5\}$ y $B = \{~4,~5,~6\}$ es $A \cap B = \{5\}$. Usando probabilidad condicional, la probabilidad de la intersección $A \cap B$ se puede calcular usando la fórmula 

$$P(A\cap B)=P(A)P(B|A)$$

Los eventos $A$ y $B$ son independientes si $P(B|A) = P(A)$ y $P(A|B) = P(B)$, lo que significa que el hecho de que ocurra $B$ no cambia la probabilidad de que
ocurra $A$, y viceversa. En este caso, la probabilidad de la intersección es 

$$P(A \cap B)=P(A)P(B)$$

Por ejemplo, al robar una carta de una baraja, los eventos $A = \text{el palo es de tréboles}$ y $B=\text{el valor es cuatro}$ son independientes. De ahí el evento $A\cap B = \text{la carta es el cuatro de tréboles}$ sucede con probabilidad $$P(A \cap B) = P(A)P(B) = \dfrac{1}{4} \times \dfrac{1}{13} = \dfrac{1}{52}$$


\textbf{Las reglas básicas de probabilidad:}

\begin{itemize}
	\item $P(A) + P(\bar{A}) = 1$
	\item $P(A \cap B) + P(A \cap \bar{B}) = P(A)$
	\item $P(A \cup B) = 1 - P(\bar{A} \cap \bar{B})$
	\item $P(A \cup B) = P(A) + P(B) - P(A \cap B)$ 
\end{itemize} 






