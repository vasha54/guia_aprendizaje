\subsection{C++}
\subsubsection{Máximo camino}

\subsubsection{Cantidad de caminos}
\begin{lstlisting}[language=C++]

//Recursividad

int uniquePathHelper(int i, int j, int r, int c, vector<vector<int>>& A){
   if(i == r || j == c) return 0 ;
	
   if(A[i][j] == 1) return 0 ;
	
   if(i == r-1 && j == c-1) return 1 ;
	
   int left = uniquePathHelper(i+1,j,r,c,A);
   int rigth = uniquePathHelper(i,j+1,r,c,A); 	
   return  left+ rigth;
}

/*
* Recibe una matriz donde el valor 1 significa obstaculo
* y 0 significa libre
*/
int countPathsRecursive(vector<vector<int>>& A){
   int r = A.size(), c = A[0].size();
   return uniquePathHelper(0,0,r,c,A);
}

//Top-Down
int uniquePathHelper(int i,int j,int r,int c,
					vector<vector<int> >& A,vector<vector<int> >& paths){
   if (i == r || j == c) return 0;
   if (A[i][j] == 1) return 0;
   if (i == r - 1 && j == c - 1) return 1;
	
   if (paths[i][j] != -1) return paths[i][j];
   
   int down = uniquePathHelper(i + 1, j, r, c, A, paths);
   int right = uniquePathHelper(i, j + 1, r, c, A, paths); 	
   return paths[i][j] = down + right;
}

/*
* Recibe una matriz donde el valor 1 significa obstaculo
* y 0 significa libre
*/
int countPathsTopDown(vector<vector<int> >& A){
   int r = A.size(), c = A[0].size();
   vector<vector<int> > paths(r, vector<int>(c, -1));
   return uniquePathHelper(0, 0, r, c, A, paths);
}


//Bottom-Up
/*
* Recibe una matriz donde el valor 1 significa obstaculo
* y 0 significa libre
*/
int countPathsBottomUp(vector<vector<int>>& A){
   int r = A.size(), c = A[0].size();
	
   vector<vector<int>> paths(r, vector<int>(c, 0));
   
   if (A[0][0] == 0)
      paths[0][0] = 1;
	
   for(int i = 1; i < r; i++){
      if (A[i][0] == 0) paths[i][0] = paths[i-1][0];
   }
	
   for(int j = 1; j < c; j++){
      if (A[0][j] == 0) paths[0][j] = paths[0][j - 1];
   }  
	
   for(int i = 1; i < r; i++){
      for(int j = 1; j < c; j++){
         if (A[i][j] == 0)
            paths[i][j] = paths[i - 1][j] + paths[i][j - 1];
      } 
   }
   return paths[r-1][c-1];
}


//Optimizacion del espacio de la solucion DP
/*
* Recibe una matriz donde el valor 1 significa obstaculo
* y 0 significa libre
*/
int countPathsSpaceOptimizationDP(vector<vector<int> >& A){
   int r = A.size();
   int c = A[0].size();
   
   if (A[0][0]) return 0;
	
   A[0][0] = 1;
   
   for(int j = 1; j < c; j++){
      if(A[0][j] == 0) {
         A[0][j] = A[0][j - 1];
      } else {
         A[0][j] = 0;
      }
   }
   
   for (int i = 1; i < r; i++) {
      if (A[i][0] == 0) {
         A[i][0] = A[i - 1][0];
      } else {
         A[i][0] = 0;
      }
   }
	
   for (int i = 1; i < r; i++) {
      for (int j = 1; j < c; j++) {
         if (A[i][j] == 0) {
            A[i][j] = A[i - 1][j] + A[i][j - 1];
         }
         else {
            A[i][j] = 0;
         }
      }
   }
   
   return A[r-1][c-1];
}

//El enfoque 2D DP
/*
* Recibe una matriz donde el valor 1 significa obstaculo
* y 0 significa libre
*/
#define int long long
using namespace std;
int n, m;
int path(vector<vector<int> >& dp, vector<vector<int> >& grid, int i, int j){
   if (i < n && j < m && grid[i][j] == 1) return 0;
   if (i == n - 1 && j == m - 1) return 1;
   if (i >= n || j >= m) return 0;
   if (dp[i][j] != -1) return dp[i][j];
   
   int left = path(dp, grid, i + 1, j);
   int right = path(dp, grid, i, j + 1);
   return dp[i][j] = left + right;
}

int countPath2DDP(vector<vector<int> >& grid){
   n = grid.size();
   m = grid[0].size();
   if (n == 1 && m == 1 && grid[0][0] == 0) return 1;
   if (n == 1 && m == 1 && grid[0][0] == 1) return 0;
   vector<vector<int> > dp(n, vector<int>(m, -1));
   path(dp, grid, 0, 0);
   if (dp[0][0] == -1) return 0;
   return dp[0][0];
}


\end{lstlisting}

\subsection{Java}
\subsubsection{Máximo camino}

\subsubsection{Cantidad de caminos}