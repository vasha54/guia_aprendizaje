\subsection{C++}

\subsubsection{LCA usando arbol de rango}
\begin{lstlisting}[language=C++]
struct LCA {
   vector<int> height, euler, first, rtree;
   vector<bool> visited;
   int n;
   
   LCA(vector<vector<int>> &adj, int root = 0) {
      n = adj.size(); height.resize(n);
      first.resize(n); euler.reserve(n * 2);
      visited.assign(n, false); dfs(adj, root);
      int m = euler.size(); rtree.resize(m * 4);
      build(1, 0, m - 1);
   }
	
   void dfs(vector<vector<int>> &adj, int node, int h = 0) {
      visited[node] = true; height[node] = h;
      first[node] = euler.size();
      euler.push_back(node);
      for (auto to : adj[node]) {
         if(!visited[to]) {
            dfs(adj, to, h + 1); euler.push_back(node);
         }
      }
   }
	
   void build(int node, int b, int e) {
      if(b==e){ segtree[node] = euler[b];} 
      else {
         int mid = (b + e) / 2;
         build(node << 1, b, mid); 
         build(node << 1 | 1, mid + 1, e);
         int l = rtree[node << 1], r = rtree[node << 1 | 1];
         rtree[node] = (height[l] < height[r]) ? l : r;
      }
   }
	
   int query(int node, int b, int e, int L, int R) {
      if (b > R || e < L) return -1;
      if (b >= L && e <= R) return rtree[node];
      int mid = (b + e) >> 1;
      
      int left = query(node << 1, b, mid, L, R);
      int right = query(node << 1 | 1, mid + 1, e, L, R);
      if (left == -1) return right;
      if (right == -1) return left;
      return height[left] < height[right] ? left : right;
   }
	
   int lca(int u, int v) {
      int left = first[u], right = first[v],tmp;
      if (left > right){ tmp =left;left=right;right=tmp};
      return query(1, 0, euler.size() - 1, left, right);
   }
};

\end{lstlisting}

\subsubsection{LCA usando elevación binaria}
\begin{lstlisting}[language=C++]
int n, l;
vector<vector<int>> adj;

int timer;
vector<int> tin, tout;
vector<vector<int>> up;

void dfs(int v, int p){
   tin[v] = ++timer; up[v][0] = p;
   for (int i = 1; i <= l; ++i)
      up[v][i] = up[up[v][i-1]][i-1];
   for (int u : adj[v]) {
      if (u != p) dfs(u, v);
   }
   tout[v] = ++timer;
}

bool is_ancestor(int u, int v){
   return tin[u] <= tin[v] && tout[u] >= tout[v];
}

int lca(int u, int v) {
   if (is_ancestor(u, v)) return u;
   if (is_ancestor(v, u)) return v;
   for (int i = l; i >= 0; --i) {
      if (!is_ancestor(up[u][i], v)) u = up[u][i];
   }
   return up[u][0];
}

void preprocess(int root) {
   tin.resize(n); tout.resize(n);
   timer = 0; l = ceil(log2(n));
   up.assign(n, vector<int>(l + 1));
   dfs(root, root);
}
\end{lstlisting}

\subsubsection{LCA usando algoritmo Tarjan offline}

No se ha incluido la implementación de DSU, ya que se puede utilizar la clásica sin modificaciones.
\begin{lstlisting}[language=C++]
vector<vector<int>> adj, queries;
vector<int> ancestor;
vector<bool> visited;

void dfs(int v){
   visited[v] = true; ancestor[v] = v;
   for (int u : adj[v]) {
      if (!visited[u]) {
         dfs(u); union_sets(v, u);
         ancestor[find_set(v)] = v;
      }
   }
   for (int other_node : queries[v]) {
      if (visited[other_node])
         cout << "LCA de " << v << " y " << other_node
         << " es " << ancestor[find_set(other_node)] << ".\n";
   }
}

void compute_LCAs() {
   // initialize n, adj and DSU
   // for (each query (u, v)) {
   //    queries[u].push_back(v);
   //    queries[v].push_back(u);
   // }
   ancestor.resize(n); visited.assign(n, false); dfs(0);
}
\end{lstlisting}

\subsubsection{LCA usando algoritmo Farach-Colton y Bender}
\begin{lstlisting}[language=C++]
int n;

vector<vector<int>> adj;

int block_size, block_cnt;
vector<int> first_visit;
vector<int> euler_tour;
vector<int> height;
vector<int> log_2;
vector<vector<int>> st;
vector<vector<vector<int>>> blocks;
vector<int> block_mask;

void dfs(int v, int p, int h) {
   first_visit[v] = euler_tour.size(); euler_tour.push_back(v);
   height[v] = h;
   for (int u : adj[v]) {
      if (u == p) continue;
      dfs(u, v, h + 1);
      euler_tour.push_back(v);
   }
}

int min_by_h(int i, int j) {
   return height[euler_tour[i]] < height[euler_tour[j]] ? i : j;
}

void precompute_lca(int root) {
   // obtener el recorrido de euler y 
   // los indices de las primeras ocurrencias
   first_visit.assign(n, -1); height.assign(n, 0);
   euler_tour.reserve(2 * n); dfs(root, -1, 0);
	
   // precalcular todos los valores de log
   int m = euler_tour.size(); log_2.reserve(m + 1);
   log_2.push_back(-1);
   for (int i = 1; i <= m; i++)
      log_2.push_back(log_2[i / 2] + 1);
	
   block_size = max(1, log_2[m] / 2); 
   block_cnt = (m + block_size - 1) / block_size;
	
   // precalcula el minimo de cada bloque y 
   // construya una tabla dispersa
   st.assign(block_cnt, vector<int>(log_2[block_cnt] + 1));
   for (int i = 0, j = 0, b = 0; i < m; i++, j++) {
      if (j == block_size) j = 0, b++;
      if (j == 0 || min_by_h(i, st[b][0]) == i) st[b][0] = i;
   }
   for (int l = 1; l <= log_2[block_cnt]; l++) {
      for (int i = 0; i < block_cnt; i++) {
         int ni = i + (1 << (l - 1));
         if (ni >= block_cnt) st[i][l] = st[i][l-1];
         else st[i][l]=min_by_h(st[i][l-1],st[ni][l-1]);
      }
   }
   // mascara de calculo previo para cada bloque
   block_mask.assign(block_cnt, 0);
   for (int i = 0, j = 0, b = 0; i < m; i++, j++) {
      if (j == block_size) j = 0, b++;
      if (j>0 && (i>=m || min_by_h(i-1,i)==i-1))block_mask[b]+=1<<(j-1);
   }
   // precalcular RMQ para cada bloque unico
   int possibilities = 1 << (block_size - 1);
   blocks.resize(possibilities);
   for (int b = 0; b < block_cnt; b++) {
      int mask = block_mask[b];
      if (!blocks[mask].empty()) continue;
      blocks[mask].assign(block_size, vector<int>(block_size));
      for (int l = 0; l < block_size; l++) {
         blocks[mask][l][l] = l;
         for (int r = l + 1; r < block_size; r++) {
            blocks[mask][l][r] = blocks[mask][l][r - 1];
            if (b * block_size + r < m) 
               blocks[mask][l][r]=min_by_h(b*block_size + blocks[mask][l][r],
                                           b*block_size +r)-b*block_size;
         }
      }
   }
}

int lca_in_block(int b, int l, int r) {
   return blocks[block_mask[b]][l][r] + b * block_size;
}

int lca(int v, int u) {
   int l = first_visit[v]; int r = first_visit[u];
   if (l > r){int tmp = l;l=r;r=tmp;}
   int bl = l / block_size;
   int br = r / block_size;
   if(bl==br)return euler_tour[lca_in_block(bl,l%block_size,r%block_size)];
   int ans1 = lca_in_block(bl,l%block_size,block_size-1);
   int ans2 = lca_in_block(br,0,r%block_size);
   int ans = min_by_h(ans1, ans2);
   if (bl + 1 < br) {
     int l = log_2[br - bl - 1]; int ans3 = st[bl+1][l];
     int ans4 = st[br - (1 << l)][l];
     ans = min_by_h(ans, min_by_h(ans3, ans4));
   }
   return euler_tour[ans];
}

\end{lstlisting}

\subsection{Java}

\subsubsection{LCA usando arbol de rango}
\begin{lstlisting}[language=Java]
private class LCA {
   private int[] depth, dfs_order,first,minPos;
   private int cnt,n;

   private void dfs(List<Integer>[] tree, int u, int d) {
      depth[u] = d; dfs_order[cnt++] = u;
      for (int v : tree[u])
         if (depth[v] == -1) {
            dfs(tree, v, d + 1); dfs_order[cnt++] = u;
      }
    }
	
   private void buildTree(int node, int left, int right) {
      if(left==right){ minPos[node]=dfs_order[left]; return;}
      int mid = (left + right) >> 1;
      buildTree(2 * node + 1, left, mid);
      buildTree(2 * node + 2, mid + 1, right);
      minPos[node]=depth[minPos[2*node +1]]<depth[minPos[2*node+2]]? minPos[2*node+1]:minPos[2*node+2];
   }
	
   public LCA(List<Integer>[] tree, int root) {
      int nodes = tree.length; depth = new int[nodes];
      Arrays.fill(depth, -1);
      n = 2 * nodes - 1; dfs_order = new int[n];
      cnt = 0; dfs(tree, root, 0);
      minPos = new int[4 * n];
      buildTree(0, 0, n - 1);
      first = new int[nodes]; Arrays.fill(first, -1);
      for (int i = 0; i < dfs_order.length; i++)
         if (first[dfs_order[i]] == -1) first[dfs_order[i]] = i;
   }
	
   public int lca(int a, int b) {
      return minPos(Math.min(first[a],first[b]),Math.max(first[a],first[b]),0,0,n-1);
   }
	
   private int minPos(int a, int b, int node, int left, int right) {
      if(a == left && right == b) return minPos[node];
      int mid = (left + right) >> 1;
      if (a <= mid && b > mid){ 
         int p1 = minPos(a,Math.min(b,mid),2*node+1,left,mid);
         int p2 = minPos(Math.max(a,mid+1),b,2*node+2,mid+1,right);
         return depth[p1] < depth[p2] ? p1 : p2;
      } else if (a <= mid) {
         return minPos(a,Math.min(b,mid),2*node+1,left,mid);
      } else if (b > mid) {
         return minPos(Math.max(a,mid+1),b,2*node+2,mid+1,right);
      } else { throw new RuntimeException(); }
   }
}
\end{lstlisting}