

\begin{tabular}{|c|p{11cm}|}
	\hline
	\textbf{Operador} & \textbf{Descripción}  \\
	\hline
	sizeof & Devuelve el tamaño, en bytes, del operando situado a su derecha. El operando
	puede ser un especificador de tipo, en cuyo caso se emplean paréntesis; por ejemplo,
	sizeof(float). Puede ser también el nombre de una variable con concreta o de un array,
	en cuyo caso no se emplean paréntesis: sizeof foto. Solo presente en C++\\
	\hline
	(tipo) &  Operador de moldeado, convierte el valor que vaya a continuación en el tipo especifi-
	cado por la palabra clave encerrada entre los paréntesis. Por ejemplo, (float)9 convierte
	el entero 9 en el número de punto flotante 9.0. \\
	\hline
	, &  El operador coma une dos expresiones en una, garantizando que se evalúa en primer
	lugar la expresión situada a la izquierda, una aplicación típica es la inclusión de más
	información de más información en la expresión de control de un bucle for: \\
	\hline
	
\end{tabular}