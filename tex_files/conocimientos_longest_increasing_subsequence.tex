\subsection{Función recursiva}
La recursividad es una técnica de programación que se utiliza para realizar una llamada a una
función desde ella misma, de allí su nombre. Un algoritmo recursivo es un algoritmo que expresa la solución de un problema en términos
de una llamada a sí mismo. La llamada a sí mismo se conoce como llamada recursiva o recurrente.


\subsection{Memorización}
En Informática, el término memorización (del inglés memoization) es una técnica de optimización que se usa principalmente para acelerar los tiempos de cálculo, almacenando los resultados de la llamada a una subrutina en una memoria intermedia o búfer y devolviendo esos mismos valores cuando se llame de nuevo a la subrutina o función con los mismos parámetros de entrada. 


\subsection{Programación Dinámica}
La Programación Dinámica la cual es una técnica que combina la corrección de la búsqueda completa y la eficiencia de los algoritmos golosos.

\subsection{Búsqueda Binaria}
La búsqueda binaria es un algoritmo eficiente para encontrar un elemento en una lista ordenada de elementos. Funciona al dividir repetidamente a la mitad la porción de la lista que podría contener al elemento, hasta reducir las ubicaciones posibles a solo una. 