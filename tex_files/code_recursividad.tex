Como lo abordado no es un algoritmo sino una técnica de progrmación veamos algunos ejemplos de la implementación de esta técnica para resolver determinados problemas.

\begin{lstlisting}[language=C++]

/* Ejemplo de recursividad de indirecta o mutua para determinar si un numero positivo es
par o impar */

int par(int n){
  if (n==0) return 1;
  else return (impar(n-1));
}
int impar(int n){
  if (n==0) return 0; 
  else return(par(n-1));
}

/* Calcular el ensismo termino de Fibanacci*/
long fibonacci(int n)
{
  if(1 == n || 2 == n) {
    return 1;
  } else {
    return (fibonacci(n-1) + fibonacci(n-2));
  }
}

/*Calcular el maximo comun divisor*/
long mcd(long a, long b){
  if (a==b) return a;
  else if (a<b) return mcd(a,b-a);
  else return mcd(a-b,b);
}

/*Calcular el factorial*/
int factorial(int numero){
   if (numero > 1) return numero*factorial(numero-1);
   else return 1;
}
\end{lstlisting} 