La función phi de Euler, denotada como $\phi(n)$, es una función aritmética que cuenta la cantidad de números enteros positivos menores o iguales a n que son coprimos con n (es decir, que no comparten ningún factor primo con n). Algunas aplicaciones de la función phi de Euler son:

\begin{enumerate}
	\item \textbf{Teorema de Euler:} La función phi de Euler está relacionada con el teorema de Euler, que establece que $a^{\phi(n)} \equiv 1 (\mod n)$ para todo entero a coprimo con $n$.
	
	\item \textbf{Criptografía:} La función phi de Euler es utilizada en criptografía para generar claves públicas y privadas en sistemas criptográficos basados en el algoritmo RSA.
	
	\item \textbf{Teoría de números:} La función phi de Euler es fundamental en la teoría de números y se utiliza en diversos problemas relacionados con la factorización de enteros, congruencias y otros conceptos matemáticos.
	
	\item \textbf{Generación de números primos:} La función phi de Euler se utiliza en la generación de números primos y en la implementación de algoritmos eficientes para encontrar primos relativos a un número dado.
	
\end{enumerate}

