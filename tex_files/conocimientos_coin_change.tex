\subsection{Recursivad}

La recursividad es una técnica de programación que se utiliza para realizar una llamada a una
 función desde ella misma, de allí su nombre. Un algoritmo recursivo es un algoritmo que expresa
la solución de un problema en términos de una llamada a sí mismo. La llamada a sí mismo se
conoce como llamada recursiva o recurrente.

\subsection{Algoritmos golosos (\emph{Greedy})}
Un algoritmo goloso construye una solución al problema al tomar siempre una decisión que
 se ve mejor en este momento. Un algoritmo goloso nunca recupera sus opciones, pero construye
 directamente la solución final. Por esta razón, los algoritmos golosos suelen ser muy eficientes.

La dificultad para diseñar algoritmos golosos es encontrar una estrategia codiciosa que siempre
 produzca una solución óptima al problema. Las opciones localmente óptimas en un algoritmo
goloso también deben ser globalmente óptimos. A menudo es difícil argumentar que funciona un algoritmo goloso.

\subsection{Memorización}

La memorización es una técnica de optimización utilizada principalmente para hacer que los programas informáticos sean más rápidos al almacenar los resultados de las llamadas a funciones en la memoria caché y devolver los resultados almacenados en la memoria caché la próxima vez que se necesiten en lugar de volver a calcularlos. 

En el caso de la programación dinámica consiste en almacenar las soluciones de los subproblemas en alguna estructura de datos para reutilizarlas posteriormente. De esa forma, se consigue un algoritmo más eficiente que la fuerza bruta, que resuelve el mismo subproblema una y otra vez.