La intersección $A \cap B$ significa \textbf{A y B suceden}. Por ejemplo, la intersección de $A = \{~2,~5\}$ y $B = \{~4,~5,~6\}$ es $A \cap B = \{5\}$. Usando probabilidad condicional, la probabilidad de la intersección $A \cap B$ se puede calcular usando la fórmula 

$$P(A\cap B)=P(A)P(B|A)$$

Los eventos $A$ y $B$ son independientes si $P(B|A) = P(A)$ y $P(A|B) = P(B)$, lo que significa que el hecho de que ocurra $B$ no cambia la probabilidad de que
ocurra $A$, y viceversa. En este caso, la probabilidad de la intersección es 

$$P(A \cap B)=P(A)P(B)$$

Por ejemplo, al robar una carta de una baraja, los eventos $A = \text{el palo es de tréboles}$ y $B=\text{el valor es cuatro}$ son independientes. De ahí el evento $A\cap B = \text{la carta es el cuatro de tréboles}$ sucede con probabilidad $$P(A \cap B) = P(A)P(B) = \dfrac{1}{4} \times \dfrac{1}{13} = \dfrac{1}{52}$$