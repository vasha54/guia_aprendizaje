\subsection{Matriz}
La matriz es un objeto o estructura matemática muy popular. Basicamente es una tabla de números con dos dimensiones. Usualmente las matrices son representadas de la siguiente manera:



$$A=\begin{bmatrix}
	a_{11} & a_{12} & \ldots & a_{1m} \\ 
	a_{21} & a_{22} & \ldots & a_{2m} \\ 
	\vdots & \vdots & \ddots & \vdots \\ 
	a_{n1} & \ldots & \ldots & a_{nm}
\end{bmatrix} $$ 

\vspace*{0.3in}

$A$ es una matriz con $n$ filas y $m$ columnas. Cada elemento en $A$ es representado como $a_{ij}$ donde $i$ es la fila enumeradas de 1 a $n$, $j$ la columna de igual forma enumeradas de 1 a $m$.

\subsection{Fórmula de Cayley}
En teoría de grafos, la fórmula de Cayley es un resultado llamado así en honor a Arthur Cayley, que establece que para cualquier entero positivo n, el número de árboles en $n$ vértices etiquetados  $n^{n-2}$. Equivalentemente, la fórmula cuenta el número de árboles de expansión de un grafo completo con vértices etiquetados. 

\subsection{Teorema de Kirchhoff}
En el campo matemático de la teoría de grafos, el teorema de Kirchhoff, nombrado por Gustav Kirchhoff es un teorema sobre el número de árboles de expansión en un grafo, mostrando que ese número puede ser computado en tiempo polinomial como el determinante de una matriz derivada del grafo. Es una generalización de la fórmula de Cayley que provee el número total de árboles de expansión en un grafo completo. 

