Como es evidente el algoritmo no permite el hallar árbol de expansión mínima de un grafo si este fuera inicalmente conexo sino hallaríamos el árbol de expasión mínima de cada componente conexa de que se compone el grafo.

El algoritmo de Kruskal también ha sido aplicado para hallar soluciones en diversas áreas como es el diseño de redes de transporte, telecomunicaciones, TV por cable, sistemas distribuidos, interpretación de datos climatológicos, visión artificial, entre otros.