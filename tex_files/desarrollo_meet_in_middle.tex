La idea es dividir primero los números en dos conjuntos y revisar los subconjuntos de estos conjuntos, lo que lleva un tiempo O ($2^{n/2}$). En este paso, se calcula la suma de cada subconjunto y se ordenan las sumas. Después de esto, basta con revisar los subconjuntos del primer conjunto y calcular para cada subconjunto de cuántas maneras se puede seleccionar un subconjunto del segundo conjunto para que la suma sea $x$. Cuando la suma de un subconjunto del primer conjunto es $a$, se debe elegir del segundo conjunto un subconjunto cuya suma sea $x-a$. Esto se hace de forma eficaz cuando los subconjuntos están en orden.

Como ejemplo, considere un problema en el que se nos da una lista de $n$ números y un número $x$, y queremos saber si es posible elegir algunos números de la lista para que su suma sea $x$. Por ejemplo, dada la lista $[2, 4, 5, 9]$ y $x = 15$, podemos elegir los números $[2, 4, 9]$ para obtener $2 + 4 + 9 = 15$. Sin embargo, si $x = 10$ para la misma lista, no es posible formar la suma.

Un algoritmo simple para el problema es revisar todos los subconjuntos de elementos y verificar si la suma de alguno de los subconjuntos es $x$. El tiempo de ejecución de dicho algoritmo es O($2^n$), porque hay $2^n$ subconjuntos. Sin embargo, utilizando la técnica de encuentro en el medio, podemos lograr un algoritmo de tiempo O($2^{\frac{n}{2}}$) más eficiente. Note que O($2^n$) y O ($2^{\frac{n}{2}}$) son de diferente complejidad porque $2^{\frac{n}{2}}$ es igual a $\sqrt{2^n}$.

La idea es dividir la lista en dos listas $A$ y $B$ de modo que ambas listas contengan aproximadamente la mitad de los números. La primera búsqueda genera todos los subconjuntos de $A$ y almacena sus sumas en una lista $S_A$. En consecuencia, la segunda búsqueda crea una lista $S_B$ a partir de $B$ . Después de esto, basta comprobar si es posible elegir un elemento de $S_A$ y otro elemento de $S_B$ tales que su suma sea $x$ . Esto es posible exactamente cuando hay una manera de formar la suma $x$ usando los números de la lista original.

Por ejemplo, supongamos que la lista es $[2, 4, 5, 9]$ y $x = 15$. Primero, dividimos la lista en $A = [2, 4]$ y $B = [5, 9]$. Después de esto, creamos las listas $S_A = [0, 2, 4, 6]$ y $S_B = [0, 5, 9, 14]$. En este caso, es posible formar la suma $x = 15$, porque $S_A$ contiene la suma $6$, $S_B$ contiene la suma $9$ y $6 + 9 = 15$. Esto corresponde a la solución $[2, 4, 9]$.

\emph{Meet in the middle} es una técnica en la que el espacio de búsqueda se divide en dos partes de aproximadamente el mismo tamaño. Se realiza una búsqueda separada para ambas partes y finalmente se combinan los resultados de las búsquedas.