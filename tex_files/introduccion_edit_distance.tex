La distancia de edición o distancia de Levenshtein es el número mínimo de operaciones de edición necesarias para transformar una cadena en otra cadena. Las operaciones de edición permitidas son las siguientes:

\begin{itemize}
	\item Insertar un caracter( ABC $\rightarrow$ ABCA )
	\item Remover un caracter ( ABC $\rightarrow$ AB)
	\item Modificar un caracter ( ABC $\rightarrow$ ADC)
\end{itemize}

Por ejemplo, la distancia de edición entre LOVE y MOVIE es 2, porque primero podemos realizar la operación LOVE $\rightarrow$ MOVE (modificar) y luego la operación MOVE $\rightarrow$ MOVIE (insertar). Este es el menor número posible de operaciones, porque está claro que una sola operación no es suficiente.