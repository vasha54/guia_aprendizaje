Algunas aplicaciones del algoritmo extendido de Euclides son:

\begin{enumerate}
	\item \textbf{Encriptación RSA:} El algoritmo extendido de Euclides se utiliza en el proceso de generación de claves para el algoritmo de encriptación RSA, donde se necesitan encontrar números primos p y q y calcular la clave privada y pública.
	\item \textbf{Resolución de ecuaciones diofánticas:} Las ecuaciones diofánticas son ecuaciones en las que se buscan soluciones enteras. El algoritmo extendido de Euclides se puede utilizar para encontrar soluciones enteras de ecuaciones diofánticas lineales. 
	\item \textbf{Reducción de fracciones:} El algoritmo extendido de Euclides se puede utilizar para simplificar fracciones, es decir, para encontrar la fracción irreducible equivalente a una fracción dada.
	\item \textbf{Teoría de números:} El algoritmo extendido de Euclides es una herramienta fundamental en la teoría de números para estudiar propiedades de los números enteros, como la existencia de soluciones enteras para ciertas ecuaciones.
\end{enumerate}

En resumen, el algoritmo extendido de Euclides tiene diversas aplicaciones en matemáticas y en campos como la criptografía y la teoría de números.