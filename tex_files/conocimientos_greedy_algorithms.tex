\subsection{Heurística}
En computación, dos objetivos fundamentales son encontrar algoritmos con buenos tiempos de ejecución y buenas soluciones, usualmente las óptimas. Una heurística es un algoritmo que abandona uno o ambos objetivos; por ejemplo, normalmente encuentran buenas soluciones, aunque no hay pruebas de que la solución no pueda ser arbitrariamente errónea en algunos casos; o se ejecuta razonablemente rápido, aunque no existe tampoco prueba de que siempre será así. Las heurísticas generalmente son usadas cuando no existe una solución óptima bajo las restricciones dadas (tiempo, espacio, etc.), o cuando no existe del todo.

\subsection{Programación Dinámica}
La Programación Dinámica la cual es una técnica que combina la corrección de la búsqueda completa y la eficiencia de los algoritmos golosos.