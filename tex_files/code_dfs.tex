\subsection{C++}

Variante no recursiva

\begin{lstlisting}[language=C++]
#include <stack>
#include <vector>
#define MAX_N 5001
using namespace std;
	
struct Node{
   vector<int> adj;
};
Node graf[MAX_N];
bool mark[MAX_N];
	
inline void DFS(int start){
   stack<int> dfs_stek;
   dfs_stek.push(start);
   while(!dfs_stek.empty()){
      int xt = dfs_stek.top();
	  dfs_stek.pop();
      mark[xt] = true;
      for (int i=0;i<graf[xt].adj.size();i++){
         if (!mark[graf[xt].adj[i]]){
            dfs_stek.push(graf[xt].adj[i]);
            mark[graf[xt].adj[i]] = true;
         }
      }
    }
}
\end{lstlisting} 

Variante recursiva

\begin{lstlisting}[language=C++]
vector< vector<int> > adj; // el grafo representado como una lista de adyacencia
int n; // numeros de vertices

vector<bool> visited;

void dfs(int v){
  visited[v] = true;
  for (int u : adj[v]){
     if (!visited[u])
       dfs(u);
  }
}
\end{lstlisting}

Esta es la implementación más simple de DFS. Como se describe en las aplicaciones, también podría ser útil calcular los tiempos de entrada y salida y el color del vértice. Colorearemos todos los vértices con el color 0, si no los hemos visitado, con el color 1 si los visitamos, y con el color 2, si ya salimos del vértice.

Aquí hay una implementación genérica que además los calcula:

\begin{lstlisting}[language=C++]
vector< vector<int> > adj; // el grafo representado como una lista de adyacencia
int n; // numeros de vertices

vector<int> color;

vector<int> time_in, time_out;
int dfs_timer = 0;

void dfs(int v){
   time_in[v] = dfs_timer++;
   color[v] = 1;
   for (int u : adj[v])
      if (color[u] == 0)
         dfs(u);
   color[v] = 2;
   time_out[v] = dfs_timer++;
}
\end{lstlisting}

\subsection{Java}
\begin{lstlisting}[language=Java]
private class Graph{
   public List<ArrayList<Integer> > lAdyancencia;
   public int nodes;
   
   public Graph(int _nodes){
      this.nodes =_nodes;
      lAdyancencia =new ArrayList<ArrayList<Integer> >(this.nodes+1);
      for(int i=0;i<this.nodes+1;i++) {
         lAdyancencia.add(new ArrayList<Integer>());
      }
      boolean [] visited = new boolean [this.nodes+1];
      Arrays.fill(visited,false);
   }
   
   private void dfsRecursive(int node){
      visited[node]=true;
      int nvecinos = lAdyancencia.get(node).size(),vecino;
      for(int i=0;i<nvecinos;i++){
         vecino = lAdyancencia.get(node).get(i);
         if(visited[vecino]==false){
            dfsRecursive(vecino, access, visited);
         }
      }
   }

   private void dfsStack(int node){
      Stack<Integer> dfsStack =new Stack<Integer>();
      boolean [] visited =new boolean [this.nodes+1];
      
      Arrays.fill(visited, false);
      int nvecinos,vecino;
      
      dfsStack.add(node);
	
      while(dfsStack.empty()==false){
        node = dfsStack.pop();
        visited[node] =true;
        nvecinos = lAdyancencia.get(node).size();
        for(int i=0;i<nvecinos;i++) {
           vecino = lAdyancencia.get(node).get(i);
           if(visited[vecino] == false) {
              dfsStack.add(vecino);
              visited[vecino] = true;
           }
         }
       }
   }

   public void addEgde(int a,int b) {
      lAdyancencia.get(a).add(b);
   }
}	
\end{lstlisting}