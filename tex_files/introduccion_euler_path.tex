En la teoría de grafos, un camino euleriano es un camino que pasa por cada arista una y solo una vez. Un ciclo o circuito euleriano es un camino cerrado que recorre cada arista exactamente una vez. El problema de encontrar dichos caminos fue discutido por primera vez por Leonhard Euler, en el famoso \href{https://es.wikipedia.org/wiki/Problema_de_los_puentes_de_K%C3%B6nigsberg}{problema de los puentes de Königsberg}. 
	
Sobre como determinar si un grafo presenta o no un camino de euler y como encontrarlo lo veremos en esta guía.