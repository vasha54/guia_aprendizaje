\subsection{C++}

\subsubsection{Saber si dos numeros son comprimos}
\begin{lstlisting}[language=C++]
// En versiones modernas de C++ puede utilizar la funcion __gcd(a,b)
int gcd(int a,int b){
   while (b > 0){
      a=a%b; a^=b; 
      b^=a;  a^=b; 
   } 
   return a;
}

bool isCoprime(int a,int b){
   return 1==gcd(a,b)
}
\end{lstlisting}

\subsubsection{Calcular la cantidad de comprimos de N menores que este}

\begin{lstlisting}[language=C++]
int phi(int n) {
   int result = n;
   for (int i = 2; i * i <= n; i++) {
      if(n % i == 0) {
         while (n % i == 0) n /= i;
         result -= result / i;
      }
   }
   if (n > 1) result -= result / n;
   return result;
}
\end{lstlisting}

\subsubsection{Calcular la cantidad de comprimos de todos los valores hasta N}

\begin{lstlisting}[language=C++]
	
// Idea de la criba de Eratostenes
vector<int> phi_1_to_n(int n) {
   vector<int> phi(n + 1);
   for (int i = 0; i <= n; i++) phi[i] = i;
   for (int i = 2; i <= n; i++) {
      if (phi[i] == i) {
         for (int j = i; j <= n; j += i)
            phi[j] -= phi[j] / i;
      }
   }
   return phi;
}

// Idea suma del divisor
vector<int> phi_1_to_n(int n) {
   vector<int> phi(n + 1);
   phi[0] = 0;
   phi[1] = 1;
   for (int i = 2; i <= n; i++) phi[i] = i - 1;
   for (int i = 2; i <= n; i++)
      for (int j = 2 * i; j <= n; j += i)
         phi[j] -= phi[i];
   return phi
}
\end{lstlisting}

\subsection{Java}

\subsubsection{Saber si dos numeros son comprimos}

\begin{lstlisting}[language=Java]
public int gcd(int a,int b){
   while (b > 0){
      a=a%b; a^=b; 
      b^=a;  a^=b; 
   } 
   return a;
}

public boolean isCoprime(int a,int b){
   return 1==gcd(a,b)
}
\end{lstlisting}

\subsubsection{Calcular la cantidad de comprimos de N menores que este}

\begin{lstlisting}[language=Java]
public int phi(int n) {
   int result = n;
   for (int i = 2; i * i <= n; i++) {
      if(n % i == 0) {
         while (n % i == 0) n /= i;
         result -= result / i;
      }
   }
   if (n > 1) result -= result / n;
   return result;
}
\end{lstlisting}

\subsubsection{Calcular la cantidad de comprimos de todos los valores hasta N}

\begin{lstlisting}[language=Java]
// Idea de la criba de Eratostenes
public int [] phi_1_to_n(int n) {
   int [] phi =new int [n + 1];
   for (int i = 0; i <= n; i++) phi[i] = i;
   for (int i = 2; i <= n; i++) {
      if (phi[i] == i) {
         for (int j = i; j <= n; j += i)
            phi[j] -= phi[j] / i;
      }
   }
   return phi;
}

// Idea suma del divisor
public int [] phi_1_to_n(int n) {
   int [] phi =new int [n + 1];
   phi[0] = 0;
   phi[1] = 1;
   for (int i = 2; i <= n; i++) phi[i] = i - 1;
   for (int i = 2; i <= n; i++)
      for (int j = 2 * i; j <= n; j += i)
         phi[j] -= phi[i];
   return phi
}
\end{lstlisting}