\subsection{Cola}

Una cola  es una estructura de datos, caracterizada por ser una secuencia de elementos en la que la operación de inserción push se realiza por un extremo y la operación de extracción pull por el otro. También se le llama estructura FIFO (del inglés First In First Out), debido a que el primer elemento en entrar será también el primero en salir.

La particularidad de una estructura de datos de cola es el hecho de que solo podemos acceder al primer y al último elemento de la estructura. Así mismo, los elementos solo se pueden eliminar por el principio y solo se pueden añadir por el final de la cola.

\subsubsection{C++}

En el caso de C++ para utilizar la cola incluimos la biblioteca \emph{queue} que nos permite utilizar la cola propia del lenguaje la cual cuenta con las siguientes funcionalidades

\begin{itemize}
	\item \textbf{queue::empty():} Devuelve si la cola está vacía.
	\item \textbf{queue::size():} Devuelve el tamaño de la cola.
	\item \textbf{queue::swap():} Intercambia el contenido de dos colas, pero las colas deben ser del mismo tipo, aunque los tamaños pueden diferir.
	\item \textbf{queue::emplace():} Inserta un nuevo elemento en el contenedor de la cola, el nuevo elemento se agrega al final de la cola.
	\item \textbf{queue::front():} Devuelve una referencia al primer elemento de la cola.
	\item \textbf{queue::back():} Devuelve una referencia al último elemento de la cola.
	\item \textbf{queue::push(g):} Agrega el elemento 'g' al final de la cola.
	\item \textbf{queue::pop():} Elimina el primer elemento de la cola.
\end{itemize}

\begin{lstlisting}[language=C++]
#include <iostream>
#include <queue>
	
using namespace std;
	
void showq(queue<int> gq){
   queue<int> g = gq;
   while (!g.empty()) {
     cout << '\t' << g.front();
     g.pop();
   }
   cout << '\n';
}
	
int main(){
  queue<int> gquiz;
  gquiz.push(10);
  gquiz.push(20);
  gquiz.push(30);
		
  cout << "The queue gquiz is : ";
  showq(gquiz);
		
  cout << "\ngquiz.size() : " << gquiz.size();
  cout << "\ngquiz.front() : " << gquiz.front();
  cout << "\ngquiz.back() : " << gquiz.back();
  cout << "\ngquiz.pop() : ";
  gquiz.pop();
  showq(gquiz);
  return 0;
}
\end{lstlisting}

\subsubsection{Java}

En Java, Queue es una interfaz que forma parte del paquete java.util. La interfaz de Queue amplía la interfaz de Colllection de Java.

Para usar una cola en Java, primero debemos importar la interfaz de la cola de la siguiente manera:

\begin{lstlisting}[language=Java]
importar java.util.queue;
\end{lstlisting}

O

\begin{lstlisting}[language=Java]
importar java.util.*;
\end{lstlisting}

Una vez importado, podemos crear una cola como se muestra a continuación:

\begin{lstlisting}[language=Java]
Queue<String> str_queue = new LinkedList<> ();
\end{lstlisting}


Como Queue es una interfaz, usamos una clase LinkedList que implementa la interfaz Queue para crear un objeto de cola.

Del mismo modo, podemos crear una cola con otras clases concretas.

\begin{lstlisting}[language=Java]
Queue<String> str_pqueue = new PriorityQueue<> ();
Queue<Integer> int_queue = new ArrayDeque<> ();
\end{lstlisting}

Ahora que se creó el objeto de la cola, podemos inicializar el objeto de la cola al proporcionarle los valores a través del método de agregar, como se muestra a continuación.

\begin{lstlisting}[language=Java]
str_queue.add("uno");
str_queue.add("dos");
str_queue.add("tres");
\end{lstlisting}

\begin{itemize}
	\item \textbf{add boolean add(E e):} Agrega el elemento e a la cola al final (cola) de la cola sin violar las restricciones de capacidad. Devuelve verdadero si tiene éxito o IllegalStateException si la capacidad está agotada.
	\item \textbf{peek E peek():} Devuelve la cabeza (frente) de la cola sin eliminarla.
	\item \textbf{element E element():} Realiza la misma operación que el método peek(). Lanza NoSuchElementException cuando la cola está vacía.
	\item \textbf{remove E remove():} Elimina la cabeza de la cola y la devuelve. Lanza NoSuchElementException si la cola está vacía.
	\item \textbf{poll E poll():} Elimina la cabeza de la cola y la devuelve. Si la cola está vacía, devuelve nulo.
	\item \textbf{size int size():} Devuelve el tamaño o el número de elementos en la cola.
	\item \textbf{Offer	boolean offer(E e):} Inserta el nuevo elemento e en la cola sin violar las restricciones de capacidad.
\end{itemize}

Puede tener otro grupo de operaciones las cuales va a depender de con que clase se instancie la interfaz Queue en si.

\begin{lstlisting}[language=Java]
import java.util.*;
	
public class Main {
  public static void main(String[] args) {
	Queue<String> str_queue = new LinkedList<>();
	str_queue.add("one");
	str_queue.add("two");
	str_queue.add("three");
	str_queue.add("four");
			
	System.out.println("The Queue contents:" + str_queue);
  }
}
\end{lstlisting}







