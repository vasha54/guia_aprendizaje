Dentro de las sucesiones conocidas la de Fibonacci es de las más utilizadas en  problemas de concursos siempre con alguna que otra variación o con la aplicación de algunas de las propiedades que ella presente es por ello la importancia de concerla y las formas de hallar determinado término de la sucesión también tiene diversas aplicaciones en diferentes campos, algunas de ellas son:
\begin{enumerate}
	\item \textbf{Matemáticas:} La secuencia de Fibonacci es utilizada en matemáticas para estudiar patrones de crecimiento y propiedades numéricas.
	\item \textbf{Computación:} La secuencia de Fibonacci es utilizada en algoritmos y programas informáticos, como por ejemplo en la optimización de códigos y en la programación dinámica.
	\item \textbf{Biología:} La secuencia de Fibonacci se encuentra en la naturaleza, en patrones de crecimiento de plantas, en la disposición de las hojas en las ramas, en la estructura de las conchas y en la distribución de semillas en los girasoles, entre otros.
	\item \textbf{Finanzas:} La secuencia de Fibonacci es utilizada en el análisis técnico de los mercados financieros para predecir tendencias y movimientos de precios.
	\item \textbf{Arte y diseño:} La secuencia de Fibonacci es utilizada en arte y diseño para crear composiciones equilibradas y armoniosas, ya que se considera que los números de Fibonacci representan proporciones estéticamente agradables.
\end{enumerate}
