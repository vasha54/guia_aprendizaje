\subsection{Versión recursiva}

La función recursiva siguiente devuelve el GCD y los valores de los coeficientes a $x$ y $y$ (que se pasan por referencia a la función). Esta implementación del algoritmo euclidiano extendido también produce resultados correctos para números enteros negativos.

\subsubsection{C++}
\begin{lstlisting}[language=C++]
int gcd_recursive(int a, int b, int& x, int& y) {
   if (b == 0) {
      x = 1; y = 0;
      return a;
   }
   int x1, y1;
   int d = gcd_recursive(b, a % b, x1, y1);
   x = y1;
   y = x1 - y1 * (a / b);
   return d;
}

\end{lstlisting}

\subsubsection{Java}
\begin{lstlisting}[language=Java]
// retorna { gcd(a,b), x, y } tal que gcd(a,b) = a*x + b*y
public static long[] gcd_recursive(long a, long b) {
   if (b == 0) return a > 0 ? new long[]{a, 1, 0} : new long[]{-a, -1, 0};
   long[] r = gcd_recursive(b, a % b);
   return new long[]{r[0], r[2], r[1] - a / b * r[2]};
}
\end{lstlisting}

\subsection{Versión iterativa}
También es posible escribir el algoritmo euclidiano extendido de forma iterativa. Como evita la recursividad, el código se ejecutará un poco más rápido que el recursivo.

Si observa detenidamente las variables $a_1$ y $b_1$ , podrá notar que toman exactamente los mismos valores que en la versión iterativa del algoritmo euclidiano normal . Entonces el algoritmo al menos calculará el GCD correcto.

Para ver por qué el algoritmo también calcula los coeficientes correctos, puede verificar que las siguientes invariantes se mantendrán en cualquier momento (antes del ciclo while y al final de cada iteración): $x \cdot a + y \cdot b = a_1$ y $x_1 \cdot a + y_1 \cdot b = b_1$ . Es trivial ver que estas dos ecuaciones se satisfacen al principio. Y puede comprobar que la actualización en la iteración del bucle seguirá manteniendo válidas esas igualdades.

Al final sabemos que $a_1$ contiene el GCD, por lo que $x \cdot a + y \cdot b = g$. Lo que significa que hemos encontrado los coeficientes requeridos.

Incluso puedes optimizar más el código y eliminar la variable $a_1$ y $b_1$ del código, y simplemente reutilizar $a$ y $b$. Sin embargo, si lo hace, perderá la capacidad de discutir sobre las invariantes.

\subsubsection{C++}
\begin{lstlisting}[language=C++]
int gcd_iterative(int a, int b, int& x, int& y) {
   x = 1, y = 0;
   int x1 = 0, y1 = 1, a1 = a, b1 = b;
   while (b1) {
      int q = a1 / b1;
      tie(x, x1) = make_tuple(x1, x - q * x1);
      tie(y, y1) = make_tuple(y1, y - q * y1);
      tie(a1, b1) = make_tuple(b1, a1 - q * b1);
   }
   sreturn a1;
}
\end{lstlisting}

\subsubsection{Java}
\begin{lstlisting}[language=Java]
// retorna { gcd(a,b), x, y } tal que gcd(a,b) = a*x + b*y
public static long[] gcd_iterative(long a, long b) {
   long x = 1, y = 0, x1 = 0, y1 = 1, t;
   while (b != 0) {
      long q = a / b; t = x;
      x = x1; x1 = t - q * x1;
      t = y; y = y1;
      y1 = t - q * y1; t = b;
      b = a - q * b; a = t;
   }
   return a > 0 ? new long[]{a, x, y} : new long[]{-a, -x, -y};
}
\end{lstlisting}
