Un tipo de estructura de datos que se utiliza comúnmente en informática es la estructura de datos basada en políticas. Una estructura de datos basada en políticas es una estructura de datos que permite al programador definir un conjunto de reglas o políticas que gobiernan el comportamiento de la estructura de datos.

Por ejemplo, suponga que tiene un conjunto de datos que desea almacenar en una estructura de datos. Podría utilizar una estructura de datos estándar como una matriz o una lista vinculada para almacenar los datos. Sin embargo, si desea aplicar ciertas políticas sobre los datos, como permitir que solo se almacenen ciertos valores o garantizar que los datos siempre estén ordenados en un orden particular, es posible que necesite utilizar una estructura de datos basada en políticas.

Las estructuras de datos basadas en políticas, como el árbol de estadísticas de pedidos, son estructuras de datos especializadas que permiten consultar y actualizar datos de manera eficiente en función de políticas o reglas específicas. 