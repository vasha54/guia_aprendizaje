\subsection{C++}
\begin{lstlisting}[language=C++]
#include <iostream>
#include <queue>
using namespace std ;
int main (){
   //Creacion de una cola de prioridad
   priority_queue <int> pq ;
   //De esta manera la prioridad sera inversa al valor del elemento.
   priority_queue < int , vector < int > , greater < int > > pqg ;
   //Adicionamos a la cola de prioridad los elementos
   pq.push(100); pq.push(10); pq.push(20); pq.push (100);
   pq.push(70); pq.push(2000);
   if( pq.empty() ){ cout<<" La cola de prioridad esta vacia\n"; } 
   else{
      cout<< " La cola de prioridad lleva " ;
      cout<< pq.size() <<" elementos\n" ;
   }
   //Salida : La cola de prioridad tiene 6 elementos
   while (!pq.empty()){
      int a = pq.top();//O(1) el acceso es constante
      pq.pop(); // O(log n) la eliminacion no es constante
      cout<<a<< "\n" ;
   }//Salida: 2000 100 100 70 20 10
   return 0;
}
\end{lstlisting}

\subsection{Java}
\begin{lstlisting}[language=C++]
import java.util.*;
public class Main{
   public static void main(String[] args) {
      PriorityQueue<String> cp = new PriorityQueue<String>();
      cp.add("Juan"); cp.add("Maria"); cp.add("Jose");
      cp.add("Laura");
      System.out.println("Sacando elementos...");
      String datos;
      while (!cp.isEmpty()) {
         datos = cp.remove();
         System.out.println(datos);
      }
   }
}
\end{lstlisting}	