En el diseño e implementación de un algoritmo el mismo no siempre va ser de forma lineal (que se ejecuten cada instrucción una detrás de la otra) sino que llegado determinado paso del algoritmo el mismo debe ser capaz repetir una o un cojunto de instrucciones mientras se cumpla una determinada condición y cuando esta deje de cumplirse seguir con el resto de las instrucciones del programa. Para poder realizar es necesario el uso de estructura de control las cuales permiten modificar el flujo de ejecución de las instrucciones de un algoritmo.

Vamos analizar dentros de la estructura de control las sentencias de bucle que
realizan una pregunta la cual retorna verdadero o falso (evalúa una condición) y realizan un grupo de instrucciones repetitivamente mientras la respuesta a la pregunta sea verdadera. Especificamente veremos la instrucción \textbf{while}