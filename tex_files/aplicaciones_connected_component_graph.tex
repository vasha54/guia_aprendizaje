Las aplicaciones de saber detectar las componentes conexa de un grafo son varias sobre todo en el desarrollo de software que se basan en la detección de contornos. En el caso de los ejercicios y problemas es bastante aplicable sobre grafos que se representan en tableros o matrices. Lo más común en los ejercicios abordan esta temática puramente es saber la cantidad de componentes conexa, cual es la componente conexa con mayor o menor cantidad de nodos, existen otros ejercicios donde el comienzo de la solución pasa por saber detectar las componentes conexa.