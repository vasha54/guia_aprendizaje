La implementación acaba siendo sorprendentemente breve y expresiva. Este es un algoritmo en línea, es decir, procesa los datos a medida que llegan; por ejemplo, puede leer los caracteres de la cadena uno por uno. y procesarlos inmediatamente, encontrando el valor de la función de prefijo para cada carácter siguiente. El algoritmo aún requiere almacenar la
cadena en sí y los valores de la función de prefijo calculados previamente, pero si conocemos de antemano el valor máximo $M$ la función de prefijo puede tomar la cadena, solo podemos almacenar $M + 1$ primeros caracteres de la cadena y el mismo número de valores de la función de prefijo.

\subsection{C++}

\subsubsection{Algoritmo trivial}
\begin{lstlisting}[language=C++]
vector<int> prefix_function_trivial(string s) {
   int n = (int)s.length();
   vector<int> pi(n);
   for (int i = 0; i < n; i++)
     for (int k = 0; k <= i; k++)
        if (s.substr(0, k) == s.substr(i-k+1, k)) pi[i] = k;
   return pi;
}
\end{lstlisting}

\subsubsection{Algoritmo eficiente} 
\begin{lstlisting}[language=C++]
vector<int> prefix_function_efficient(string s) {
   int n = (int)s.length();
   vector<int> pi(n);
   for (int i = 1; i < n; i++) {
      int j = pi[i-1];
      while (j > 0 && s[i] != s[j]) j = pi[j-1];
      if (s[i] == s[j]) j++;
      pi[i] = j;
   }
   return pi;
}
\end{lstlisting}

\subsubsection{KMP} 
\begin{lstlisting}[language=C++]
vector<int> KMP(string sequence, string pattern){
   int n = sequence.size();
   int m = pattern.size();
   vector<int> matches;
   vector<int> P(m+1);
   for (int i=0;i<m;i++) P[i] = -1;
   for (int i=0, j=-1;i<m;){
      while (j > -1 && pattern[i] != pattern[j]) j = P[j];
      i++; j++; P[i] = j;
   }
   for (int i=0, j=0;i<n;){
      while (j > -1 && sequence[i] != pattern[j]) j = P[j];
      i++; j++;
      if (j == m){
         matches.push_back(i - m);
         j = P[j];
      }
   }
   return matches;
}
\end{lstlisting}


\subsection{Java}

\subsubsection{Algoritmo trivial}
\begin{lstlisting}[language=Java]
public static  int [] prefix_function_trivial(String s) {
   int n = s.length();
   int [] pi = new int[n];
   Arrays.fill(pi,0);
   for (int i = 0; i < n; i++)
      for (int k = 0; k <= i; k++)
         if (s.substring(0, k) == s.substring(i-k+1, k)) pi[i] = k;
   return pi;
}
\end{lstlisting}

\subsubsection{Algoritmo eficiente} 
\begin{lstlisting}[language=Java]
public static  int [] prefix_function_efficient(String s) {
   int n = s.length();
   int [] pi = new int[n];
   Arrays.fill(pi,0);
   for (int i = 1; i < n; i++) {
      int j = pi[i-1];
      while (j > 0 && s.charAt(i) != s.charAt(j)) j = pi[j-1];
      if (s.charAt(i) == s.charAt(j)) j++;
      pi[i] = j;
   }
   return pi;
}
\end{lstlisting}

\subsubsection{KMP} 
\begin{lstlisting}[language=Java]
ArrayList<Integer> KMP(String sequence, String pattern){
   int n = sequence.length();
   int m = pattern.length();
   ArrayList<Integer> matches = new ArrayList<Integer>();
   int [] P = new int[m+1];
   for (int i=0;i<m;i++) P[i] = -1;
   for (int i=0, j=-1;i<m;){
      while (j > -1 && pattern.charAt(i) != pattern.charAt(j)) j = P[j];
      i++; j++; P[i] = j;
   }
   for (int i=0, j=0;i<n;){
      while (j > -1 && sequence.charAt(i) != pattern.charAt(j)) j = P[j];
      i++; j++;
      if (j == m){
         matches.add(i - m);
         j = P[j];
      }
   }
   return matches;
}
\end{lstlisting}
