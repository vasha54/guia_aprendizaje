\subsection{Biblioteca de funciones matemáticas}
Todos los lenguajes de programación presentan una biblioteca como es el el caso de C++ con \emph{math.h} o una clase como es Java con \emph{Math} donde se agrupan todas las implementación de operaciones y constantes matemáticas básicas.

\subsection{Funciones trigonométricas}
Las funciones trigonométricas se pueden definir como el cociente entre dos lados de un triángulo rectángulo, asociado a sus ángulos. Las funciones trigonométricas son funciones cuyos valores son extensiones del concepto de razón trigonométrica en un triángulo rectángulo trazado en una circunferencia unitaria (de radio unidad). Definiciones más modernas las describen como series infinitas o como la solución de ciertas ecuaciones diferenciales, permitiendo su extensión a valores positivos y negativos, e incluso a números complejos. 

Existen seis funciones trigonométricas básicas. Las últimas cuatro, se definen en relación de las dos primeras funciones, aunque se pueden definir geométricamente o por medio de sus relaciones. 