Dentro de los ejercicios o problemas que tienen una solución usando DP se puede agrupar o clasificar en dos grupos:

\begin{enumerate}
	\item \textbf{Los clásicos:} Son aquellos problemas que para dicha situación ya se conoce un algoritmo de DP bien definido que le da una solución. Aunque los problemas de tipo programación dinámica son muy popular con una alta frecuencia de aparición en concursos de programación recientes, los problemas clásicos de programación dinámica en su forma pura por lo general ya no aparecen en los IOI o ICPC modernos. A pesar de esto es necesario su estudio ya que nos permite entender la DP y como poder resolver aquellos problema de DP clasificados como \textbf{no-clásicos} e incluso nos permite desarrollar nuestras \emph{habilidades de programación dinámica} en el proceso. Dentro de los problemas clásicos de DP podemos mencionar:
	
	\begin{itemize}
		\item Suma máxima de rango dentro de un arreglo.
		\item Suma máxima de rango dentro de una matriz.
		\item Subsecuencia creciente más larga (LIS)
		\item 0-1 Mochila 
		\item Cambio de moneda
		\item Problema del vendedor ambulante
		\item Caminos en una cuadrícula
		\item La distancia de edición o distancia de Levenshtein
		\item Contando mosaicos
		\item Multiplicación de matrices
		\item Subsecuencia común más larga
	\end{itemize}
	
	\item \textbf{Los no clásicos:} Es evidente que aquí están los problemas que aún no se cuenta con un algoritmo bien descripto que solucione el problema o que su solución parte de la combinación de varios clásicos. Otro factor por lo que se considera no-clásico es por la frecuencia de aparición de este tipo de problema en los concursos. Es muy común que si la frecuencia de aparición del problema en los concursos y competencia aumente deje ser no clásico y pase a clásico.   
\end{enumerate}

 