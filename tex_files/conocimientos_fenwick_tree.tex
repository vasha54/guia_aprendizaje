\subsection{Función asociativa binaria}
Sea A un conjunto no vacío y * una operación binaria en A. Se dice que * es asociativa si, solo si: (a*b)*c = a*(b*c). Por ejemplo la adición y la multiplicación con números pares son asociativas

\subsection{Vector}
Vector es una clase
genérica y está pensada para operar con arreglos unidimensionales de datos del mismo tipo.

\subsection{Operador AND (\emph{ $\&$ })}

El AND bit a bit, toma dos números enteros y realiza la operación AND lógica en cada par
correspondiente de bits. El resultado en cada posición es 1 si el bit correspondiente de los dos
operandos es 1, y 0 de lo contrario. La operacion AND se representa con el signo \&.

\begin{tabular}{c|c|c}
a	& b &  a \& b \\
	\hline
0	& 0 & 0 \\

0	& 1 & 0 \\

1	& 0 & 0  \\

1	& 1 & 1 \\
\end{tabular} 

Ejemplo 21 ($10101_2$) \& 11 ($01011_2$) :

\begin{tabular}{c|c|}
	\&	& 10101  \\
		& 01011 \\
	\hline
		& 00001  \\
\end{tabular} 

\subsection{Operador OR (\emph{ $|$ })}

Una operación OR de bit a bit, toma dos números enteros y realiza la operación OR inclusivo en
cada par correspondiente de bits. El resultado en cada posición es 0 si el bit correspondiente de los
dos operandos es 0, y 1 de lo contrario. La operacion OR se representa con el signo $|$. 

\begin{tabular}{c|c|c}
	a	& b &  a $|$ b \\
	\hline
	0	& 0 & 0 \\
	
	0	& 1 & 1 \\
	
	1	& 0 & 1  \\
	
	1	& 1 & 1 \\
\end{tabular}

Ejemplo 21 ($10101_2$) \& 11 ($01011_2$) :

\begin{tabular}{c|c|}
	$|$	& 10101  \\
	    & 01011 \\
	\hline
	    & 11111  \\
\end{tabular} 