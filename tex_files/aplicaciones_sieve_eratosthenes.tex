La criba de Eratóstenes es un algoritmo utilizado para encontrar todos los números primos hasta un número dado. Algunas aplicaciones de este algoritmo son:

\begin{enumerate}
	\item \textbf{Encriptación de datos:} La criba de Eratóstenes se puede utilizar en criptografía para generar claves públicas y privadas basadas en números primos.
	\item \textbf{Optimización de algoritmos:} Este algoritmo se puede utilizar para optimizar algoritmos que requieran el uso de números primos, como el algoritmo de Euclides para encontrar el máximo común divisor.
	\item \textbf{Seguridad informática:} La criba de Eratóstenes se puede utilizar en la generación de números primos aleatorios para mejorar la seguridad en sistemas informáticos.
	\item \textbf{Matemáticas computacionales:} Este algoritmo se utiliza en la teoría de números computacionales para encontrar números primos de manera eficiente.   
\end{enumerate}

En resumen, la criba de Eratóstenes tiene diversas aplicaciones en campos como la criptografía, la seguridad informática, la optimización de algoritmos y las matemáticas computacionales.