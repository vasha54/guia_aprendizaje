Podemos escribir un número compuesto impar $n = p \cdot q$ como la diferencia de dos cuadrados $n = a^2 - b^2$:

$$n = \left(\frac{p + q}{2}\right)^2 - \left(\frac{p - q}{2}\right)^2$$

El método de factorización de Fermat intenta explotar este hecho adivinando el primer cuadrado 
$a^2$, y comprobando si la parte restante, $b^2 = a^2 - n$, también es un número cuadrado. Si es 
así, entonces hemos encontrado los factores $a-b$ y $a+b$ de $n$.