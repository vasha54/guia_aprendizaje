Aquí las  implementaciones que asume que el grafo es acíclico, es decir, existe el ordenamiento topológico deseado. 

\subsection{C++}
\begin{lstlisting}[language=C++]
int n; // numero de vertices
vector< vector<int> > adj; //lista de adyacencia del grafo
vector<bool> visited;
vector<int> ans;

void dfs(int v){
   visited[v] = true;
   for (int u : adj[v]) {
      if (!visited[u]) dfs(u);
   }
   ans.push_back(v);
}

void topological_sort() {
   visited.assign(n, false);
   ans.clear();
   for (int i = 0; i < n; ++i) {
      if (!visited[i]) dfs(i);
   }
   reverse(ans.begin(), ans.end());
}
\end{lstlisting}

La función principal de la solución es \emph{topological\_sort}, que inicializa las variables DFS, inicia DFS y recibe la respuesta en el vector \emph{ans}.

\subsection{Java}
\begin{lstlisting}[language=Java]
public void dfs(List<Integer>[] graph, boolean[] used,
                List<Integer> order, int u){
   used[u] = true;
   for (int v : graph[u])
      if (!used[v])
         dfs(graph, used, order, v);
   order.add(u);
}

public List<Integer> topologicalSort(List<Integer>[] graph) {
   int n = graph.length;
   boolean[] used = new boolean[n];
   List<Integer> order = new ArrayList<>();
   for(int i = 0; i < n; i++)
      if(!used[i])
         dfs(graph, used, order, i);
   Collections.reverse(order);
   return order;
}
\end{lstlisting}