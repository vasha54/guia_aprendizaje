Vayamos más allá, queremos deshacernos de las comparaciones de cadenas. Para lograr esto, tenemos que utilizar toda la información calculada en
los pasos anteriores.

Entonces, calculemos el valor de la función de prefijo $\phi$ para $i+1$. Si $s[i+1]= s[\phi[i]]$, entonces podemos decir con certeza que
$\phi[i + 1] = \phi[i] + 1$ , ya que sabemos que el sufijo en la posición $i$ de longitud $\phi[i]$ es igual al prefijo de longitud $\phi[i]$. Esto se ilustra nuevamente con un ejemplo.

$$\underbrace{\overbrace{s_0 ~ s_1 ~ s_2}^{\pi[i]} ~ \overbrace{s_3}^{s_3 = s_{i+1}}}_{\pi[i+1] = \pi[i] + 1} ~ \dots ~ \underbrace{\overbrace{s_{i-2} ~ s_{i-1} ~ s_{i}}^{\pi[i]} ~ \overbrace{s_{i+1}}^{s_3 = s_{i + 1}}}_{\pi[i+1] = \pi[i] + 1}$$


Si este no es el caso, $s[i+1] \neq s[\pi[i]]$, entonces debemos probar con una cadena más corta. Para acelerar las cosas, nos gustaría pasar
inmediatamente a la longitud más larga $j < \pi[i]$ , de modo que la propiedad del prefijo en la posición $i$ sostiene, es decir
$s[0 \dots j-1] = s[i-j+1 \dots i]$ :

$$\overbrace{\underbrace{s_0 ~ s_1}_j ~ s_2 ~ s_3}^{\pi[i]} ~ \dots ~ \overbrace{s_{i-3} ~ s_{i-2} ~ \underbrace{s_{i-1} ~ s_{i}}_j}^{\pi[i]} ~ s_{i+1}$$


De hecho, si encontramos tal longitud j , entonces nuevamente solo necesitamos comparar los personajes $s[i + 1]$ y $s[j]$ . Si son iguales entonces
podemos asignar $\pi[i+1] = j + 1$ . De lo contrario necesitaremos encontrar el valor más grande menor que $j$ , para el cual se cumple la propiedad
de prefijo, y así sucesivamente. Puede suceder que esto dure hasta $j = 0$. Si entonces $s[i + 1] = s[0]$ , asignamos $\phi[i + 1] = 1$ , y $\phi[i + 1] = 0$ de
lo contrario.


Entonces ya tenemos un esquema general del algoritmo. La única pregunta que queda es ¿cómo encontramos efectivamente las longitudes para $j$ .
Recapitulemos: para la duración actual $j$ en la posición $i$ para el cual se cumple la propiedad del prefijo, es decir $s[0 \dots j-1] = s[i-j+1 \dots i]$ ,
queremos encontrar el mayor $k < j$ , para el cual se cumple la propiedad del prefijo.

$$\overbrace{\underbrace{s_0 ~ s_1}_k ~ s_2 ~ s_3}^j ~ \dots ~ \overbrace{s_{i-3} ~ s_{i-2} ~ \underbrace{s_{i-1} ~ s_{i}}_k}^j ~s_{i+1}$$

La ilustración muestra que este tiene que ser el valor de $\phi[j-1]$ , que ya calculamos anteriormente.