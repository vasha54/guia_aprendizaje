En general, un árbol de rangos es una estructura de datos muy flexible y se puede resolver una gran cantidad de problemas con él. Además, también es posible aplicar operaciones más complejas y responder consultas más complejas. En particular, el árbol de segmentos se puede generalizar fácilmente a dimensiones más grandes. Por ejemplo, con un árbol de rangos bidimensional, puede responder consultas de suma o mínimo sobre algún subrectángulo de una matriz dada. Sin embargo, solo en tiempo O($\log^{2}N$).