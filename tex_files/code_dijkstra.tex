Veamos la implementación de este algoritmo en ambos lenguajes

\subsection{C++}

\begin{lstlisting}[language=C++]
#include <vector>
#include <algorithm>
#include <queue>
#include <stack>
#define MAX_N 100001
#define INF 987654321
using namespace std;

int nnodes;

struct Node{
   int dist;
   vector<int> adj;
   vector<int> weight;
};
Node graf[MAX_N];
bool mark[MAX_N];


struct pq_entry{
   int node, dist;
   bool operator <(const pq_entry &a) const{
      if (dist != a.dist) return (dist > a.dist);
         return (node > a.node);
   }
};

inline void dijkstra(int source){
   priority_queue<pq_entry> pq;
   pq_entry P;
   for (int i=0;i<nnodes;i++){
      if (i == source){
         graf[i].dist = 0;
         P.node = i;
         P.dist = 0;
         pq.push(P);
      }
      else graf[i].dist = INF;
   }
   while (!pq.empty()){
      pq_entry curr = pq.top();
      pq.pop();
      int nod = curr.node;
      int dis = curr.dist;
      for(int i=0;i<graf[nod].adj.size();i++){
         if(!mark[graf[nod].adj[i]]){
            int nextNode = graf[nod].adj[i];
            if(dis + graf[nod].weight[i] < graf[nextNode].dist){
               graf[nextNode].dist = dis + graf[nod].weight[i];
               P.node = nextNode;
               P.dist = graf[nextNode].dist;
               pq.push(P);
            }
         }
      }
      mark[nod] = true;
   }
}

int main(){
   n = 4;
   graf[0].adj.push_back(1);
   graf[0].weight.push_back(5);
   graf[1].adj.push_back(0);
   graf[1].weight.push_back(5);
	
   graf[1].adj.push_back(2);
   graf[1].weight.push_back(5);
   graf[2].adj.push_back(1);
   graf[2].weight.push_back(5);
	
   graf[2].adj.push_back(3);
   graf[2].weight.push_back(5);
   graf[3].adj.push_back(2);
   graf[3].weight.push_back(5);
	
   graf[3].adj.push_back(1);
   graf[3].weight.push_back(6);
   graf[1].adj.push_back(3);
   graf[1].weight.push_back(6);
	
   dijkstra(0);
	
   cout<<graf[3].dist<<endl;
   return 0;
}
\end{lstlisting}

\subsection{Java}

\begin{lstlisting}[language=Java]
public class Main {
	
   public static void shortestPaths(List<Edge>[] edges, int s, int[] prio, int[] pred){
      Arrays.fill(pred, -1);
      Arrays.fill(prio, Integer.MAX_VALUE);
      prio[s] = 0;
      PriorityQueue<Long> q = new PriorityQueue<>();
      q.add((long) s);
      while (!q.isEmpty()) {
      	long cur = q.remove();
      	int curu = (int) cur;
        if (cur >>> 32 != prio[curu])
        continue;
        for(Edge e : edges[curu]){
           int v = e.t;
           int nprio = prio[curu]+e.cost;
           if(prio[v] > nprio){
              prio[v] = nprio;
              pred[v] = curu;
              q.add(((long) nprio << 32) + v);
           }
        }
      }
   }
	
   static class Edge{
      int t, cost;
      public Edge(int t, int cost){
         this.t = t;
         this.cost = cost;
      }
   }
	
   // Ejemplo de uso
   public static void main(String[] args) {
      int[][] cost = { { 0, 3, 2 }, { 0, 0, -2 }, { 0, 0, 0 } };
      int n = cost.length;
      List<Edge>[] edges = new List[n];
      for(int i = 0; i < n; i++) {
         edges[i] = new ArrayList<>();
         for(int j = 0; j < n; j++) {
            if(cost[i][j] != 0) {
               edges[i].add(new Edge(j, cost[i][j]));
            }
         }
      }
      int[] dist = new int[n];
      int[] pred = new int[n];
      shortestPaths(edges, 0, dist, pred);
   }
}
\end{lstlisting}

En ambas implementaciones se asume que el grafo en ponderado no dirigido.