Lo interesante del algoritmo Manacher es que lo realiza en un tiempo O($n$) lo cual los hace muy eficiente. Mientras el mismo problema se puede resolver con Hashing se puede resolver en $O(n\cdot\log n)$, y con Suffix Trees y LCA rápido, este problema se puede resolver en $O(n)$. Pero el método descrito aquí es suficientemente más simple y tiene menos constante oculta en el tiempo y la complejidad de la memoria. En el caso del algoritmo trivial es lento, puede calcular la respuesta solo en $O(n^2)$.

A primera vista, no es obvio que el Manacher tenga una complejidad de tiempo lineal, porque a menudo ejecutamos el algoritmo ingenuo mientras buscamos la respuesta para una posición en particular. Sin embargo, un análisis más cuidadoso muestra que el algoritmo es lineal. Podemos notar que cada iteración del algoritmo trivial aumenta $r$ por uno. También $r$ no se puede disminuir durante el algoritmo. Entonces, el algoritmo trivial hará $O(n)$ iteraciones en total. Otras partes del algoritmo de Manacher funcionan obviamente en tiempo lineal. Así, obtenemos $O(n)$ complejidad del tiempo. 