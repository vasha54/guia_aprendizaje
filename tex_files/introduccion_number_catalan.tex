En combinatoria, los \textbf{números de Catalan} forman una secuencia de números naturales que aparece en varios problemas de conteo que habitualmente son recursivos. 

La secuencia de números de Catalan fue descrita en el siglo XVIII por Leonhard Euler. La secuencia lleva el nombre de Eugène Charles Catalan, quien descubrió la conexión con las expresiones entre paréntesis durante su exploración del rompecabezas de las torres de Hanói.

En 1988, salió a la luz que la secuencia numérica de Catalan había sido utilizada en China por el matemático mongol Minggatu hacia 1730. Fue cuando comenzó a escribir su libro Ge Yuan Mi Lu Jie Fa \emph{El método rápido para obtener la relación precisa de división de un círculo}, que fue completado por su alumno Chen Jixin en 1774, pero publicado sesenta años después. 

De como calcular esta secuencia y su utilización en determinados problemas de programación competitiva tratará la siguiente guía.