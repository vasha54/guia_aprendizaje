\subsection{Programación dinámica}
La programación dinámica es una técnica de optimización que se utiliza para resolver problemas complejos dividiéndolos en subproblemas más pequeños y resolviéndolos de manera recursiva. La idea principal detrás de la programación dinámica es almacenar los resultados de los
subproblemas resueltos para evitar repetir los cálculos y mejorar la eficiencia del algoritmo.

\subsection{Principio de óptimo}
Enunciado por Bellman en 1957 y que dice \emph{En una secuencia de decisiones óptima toda subsecuencia ha de ser también
óptima}. Hemos de observar que aunque este principio parece evidente no siempre es
aplicable y por tanto es necesario verificar que se cumple para el problema en
cuestión