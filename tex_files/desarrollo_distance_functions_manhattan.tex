La geometría del taxista, considerada por Hermann Minkowski en el siglo XIX, es una forma de geometría en la que la métrica usual de la geometría euclidiana es reemplazada por una nueva métrica en la que la distancia entre dos puntos es la suma de las diferencias (absolutas) de sus coordenadas. La métrica del taxista (en inglés se denomina geometría Taxicab) también se conoce como distancia rectilínea, distancia L1 o norma $\ell$, distancia de ciudad, distancia Manhattan, o longitud Manhattan, con las correspondientes variaciones en el nombre de la geometría. El último nombre alude al diseño en cuadrícula de la mayoría de las calles de la isla de Manhattan, lo que causa que el camino más corto que un auto puede tomar entre dos puntos de la ciudad tengan la misma distancia que dos puntos en la geometría del taxista. 


Una función de distancia alternativa es la distancia de Manhattan donde es la distancia entre los puntos ($x_1, y_1$) y ($x_2, y_2$) es:

$$|x_1-x_2| + |y_1-y_2|$$