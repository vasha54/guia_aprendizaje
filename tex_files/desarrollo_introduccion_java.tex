Sun Microsystems patrocinó en 1991 un proyecto interno de investigación
 denominado Green, el cual desembocó en el desarrollo de un lenguaje basado en C++ al que su creador, James
 Gosling, llamó Oak debido a un roble que tenía a la vista desde su ventana en las oficinas de Sun. Posteriormente
se descubrió que ya existía un lenguaje de computadora con el mismo nombre. Cuando un grupo de gente de Sun
visitó una cafetería local, sugirieron el nombre Java (una variedad de café) y así se quedó.

Pero el proyecto Green tuvo algunas dificultades. El mercado para los dispositivos electrónicos inteligentes de
uso doméstico no se desarrollaba tan rápido a principios de los noventa como Sun había anticipado. El proyecto
corría el riesgo de cancelarse. Pero para su buena fortuna, la popularidad de World Wide Web explotó en 1993
y la gente de Sun se dio cuenta inmediatamente del potencial de Java para agregar contenido dinámico, como
interactividad y animaciones, a las páginas Web. Esto trajo nueva vida al proyecto.

Sun anunció formalmente a Java en una importante conferencia que tuvo lugar en mayo de 1995. Podemos decir que el motivo de ser tan conocido e innovador es que tiene una filosofía única: "escribe una vez y ejecuta en cualquier lugar" ("write once, run everywhere"). En otras palabras, lo programas una vez y lo ejecutas en cualquier sistema. 

Las principales herramientas necesarias para escribir un programa en Java son las siguientes:

\begin{itemize}
	\item Un equipo ejecutando un sistema operativo (Una computadora con Windows, Mac o alguna distribución de Linux).
	\item Java Virtual Machine (JVM). Java es un lenguaje multiplataforma, que se ejecuta en cualquier máquina. Esto es gracias a la JVM (Java Virtual Machine) que nos permite ejecutar el código de Java en cualquier lugar para el que se haya creado dicha máquina virtual. JVM es el secreto (no tan secreto porque todo el mundo lo sabe) y la clave de Java como lenguaje multiplataforma.
	\item Java Runtime Environment (JRE). El entorno en tiempo de ejecución de Java está conformado por una Máquina Virtual de Java o JVM, un conjunto de bibliotecas Java y otros componentes necesarios para que una aplicación escrita en lenguaje Java pueda ser ejecutada. El JRE actúa como un "intermediario" entre el sistema operativo y Java. 
	\item Java Development Kit (JDK). Es un software gratuito que contiene todo aquello que requiere tu máquina para trabajar con el lenguaje, tanto la JVM como las librerías para realizar programas de Java, desde los más básicos hasta los más complejos y específicos. 
	\item Un entorno de desarrollo (IDE) de igual manera puede variar dependiendo del sistema operativo. A continuación de listan algunos de los mas usados por los concursantes.
	\begin{itemize}
		\item NetBeans 
		\item Eclipse
		\item IntelliJ IDEA
	\end{itemize}
	\item Tiempo para practicar 
	\item Paciencia, pero mucha paciencia y perseverancia.
\end{itemize}

Adicionalmente deberías concocer

\begin{itemize}
	\item Inglés (Recomendado). Casi toda la documentación y los problemas de la competencias internacionales se escriben en este idioma. Por tanto al menos tener la habilidad de lectura y compresión en este idioma es fundamental.
	
\end{itemize}

