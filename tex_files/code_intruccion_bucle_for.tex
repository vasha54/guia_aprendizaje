La implementación de esta estructura es similar en ambos lenguajes no existe ninguna diferencia.

\subsection{C++}
\begin{lstlisting}[language=C++]
for ( int i =1; i <=10; i++){
   cout<<" Hola Mundo " ;
}
\end{lstlisting}

Esto indica que el contador \textbf{i} inicia desde \textbf{1} y continuará iterando mientras \textbf{i} sea menor o igual a \textbf{10}
(en este caso llegará hasta 10) e \textbf{i++} realiza la suma por unidad lo que hace que el \textbf{for} y el contador
se sumen repitiendo 10 veces \emph{Hola Mundo} en pantalla.

\begin{lstlisting}[language=C++]
for( int i=10; i>=0; i--){
   cout<<" Hola Mundo " ;
}
\end{lstlisting}

Este ejemplo hace lo mismo que el primero, salvo que el contador se inicializa a 10 en lugar de 1; y
 por ello cambia la condición que se evalúa así como que el contador se decrementa en lugar de ser
incrementado.

La condición también puede ser independiente del contador:

\begin{lstlisting}[language=C++]
int j=20;
for( int i=0;j>0;i++){
   cout<<" Hola "<<i<<" - "<<j<<endl;
   j--;
}
\end{lstlisting}

En este ejemplo las iteraciones continuaran mientras i sea mayor que 0, sin tener en cuenta el valor
que pueda tener i.

\subsection{Java}

Ahora vamos a ver  los mismos ejemplos en Java

\begin{lstlisting}[language=Java]
for ( int i =1; i <=10; i++){
   System.out.print(" Hola Mundo ") ;
}
\end{lstlisting}

\begin{lstlisting}[language=Java]
for( int i=10; i>=0; i--){
   System.out.print(" Hola Mundo ") ;
}
\end{lstlisting}

\begin{lstlisting}[language=Java]
int j=20;
for( int i=0;j>0;i++){
   Syste.out.println(" Hola "+i+" - "+j);
   j--;
}
\end{lstlisting}

