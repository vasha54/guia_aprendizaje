En muchos ejercicios parte del algoritmo solución o el mismo algoritmo solución radica en saber si un elemento esta dentro de los valores almacenados en una estructura de datos. Para solucionar lo anterior en muchos casos realizamos lo que comúnmente se llama una búsqueda lineal. 

La búsqueda lineal radica en recorrer todos los elementos de la coleción en busca de la existencia o no de un determinado valor. Dicho algoritmo en el peor de los casos puede tener una complejidad de $O(n)$ ya que bien el elemento bien no pudiera estar en la colección o estar en la última posición que se visite en el recorrido.

Ahora si antes de buscar un elemento dentro de una coleción conocemos de antemano dicha que coleción sus valores están ordenados. Podría esto ayudar a mejorar el tiempo de búsqueda ? .