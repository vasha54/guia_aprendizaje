Aunque hay muchos algoritmos conocidos con tiempo de ejecución sublineal (es decir,$O(N)$), el algoritmo descrito a continuación es interesante por su simplicidad: no es más complejo que el clásico criba de Eratóstenes.

Además, el algoritmo dado aquí calcula factorizaciones de todos los números en el segmento$[2,N]$ como efecto secundario, y que puede ser útil en muchas aplicaciones prácticas.

La debilidad del algoritmo dado es que usa más memoria que la criba clásica de Eratóstenes: requiere una matriz de números, mientras que para criba clásico de Eratóstenes basta con tener
bits de memoria (que es 32 veces menos).

Por lo tanto, tiene sentido usar el algoritmo descrito solo hasta que para números de orden
y no mayor $10^{7}$.

La autoría del algoritmo parece pertenecer a Gries \& Misra (Gries, Misra, 1978: ver referencias al final del artículo). Sin embargo, también se puede atribuir a Euler, y también se le conoce como el tamiz de Euler, quien ya utilizó una versión similar durante su trabajo.

El algoritmo se centra en calcular el factor primo mínimo para cada número en el segmento $[2,N]$ . 

Además, necesitamos almacenar la lista de todos los números primos encontrados, llamémoslo $pr[]$

Inicializaremos los valores $lp[i]$ con ceros, lo que significa que asumimos que todos los números son primos. Durante la ejecución del algoritmo, este vector se llenará gradualmente.

Ahora repasaremos los números del 2 al $N$. Tenemos dos casos para el número actual $i$:

\begin{enumerate}
	\item $lp[i] = 0$ - eso significa que i es primo, es decir, no hemos encontrado factores menores para él. Por lo tanto, asignamos $lp [i] = i$ y agregamos $i$ al final de la lista $pr[]$. 
	\item $lp[i] != 0$ - eso significa que i es compuesto y que su minimo factor primo es $lp[i]$
\end{enumerate}  

En ambos casos actualizamos los valores $lp[]$ de para los números que son divisibles por $i$. Sin embargo, nuestro objetivo es aprender a hacerlo para establecer un valor como máximo una vez para cada número $lp[]$. Podemos hacerlo de la siguiente manera:

Consideremos los números $x_j = i \cdot p_j$ donde $p_j$ son todos los números primos menores o iguales que $lp [i]$ (por eso necesitamos almacenar la lista de todos los números primos).

Estableceremos un nuevo valor $lp [x_j] = p_j$ para todos los números de esta forma.

Aunque el tiempo de ejecución de $O(n)$ es mejor que $O(n \log \log n)$ de la criba clásica de Eratóstenes, la diferencia entre ambos no es tan grande. En la práctica, la criba lineal funciona tan rápido como una implementación típica de la criba de Eratóstenes.

En comparación con las versiones optimizadas del tamiz de Eratóstenes, p. el tamiz segmentado, es mucho más lento.

Teniendo en cuenta los requisitos de memoria de este algoritmo: un vector $lp []$ de longitud $n$ y un vector de $pr []$ de longitud $\frac n {\ln n}$, este algoritmo parece peor que la criba clásicá en todos los sentidos.

Sin embargo, su cualidad redentora es que este algoritmo calcula un vector $lp []$, lo que nos permite encontrar la factorización de cualquier número en el segmento $[2; n]$ en el tiempo del orden de tamaño de esta factorización. Además, usar solo una matriz adicional nos permitirá evitar divisiones al buscar factorización.

Conocer las factorizaciones de todos los números es muy útil para algunas tareas, y este algoritmo es uno de los pocos que permiten encontrarlos en tiempo lineal.