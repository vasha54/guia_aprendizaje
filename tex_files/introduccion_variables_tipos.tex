Cuando se codifica un algoritmo independientemente del lenguaje que se use se tiene que hace uso de un concepto que tomamos de matemática como es variable.

Cuando en matemática hacemos uso de este concepto es para referirnos a un valor que desconocemos pero que trabajaremos con ella aún si conocer el valor real. 

En programación tambien vamos a apoyarnos en ese concepto para poder trabajar con aquellos valores que desconocemos bien porque son valores introducidos por el usuario en la entrada del programa o valores que son resultados de operaciones realizadas con los datos introducidos por el usuario.

La gran diferencia entre una variable en matemática a una variable en programación es que la de matemática siempre va ser referencia a un valor númerico desconocido, mientras la de programación hace alusión a un valor desconocido pero este no tiene  que ser necesariamente númerico.