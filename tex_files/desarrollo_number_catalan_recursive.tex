Los números catalanes se pueden calcular mediante la fórmula:

$$C_0 = C_1 = 1$$

$$C_n = \sum_{i = 0}^{n-1} C_i C_{n-1-i} , {n} \geq 2$$

La suma pasa por las formas de dividir la expresión en dos partes de modo que ambas partes sean expresiones válidas y la primera parte sea lo más corta posible pero no vacía. Para cualquier i, la primera parte contiene i + 1 pares de paréntesis y el número de expresiones es el producto de los siguientes valores:

\begin{itemize}
	\item $C_i$: el número de formas de construir una expresión usando los paréntesis de la primera parte, sin contar los paréntesis más externos.
	\item $C_{n-i-1}$: el número de formas de construir una expresión usando los paréntesis de la segunda parte.
\end{itemize}

El caso base es $C_0 = 1$, porque podemos construir una expresión entre paréntesis vacía usando cero pares de paréntesis.

La fórmula de recurrencia se puede deducir fácilmente del problema de la secuencia correcta de corchetes.

El paréntesis de apertura más a la izquierda $l$ corresponde a cierto corchete de cierre $r$, que 
divide la secuencia en 2 partes que a su vez deberían ser una secuencia correcta de corchetes. 
Así, la fórmula también se divide en 2 partes. Si denotamos $k = {r - l - 1}$, entonces para $r$ 
fijo, habrá exactamente $C_i C_{n-1-i}$ tales secuencias de corchetes. Sumando esto sobre todos 
los $i$ admisibles, obtenemos la relación de recurrencia en $C_n$.

También puedes pensarlo de esta manera. Por definición, $C_n$ denota el número de secuencias de 
corchetes correctas. Ahora, la secuencia se puede dividir en 2 partes de longitud $i$ y ${n-i}$, cada una de las cuales debe ser una secuencia de corchetes correcta. Ejemplo:

$()(())$ se puede dividir en $()$ y $(())$, pero no se puede dividir en $()($ y $())$. Nuevamente 
sumando todos los $i$ admisibles, obtenemos la relación de recurrencia en $C_n$.