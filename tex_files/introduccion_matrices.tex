En varios problemas debemos trabajar con varias informaciones de forma grupal y dichas in-
formaciones tienen en común que comporten el mismo tipo de dato. Una variante bastante trivial
es declarar un variable por cada información con que necesito trabajar. Por ejemplo si tengo el
salario mensual de un trabajador durante un año con declarar 12 variables me sería suficiente
para luego realizar un grupo de operaciones como el promedio salarial, el mínimo y máximo sa-
lario del trabajador. Estas operaciones aunque se pueden implementar no cabe duda que tienen
su complejidad en cuanto a la implementación que puede ser un tanto tediosas. Bueno imagine-
mos que ahora tengamos que hacer ese mismo trabajo con un quinquenio o decada de trabajo del
trabajador la complejidad de implementación aumentería casi que literal en cino o diez veces más.

Ya habíamos tratado en otras guías que la solución a estos era el trabajo con arreglo. Pero que pasaría si quiero almacenar el salario mensual de varios años de forma que pueda luego calcular de forma cómoda el promedio del salario anual. Veremos en esta guía que una forma muy comoda de organizar los datos bajo determinadas circunstancia es utilizando las \textbf{matrices}.   