C++ y Java disponen también de un conjunto de operadores que actúan a nivel de bits. Las operaciones de
bits se utilizan con frecuencia para definir señales o flags, esto es, variables de tipo entero en las que
cada uno de sus bits indican si una opción está activada o no.

Estos operadores sólo pueden usarse con los tipos int y char y funcionan bit a bit. El operador de desplazamiento se puede utilizar para realizar multiplicaciones o divisiones rápidas, pues cada desplazamiento a la izquierda multiplica por 2, y cada desplazamiento a la derecha divide por 2.

\begin{longtable}{|c|c|p{11cm}|}
	\hline
	\textbf{Operador}	& \textbf{Uso} & \textbf{Resultado}  \\
	\hline
	$>>$ & op1 $>>$ op2 & Desplaza los bits de op1 a la derecha una distancia op2. El desplazamiento a la derecha es un operador que desplaza los bits del operando
	situado a su izquierda hacia la derecha el número de sitios marcado por el operando
	situado a su derecha. Los bits que superan el extremo derecho del byte se pierden.
	En tipos unsigned, los lugares vacantes a la izquierda se rellenan con ceros. En tipos
	con signo el resultado depende del ordenador utilizado; los lugares vacantes se pueden
	rellenar con ceros o bien con copias del signo (bit extremo izquierdo). En un valor sin
	signo tendremos (10001010)$>>$2 == (00100010) en el que cada bit se ha movido dos lugares hacia la derecha. \\
	\hline
	$<<$ & op1 $<<$ op2 & Desplaza los bits de op1 a la izquierda una distancia op2. El desplazamiento a la izquierda es un operador que desplaza los bits del operando
	izquierdo a la izquierda el número de sitios indicando por el operados de su derecha.
	Las posiciones vacantes se rellenan con ceros, y los bits que superan el límite del byte
	se pierden. Así: (10001010)$<<$2 == (1000101000) cada uno de los bits se ha movido dos lugares hacia la izquierda \\
	\hline
	$>>>$ & op1 $>>>$ op2 & Desplaza los bits de op1 a la derecha una distancia op2 (positiva) \\
	\hline
	\& & op1 \& op2 & Operador AND a nivel de bits. El and de bits es un operador que hace la comparación bit por bit entre dos operandos. Para
	cada posición de bit, el bit resultante es 1 únicamente cuando ambos bits de los operandos
	sean 1. En terminología lógica diríamos que el resultado es cierto si, y sólo si, los dos bit
	que actúan como operandos lo son también. Por tanto (10010011) \& (00111101) == (00010001) ya que únicamente en los bits 4 y 0 existe un 1 en ambos operandos.\\
	\hline
	| & op1 | op2 & Operador OR a nivel de bits. El or para bits es un operador binario realiza una comparación bit por bit entre dos operandos.
	En cada posición de bit, el bit resultante es 1 si alguno de los operandos o ambos contienen
	un 1. En terminología lógica, el resultado es cierto si uno de los bits de los operandos es
	cierto, o ambos lo son. Así: (10010011) | (00111101) == (10111111) porque todas las posiciones de bits con excepción del bit número 6 tenían un valor 1 en, por
	lo menos, uno de los operandos. \\
	\hline
	$\wedge$ & op1 $\wedge$ op2 & Operador XOR a nivel de bits (1 si sólo uno de los operandos es 1). El or exclusivo de bits es un operador binario que realiza una comparación bit por bit entre
	dos operandos. Para cada posición de bit, el resultante es 1 si alguno de los operandos
	contiene un 1; pero no cuando lo contienen ambos a la vez. En terminología, el resultado es
	cierto si lo es el bit u otro operando, pero no si lo son ambos. Por ejemplo: (10010011) $\wedge$ (00111101) == (10101110) Observe que el bit de posición 0 tenía valor 1 en ambos operandos; por tanto, el bit resultante
	ha sido 0. \\
	\hline
	$\sim$ & $\sim$op2 & Operador complemento (invierte el valor de cada bit). El complemento a uno o negación en bits es un operador unario, cambia todos los 1 a 0 y
	los 0 a 1. Así: $\sim$(10011010) == 01100101 \\
	\hline
\end{longtable}

En binario, las potencias de dos se representan con un único bit activado. Por ejemplo, los
números (1, 2, 4, 8, 16, 32, 64, 128) se representan respectivamente de modo binario en la forma
(00000001, 00000010, 00000100, 00001000, 00010000, 00100000, 01000000, 10000000),
utilizando sólo 8 bits. La suma de estos números permite construir una variable flags con los bits
activados que se deseen. Por ejemplo, para construir una variable flags que sea 00010010 bastaría
hacer flags=2+16.


\begin{tabular}{|c|c|c|}
	\hline
	\textbf{Operador}	& \textbf{Utilización} &  \textbf{Expresión equivalente} \\
	\hline
	\&= & op1 \&= op2 & op1 = op1 \& op2 \\
	\hline
	|= & op1 |= op2 & op1 = op1 | op2 \\
	\hline
	$\wedge$ = & op1 $\wedge$ = op2 & op1 = op1 $\wedge$  op2 \\
	\hline
	$>>$= & op1 $>>$= op2 & op1 = op1 $>>$ op2 \\
	\hline
	$<<$= & op1 $<<$= op2 & op1 = op1 $<<$  op2 \\
	\hline
	$>>>$= & op1 $>>>$= op2 & op1 = op1 $>>>$  op2 \\
	\hline
\end{tabular}