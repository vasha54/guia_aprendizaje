\begin{lstlisting}[language=C++]
#include <iostream>
#include <set>
#include <vector>
using namespace std ;

bool comvec(vector<int> v1 ,vector<int> v2){//Esta es el comparador
   return v1.size()<v2.size();
}

int main() {
   //Creacion de un conjunto con 'vector<int>' como Tipo de Dato
   set<vector<int>, bool(*)( vector<int>, vector<int>)> conjvec (comvec);
	
   set<int> conj ;
   //Adicionamos al conjunto los elementos
   conj.insert(69); conj.insert(80); conj.insert(77); 
   conj.insert(82); conj.insert(75); conj.insert(81);
   conj.insert(78);
   if(conj.empty()){cout<<" El conjunto esta vacio\n" ;} 
   else{
      cout<<" El conjunto lleva " ;
      cout<<conj.size()<<" elementos\n " ;
   }//Salida : El conjunto lleva 7 elementos
   for(set<int>::iterator it = conj.begin(); it!=conj.end(); it++){
      cout <<*it<<" ";
   } 
   cout<<"\n" ;//Resultado: 69 75 77 80 81 82
   
   if(conj.count(77)==1) cout<<" 77 esta en el conjunto\n";
   else cout<<"77 no esta en el conjunto\n";
   //Resultado: 77 esta en el conjunto
   
   if(conj.find(100)!=conj.end()) cout<<"100 esta en el conjunto\n";
   else cout<<" 100 no esta en el conjunto\n";
   //Resultado: 100 no esta en el conjunto

   if(conj.lower_bound(80)!=conj.upper_bound(80))
      cout<<" 80 esta en el conjunto\n";
   else cout<<" 80 no esta en el conjunto\n" ;
   //Resultado: 80 esta en el conjunto
   
   if(conj.lower_bound(70)==conj.upper_bound(70))
      cout<<" 70 no esta en el conjunto\n";
   else cout<<" 70 esta en el conjunto \n" ;
   //Resultado: 70 no esta en el conjunto
   
   set<int> c1, c2;
   vector<int> unionc, difc, interc;
   
   c1.insert(1);c1.insert(2);c1.insert(3);c1.insert(4);
   c1.insert(5);c1.insert(6);c1.insert(7);
   
   c2.insert(4);c2.insert(3);c2.insert(6);c2.insert(1);
   c2.insert(9);c2.insert(10);c2.insert(7);
   
   set_union(c1.begin(),c1.end(),c2.begin(),c2.end(),back_inserter(unionc));
   set_difference(c1.begin(),c1.end(),c2.begin(),c2.end(),back_inserter(difc));
   set_union(c1.begin(),c1.end(),c2.begin(),c2.end(),back_inserter(interc));
   
   cout<<"Union :";for(auto x : unionc) cout<<x<<" "; cout<<endl;
   cout<<"Diferencia :";for(auto x : difc) cout<<x<<" "; cout<<endl;
   cout<<"Interseccion :";for(auto x : interc)cout<<x<<" ";cout<<endl;
   //Union :1 2 3 4 5 6 7 9 10 
   //Diferencia :2 5 
   //Interseccion :1 2 3 4 5 6 7 9 10
   
   return 0;
}
\end{lstlisting}