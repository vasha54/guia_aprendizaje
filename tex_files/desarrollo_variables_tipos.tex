\subsection{Variable}

Una variable es un nombre que contiene un valor que puede cambiar a lo largo del programa. De
 acuerdo con el tipo de información que contienen, en C++ y Java hay dos tipos principales de variables:

\begin{enumerate}
	\item Variables de tipos primitivos. Están definidas mediante un valor único que puede ser
entero, de punto flotante, carácter o booleano.
	\item Variables referencia. Las variables referencia son referencias o nombres de una
información más compleja: arrays u objetos de una determinada clase.
\end{enumerate}

Desde el punto de vista del papel o misión en el programa, las variables pueden ser:

\begin{enumerate}
	\item Variables \textbf{miembro} de una clase: Se definen en una clase, fuera de cualquier método;
pueden ser tipos primitivos o referencias.
	\item Variables \textbf{locales}: Se definen dentro de un método o más en general dentro de cualquier
bloque entre llaves {}. Se crean en el interior del bloque y se destruyen al finalizar dicho
bloque. Pueden ser también tipos primitivos o referencias.
	
	\item Variables \textbf{globales}: Caso especial para el tipo de estructura de solución en C++, se define  fuera de cualquier método y siempre por encima de cualquiera de ellos; pueden ser tipos primitivos o referencias.
\end{enumerate}

Los nombres de variables se pueden crear con mucha libertad. Pueden ser cualquier
conjunto de caracteres numéricos y alfanuméricos, sin algunos caracteres especiales utilizados por los lenguajes de programación como lo operadores aritméticos. 

Existe una serie de palabras reservadas las cuales tienen un significado especial tanto para C++ como para Java y
por lo tanto no se pueden utilizar como nombres de variables.

\subsection{Tipos Primitivos de Variable}

Se llaman tipos primitivos de variables de Java a aquellas variables sencillas que contienen los tipos
de información más habituales: valores boolean, caracteres y valores numéricos enteros o de punto
flotante.

Java dispone de ocho tipos primitivos de variables: un tipo para almacenar valores true y
false (boolean); un tipo para almacenar caracteres (char), y 6 tipos para guardar valores numéricos,
cuatro tipos para enteros (byte, short, int y long) y dos para valores reales de punto flotante (float y
double). Los rangos y la memoria que ocupa cada uno de estos tipos se muestran en la la siguiente tabla

\begin{tabular}{|p{2.5cm}|p{12.3cm}|}
	\hline 
	\textbf{Declaración}  & \textbf{Rango}  \\ 
	\hline 
	\textbf{boolean} & true - false   \\ 
	\hline	
	\textbf{byte} & [-128 .. 127]   \\ 
	\hline 
	\textbf{short} & [-32,768 .. 32,767]  \\ 
	\hline 
	\textbf{int} & [-2$^{31}$ .. 2$^{31}$-1]   \\ 
	\hline	
	\textbf{long} & [-2$^{63}$ .. 2$^{63}$-1]    \\ 
	\hline	
	\textbf{float} & [$\pm3.4*10^{-38}$ .. $\pm3.4*10^{38}$]   \\ 
	\hline	
	\textbf{double} & [$\pm1.7*10^{-308}$ .. $\pm1.7*10^{308}$]  \\ 
	\hline	
	\textbf{char} & ['u0000' .. 'uffff'] o [0 .. 65.535]  \\ 
	\hline
	\textbf{String} & Para almacenar una secuencia de caracteres  \\ 
	\hline	
\end{tabular}

En caso de C++ la tabla quedaría de la siguiente manera:

\begin{tabular}{|p{2.5cm}|p{12.3cm}|}
	\hline 
	\textbf{Declaración}  & \textbf{Rango}   \\ 	
	\hline 
	\textbf{bool}  & true - false   \\ 
	\hline	
	\textbf{short}     & [-2$^{15}$ .. 2$^{15}$-1]   \\ 
	\hline	
	\textbf{int}    & [-2$^{31}$ .. 2$^{31}$-1]   \\ 
	\hline	
	\textbf{long long}  & [-2$^{63}$ .. 2$^{63}$-1]    \\ 
	\hline	
	\textbf{float}   & [$\pm1.18e-38 $ .. $\pm3.40e38$] Precisión científica ( 7-dígitos)  \\ 
	\hline	
	\textbf{double}     & [$\pm2.23e-308$ .. $\pm1.79e308$] Precisión científica (15-dígitos)  \\ 
	\hline	
	\textbf{long double}     & [$\pm3.37e-4932$ .. $\pm1.18e4932$]  Precisión científica (18-dígitos) \\ 
	\hline	
	\textbf{char} & [-128 ... 127]  \\ 
	\hline	
	\textbf{unsigned char} & [0 ... 255]  \\ 
	\hline
	\textbf{string} & Para almacenar una secuencia de caracteres  \\ 
	\hline
\end{tabular}
\vspace*{1em}

En casos de los tipos de datos que soportan datos númericos de tipo entero si se coloca delante del tipo dato el modificador \textbf{unsigned} indica que la variable solo soportará enteros neutros y positivos desplazando el rango de 0 a 2$^{16}$-1 , 2$^{32}$-1 o 2$^{64}$-1  según sea el caso.

\subsection{Cómo se definen e inicializan las variables}

Una variable se define especificando el tipo y el nombre de dicha variable. Estas variables pueden
ser tanto de tipos primitivos como referencias a objetos de alguna clase perteneciente al lenaguaje de programación o generada por el usuario. Si no se especifica un valor en su declaración, las variable
primitivas se inicializan a cero (salvo boolean y char, que se inicializan a false y $'\\0'$).
Análogamente las variables de tipo referencia son inicializadas por defecto a un valor especial:
null.

\begin{lstlisting}[language=C++]
int x; //Declaracion de la variable x. Se inicializa a 0
int y = 5; //Declaracion de la variable y. Se inicializa a 5
\end{lstlisting} 

\subsection{Constante}

Contrario a una variable, una constante es un determinado objeto
cuyo valor no puede ser alterado durante el proceso de una tarea específica. En C, C++ para declarar
variables no existe una palabra especial, es decir, las variables se declaran escribiendo el tipo seguido
de uno o más identificadores o nombres de variables. Por otro lado, para declarar constantes existe
la palabra reservada const, así como la directiva $\sharp define$. A continuación se muestran ejemplos de declaración de variables y constantes.

\begin{lstlisting}[language=C++]
//Constante primera variante
#define A 100

//Constante segunda variante
const double b = 100;

//Variables 
int a,c;
\end{lstlisting}

A diferencia de las constantes declaradas con la palabra const los símbolos definidos con $\sharp define$ no
ocupan espacio en la memoria del código ejecutable resultante.
El tipo de la variable o constante puede ser cualquiera de los listados en Tipos primitivos, o bien de
 un tipo definido por el usuario.

En el caso de Java para definir una constante basta con usar la palabra reservada \textbf{final}

\begin{lstlisting}[language=C++]
final double PI=3.14159265358979323846;
\end{lstlisting}

Las constantes son usadas a menudo con un doble propósito, el primero es con el fin de hacer más
 legible el código del programa
