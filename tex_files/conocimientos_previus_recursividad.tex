\subsection{Concepto de función o método}
Las aplicaciones informáticas que habitualmente se utilizan, incluso a nivel de informática
 personal, suelen contener decenas y aún cientos de miles de líneas de código fuente. A medida
que los programas se van desarrollando y aumentan de tamaño, se convertirían rápidamente en
sistemas poco manejables si no fuera por la modularización, que es el proceso consistente en
dividir un programa muy grande en una serie de módulos mucho más pequeños y manejables.
A estos módulos se les ha solido denominar de distintas formas (subprogramas, subrutinas,
procedimientos, funciones, métodos etc.) según los distintos lenguajes. Sea cual sea la nomenclatura, la idea es sin embargo siempre
la misma: dividir un programa grande en un conjunto de subprogramas o funciones más
pequeñas que son llamadas por el programa principal; éstas a su vez llaman a otras funciones
más específicas y así sucesivamente.

La división de un programa en unidades más pequeñas o funciones presenta –entre
otras– las ventajas siguientes:

\begin{enumerate}
	\item \textbf{Modularización} Cada función tiene una misión muy concreta, de modo que nunca tiene
un número de líneas excesivo y siempre se mantiene dentro de un tamaño manejable.
Además, una misma función (por ejemplo, un producto de matrices, una resolución de
un sistema de ecuaciones lineales, ...) puede ser llamada muchas veces en un mismo
programa, e incluso puede ser reutilizada por otros programas. Cada función puede ser
desarrollada y comprobada por separado.
	\item \textbf{Ahorro de memoria y tiempo de desarrollo} En la medida en que una misma función es
utilizada muchas veces, el número total de líneas de código del programa disminuye, y
también lo hace la probabilidad de introducir errores en el programa.
	\item \textbf{Independencia de datos y ocultamiento de información} Una de las fuentes más
comunes de errores en los programas de computador son los efectos colaterales o
perturbaciones que se pueden producir entre distintas partes del programa. Es muy
frecuente que al hacer una modificación para añadir una funcionalidad o corregir un
error, se introduzcan nuevos errores en partes del programa que antes funcionaban
correctamente. Una función es capaz de mantener una gran independencia con el resto
del programa, manteniendo sus propios datos y definiendo muy claramente la interfaz o
comunicación con la función que la ha llamado y con las funciones a las que llama, y no
teniendo ninguna posibilidad de acceso a la información que no le compete.
\end{enumerate}

Una función de C++ o Java es una porción de código o programa que realiza una determinada tarea.
Una función está asociada con un \textbf{identificador o nombre}, que se utiliza para referirse a ella
desde el resto del programa. En toda función utilizada en C++ o Java hay que distinguir entre su \textbf{definición}, su \textbf{declaración} su \textbf{llamada}, su \textbf{valor de retorno} y sus \textbf{argumentos}. En algunos casos tanto la \textbf{definición} y su \textbf{declaración} se realizan en conjunto.





\subsection{La pila de recursión}
La memoria de un ordenador se divide en 4 segmentos:

\begin{itemize}
	\item Segmento de código: almacena las instrucciones del programa en código máquina
	\item Segmento de datos: almacena las variables estáticas o constantes.
	\item Montículo: almacena las variables dinámicas
	\item Pila del programa: Parte destinada a las variables locales y parámetros de la función que se está ejecutando.
\end{itemize}