\subsection{Recorrido en Profundidad (\emph{DFS})}
Búsqueda en profundidad. Una Búsqueda en profundidad (en inglés \emph{DFS} o \emph{Depth First Search}) es un algoritmo que permite recorrer todos los nodos de un grafo o árbol (teoría de grafos) de manera ordenada, pero no uniforme. Su funcionamiento consiste en ir expandiendo todos y cada uno de los nodos que va localizando, de forma recurrente, en un camino concreto. Cuando ya no quedan más nodos que visitar en dicho camino, regresa (Backtracking), de modo que repite el mismo proceso con cada uno de los hermanos del nodo ya procesado.

\subsection{Recorrido a lo Ancho (\emph{BFS})}
En Ciencias de la Computación, Búsqueda a lo ancho (en inglés \emph{BFS - Breadth First SearchRe}) es un algoritmo de búsqueda no informada utilizado para recorrer o buscar elementos en un grafo (usado frecuentemente sobre árboles). Intuitivamente, se comienza en la raíz (eligiendo algún nodo como elemento raíz en el caso de un grafo) y se exploran todos los vecinos de este nodo. A continuación para cada uno de los vecinos se exploran sus respectivos vecinos adyacentes, y así hasta que se recorra todo el árbol. 

\subsection{Programación Dinámica}
La programación dinámica es un método para reducir el tiempo de ejecución de un algoritmo mediante la utilización de subproblemas superpuestos y subestructuras óptimas. 

La teoría de programación dinámica se basa en una estructura de optimización, la cual consiste en descomponer el problema en subproblemas (más manejables). Los cálculos se realizan entonces recursivamente donde la solución óptima de un subproblema se utiliza como dato de entrada al siguiente problema. Por lo cual, se entiende que el problema es solucionado en su totalidad, una vez se haya solucionado el último subproblema. Dentro de esta teoría, Bellman desarrolla el Principio de Optimalidad, el cual es fundamental para la resolución adecuada de los cálculos recursivos. Lo cual quiere decir que las etapas futuras desarrollan una política óptima independiente de las decisiones de las etapas predecesoras. Es por ello, que se define a la programación dinámica como una técnica matemática que ayuda a resolver decisiones secuenciales interrelacionadas, combinándolas para obtener de la solución óptima. 