Vamos a considerar otra versión del problema de las monedas en donde la tarea es calcular el total de maneras de producir un total $N$ utilizando las monedas. Para este ejemplo donde las denominaciones de las monedas son ${1,3,4}$ y $N=5$ existen un total de 6 maneras:
\begin{enumerate}
	\item 1 + 1 + 1 + 1 + 1 = 5
	\item 3 + 1 + 1 = 5
	\item 1 + 1 + 3 = 5
	\item 1 + 3 + 1 = 5
	\item 1 + 4 = 5
	\item 1 + 4 = 5 
\end{enumerate}

Otra vez vamos a resolver el  problema recursivamente. La función $solve(x)$ calcula la cantidad de formas que se obtiene una suma igual $x$. Por ejemplo, si la denominaciones de las monedas son $c={1,3,4}$, entonces $solve(5)=6$ y su fómula recursiva es:

$$\begin{matrix}
	solve(x) = & solve(x-1)+ \\
	& solve(x-3)+ \\
	& solve(x-4)
\end{matrix}$$

Entonces la fórmula general recursiva es la siguiente:

$$solve(x) = \begin{cases}
	0 & x<0 \\ 
	1 & x=0 \\
	\sum_{c \in monedas} solve(x-c)+1 & x>0 
\end{cases}$$

Si $x < 0$, el valor es 0, porque no hay soluciones. Si $x = 0$, el valor es 1, porque solo hay una forma de formar una suma vacía. De lo contrario, calculamos la suma de todos los valores de la forma $solve(x-c)$ donde c está en monedas.

A menudo, el número de soluciones es tan grande que no es necesario calcular el número exacto, pero es suficiente para dar la respuesta módulo m donde, por ejemplo, $m = 10^9 + 7$. Esto se puede hacer cambiando el código para que todos los cálculos se realicen módulo $m$.

Aún así, esta función no es eficiente, porque puede haber una exponencial
número de maneras de devolver la suma. Sin embargo, aplicando el mismo proceder del caso anterior veremos cómo hacer el función eficiente con el uso de la memorización. 

En este problema, usamos arreglos donde $ready[x]$ indica si el valor de $solve(x)$ ha sido calculado, y si es así, $counts[x]$ contiene este valor. La constante $N$ ha sido elegida para que todos los valores requeridos quepan en los arreglos.

La función maneja los casos base $x < 0$ y $x = 0$ como antes. Luego, la función verifica desde  $ready[x]$ es verdadero si lo es implica que $solve(x)$ ya se ha almacenado en el $count[x]$, y si es así, la función lo devuelve directamente. De lo contrario, la función calcula el valor de $solve(x)$ de forma recursiva y lo almacena en el $value[x]$.

Esta función funciona de manera eficiente, porque la respuesta para cada parámetro $x$ se calcula recursivamente solo una vez. Después de que un valor de $solve(x)$ se haya almacenado en $count[x]$, se puede recuperar de manera eficiente cada vez que se vuelva a llamar a la función con el parámetro $x$. 




 