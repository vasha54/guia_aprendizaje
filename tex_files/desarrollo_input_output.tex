Para la lectura e impresión de datos se puede realizar por dos métodos por \textbf{cin} y \textbf{cout} primitivos
de C++ o por \textbf{printf} y \textbf{scanf} primitivos de C, en el programa se pueden usar ambos métodos si es
necesario.

\subsection{Primitivos de C++}

En sus programas, si usted desea hacer uso de los objetos  \textbf{cin} y \textbf{cout} tendrá que incluir el uso de la
biblioteca iostream ( por medio de la directiva \#include ). La iostream es la biblioteca estándar en
C++ para poder tener acceso a los dispositivos estándar de entrada y/o salida.
Si usted usa la directiva \#include <iostream.h> \quad o \#include <iostream> \quad en sus programas, automáticamente la iostream pone a su disposición los objetos \textbf{cin} y \textbf{cout} en el ámbito estándar (std), de tal
manera que usted puede comenzar a enviar o recibir información a través de los mismos sin siquiera
preocuparse de su creación.

Para el manejo de \textbf{cin} y \textbf{cout} se necesita también el uso de los \textbf{operadores de direccionamiento}
$<<$ y $>>$. Los operadores de direccionamiento son los encargados de manipular el flujo de datos
desde o hacia el dispositivo referenciado por un stream específico. El operador de direccionamiento
para salidas es una pareja de símbolos de "menor que" $<<$, y el operador de direccionamiento
para entradas es una pareja de símbolos de "mayor que" $>>$. Los operadores de direccionamiento
se colocan entre dos operandos, el primero es el stream y el segundo es una variable o constante
que proporciona o recibe los datos de la operación.

\subsubsection{Funciones miembro get y getline}

La función miembro \textbf{get} sin argumentos recibe como entrada un carácter del flujo designado (incluyendo caracteres de espacio en blanco y otros caracteres no gráfi cos, como la secuencia de teclas que representa el fi n de archivo) y lo devuelve como el
valor de la llamada a la función. Esta versión de get devuelve EOF cuando se encuentra el fin del archivo en el flujo.

La función miembro \textbf{getline} opera de manera similar, inserta un carácter nulo después de la línea en el arreglo de caracteres. La función \textbf{getline} elimina el delimitador del flujo (es decir, lee
el carácter y lo descarta), pero no lo almacena en el arreglo de caracteres.

\subsubsection{Precisión de punto flotante ( precision , setprecision )}

Para controlar la precisión de los números de punto flotante (es decir, el número de dígitos a la derecha del punto
decimal), podemos usar el manipulador de flujo \textbf{setprecision} o la función miembro \textbf{precision} de ios\_base . Una
llamada a uno de estos miembros establece la precisión para todas las operaciones de salida subsecuentes, hasta la siguiente llamada para establecer la precisión. Una llamada a la función miembro \textbf{precision} sin argumento devuelve la opción
de precisión actual (esto es lo que necesitamos usar para poder restaurar la precisión original en un momento dado, una
vez que ya no sea necesaria una opción pegajosa).

\subsubsection{Anchura de campos ( width , setw )}

La función miembro \textbf{width} (de la clase base ios\_base ) establece la anchura de campo (es decir, el número de posiciones
de caracteres en los que debe imprimirse un valor, o el número máximo de caracteres que deben introducirse) y devuelve
la anchura anterior. Si los valores que se imprimen son menos que la anchura de campo, se insertan caracteres de relleno
como relleno ( padding). Un valor más ancho que la anchura designada no se truncará; se imprimirá el número completo.
La función \textbf{width} sin argumento devuelve la configuración actual.

\subsubsection{Estados de formato de flujos y manipuladores de flujos}

Se pueden utilizar varios manipuladores de flujos para especificar los tipos de formato a realizar durante las operaciones
de E/S de flujos. Los manipuladores de flujos controlan la configuración del formato de la salida. Todos estos manipuladores pertenecen a
la clase ios\_base.

\begin{longtable}{|c|p{11cm}|}
	\hline
  \textbf{Manipulador de flujo}	& \textbf{Descripción}  \\
	\hline
	skipws & Omite los caracteres de espacio en blanco en un flujo de entrada. Esta opción se restablece con el
manipulador de flujo noskipws . \\
	\hline
	left & Justifica la salida a la izquierda en un campo. Si es necesario, aparecen caracteres de relleno a la
derecha.  \\
	\hline
	right & Justifica la salida a la derecha en un campo. Si es necesario, aparecen caracteres de relleno a la izquierda.  \\
	\hline
	internal & Indica que el signo de un número debe justifi carse a la izquierda en un campo, y que la magnitud
del número se debe justificar a la derecha en ese mismo campo (es decir, deben aparecer caracteres de
relleno entre el signo y el número). \\
	\hline
	dec & Especifica que los enteros deben tratarse como valores decimales (base 10). \\
	\hline
	oct & Especifica que los enteros se deben tratar como valores octales (base 8). \\
	\hline
	hex & Especifica que los enteros se deben tratar como valores hexadecimales (base 16). \\
	\hline
	showbase & Especifica que la base de un número se debe imprimir adelante del mismo (un 0 a la izquierda para
los valores octales; 0x o 0X a la izquierda para los valores hexadecimales). Esta opción se restablece
con el manipulador de flujo noshowbase . \\
	\hline
	showpoint & Especifica que los números de punto f lotante se deben imprimir con un punto decimal. Esto se usa
generalmente con fixed para garantizar cierto número de dígitos a la derecha del punto decimal, aun
y cuando sean ceros. Esta opción se restablece con el manipulador de flujo noshowpoint. \\
	\hline
	uppercase & Especifica que deben usarse letras mayúsculas (es decir, X y de la A a la F ) en un entero hexadecimal,
y que se debe usar la letra E al representar un valor de punto flotante en notación científica. Esta
opción se restablece con el manipulador de flujo nouppercase . \\
	\hline
	showpos & Especifica que a los números positivos se les debe anteponer un signo positivo ( + ). Esta opción se
restablece con el manipulador de flujo noshowpos . \\
	\hline
	scientific & Especifica la salida de un valor de punto flotante en notación científica. \\
	\hline
	fixed & Especifica la salida de un valor de punto flotante en notación de punto fijo, con un número específico de dígitos a la derecha del punto decimal. \\
	\hline
\end{longtable}

\subsection{Primitivos de C}

En sus programas, si usted desea hacer uso de los objetos \textbf{scanf} y \textbf{printf} tendrá que incluir el uso de
la biblioteca cstdio ( por medio de la directiva \#include <cstdio> ). Cstdio es la biblioteca estándar en C para
poder tener acceso a los dispositivos estándar de entrada y/o salida.

Por defecto las funciones de entrada/salida de C (Input/Output) son un conjunto de funciones, incluidas
con el compilador, que permiten a un programa recibir y enviar datos al exterior. Para su
utilización es necesario incluir, al comienzo del programa, el archivo stdio.h en el que están
definidos sus prototipos \#include <stdio.h>
donde stdio proviene de \emph{standard-input-output}.

Generalmente, \textbf{printf()} y \textbf{scanf()} funcionan utilizando cada una de ellas una \emph{tira de caracteres decontrol} y una lista de \emph{argumentos}. Veremos estas características; en primer lugar en \textbf{printf()}, y a
continuación en \textbf{scanf()}.



Las instrucciones que se han de dar a \textbf{printf()} cuando se desea imprimir una variable dependen del
tipo de variable de que se trate. Así, tendremos que utilizar la notación \%d para imprimir un entero,
y \%c para imprimir un carácter, como ya se ha dicho. A continuación damos la lista de todos los
identificadores que emplea la función \textbf{printf()} y el tipo de salida que imprimen. La mayor parte de
sus necesidades queda cubierta con los ocho primeros; de todas formas, ahí están los dos restantes
por si desea emplearlos.

\begin{tabular}{|c|p{12cm}|}
	\hline
	\textbf{Identificador} & \textbf{Salida} \\
	\hline
	\%c & Carácter o entero pequeño \\
	\hline
	\%s &  Tira de caracteres \\
	\hline
	\%d, \%i &  Entero decimal \\
	\hline
	\%u &  Entero decimal sin signo\\
	\hline
	\%lld &  Entero largo\\
	\hline
	\%llu &  Entero largo sin signo\\
	\hline
	\%f &  Número de punto flotante en notación decimal\\
	\hline
	\%lf &  Número de punto flotante en notación decimal con doble presición\\
	\hline
	\%o &  Entero octal sin signo\\
	\hline
	\%x &  Entero hexadecimal sin signo\\
	\hline
\end{tabular}

El formato para uso de \textbf{printf()} es éste:

\qquad \textbf{ printf(Control, item1, item2, ....);} 

\textbf{item1}, \textbf{item2}, etc., son las distintas variables o constantes a imprimir. Pueden también ser expresiones, las cuales se evalúan antes de imprimir el resultado. \textbf{Control} es una tira de caracteres que
describen la manera en que han de imprimirse los items.

También se debe hablar de \textbf{modificadores de especificaciones de conversión} en printf(), estos son
apéndices que se agregan a los especificadores de conversión básicos para modificar la salida. Se
colocan entre el símbolo \textbf{\%} y el carácter que define el tipo de conversión. A continuación se da una
lista de los símbolos que está permitido emplear. Si se utiliza más de un modificador en el mismo
sitio, el orden en que se indican deberá ser el mismo que aparece en la tabla. Tenga presente que no
todas las combinaciones son posibles.

\begin{enumerate}
	\item \textbf{-}: El ítem correspondiente se comenzará a escribir empezando en el extremo izquierdo del
campo que tenga asignado. Normalmente se escribe el ítem de forma que acabe a la
derecha del campo.
Ejemplo: \%-10d
	\item \textbf{número}: Anchura mínima del campo. En el caso de que la cantidad a imprimir (o la tira de caracteres) no quepa en el lugar asignado, se usará automáticamente un campo mayor. Ejemplo: \%4d
	\item \textbf{.número}: Precisión. En tipos flotantes es la cantidad de cifras que se han de imprimir a la derecha
del punto (es decir, el número de decimales). En el caso de tiras, es el máximo número de caracteres que se ha de imprimir. Ejemplo: \%.2f (dos decimales )
\end{enumerate}

Para el uso de \textbf{scanf()} se utilizan los mismo identificadores que en \textbf{printf()}, y al igual que \textbf{printf()},
\textbf{scanf()} emplea una tira de caracteres de control y una lista de argumentos. La mayor diferencia
entre ambas está en esta última; \textbf{printf()} utiliza en sus listas nombres de variables, constantes y
expresiones; \textbf{scanf()} usa punteros a variable. Afortunadamente no se necesita saber mucho de
punteros para emplear esta expresión; se trata simplemente de seguir las dos reglas que se dan a
continuación:

\begin{itemize}
	\item Si se desea leer un valor perteneciente a cualquier de los tipos básicos coloque el nombre de la
variable precedido por un \&.
	\item Si lo que desea es leer una variable de tipo tira de caracteres, no use \&.
\end{itemize}

Otro detalle a tomar en cuenta es que \textbf{scanf()} considera que dos ítems de entrada son diferentes
 cuando están separados por blancos, tabulados o espacios. Va encajando cada especificador de
conversión con su campo correspondiente, ignorando los blancos intermedios. La única excepción
es la especificación \%c, que lee el siguiente caracter, sea blando o no.

\subsubsection{Macros getchar() y putchar()}

Las macros \textbf{getchar()} y \textbf{putchar()} permiten respectivamente leer e imprimir un sólo carácter
cada vez, en la entrada o en la salida estándar. La macro \textbf{getchar()} recoge un carácter
introducido por teclado y lo deja disponible como valor de retorno. La macro \textbf{putchar()}
escribe en la pantalla el carácter que se le pasa como argumento.
