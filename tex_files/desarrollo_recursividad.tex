Lo primero que debemos entender las diferencias de como funciona un metodo lineal o tradicional a uno recursivo. 

Cuando llamamos a una función (o método) el proceso es el siguiente:

\begin{enumerate}
	\item Se reserva espacio en la pila para los parámetros de la función y sus variables locales.
	\item Se guarda en la pila la dirección de la línea del código desde donde se ha llamado al método.
	\item Se almacenan los parámetros de la función y sus valores en la pila.
	\item Finalmente se libera la memora asignada en la pila cuando la función termina y se vuelve a la llamada de código original.
\end{enumerate}

En cambio, cuando la función es recursiva:

\begin{itemize}
	\item Cada llamada genera una nueva llamada a una función con los correspondientes objetos locales
	\item Volviéndose a ejecutar completamente, hasta la llamada a si misma. Donde vuelve a crear en la pila los nuevos parámetros y variables locales. Tantos como llamadas recursivas generemos.
	\item Al terminar, se van liberando la memoria en la pila, empezando desde la última función creada hasta la primera, la cual será la ultima en liberarse.
\end{itemize}

En la implementación de un algoritmo recursivo consta de dos partes:
\begin{itemize}
	\item Caso base: Es la resolución del problema de manera
directa
	\item Caso Recursivo: Caso en el que el problema se divide en
versiones más pequeñas de si mismo.
\end{itemize}

Es importante hacer notar que en la implementación de los algoritmos recursivos puede existir más de un caso base y más de un caso recursivo lo que siempre debe existir uno de cada tipo porque la ausencia de uno de estos casos significa que o tenemos un algoritmo no recursivo (si falta el caso recursivo) o un algortimo que provoca un bucle sin fin (si falta el caso base)

\subsection{Cómo diseñar un algoritmo recursivo}

\begin{enumerate}
	\item Reconocer el caso base y proporcionar una solución
para él.
	\item Diseñar una estrategia para dividir el problema en
versiones más pequeñas del mismo considerando
avanzar hacia el caso base.
	\item Combine las soluciones de los problemas más
pequeños para obtener la solución del problema
original.
\end{enumerate}


Según el modo en que se realiza la llamada recursiva se puede clasificar o agrupar como:

\begin{itemize}
	\item \textbf{Directa:} Cuando un método/función se invoca así
mismo.
	\item \textbf{Indirecta o mutua :} Cuando un método/función puede
invocar a una segunda función/método que a su
vez invoca a la primera
	\item \textbf{Recursión lineal no final:} En la recursión lineal no final el resultado de la llamada recursiva se combina en una
	expresión para dar lugar al resultado de la función que llama.
	\item \textbf{Recursión lineal final:} En la recursión lineal final el resultado que es devuelto es el resultado de ejecución de la
	última llamada recursiva.
	\item \textbf{Recursión múltiple:} Alguna llamada recursiva puede generar más de una llamada a la función.
\end{itemize}


