\subsection{Bifurcación if}
Esta estructura permite ejecutar un conjunto de sentencias en función del valor que tenga la
expresión de comparación (se ejecuta si la expresión de comparación tiene valor true). Tiene la
forma siguiente:

\begin{lstlisting}[language=C++]
if (booleanExpression) {
   statements;
}
	
\end{lstlisting}

Las llaves \{\} sirven para agrupar en un bloque las sentencias que se han de ejecutar, y no son
necesarias si sólo hay una sentencia dentro del if.

\subsection{Bifurcación if else}

Análoga a la anterior, de la cual es una ampliación. Las sentencias incluidas en el else se ejecutan en
el caso de no cumplirse la expresión de comparación (false),

\begin{lstlisting}[language=C++]
if (booleanExpression) {
   statements1;
} else {
   statements2;
}	
\end{lstlisting}

\subsection{Bifurcación if elseif else}

Permite introducir más de una expresión de comparación. Si la primera condición no se cumple, se
compara la segunda y así sucesivamente. En el caso de que no se cumpla ninguna de las
comparaciones se ejecutan las sentencias correspondientes al else.

\begin{lstlisting}[language=C++]
if (booleanExpression1) {
   statements1;
} else if (booleanExpression2) {
   statements2;
} else if (booleanExpression3) {
   statements3;
} else {
   statements4;
}	
\end{lstlisting}

En esta estructura se puede omitir del bloque \textbf{else} de ser necesario. 