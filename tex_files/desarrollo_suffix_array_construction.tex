\subsubsection{Trivial}
Este es el enfoque más ingenuo. Obtenga todos los sufijos y ordénelos usando ordenación rápida o ordenación por mezcla y conserve simultáneamente sus índices originales.

\subsubsection{Variante de desplazamientos cíclicos de una cadena}

Estrictamente hablando, el siguiente algoritmo no ordenará los sufijos, sino los 
desplazamientos cíclicos de una cadena. Sin embargo, podemos derivar muy 
fácilmente un algoritmo para ordenar sufijos a partir de él: basta con añadir un 
carácter arbitrario al final de la cadena que sea más pequeño que cualquier 
carácter de la cadena. Es común utilizar el símbolo \$. Entonces el orden de los 
desplazamientos cíclicos ordenados es equivalente al orden de los sufijos 
ordenados, como se demuestra aquí con la cadena $dabb$.

$$\begin{array}{lll} 1. & abbb\$d & abbb \\ 4. & b\$dabb & b \\ 3. & bb\$dab & bb \\ 2. & bbb\$da & bbb \\ 0. & dabbb\$ & dabbb \end{array}$$

Como vamos a ordenar los desplazamientos cíclicos, consideraremos \textbf{subcadenas cíclicas}. Usaremos la notación $s[i \dots j]$ para la subcadena de $s$
incluso si $i>j$. En este caso en realidad nos referimos a la cadena $s[i \dots n-1]+s[0 \dots j]$. Además tomaremos todos los índices módulo la 
longitud de $s$ y omitirá la operación del módulo por simplicidad.

El algoritmo que analizamos funcionará $\lceil \log n \rceil + 1$ iteraciones. En el $k$-ésima iteración ($k = 0 \dots \lceil \log n \rceil$) ordenamos el
$n$ subcadenas cíclicas de $s$ de longitud $2^k$. Después de la $\lceil \log n \rceil$-ésima iteración de las subcadenas de longitud
$2^{\lceil \log n \rceil} \ge n$ se ordenarán, por lo que esto equivale a ordenar los cambios cíclicos por completo.

En cada iteración del algoritmo, además de la permutación $p[0 \dots n-1]$, dónde $p[i]$ es el índice de la $i$-ésima subcadena (comenzando en $i$ y con
longitud $2^k$) en el orden ordenado, también mantendremos una matriz $c[0 \dots n-1]$, dónde $c[i]$ corresponde a la \textbf{clase de equivalencia} a la que 
pertenece la subcadena. Porque algunas de las subcadenas serán idénticas y el algoritmo debe tratarlas por igual. Para mayor comodidad, las clases se
etiquetarán con números que comienzan desde cero. Además los números $c[i]$ se asignarán de tal manera que conserven información sobre el orden: si una subcadena es más pequeña que la otra, entonces también debería tener una etiqueta de clase más pequeña. El número de clases de equivalencia se almacenará en 
una variable $\text{clases}$.

Veamos un ejemplo. Considere la cuerda $s = aaba$. Las subcadenas cíclicas y las matrices correspondientes $p[]$ y $c[]$ se dan para cada iteración:

$$\begin{array}{cccc} 0: & (a,~ a,~ b,~ a) & p = (0,~ 1,~ 3,~ 2) & c = (0,~ 0,~ 1,~ 0)\\ 1: & (aa,~ ab,~ ba,~ aa) & p = (0,~ 3,~ 1,~ 2) & c = (0,~ 1,~ 2,~ 0)\\ 2: & (aaba,~ abaa,~ baaa,~ aaab) & p = (3,~ 0,~ 1,~ 2) & c = (1,~ 2,~ 3,~ 0)\\ \end{array}$$

Vale la pena señalar que los valores de $p[]$ puede ser diferente. Por ejemplo en el $0$-ésima iteración la matriz también podría ser $p=(3,~1,~0,~2)$ o
$p=(3,~0,~1,~2)$. Todas estas opciones permutan las subcadenas en un orden ordenado. Entonces todos son válidos. Al mismo tiempo la matriz $c[]$ es fijo, no 
puede haber ambigüedades.

Centrémonos ahora en la implementación del algoritmo. Escribiremos una función que tome una cadena $s$ y devuelve las permutaciones de los cambios cíclicos ordenados.

Al principio (en el $0$-ésima iteración) debemos ordenar las subcadenas cíclicas de longitud $1$, es decir, tenemos que ordenar todos los caracteres de la 
cadena y dividirlos en clases de equivalencia (los mismos símbolos se asignan a la misma clase). Esto se puede hacer de forma trivial, por ejemplo, utilizando 
la ordenación por conteo. Para cada carácter contamos cuántas veces aparece en la cadena y luego usamos esta información para crear la matriz $p[]$. Después 
de eso pasamos por la matriz $p[]$ y construir $c[]$ comparando caracteres adyacentes.

Ahora tenemos que hablar del paso de iteración. Supongamos que ya hemos realizado el $k-1$-ésimo paso y calculó los valores de las matrices $p[]$ y $c[]$ para ello. Queremos calcular los valores de $k$-ésimo paso en $O(n)$ tiempo. Ya que realizamos este paso $O(\log n)$ veces, el algoritmo completo tendrá una complejidad temporal de $O(n \log n)$.

Para hacer esto, tenga en cuenta que las subcadenas cíclicas de longitud $2^k$ consta de dos subcadenas de longitud $2^{k-1}$ que podemos comparar entre sí en $O(1)$ utilizando la información de la fase anterior - los valores de las clases de equivalencia $c[]$. Así, para dos subcadenas de longitud $2^k$ comenzando en la posición $i$ y $j$, toda la información necesaria para compararlos está contenida en los pares $(c[i],~ c[i+2^{k-1}])$ y $(c[j],~c[j+2^{k-1}])$.

$$\dots \overbrace{ \underbrace{s_i \dots s_{i+2^{k-1}-1}}_{\text{length} = 2^{k-1},~ \text{class} = c[i]} \quad \underbrace{s_{i+2^{k-1}} \dots s_{i+2^k-1}}_{\text{length} = 2^{k-1},~ \text{class} = c[i + 2^{k-1}]} }^{\text{length} = 2^k} \dots \overbrace{ \underbrace{s_j \dots s_{j+2^{k-1}-1}}_{\text{length} = 2^{k-1},~ \text{class} = c[j]} \quad \underbrace{s_{j+2^{k-1}} \dots s_{j+2^k-1}}_{\text{length} = 2^{k-1},~ \text{class} = c[j + 2^{k-1}]} }^{\text{length} = 2^k} \dots$$

Esto nos da una solución muy simple: ordenar las subcadenas de longitud $2^k$ por estos pares de números. Esto nos dará el orden requerido $p[]$. Sin embargo, se ejecuta una clasificación normal $O(n \log n)$ tiempo, con el que no estamos satisfechos. Esto sólo nos dará un algoritmo para construir una matriz de sufijos en $O(n\log^2n)$ veces.

¿Cómo realizamos rápidamente tal clasificación de los pares? Dado que los elementos de los pares no exceden $n$, podemos usar la ordenación por conteo 
nuevamente. Sin embargo, ordenar pares con ordenación por conteo no es lo más eficiente. Para lograr una constante mejor oculta en la complejidad, usaremos 
otro truco.

Usamos aquí la técnica en la que se basa la clasificación por base: para ordenar los pares primero los clasificamos por el segundo elemento y luego por el 
primer elemento (con una clasificación estable, es decir, sin romper el orden relativo de elementos iguales). Sin embargo, los segundos elementos ya estaban 
ordenados en la iteración anterior. Así, para ordenar los pares por los segundos elementos, sólo necesitamos restar $2^{k-1}$ de los índices en $p[]$ (por 
ejemplo, si la subcadena más pequeña de longitud $2^{k-1}$ comienza en la posición $i$, entonces la subcadena de longitud $2^k$ con la segunda mitad más 
pequeña comienza en $i-2^{k-1}$).

Así que sólo mediante restas simples podemos ordenar los segundos elementos de los pares en $p[]$. Ahora necesitamos realizar una clasificación estable por
los primeros elementos. Como ya se mencionó, esto se puede lograr con la clasificación por conteo.

Lo único que queda es calcular las clases de equivalencia $c[]$, pero como antes, esto se puede hacer simplemente iterando sobre la permutación ordenada
$p[]$ y comparar pares vecinos.

Aquí está la implementación restante. Usamos matrices temporales $pn[]$ y $cn[]$ para almacenar la permutación por los segundos elementos y los nuevos índices 
de clase equivalentes.

El algoritmo requiere $O(n\log n)$ tiempo y $O(n)$ memoria. Por simplicidad utilizamos el rango ASCII completo como alfabeto.

Si se sabe que la cadena sólo contiene un subconjunto de caracteres, por ejemplo, sólo letras minúsculas, entonces la implementación se puede optimizar, pero 
el factor de optimización probablemente sería insignificante, ya que el tamaño del alfabeto sólo importa en la primera iteración. Cada dos iteraciones dependen 
del número de clases de equivalencia, que pueden alcanzar rápidamente $O(n)$ incluso si inicialmente era una cadena sobre el alfabeto de tamaño $2$.

Tenga en cuenta también que este algoritmo solo ordena los cambios de ciclo. Como se mencionó al principio de esta sección, podemos generar el orden de los 
sufijos agregando un carácter que sea más pequeño que todos los demás caracteres de la cadena y ordenando esta cadena resultante por cambios de ciclo, por  ejemplo, ordenando los cambios de ciclo de \$s.+\$\$. Obviamente, esto dará la matriz de sufijos de $s$, sin embargo antepuesto con $|s|$.



