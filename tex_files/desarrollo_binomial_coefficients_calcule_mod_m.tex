Muy a menudo te encuentras con el problema de calcular coeficientes binomiales módulo algunos $m$.

\subsubsection{Coeficiente de binomial para pequeños $n$}

El enfoque previamente discutido del triángulo de Pascal se puede utilizar para calcular todos los valores de $\binom n  k  \bmod m$ por un tamaño razonablemente pequeño $n$, ya que requiere complejidad de tiempo O$(n^2)$. Este enfoque puede manejar cualquier módulo, ya que solo se utilizan operaciones de suma.

\subsubsection{Coeficiente binomial módulo primo grande}

La fórmula para los coeficientes binomiales es:

$$\binom n k = \frac {n!} {k!(n-k)!},$$

entonces si queremos calcularlo módulo algún primo $m > n$ obtenemos:

$$\binom n k \equiv n! \cdot (k!)^{-1} \cdot ((n-k)!)^{-1} \mod m$$

Primero calculamos previamente todos los módulos factoriales $m$ hasta $\text{MAXN}!$ en O($\text{MAXN}$) tiempo. Y luego podemos calcular el coeficiente binomial en O$(\log m)$ tiempo. 
Incluso podemos calcular el coeficiente binomial en $O(1)$ tiempo si precalculamos los inversos multiplicativos de todos los factoriales en O$(\text{MAXN} \log m)$ utilizando el método habitual 
para calcular el inverso multiplicativo, o incluso en O$(\text{MAXN})$ tiempo usando la congruencia $(x!)^{-1} \equiv ((x-1)!)^{-1} \cdot x^{-1}$ y el método para calcular todas todos los 
inversos en O$(n)$.

\subsubsection{Coeficiente binomial módulo potencia prima}

Aquí queremos calcular el coeficiente binomial módulo de alguna potencia prima, es decir $m = p^b$ por alguna prima $p$. Si $p > \max(k,nk)$, entonces podemos usar el mismo método descrito en la sección anterior. Pero si $p \le \max(k, nk)$, entonces al menos uno de $k!$ y $(nk)!$ no son coprimos con $m$, y por lo tanto no podemos calcular las inversas: no existen. Sin embargo, podemos calcular el coeficiente binomial.

La idea es la siguiente: Calculamos para cada $x!$ el mayor exponente $c$ tal que $p^c$ divide $x!$, es decir $p^c ~|~ x!$. Dejar $c(x)$ sea es número. Y deja $g(x) := \frac{x!}{p^{c(x)}}$. Entonces podemos escribir el coeficiente binomial como:

$$\binom n k = \frac {g(n) p^{c(n)}} {g(k) p^{c(k)} g(n-k) p^{c(n-k)}} = \frac {g(n)} {g(k) g(n-k)}p^{c(n) - c(k) - c(n-k)}$$

Lo interesante es que $g(x)$ ahora está libre del divisor primo $p$. Por lo tanto $g(x)$ es coprimo de m, y podemos calcular los inversos modulares de $g(k)$ y $g(nk)$.

Después de precalcular todos los valores para $g$ y $c$, que se puede hacer de manera eficiente usando programación dinámica en O($n$), podemos calcular el coeficiente binomial en $O(\log m)$
tiempo. O precalcular todas las inversas y todas las potencias de $p$ y luego calcular el coeficiente binomial en $O(1)$.

Aviso, si $c(n) - c(k) - c(nk) \ge b$, que $p^b ~|~ p^{c(n) - c(k) - c(nk)}$, y el coeficiente binomial es $0$

\subsubsection{Módulo de coeficiente binomial un número arbitrario}

Ahora calculamos el módulo del coeficiente binomial algún módulo arbitrario $m$.

Sea la factorización prima de $m$ ser $m=p_1^{e_1} p_2^{e_2} \cdots p_h^{e_h}$. Podemos calcular el módulo del coeficiente binomial $p_i^{e_i}$ para cada $i$. esto nos da $h$ diferentes 
congruencias. Dado que todos los módulos $p_i^{e_i}$ son coprimos, podemos aplicar el teorema chino del resto para calcular el módulo del coeficiente binomial, el producto de los módulos, que es 
el módulo del coeficiente binomial deseado $m$.

\subsubsection{Coeficiente binomial para grandes $n$ y módulo pequeño}

Cuando $n$ es demasiado grande, el O($n$) los algoritmos discutidos anteriormente se vuelven poco prácticos. Sin embargo, si el módulo $m$ es pequeño todavía hay maneras de calcular $\binom{n}{k} \bmod m$ cuando el módulo $m$ es primo, hay 2 opciones:

\begin{itemize}
	\item Se puede aplicar el teorema de Lucas, lo que resuelve el problema de la informática $\binom{n}{k} \bmod m$ en $\log_mn$ problemas de la forma $\binom{x_i}{y_i} \bmod m$
	dónde $x_i, y_i < m$. Si cada coeficiente reducido se calcula utilizando factoriales precalculados y factoriales inversos, la complejidad es O($m + \log_m n$).
	\item El método de calcular el módulo factorial $P$ se puede utilizar para obtener el valor requerido $g$ y $c$ valores y utilícelos como se describe en la sección de módulo de potencia 
	primaria esto toma O$(m \log_m n)$.
\end{itemize}

Cuando $m$ no está libre de cuadrados, se puede aplicar una generalización del teorema de Lucas para potencias primas en lugar del teorema de Lucas.

Cuando $m$ no es primo sino libre de cuadrados, los factores primos de $m$ se puede obtener y el módulo de coeficiente de cada factor primo se puede calcular utilizando cualquiera de los métodos anteriores, y la respuesta general se puede obtener mediante el teorema del resto chino.


