\subsection{C++}
\begin{lstlisting}[language=C++]
vector<int> parent, depth, heavy, head, pos;
int cur_pos;

int dfs(int v, vector<vector<int>> const& adj) {
   int size = 1;
   int max_c_size = 0;
   for (int c : adj[v]) {
      if (c != parent[v]) {
         parent[c] = v, depth[c] = depth[v] + 1;
         int c_size = dfs(c, adj);
         size += c_size;
         if(c_size > max_c_size)
            max_c_size = c_size, heavy[v] = c;
      }
   }
   return size;
}

void decompose(int v, int h, vector<vector<int>> const& adj) {
   head[v] = h, pos[v] = cur_pos++;
   if (heavy[v] != -1) decompose(heavy[v], h, adj);
   for (int c : adj[v]) {
      if (c != parent[v] && c != heavy[v])
         decompose(c, c, adj);
   }
}

void init(vector<vector<int>> const& adj, int root) {
   int n = adj.size(); parent = vector<int>(n);
   depth = vector<int>(n); heavy = vector<int>(n, -1);
   head = vector<int>(n); pos = vector<int>(n);
   cur_pos = 1; dfs(root, adj);
   decompose(root, 0, adj);
}
\end{lstlisting}

La lista de adyacencia del árbol y su raiz debe pasarse a la función \emph{init}. La función \emph{dfs} se usa para calcular $heavy[v]$, el hijo en el otro extremo de la arista pesada desde $v$, para cada vértice $v$. Además, \emph{dfs} también almacena el padre y la profundidad de cada vértice, lo que será útil más adelante durante las consultas. La función \emph{decompose} asigna para cada vértice $v$ los valores $head[v]$ y $pos[v]$, que son, respectivamente, la cabeza del camino pesado al que $v$ pertenece y la posición de $v$ en el árbol de rango único que cubre todos los vértices.

\begin{lstlisting}[language=C++]
struct segtree {
   struct node {
      // Valor inicial de los nodos hojas
      long long mx = LLONG_MIN;
      long long mn = LLONG_MAX;
      long long sum = 0;
      long long add = 0;
      
      void apply(int l, int r, long long v) {
         mx = v; mn = v;
         sum += (r - l + 1) * v;
         add += v;
     }
   };
   
   int nnodes;
   vector<node> tree;
   
   static node unite(const node &a, const node &b) {
      node res; res.mx = max(a.mx, b.mx);
      res.mn = min(a.mn, b.mn); res.sum = a.sum + b.sum;
      return res;
   }
   
   segtree(int _n) : nnodes(_n) {
      tree.resize(2 * nnodes - 1); build(0, 0, nnodes - 1);
   }
	
   void pull(int x, int y) {
      tree[x] = unite(tree[x + 1], tree[y]);
   }
	
   void push(int x, int l, int r) {
      int m = (l + r) >> 1; int y = x + ((m - l + 1) << 1);
      if (tree[x].add != 0) {
         tree[x + 1].apply(l, m, tree[x].add);
         tree[y].apply(m + 1, r, tree[x].add);
         tree[x].add = 0;
      }
   }
	
   void build(int x, int l, int r) {
      if (l == r) { return;}
      int m = (l + r) >> 1;
      int y = x + ((m - l + 1) << 1);
      build(x + 1, l, m); build(y, m + 1, r);
      pull(x, y);
   }
	
   node get(int x, int l, int r, int ll, int rr) {
      if (ll <= l && r <= rr){ return tree[x];}
      int m = (l+r)>>1; int y = x+((m-l+1)<<1);
      push(x, l, r);node res;
      if (rr <= m){res=get(x+1,l,m,ll,rr);} 
      else{ 
         if(ll > m){res = get(y, m + 1, r, ll, rr);}
         else{
            res=unite(get(x+1,l,m,ll,rr),get(y,m+1,r,ll,rr));
         }
      }
      pull(x, y); return res;
   }
	
   node get(int ll, int rr) {
      return get(0, 0, nnodes - 1, ll, rr);
   }
   
   void modify(int x, int l, int r, int ll, int rr, long &v) {
      if(ll <= l && r <= rr){tree[x].apply(l,r,v);return;}
      int m = (l + r) >> 1; int y = x + ((m - l + 1) << 1);
      push(x, l, r);
      if (ll <= m){ modify(x + 1, l, m, ll, rr, v);}
      if (rr > m){ modify(y, m + 1, r, ll, rr, v);}
      pull(x, y);
   }
   
   void modify(int ll, int rr, long v) {
      modify(0, 0, nnodes - 1, ll, rr, v);
   }
};

class HeavyLight {
   public:
      vector<vector<int>> tree;
      // verdadero los valores en los vertices
      // falso los valores en las aristas
      bool valuesOnVertices; 
      segtree segment_tree;
      vector<int> parent,depth,pathRoot,in;
      
      HeavyLight(const vector<vector<int>> &t,int root, bool valuesOnVertices):
         tree(t), valuesOnVertices(valuesOnVertices), segment_tree(t.size()), 
         parent(t.size()), depth(t.size()), pathRoot(t.size()), in(t.size()){
         
         int time = 0;parent[root] = -1;
         
         function<int(int)> dfs1 = [&](int u) {
            int size = 1; int maxSubtree = 0;
            for (int &v : tree[u]) {
               if (v == parent[u]) continue;
               parent[v] = u;
               depth[v] = depth[u] + 1;
               int subtree = dfs1(v);
		       if (maxSubtree < subtree) {
                  maxSubtree = subtree;
                  int t = v; v = tree[u][0]; tree[u][0] = t;
               }
               size += subtree;
             }
             return size;
          };
		
          function<void(int)> dfs2 = [&](int u) {
             in[u] = time++;
             for (int v : t[u]) {
                if (v == parent[u]) continue;
                pathRoot[v] = v == t[u][0] ? pathRoot[u] : v;
                dfs2(v);
             }
          };
          dfs1(root);dfs2(root);
      }
      
      segtree::node get(int u, int v) {
         segtree::node res;
         process_path(u, v, [this, &res](int a, int b) {
            res = segtree::unite(res, segment_tree.get(a, b));});
         return res;
      }
	
      void modify(int u, int v, long long delta) {
         process_path(u, v, [this, delta](int a, int b) {
            segment_tree.modify(a, b, delta);});
      }
	
      void process_path(int u,int v, const function<void(int x,int y)>&op){
         for(;pathRoot[u]!=pathRoot[v];v=parent[pathRoot[v]]){
            if(depth[pathRoot[u]]>depth[pathRoot[v]]){
               int t = u; u = v; v = t;
            }
            op(in[pathRoot[v]], in[v]);
         }
         if (u != v || valuesOnVertices)
            op(min(in[u],in[v])+(valuesOnVertices ? 0:1),max(in[u],in[v]));
      }
};
\end{lstlisting}

\subsection{Java}
\begin{lstlisting}[language=Java]
import java.util.function.BiConsumer;	

private class HeavyLightDescomposition {
   private List<Integer>[] tree;
   // Si es verdadero los valores en los vertices
   // falso los valores en la aristas
   private boolean valuesOnVertices;
   private SegmentTree segmentTree;
   private int[] parent;
   private int[] depth;
   private int[] pathRoot;
   private int[] in;
   private int time;
	
   public HeavyLightDescomposition(List<Integer>[] tree,int root,boolean vOV){
      this.tree = tree; this.valuesOnVertices = vOV;
      int n = tree.length; segmentTree = new SegmentTree(n);
      parent = new int[n]; depth = new int[n];
      pathRoot = new int[n]; in = new int[n];
      parent[root] = -1; dfs1(root); dfs2(root);
   }
	
   private int dfs1(int u) {
      int size = 1; int maxSubtree = 0;
      for (int i = 0; i < tree[u].size(); i++) {
         int v = tree[u].get(i);
         if (v == parent[u]) continue;
         parent[v] = u; depth[v] = depth[u] + 1;
         int subtree = dfs1(v);
         if (maxSubtree < subtree) {
            maxSubtree = subtree;
            tree[u].set(i, tree[u].set(0, v));
         }
         size += subtree;
      }
      return size;
   }
	
   void dfs2(int u) {
      in[u] = time++;
      for (int v : tree[u]) {
         if (v == parent[u]) continue;
         pathRoot[v] = v == tree[u].get(0) ? pathRoot[u] : v;
         dfs2(v);
      }
   }
	
   public Node get(int u, int v) {
       Node[] res = { new Node() };
       processPath(u, v, (a, b) -> res[0] = segmentTree.unite(res[0], segmentTree.get(a, b)));
       return res[0];
   }
   
   public void modify(int u, int v, long delta) {
      processPath(u, v, (a, b) -> segmentTree.modify(a, b, delta));
   }
	
   public void processPath(int u, int v, BiConsumer<Integer, Integer> op) {
      for (; pathRoot[u] != pathRoot[v]; v = parent[pathRoot[v]]) {
         if (depth[pathRoot[u]] > depth[pathRoot[v]]) {
            int t = u; u = v; v = t;
         }
         op.accept(in[pathRoot[v]], in[v]);
      }
      if (u != v || valuesOnVertices)
        op.accept(Math.min(in[u], in[v]) + (valuesOnVertices ? 0 : 1), Math.max(in[u], in[v]));
   }
	
   private class Node { 
     // Valor inicial para las hojas
      public long mx = Long.MIN_VALUE;
      public long mn = Long.MAX_VALUE;
      public long sum = 0;
      public long add = 0;
		
      public Node(){}
     
      public void apply(int l, int r, long v) {
         mx = v; sum += v * (r - l + 1);
         add += v; mn = v;
      }
   }
	
   private class SegmentTree {
      private int nnodes;
      private Node[] tree;
		
      public Node unite(Node a, Node b) {
         Node res = new Node(); res.mx = Math.max(a.mx, b.mx);
         res.mn = Math.min(a.mn, b.mn); res.sum = a.sum + b.sum;
         return res;
      }
      
      public SegmentTree(int n) {
         this.nnodes = n; tree = new Node[2 * n - 1];
         for (int i = 0; i < tree.length; i++) tree[i] = new Node();
         build(0, 1, n - 1);
      }
		
      private void pull(int x, int y) {
         tree[x] = unite(tree[x + 1], tree[y]);
      }
		
      void push(int x, int l, int r) {
         int m = (l + r) >> 1; int y = x + ((m - l + 1) << 1);
         if (tree[x].add != 0) {
            tree[x + 1].apply(l, m, tree[x].add);
            tree[y].apply(m + 1, r, tree[x].add); tree[x].add = 0;
         }
      }
		
      private void build(int x, int l, int r) {
         if (l == r) { return; }
         int m = (l + r) >> 1; int y = x + ((m - l + 1) << 1);
         build(x + 1, l, m); build(y, m + 1, r); pull(x, y);
      }
		
      private Node get(int ll, int rr) {
         return get(ll, rr, 0, 0, this.nnodes - 1);
      }
		
      public Node get(int ll, int rr, int x, int l, int r) {
         if (ll <= l && r <= rr) { return tree[x]; }
         int m = (l + r) >> 1; int y = x + ((m - l + 1) << 1);
         push(x, l, r); Node res;
         if (rr <= m) { res = get(ll, rr, x + 1, l, m);} 
         else {
            if(ll > m){ res = get(ll, rr, y, m + 1, r);} 
            else{
               res=unite(get(ll,rr,x+1,l,m),get(ll,rr,y,m+1,r));
            }
         }
         pull(x, y); return res;
      }
      
      public void modify(int ll, int rr, long v) {
         modify(ll, rr, v, 0, 0, this.nnodes - 1);
      }
		
      public void modify(int ll, int rr, long v, int x, int l, int r) {
         if(ll<=l && r<=rr){
            tree[x].apply(l, r, v); return;
         }
         int m = (l + r) >> 1; int y = x + ((m - l + 1) << 1);
         push(x, l, r);
         if(ll<=m){modify(ll, rr, v, x + 1, l, m);}
         if(rr>m){ modify(ll, rr, v, y, m + 1, r);}
         pull(x, y);
      }
   }
}
\end{lstlisting}