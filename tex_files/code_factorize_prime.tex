\subsection{C++}
\begin{lstlisting}[language=C++]
vector<long long> trialDivision(long long n) {
   vector<long long> factorization;
   for (long long d = 2; d * d <= n; d++) {
      while (n % d == 0) {
         factorization.push_back(d);
         n /= d;
      }
   }
   if (n > 1)
      factorization.push_back(n);
   return factorization;
}
\end{lstlisting}

\begin{lstlisting}[language=C++]
vector<long long> wheelFactorization(long long n) {
   vector<long long> factorization;
   for (int d : {2, 3, 5}) {
      while (n % d == 0) {
         factorization.push_back(d);
         n /= d;
      }
   }
   static array<int, 8> increments = {4, 2, 4, 2, 4, 6, 2, 6};
   int i = 0;
   for (long long d = 7; d * d <= n; d += increments[i++]) {
      while (n % d == 0) {
         factorization.push_back(d);
         n /= d;
      }
      if (i == 8)
         i = 0;
   }
   if (n > 1)
      factorization.push_back(n);
   return factorization;
}
\end{lstlisting}

En las dos implementaciones siguientes es evidente que se necesita precalcular todos los primos al menos hasta $N$ y tenerlos almacenados en el vector \emph{primos}

\begin{lstlisting}[language=C++]
//definir una tupla para almacenar a^b 
//(a base)=>.first  (b expo)=>.second
typedef pair<int,int>fac;
vector<fac> Factorizar(int n){
   int l = primos.size();
   //iterar sobre todos los primos
   //hasta que algun primo sea mayor que n
   vector<fac>FP;
   for(int i=0;i<l && primos[i]<=n ;i++){
      int pi = primos[i];
      int exp = 0;
      if(n%pi==0){
         while(n%pi==0){
            exp++;//contador para el exponente
            n/=pi;//divido entre la base
         }
         // pi^exp
         // {a,b}  es lo mismo que make_pair(a,b) o Pair(a,b)
        FP.push_back({pi,exp});
      }
    }
    // si n es distinto n es un primo
    if(n>1){
	   FP.push_back({n,1});
    }
    return FP;
}
\end{lstlisting}

\begin{lstlisting}[language=C++]
vector<fac> FactorizarFactorial(int n){
   int l = primos.size();
   vector<fac>FP;
   //iterar sobre todos los primos menores e iguales que n
   for(int i=0;i<l && primos[i]<=n;i++){
      //acumulador para la potencia del primo actual
      int exp = 0;
      //potencia del primo actual
      int pot = primos[i];
      //mientras la potencia sea menor que n
      while(pot<=n){
         //multiplos de la potencia menores que n
         exp+=n/pot;
         //proxima potencia del primo
         pot*=primos[i];
      }
      FP.push_back({primos[i],exp});
  }
  return FP;
}
\end{lstlisting}

\begin{lstlisting}
int fermat(int n) {
   int a = ceil(sqrt(n)); int b2 = a*a - n;
   int b = round(sqrt(b2));
   while (b * b != b2) {
      a = a + 1; b2 = a*a - n;
      b = round(sqrt(b2));
   }
   return a - b;
}
\end{lstlisting}

En la siguiente implementación comenzamos con $B=10$ y aumentar $B$ después de cada iteración.
\begin{lstlisting}[language=C++]
long long pollards_p_minus_1(long long n) {
   int B = 10; long long g = 1;
   while (B <= 1000000 && g < n) {
      long long a = 2 + rand() % (n - 3);
      g = gcd(a, n);
      if (g > 1) return g;
      // compute a^M
      for (int p : primes) {
         if (p >= B) continue;
         long long p_power = 1;
         while (p_power * p <= B)
            p_power *= p;
         a = power(a, p_power, n);
         g = gcd(a - 1, n);
         if (g > 1 && g < n) return g;
      }
      B *= 2;
   }
   return 1;
}
\end{lstlisting}

\begin{lstlisting}[language=C++]
long long mult(long long a, long long b, long long mod) {
   long long result = 0;
   while (b) {
      if (b & 1) result = (result + a) % mod;
      a = (a + a) % mod;
      b >>= 1;
   }
   return result;
}

long long mult(long long a, long long b, long long mod) {
   return (__int128)a * b % mod;
}

long long f(long long x, long long c, long long mod) {
   return (mult(x, x, mod) + c) % mod;
}

long long rho(long long n, long long x0=2, long long c=1) {
   long long x = x0; long long y = x0;
   long long g = 1;
   while (g == 1) {
      x = f(x, c, n); y = f(y, c, n); y = f(y, c, n);
      g = gcd(abs(x - y), n);
   }
   return g;
}
\end{lstlisting}

\begin{lstlisting}[language=C++]
long long brent(long long n, long long x0=2, long long c=1) {
   long long x = x0; long long g = 1;
   long long q = 1; long long xs, y;
   int m = 128; int l = 1;
   while (g == 1) {
      y = x;
      for (int i = 1; i < l; i++)
         x = f(x, c, n);
      int k = 0;
      while (k < l && g == 1) {
         xs = x;
         for (int i = 0; i < m && i < l - k; i++) {
            x = f(x, c, n); q = mult(q, abs(y - x), n);
         }
         g = gcd(q, n); k += m;
      }
      l *= 2;
   }
   if (g == n) {
      do {
         xs = f(xs, c, n); g = gcd(abs(xs - y), n);
      } while (g == 1);
   }
   return g;
}

\end{lstlisting}


\subsection{Java}
\begin{lstlisting}
private class Tuple{
   private int prime, int exponent;
	
   public Tuple(int _p,int _e) {
      this.prime =_p; this.exponent = _e;
   }
}

public Queue<Tuple> factorize(int n) {
   Queue<Tuple> factors = new ArrayDeque<Tuple>();
   Tuple p;
   for (int d = 2; d * d <= n; d++) {
      int power=0;
      while (n % d == 0) {
         n /= d; power++;
      }
      if(power!=0) {
         factors.add(new Tuple(d,power));
      }
   }
   if (n > 1) {
      factors.add(new Tuple(n,1));
   }
   return factors;
}
\end{lstlisting}