Dentro de las diferentes aplicaciones que puede tener este algoritmo en los problemas de concursos podemos citar:

\begin{itemize}
	\item \textbf{Buscar una subcadena en una cadena. El algoritmo de Knuth-Morris-Pratt:} La tarea es la aplicación clásica de la función de prefijo. dado un texto $t$ y una cadena $s$ , queremos encontrar y mostrar las posiciones de todas las apariciones de la cadena $s$ en el texto $t$. Por conveniencia lo denotamos con $n$ la longitud de la cadena $s$ y con $m$ la longitud del texto $t$. Generamos la cadena $s+ \diamond + t$ , dónde $\diamond$ es un separador que no aparece ni en $s$ ni en $t$ . Calculemos la función de prefijo para esta cadena. Ahora piense en el significado de los valores de la función de prefijo, excepto las primeras $n + 1$ caracteres (que pertenecen a la cadena $s$ y el separador). Por
	definición el valor $\pi[i]$ muestra la longitud más larga de una subcadena que termina en posición $i$ que coincide con el prefijo. Pero en nuestro caso esto no es más que el bloque más grande que coincide con $s$ y termina en la posición $i$. Esta longitud no puede ser mayor que $n$ debido al separador. Pero si la igualdad $\pi[i] = n$ se logra, entonces significa que la cadena $s$ aparece completamente en esta posición, es decir, termina en la posición $i$. Eso sí, no olvides que las posiciones están indexadas en la cadena. $s + \diamond + t$ . Así, si en alguna posición $i$ tenemos $\pi[i] = n$ , luego en la posición $i-(n+ 1)-n+1=i-2n$ en la cadena $t$ la cadena $s$ aparece. Como ya se mencionó en la descripción del cálculo de la función de prefijo, si sabemos que los valores del prefijo nunca exceden un cierto valor, entonces no necesitamos almacenar la cadena completa y la función completa, sino solo su comienzo. En nuestro caso esto significa que sólo
	necesitamos almacenar la cadena $s + \diamond$ y los valores de la función de prefijo para ello. Podemos leer un carácter a la vez de la cadena $t$ y calcular el valor actual de la función de prefijo.
	
	\item \textbf{Contando el número de apariciones de cada prefijo:} Aquí discutimos dos problemas a la vez. dada una cadena $s$ de longitud $n$ . En la primera variación del problema queremos contar el número de apariciones de cada prefijo $s[0 \dots i]$ en la misma cadena. En la segunda variación del problema otra cadena $t$ se da y queremos contar el número de apariciones de cada prefijo $s[0 \dots i]$ en t .
	
	Primero resolvemos el primer problema. Considere el valor de la función de prefijo. $\pi[i]$ en una posición $i$. Por definición significa que el prefijo de
	longitud $\pi[i]$ de la cadena $s$ ocurre y termina en la posición $i$, y ya no hay un prefijo que siga a esta definición. Al mismo tiempo, los prefijos más cortos
	pueden terminar en esta posición. No es difícil ver que tenemos la misma pregunta que ya respondimos cuando calculamos la función de prefijo en
	sí: dado un prefijo de longitud $j$ ese es un sufijo que termina en la posición $i$ , ¿cuál es el siguiente prefijo más pequeño $< j$ que también es un sufijo
	que termina en la posición $i$ . Así en la posición $i$ termina el prefijo de longitud $\pi[i]$ , el prefijo de longitud $\pi[\pi[i]-1]$ , el prefijo $\pi[\pi[\pi[i]-1]-1]$, y así sucesivamente, hasta que el índice sea cero. Por tanto, podemos calcular la respuesta de la siguiente manera:
\begin{lstlisting}[language=C++]	
vector<int> ans(n + 1);
for (int i = 0; i < n; i++) ans[pi[i]]++;
for (int i = n-1; i > 0; i--) ans[pi[i-1]] += ans[i];
for (int i = 0; i <= n; i++) ans[i]++;
\end{lstlisting}
	
	Aquí, para cada valor de la función de prefijo, primero contamos cuántas veces aparece en la matriz $\pi$ , y luego calcular las respuestas finales: si
	sabemos que el prefijo de longitud $i$ aparece exactamente $ans[i]$ veces, entonces este número debe sumarse al número de apariciones de su sufijo
	más largo que también es un prefijo. Al fnal debemos agregar 1 a cada resultado, ya que también necesitamos contar los prefijos originales.
	
	Consideremos ahora el segundo problema. Aplicamos el truco de Knuth-Morris-Pratt: creamos la cadena $s + \diamond + t$ y calcular su función de prefijo.
	La única diferencia con la primera tarea es que solo nos interesan los valores de prefijo que se relacionan con la cadena. $t$ , es decir. $\pi[i]$ para
	$i \ge n + 1$ . Con esos valores podemos realizar exactamente los mismos cálculos que en la primera tarea.
	
	
	\item \textbf{El número de subcadenas diferentes en una cadena:} Dada una cadena $s$ de longitud $n$ . Queremos calcular el número de subcadenas diferentes que aparecen en él. Resolveremos este problema de forma iterativa. Es decir, aprenderemos, conociendo el número actual de subcadenas diferentes, cómo volver a calcular este recuento añadiendo un carácter al final. 
	
	Digamos  que $k$ es el número actual de subcadenas diferentes en $s$, y agregamos el carácter $c$ hasta el final de $s$. Obviamente algunas subcadenas nuevas que terminan en $c$ aparecerá. Queremos contar estas nuevas subcadenas que no aparecieron antes.
	
	Tomamos la cadena $t = s + c$ y invertimos. Ahora la tarea se transforma en calcular cuántos prefijos hay que no aparecen en ningún otro lugar. Si calculamos el valor máximo de la función de prefijo  $\pi_{\text{max}}$ de la cadena invertida $t$ , luego el prefijo más largo que aparece en $s$ es $\pi_{\text{max}}$ largo. Claramente también aparecen en él todos los prefijos de menor longitud.
	
	Por lo tanto, el número de nuevas subcadenas que aparecen cuando agregamos un nuevo carácter $c$ is $|s| + 1 - \pi_{\text{max}}$. Entonces, para cada carácter agregado podemos calcular el número de subcadenas nuevas en O($n$) veces, lo que da una complejidad temporal de
	O($n^2$) en total.
	
	Vale la pena señalar que también podemos calcular el número de subcadenas diferentes agregando los caracteres al principio o eliminando caracteres del principio o del final.
	
	\item \textbf{Comprimir una cadena:} Dada una cadena $s$ de longitud $n$. Queremos encontrar el formato \emph{comprimido} más corto. representación de la cadena, es decir, queremos encontrar
	una cadena $t$ de longitud más pequeña tal que $s$ puede representarse como una concatenación de una o más copias de $t$.
	
	Está claro que sólo necesitamos encontrar la longitud de $t$. Conociendo la longitud, la respuesta al problema será el prefijo de $s$ con esta longitud.
	
	Calculemos la función de prefijo para $s$. Usando el último valor del mismo definimos el valor $k = n - \pi[n-1]$. Demostraremos que si $k$ divide $n$, entonces $k$ será la respuesta, de lo contrario no habrá compresión efectiva y la respuesta es $n$.
	
	Dejar $n$ ser divisible por $k$. Luego la cadena se puede dividir en bloques de la longitud $k$. Por definición de la función de prefijo, el prefijo de longitud $n-k$ será igual a su sufijo. Pero esto significa que el último bloque es igual al bloque anterior. Y el bloque anterior tiene que ser igual al bloque anterior. Etcétera. Como resultado, resulta que todos los bloques son iguales, por lo tanto podemos comprimir la cadena $s$ a la longitud $k$.
	
	Por supuesto, todavía tenemos que demostrar que esto es realmente lo óptimo. De hecho, si hubiera una compresión menor que $k$ , entonces la función de prefijo al final sería mayor que $n-k$. Por lo tanto $k$ es realmente la respuesta.
	
	Ahora supongamos que $n$ no es divisible por $k$ . Demostramos que esto implica que la longitud de la respuesta es $n$. Lo demostramos por contradicción. Suponiendo que existe una respuesta y que la compresión tiene longitud $p$ ( $p$ divide $n$ ). Entonces el último valor de la función de prefijo
	tiene que ser mayor que $n-p$ , es decir, el sufijo cubrirá parcialmente el primer bloque. Consideremos ahora el segundo bloque de la cuerda. Dado que el prefijo es igual al sufijo, y tanto el prefijo como el sufijo cubren este bloque y su desplazamiento entre sí $k$ no divide la longitud del bloque $p$ (de lo contrario $k$ divide $n$ ), entonces todos los caracteres del bloque tienen que ser idénticos. Pero entonces la cadena consta de un solo carácter repetido una y otra vez, por lo que podemos comprimirla en una cadena de tamaño $1$ , lo que da $k=1$ , y $k$ divide $n$. Contradicción.
	
	$$\overbrace{s_0 ~ s_1 ~ s_2 ~ s_3}^p ~ \overbrace{s_4 ~ s_5 ~ s_6 ~ s_7}^p$$
	$$s_0 ~ s_1 ~ s_2 ~ \underbrace{\overbrace{s_3 ~ s_4 ~ s_5 ~ s_6}^p ~ s_7}_{\pi[7] = 5}$$
	$$s_4 = s_3, ~ s_5 = s_4, ~ s_6 = s_5, ~ s_7 = s_6 ~ \Rightarrow ~ s_0 = s_1 = s_2 = s_3$$
	
	\item \textbf{Construyendo un autómata según la función de prefijo:} Volvamos a la concatenación de las dos cadenas mediante un separador, es decir para las cadenas $s$ y $t$ Calculamos la función de prefijo para la cadena. $s + \# + t$ . Obviamente, desde $\#$ es un separador, el valor de la función de prefijo nunca excederá $|s|$ . De ello se deduce que basta con almacenar únicamente la cadena $s + \#$ y los valores de la función de prefijo correspondiente, y podemos calcular la función de prefijo para todos los caracteres posteriores sobre la marcha:
	
	$$\underbrace{s_0 ~ s_1 ~ \dots ~ s_{n-1} ~ \#}_{\text{necesita almacenarlo}} ~ \underbrace{t_0 ~ t_1 ~ \dots ~ t_{m-1}}_{\text{no necesita almacenarlo}}$$
	
	De hecho, en tal situación, conocer al siguiente personaje $c \in t$ y el valor de la función de prefijo de la posición anterior es información suficiente para calcular el siguiente valor de la función de prefijo, sin utilizar ningún carácter anterior de la cadena $t$ y el valor de la función de prefijo en ellos.
	
	En otras palabras, podemos construir un autómata (una máquina de estados finitos): el estado que contiene es el valor actual del prefijo. función, y la transición de un estado a otro se realizará a través del siguiente carácter.
	
	Así, incluso sin tener la cadena $t$ , podemos construir tal tabla de transición $(\text{old}_\pi, c) \rightarrow \text{new}_\pi$  usando el mismo algoritmo que para calcular la tabla
	de transición:
	
\begin{lstlisting}[language=C++]	
void compute_automaton(string s, vector<vector<int>>& aut) {
   s += '#';
   int n = s.size();
   vector<int> pi = prefix_function(s);
   aut.assign(n, vector<int>(26));
   for (int i = 0; i < n; i++) {
      for (int c = 0; c < 26; c++) {
         int j = i;
         while (j > 0 && 'a' + c != s[j]) j = pi[j-1];
         if ('a' + c == s[j]) j++;
         aut[i][c] = j;
      }
   }
}
\end{lstlisting}

Sin embargo, en esta forma el algoritmo se ejecuta en O($n^2 26$) tiempo para las letras minúsculas del alfabeto. Tenga en cuenta que podemos aplicar programación dinámica y utilizar las partes de la tabla ya calculadas. Siempre que nos alejamos del valor $j$ al valor $\pi[j-1]$ , en realidad queremos
decir que la transición $(j, c)$ conduce al mismo estado que la transición como $(\pi[j-1],c)$ , y esta respuesta ya está calculada con precisión.

\begin{lstlisting}[language=C++]	
void compute_automaton(string s, vector<vector<int>>& aut) {
   s += '#';
   int n = s.size();
   vector<int> pi = prefix_function(s);
   aut.assign(n, vector<int>(26));
   for (int i = 0; i < n; i++) {
      for (int c = 0; c < 26; c++) {
         if (i > 0 && 'a' + c != s[i]) aut[i][c] = aut[pi[i-1]][c];
         else aut[i][c] = i + ('a' + c == s[i]);
      }
   }
}
\end{lstlisting}

Como resultado construimos el autómata en O($26n$) tiempo.

¿Cuándo es útil un autómata así? Para empezar recuerda que usamos la función de prefijo para la cadena $s + \# + t$ y sus valores principalmente para un único propósito: encontrar todas las apariciones de la cadena $s$ en la cuerda $t$.

Por lo tanto, el beneficio más obvio de este autómata es la aceleración del cálculo de la función de prefijo para la cadena $s + \# + t$ . Al construir el autómata para $s + \#$ , ya no necesitamos almacenar la cadena $s$ o los valores de la función de prefijo en él. Todas las transiciones ya están
calculadas en la tabla.

Pero hay una segunda aplicación, menos obvia. Podemos usar el autómata cuando la cadena $t$ es una cadena gigantesca construida usando algunas reglas. Pueden ser, por ejemplo, cadenas grises o una cadena formada por una combinación recursiva de varias cadenas cortas de la entrada.

Para completar, resolveremos el siguiente problema: dado un número $k \le 10^5$ y una cadena $s$ de longitud $ \le 10^5$ . Tenemos que calcular el número de ocurrencias de $s$ en el $k$-ésima cadena Gray. Recuerde que las cadenas de Gray se definen de la siguiente manera:
$$
\begin{tabular}{c l}
 
$g_1$	&= \text{a}  \\
	 
	$g_2$  &= \text{aba}\\
 
	$g_3$ &= \text{abacaba}\\
 
$g_4$  &= \text{abacabadabacaba}

\end{tabular}
$$
En tales casos, incluso construir la cadena $t$. Será imposible, debido a su longitud astronómica. El $k$-ésima cadena Gray es $2^k-1$ caracteres de largo. Sin embargo, podemos calcular el valor de la función de prefijo al final de la cadena de manera efectiva, conociendo solo el valor de la función de
prefijo al principio.

Además del autómata en sí, también calculamos valore $G[i][j]$ el valor del autómata después de procesar la cadena $g_i$ empezando por el estado $j$.
Y además calculamos valores. $K[i][j]$ el número de ocurrencias de $s$ en $g_i$, antes durante el procesamiento de $g_i$ empezando por el estado $j$. De hecho $K[i][j]$ es el número de veces que la función de prefijo tomó el valor $|s|$ mientras realiza las operaciones. La respuesta al problema será
entonces $K[k][0]$ .

¿Cómo podemos calcular estos valores? Primero los valores básicos son $G[0][j] = j$ y $K[0][j] = 0$ . Y todos los valores posteriores se pueden calcular a partir de los valores anteriores y utilizando el autómata. Para calcular el valor de algunos $i$ recordamos que la cuerda $g_i$ consiste en $g_{i-1}$, el $i$ carácter del alfabeto, y $g_{i-1}$. Así el autómata pasará al estado:

$$\text{mid} = \text{aut}[G[i-1][j]][i]$$
$$G[i][j] = G[i-1][\text{mid}]$$

los valores para $K[i][j]$. También se puede contar fácilmente.

$$K[i][j] = K[i-1][j] + (\text{mid} == |s|) + K[i-1][\text{mid}]$$

De esta manera podemos resolver el problema de las cadenas Gray y, de manera similar, también una gran cantidad de otros problemas similares. Por ejemplo, exactamente el mismo método también resuelve el siguiente problema: nos dan una cadena $s$ y algunos patrones $t_i$ , cada uno de los cuales se especifica de la siguiente manera: es una cadena de caracteres comunes y puede haber algunas inserciones recursivas de las cadenas anteriores de la forma $t_k^{\text{cnt}}$, lo que significa que en este lugar tenemos que insertar la cadena $t_k$ $\text{cnt}$ veces. Un ejemplo de tales patrones:

$$\begin{tabular}{c l} 
	$t_1$ &= abdeca\\ 
	$t_2$ &= $abc + t_1^{30} + \text{abd}$\\ 
	$t_3$ &= $t_2^{50} + t_1^{100}$\\ 
	$t_4$ &= $t_2^{10} + t_3^{100}$ 
\end{tabular}$$

Las sustituciones recursivas hacen estallar la cadena, de modo que sus longitudes pueden alcanzar el orden de $100^100$

Tenemos que encontrar el número de veces que la cadena s aparece en cada una de las cadenas.


El problema se puede resolver de la misma manera construyendo el autómata de la función de prefijo y luego calculamos las transiciones para cada patrón usando los resultados anteriores.
\end{itemize}