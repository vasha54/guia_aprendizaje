Usaremos una matriz bidimensional para almacenar las respuestas a las consultas precalculadas $\text{st}[i][j]$ almacenará la respuesta para el rango  $[j, j + 2^i-1]$  de longitud $2^i$. El tamaño de la matriz bidimensional será $(K + 1) \times MAXN$ , dónde $MAXN$ es la mayor longitud de matriz posible. K tiene que satisfacer $\text{K} \ge \lfloor \log_2 \text{MAXN} \rfloor$ porque  $2^{\lfloor \log_2 \text{MAXN} \rfloor}$ es la potencia más grande de dos rangos, que tenemos que soportar. Para arreglos con una longitud razonable ( $\le 10^7$  elementos), $K = 25$ es un buen valor. El MAXN es la segunda dimensión para permitir accesos consecutivos a la memoria (compatibles con la caché). 

Porque el rango  $[j, j + 2^i - 1]$ de longitud $2^i$ se divide muy bien en los rangos  $[j, j + 2^{i - 1} - 1]$ y $[j + 2^{i - 1}, j + 2^i - 1]$, ambos de longitud $2^{i - 1}$, podemos generar la tabla de manera eficiente usando programación dinámica. 