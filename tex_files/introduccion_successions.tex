Las sucesiones son una parte fundamental de las matemáticas y se utilizan en una amplia variedad de contextos, desde la física, la ingeniería, la economía y la informática. En esta última área esta presente en una gran variedad de ejercicios de programación competitiva. 

La cantidad de posibles sucesiones es infinita ya que depende en gran medida de las reglas que usen para conformarla. Lo anterior provoca que como concursante sea en vano intentar conocerlas todas, quizás algunas, las más utilizadas en ejercicios. Pero como elaborar un algoritmo que permita generar los términos de una sucesión que le es desconocida a usted como concursante. 

En  la presente guía abordaremos algunos aspectos sobre las sucesiones que le permitirán tener mejores herramientas a la hora de resolver ejercicios de programación competitiva referentes a este tema.  