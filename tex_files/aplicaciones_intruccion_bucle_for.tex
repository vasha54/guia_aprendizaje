La ventaja de la construcción \textbf{for} sobre la
construcción \textbf{while} equivalente está en que en la cabecera de la construcción \textbf{for} se tiene toda
la información sobre como se inicializan, controlan y actualizan las variables del bucle.

Esta estructura es muy utilizada cuando se necesita iterar sobre un rango o los valores de una estructura lineal iterativa como vectores, arreglos unidimensionales o bidimensionales pero para esta caso tendría que anidar una estructura \textbf{for} dentro de otra estructura \textbf{for}.