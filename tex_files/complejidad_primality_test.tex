La complejidad de la prueba basada en la división es O($\sqrt{N}$). La complejidad temporal de la prueba de primalidad de Fermat es (O$(k \times (\log n)^{2+\epsilon})$), donde $k$ es el número de veces que se ejecuta el test y determina la fiabilidad del test, y $n$ es el número natural mayor que $1$ que se está probando por primalidad. La complejidad temporal de la prueba de primalidad de Miller-Rabin es (O$(k \ln^{3} n)$), donde $n$ es el número que se está probando por primalidad y $k$ es el número de rondas realizadas. Este algoritmo es eficiente en términos de tiempo polinomial en relación con la cantidad de dígitos de $n$. La versión determinista de la prueba de Primalidad de Miller-Rabin tiene una complejidad temporal de O($(\log n)^2)$). 