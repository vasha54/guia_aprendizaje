\subsection{C++}
\begin{lstlisting}[language=C++]
// Datos de entradas
int n;
vector<int> a,b;

//lectura de datos
void readData(){
   a.resize(n);
   for(int i=0;i<n;i++) cin>>a[i];
}

// preprocesamiento
void init(){
   int len = (int) sqrt (n + .0) + 1; // longitud del bloque y numero de bloques
   b.resize(len);
   for (int i=0; i<n; ++i) b[i / len] += a[i];
}

int query(int l,int r){
   int sum = 0;
   int len = (int) sqrt (n + .0) + 1;
   for (int i=l; i<=r; ){
      if (i % len == 0 && i + len - 1 <= r) {
         // Si todo el bloque que comienza en i pertenece a [l, r]
         sum += b[i / len]; i += len;
      } else {
         sum += a[i]; ++i;
      }
   }
   return sum;
}

int queryv2(int l, int r){
   int len = (int) sqrt (n + .0) + 1;
   int sum = 0;
   int c_l = l / len,   c_r = r / len;
   if (c_l == c_r) for (int i=l; i<=r; ++i) sum += a[i];
   else {
      for (int i=l, ends=(c_l+1)*len-1; i<=ends; ++i) sum += a[i];
      for (int i=c_l+1; i<=c_r-1; ++i) sum += b[i];
      for (int i=c_r*len; i<=r; ++i) sum += a[i];
   }
   return sum;
}
\end{lstlisting}	

\subsection{Java}
\begin{lstlisting}[language=Java]
public class Main {
   // Datos de entradas
   int n;
   int[] a;
   int [] b;
	
   void readData(){
      a = new int [n];
      Scanner in =new Scanner(System.in);
      for(int i=0;i<n;i++) a[i]=in.nextInt();
   }
	
   // preprocesamiento
   public void init(){
      int len = (int) Math.sqrt (n + .0) + 1; // longitud del bloque y numero de bloques
      b = new int [len];
      for (int i=0; i<n; ++i) b[i / len] += a[i];
   }
	
   public int query(int l,int r){
      int sum = 0;
      int len = (int) Math.sqrt (n + .0) + 1;
      for (int i=l; i<=r; ){
         if (i % len == 0 && i + len - 1 <= r) {
            // Si todo el bloque que comienza en i pertenece a [l, r]
            sum += b[i / len]; i += len;
         } else {
            sum += a[i]; ++i;
         }
      }
      return sum;
   }
	
   public int queryv2(int l, int r){
      int len = (int) Math.sqrt (n + .0) + 1;
      int sum = 0;
      int c_l = l / len,   c_r = r / len;
      if (c_l == c_r) for (int i=l; i<=r; ++i) sum += a[i];
      else {
         for (int i=l, ends=(c_l+1)*len-1; i<=ends; ++i) sum += a[i];
         for (int i=c_l+1; i<=c_r-1; ++i) sum += b[i];
         for (int i=c_r*len; i<=r; ++i) sum += a[i];
      }
      return sum;
   }
}
\end{lstlisting}

La implementación \emph{query} tiene demasiadas operaciones de división (que son mucho más lentas que otras operaciones aritméticas). En su lugar, podemos calcular los índices de los bloques $c_l$ y $c_r$ que contienen los índices $l$ y $r$, y recorrer los bloques $c_l+1 \dots c_r-1$ con procesamiento separado de las partes en los bloques $c_l$ y $c_r$. Este enfoque corresponde a la última fórmula de la descripción y hace que el caso $c_l = c_r$ sea un caso especial en este caso tenemos la implementación \emph{queryv2}.