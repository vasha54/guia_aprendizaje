Esta estructura solo está presente en el lenguaje de programación C++.

\begin{lstlisting}[language=C++]
//Incluya los siguientes archivos de encabezado en su codigo para usar PBDS:
#include <ext/pb_ds/assoc_container.hpp>
#include <ext/pb_ds/tree_policy.hpp>
//Namespace 
using namespace __gnu_pbds;
//Plantillas
//definir plantilla cuando todos los elementos son distintos 
template <class T> using nombre_tu_estructura_v1 = tree<T, null_type,
less<T>, rb_tree_tag,tree_order_statistics_node_update>;
//definir plantilla cuando tambien se utilizan elementos duplicados
template <class T> using nombre_tu_estructura_v2 = tree<T, null_type,
less_equal<T>, rb_tree_tag,tree_order_statistics_node_update>;

nombre_tu_estructura_v1 <int> B;
nombre_tu_estructura_v1 <pair<int,int> > A;

nombre_tu_estructura_v2 <int> b;
nombre_tu_estructura_v2 <pair<int,int> > a;
\end{lstlisting} 

\begin{itemize}
	\item \textbf{T:} $T$ hace referencia al tipo de dato que va almacenar la estructura.
	
	\item \textbf{null\_type:} Es la política mapeada. Aquí es nulo usarlo como un conjunto. Si queremos obtener el mapa pero no el conjunto, como segundo argumento se debe usar el tipo mapeado.
	
	\item \textbf{less:} Es la base para la comparación de dos funciones.
	
	\item \textbf{rb\_tree\_tag:} tipo de árbol utilizado. Generalmente son árboles rojos y negros porque la inserción y eliminación tardan un tiempo $\log(N)$, mientras que otros tardan un tiempo lineal, como \emph{splay\_tree}. Hay tres clases base proporcionadas en STL para esto: \emph{rb\_tree\_tag} (árbol rojo-negro), \emph{splay\_tree\_tag} (árbol de visualización) y \emph{ov\_tree\_tag} (árbol de vectores ordenados). Lamentablemente, en las competiciones solo podemos usar árboles rojo-negro para esto porque el árbol splay y el árbol OV usan una operación de división cronometrada lineal que nos impide usarlos.
	
	\item \textbf{tree\_order\_statistics\_node\_\_update:} Está incluido en \emph{tree\_policy.hpp} y contiene varias operaciones para actualizar las variantes de nodos de un contenedor basado en árbol, de modo que podamos realizar un seguimiento de metadatos como el número de nodos en un subárbol. Por defecto está configurado en \emph{null\_node\_update}, es decir, información adicional no almacenada en los vértices. Además, C++ implementó una política de actualización \emph{tree\_order\_statistics\_node\_update}, que, de hecho, lleva a cabo las operaciones necesarias.
\end{itemize}

\begin{lstlisting}[language=C++]
#include <iostream>
#include <bits/stdc++.h>
//Biblioteca para trabajar con la estructura
#include <ext/pb_ds/assoc_container.hpp>
#include <ext/pb_ds/tree_policy.hpp>
#include <functional> // for less
using namespace std;
//namespace para trabajar con la estructura
using namespace __gnu_pbds;

typedef tree<int, null_type, less<int>, rb_tree_tag,
tree_order_statistics_node_update>
ordered_set;

int main(){
   ordered_set p;
   p.insert(5); p.insert(2); p.insert(6); p.insert(4);
   // valor en el tercer indice de la matriz ordenada.
   cout << "El valor en el tercer indice::"<< *p.find_by_order(3) << endl;
   // indice del numero 6
   cout << "El indice del numero 6::" << p.order_of_key(6)<< endl;
   // El numero 7 no esta en el conjunto, pero mostrara el
   // numero de indice si estaba alli en la matriz ordenada.
   cout << "El indice del numero siete ::"<< p.order_of_key(7) << endl;
   return 0;
}
\end{lstlisting}

Para insertar múltiples copias del mismo elemento en el conjunto ordenado, un enfoque simple es reemplazar la siguiente línea,

\begin{lstlisting}[language=C++]
typedef tree<int , null_type, less<int> , rb_tree_tag , tree_order_statistics_node_update> ordered_set;
\end{lstlisting}

Por

\begin{lstlisting}[language=C++]
typedef tree<int , null_type ,  less_equal<int> , rb_tree_tag , tree_order_statistics_node_update> ordered_multiset
\end{lstlisting}

Como se muestra en el siguiente ejemplo:

\begin{lstlisting}[language=C++]
#include <ext/pb_ds/assoc_container.hpp>
#include <ext/pb_ds/tree_policy.hpp>
#include <functional> // for less
#include <iostream>
using namespace __gnu_pbds;
using namespace std;

typedef tree<int, null_type, less_equal<int>, rb_tree_tag,
tree_order_statistics_node_update>
ordered_multiset;


int main(){
   ordered_multiset p;
   p.insert(5); p.insert(5);
   p.insert(5); p.insert(2);
   p.insert(2); p.insert(6);
   p.insert(4);
   for (int i = 0; i < (int)p.size(); i++) {
      cout << "El elemento presente en el indice " << i << " es ";
      // Elemento de impresion presente en el indice i
      cout << *p.find_by_order(i) << ' ';
      cout << '\n';
   }
   return 0;
}
\end{lstlisting}

Sin embargo, el enfoque anterior no se recomienda porque la estructura de datos basados en políticas de G++ no está diseñada para almacenar elementos en un orden estricto y débil, por lo que usar \emph{less\_equal} con ella puede provocar un comportamiento indefinido, como que \emph{std::find} proporcione resultados incorrectos y búsquedas más largas. veces. Otro enfoque es almacenar los elementos como pares donde el primer elemento es el valor del elemento y el segundo elemento del par es el índice en el que se encontró en la matriz. El código fuente a continuación demuestra cómo utilizar este enfoque junto con otros métodos disponibles en el conjunto ordenado.

\begin{lstlisting}[language=C++]
#include <ext/pb_ds/assoc_container.hpp>
#include <ext/pb_ds/tree_policy.hpp>
#include <functional> // for less
#include <iostream>
using namespace __gnu_pbds;
using namespace std;

typedef tree<pair<int, int>, null_type,
less<pair<int, int> >, rb_tree_tag,
tree_order_statistics_node_update>
ordered_multiset;

int main(){
   ordered_multiset p;
   
   // Durante la insercion, utilice {VAL, IDX}
   p.insert({ 5, 0 }); p.insert({ 5, 1 });
   p.insert({ 5, 2 }); p.insert({ 2, 3 });
   p.insert({ 2, 4 }); p.insert({ 6, 5 });
   p.insert({ 4, 6 });
   
   for (int i = 0; i < (int)p.size(); i++) {
      cout << "El elemento presente en el indice " << i << " es ";
      // Elemento de impresion presente en el indice i
      // Utilice .first para obtener el valor real
      cout << p.find_by_order(i)->first << ' '; 
      cout << '\n';
   }
   cout << "\nDeleting element of 2\n\n";
   // Siempre que busque, utilice {VAL, 0} para lower_bound
   //					 {VAL, IDX} para buscar
   //					 {VAL, INT_MAX} para upper_bound
   p.erase(p.lower_bound({ 2, 0 }));
   for (int i = 0; i < (int)p.size(); i++) {
      cout << "El elemento presente en el indice " << i << " es ";
      cout << p.find_by_order(i)->first << ' ';
      cout << '\n';
   }
   return 0;
}
\end{lstlisting}
