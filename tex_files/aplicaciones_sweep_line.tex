La aplicación de este enfoque condujo a un gran avance en la complejidad computacional de los algoritmos geométricos influyendo particularmente en la reducción de la complejidad temporal de los mismos.

El barrido topológico es una forma de barrido plano con una ordenación sencilla de los puntos de procesamiento, lo que evita la necesidad de ordenar completamente los puntos; permite que algunos algoritmos de línea de barrido se realicen de manera más eficiente.

La técnica de los calibradores giratorios para diseñar algoritmos geométricos también puede interpretarse como una forma de barrido del plano, en el dual proyectivo del plano de entrada: una forma de dualidad proyectiva transforma la pendiente de una línea en un plano en la coordenada x de un punto en el plano dual, por lo que la progresión a través de las líneas ordenadas por su pendiente como lo realiza un algoritmo de calibradores giratorios es dual a la progresión a través de los puntos ordenados por sus coordenadas x en un algoritmo de barrido plano.

El enfoque de barrido puede generalizarse a problemas de tres o más dimensiones.

Al igual que la programación dinámica, la línea de barrido es una herramienta extremadamente poderosa en el juego de herramientas de un competidor de algoritmos porque no es simplemente un algoritmo: es un patrón de algoritmo que se puede adaptar para resolver una amplia variedad de problemas, pero también problemas novedosos que pueden haber sido creados específicamente para un concurso. Las pequeñas restricciones a menudo significan que uno puede tomar atajos (como procesar cada evento desde cero en lugar de incrementarlo y en un orden arbitrario), pero el concepto de la línea de barrido sigue siendo útil para encontrar una solución.