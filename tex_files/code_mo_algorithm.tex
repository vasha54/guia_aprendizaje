\subsection{C++}
\begin{lstlisting}[language=C++]
#include <iostream>
#include <bits/stdc++.h>

using namespace std;

void remove(idx);  // Remueve el valor en idx de la estructura de datos
void add(idx);     // Adiciona el valor en idx a la estructura de datos
int get_answer();  // Retorna la respuesta actual de la estructura de datos

int block_size;
//crear una estructura de datos acorde al problema

struct Query {
   int l, r, idx;
   bool operator<(Query other) const
   {
      return make_pair(l / block_size, r) <
      make_pair(other.l / block_size, other.r);
   }
};

vector<int> mo_s_algorithm(vector<Query> queries) {
   vector<int> answers(queries.size());
   sort(queries.begin(), queries.end());
	
   //Incializar la estructura de datos
	
   int cur_l = 0;
   int cur_r = -1;
   // La estructura de datos siempre reflejar la respuesta del rango [cur_l, cur_r]
   for (Query q : queries) {
      while (cur_l > q.l) { cur_l--; add(cur_l); }
	  while (cur_r < q.r) { cur_r++; add(cur_r); }
      while (cur_l < q.l) { remove(cur_l); cur_l++; }
      while (cur_r > q.r) { remove(cur_r); cur_r--; }
      answers[q.idx] = get_answer();
  }
  return answers;
}

void remove(int idx){
   //dependera del problema
}

void add(int idx){
   //dependera del problema
}

int get_answer(){
   //dependera del problema
}
\end{lstlisting}
\subsection{Java}
\begin{lstlisting}[language=Java]
public class MosAlgorithm {
   public static class Query {
      int index;
      int a;
      int b;
		
      public Query(int a, int b) {
         this.a = a;
         this.b = b;
      }
   }
   
   static int add(int[] a, int[] cnt, int i) {
      //La implementacion depende del problema
   }
	
   static int remove(int[] a, int[] cnt, int i) {
     //La implementacion depende del problema
   }
	
   public static int[] processQueries(int[] a, Query[] queries) {
      for (int i = 0; i < queries.length; i++) queries[i].index = i;
      int sqrtn = (int) Math.sqrt(a.length);
      Arrays.sort(queries, Comparator.<Query>comparingInt(q -> q.a / sqrtn).thenComparingInt(q -> q.b));
      // inicializar estructura
      int[] res = new int[queries.length];
      int L = 1;
      int R = 0;
      int cur = 0;
      for (Query query : queries) {
         while (L < query.a) cur += remove(a, cnt, L++);
         while (L > query.a) cur += add(a, cnt, --L);
         while (R < query.b) cur += add(a, cnt, ++R);
         while (R > query.b) cur += remove(a, cnt, R--);
           res[query.index] = cur;
      }
     return res;
  }
}
\end{lstlisting}