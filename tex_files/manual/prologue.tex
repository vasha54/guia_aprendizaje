\pagestyle{fancy}
\lhead{}
\chead{}
\rhead{\MakeUppercase{\textit{ Prólogo}}}
\chapter*{Prólogo}
\addcontentsline{toc}{chapter}{\large Prólogo} 
El siguiente documento es una versión mejorada de muchos que comenzaron siendo un simple .txt donde tenía la mala costumbre de colocar el nombre del ejercicio, por la temática por la cual lo había resuelto y el jurado donde lo había solucionado. Creo por ahora este ha superado por mucho los anteriores, quizá dentro de algún tiempo lo vea horrendo. Tuve la necesidad de retomarlo por dos cuestiones. La primera, no se imaginan cuanto me pesaba tener que
implementar desde cero ciertos algoritmos sabiendo que los había hecho con anterior para otros ejercicios, así que comencé anotar por temática los ejercicio que iba resolviendo, no es lo mismo tener que hacer la rueda desde cero, a decir creo que para este ejercicio puede coger la rueda de otro y solo tengo pintarla o cambiar el tipo de goma. Lo segundo para ayudar muchachos sobre todo de la  que se iniciaban en este mundillo.

La estructura del documento esta acorde según mi punto de vista a las principales áreas que debe dominar un concursante de programación. Dentro de cada área abordaré aquellas temáticas o puntos que de cierta manera domino aunque sea lo básico y que al menos tenga un ejemplo de problema en algún jurado que lo solucione usando ese conocimiento. No obstante si alguien cree que faltó algo que debe ser incluido y se siente en condiciones de abordarlo sin problema se puede poner en contacto conmigo (luis.valido@umcc.cu, luis.valido1989@gmail.com) e incluyó su aporte a este documento.  

Cada capítulo termina con el análisis de problemas los cuales se enmarca dentro temas tratados en el capítulo. De no existir ninguna especificación se puede asumir que la solución fue hecha en C++ y el veredicto de ese análisis fue aceptado en ese momento. Bueno creo que por ahora no hay nada más que aclarar. Creo que la cantidad de puntos que pierdo por una ortografía no estandarizada es aceptable por la cantidad de puntos que pueden otorgar los conocimientos aquí expuestos (No obstante se aceptan correciones).Importante casi lo olvidaba este documento digamos que es un beta que se va ir llenando a medida de las posibilidades de tiempo de los autores así que puede encontrar temáticas con solamente el nombre, eso significa que ya se pensado abordarlo lo que aún no se ha tenido el tiempo.



