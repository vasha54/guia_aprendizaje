Aquí están los valores de $\phi(n)$ para los primeros números enteros positivos:

$$
\begin{array}{|c|c|c|c|c|c|c|c|c|c|c|c|c|c|c|c|c|c|c|c|c|c|}
	\hline
	n & 1 & 2 & 3 & 4 & 5 & 6 & 7 & 8 & 9 & 10 & 11 & 12 & 13 & 14 & 15 & 16 & 17 & 18 & 19 & 20 & 21 \\ \hline
	\phi(n) & 1 & 1 & 2 & 2 & 4 & 2 & 6 & 4 & 6 & 4 & 10 & 4 & 12 & 6 & 8 & 8 & 16 & 6 & 18 & 8 & 12 \\ \hline
\end{array}
$$

\textbf{Propiedades:}

Las siguientes propiedades de la función totiente de Euler son suficientes para calcularla para cualquier número:

\begin{enumerate}
	\item 
	Si $p$ es un número primo, entonces $\gcd(p, q) = 1$ para todo $1 \le q < p$ . Por lo tanto tenemos:
	
	$$\phi (p) = p - 1.$$ 
	
	\item Si $p$ es un número primo y $k \ge 1$ , entonces hay exactamente $p^k / p$ números entre $1$ y $p^k$ que son divisibles por $p$ . Lo que nos da:
	
	$$\phi(p^k) = p^k - p^{k-1}.$$ 
	
	\item Si $a$ y $b$ son primos relativos, entonces:
	
	$$\phi(ab) = \phi(a) \cdot \phi(b).$$ 
	
	Esta relación no es trivial de ver. Se deduce del teorema del resto chino . El teorema del resto chino garantiza que para cada $0 \le x < a$ y cada $0 \le y < b$ , existe un $0 \le z < ab$ único con $z \equiv x \pmod{a}$ y $z \equiv y \pmod{b}$ . No es difícil demostrar que $z$ es coprimo con $ab$ si y solo si $x$ es coprimo con $a$ y $y$ es coprimo con $b$ . Por lo tanto, la cantidad de números enteros coprimos de $ab$ es igual al producto de las cantidades de $a$ y $b$ .
	
	\item En general, para $a$ y $b$ no coprimos , la ecuación:
	
	$$\phi(ab) = \phi(a) \cdot \phi(b) \cdot \dfrac{d}{\phi(d)}$$ 
	
	con $d = \gcd(a, b)$ se mantiene.
	
\end{enumerate}

Por lo tanto, usando las primeras tres propiedades, podemos calcular $\phi(n)$ mediante la factorización de $n$ (descomposición de $n$ en un producto de sus factores primos). Si $n = {p_1}^{a_1} \cdot {p_2}^{a_2} \cdots {p_k}^{a_k}$ , donde $p_i$ son factores primos de $n$.


\begin{align}
	\phi (n) &= \phi ({p_1}^{a_1}) \cdot \phi ({p_2}^{a_2}) \cdots  \phi ({p_k}^{a_k}) \\\\
	&= \left({p_1}^{a_1} - {p_1}^{a_1 - 1}\right) \cdot \left({p_2}^{a_2} - {p_2}^{a_2 - 1}\right) \cdots \left({p_k}^{a_k} - {p_k}^{a_k - 1}\right) \\\\
	&= p_1^{a_1} \cdot \left(1 - \frac{1}{p_1}\right) \cdot p_2^{a_2} \cdot \left(1 - \frac{1}{p_2}\right) \cdots p_k^{a_k} \cdot \left(1 - \frac{1}{p_k}\right) \\\\
	&= n \cdot \left(1 - \frac{1}{p_1}\right) \cdot \left(1 - \frac{1}{p_2}\right) \cdots \left(1 - \frac{1}{p_k}\right)
\end{align}


\subsection{Función del totiente de Euler de $1$ a $n$}

Si necesitamos todos los tocientes de todos los números entre $1$ y $n$ , entonces factorizar todos los $n$ números no es eficiente. Podemos utilizar la misma idea que la Criba de Eratóstenes . Todavía se basa en la propiedad que se muestra arriba, pero en lugar de actualizar el resultado temporal de cada factor primo para cada número, encontramos todos los números primos y para cada uno actualizamos los resultados temporales de todos los números que son divisibles por ese número primo.


\subsection{Propiedad de la suma divisoria}

Esta interesante propiedad fue establecida por Gauss:
$$ \sum_{d|n} \phi{(d)} = n$$

Aquí la suma es sobre todos los divisores positivos $d$ de $n$ .

Por ejemplo, los divisores de 10 son 1, 2, 5 y 10. Por lo tanto, $\phi{(1)} + \phi{(2)} + \phi{(5)} + \phi{(10)} = 1. + 1 + 4 + 4 = 10$ .


\subsubsection{Encontrar el totiente del 1 al $n$ usando la propiedad de la suma del divisor}

La propiedad de la suma del divisor también nos permite calcular el tociente de todos los números entre 1 y $n$ . Esta implementación es un poco más simple que la implementación anterior basada en la criba de Eratóstenes, sin embargo también tiene una complejidad ligeramente peor: O$(n \log n)$



\subsection{Aplicación en el teorema de Euler}

La propiedad más famosa e importante de la función totiente de Euler se expresa en el teorema de Euler :
$$a^{\phi(m)} \equiv 1 \pmod m \quad \text{si } a \text{ y } m \text{ son primos relativos.}$$

En el caso particular en el que $m$ es primo, el teorema de Euler se convierte en el pequeño teorema de Fermat :
$$a^{m - 1} \equiv 1 \pmod m$$

El teorema de Euler y la función totiente de Euler ocurren con bastante frecuencia en aplicaciones prácticas; por ejemplo, ambos se utilizan para calcular el inverso multiplicativo modular .

Como consecuencia inmediata también obtenemos la equivalencia:
$$a^n \equiv a^{n \bmod \phi(m)} \pmod m$$

Esto permite calcular $x^n \bmod m$ para $n$ muy grandes , especialmente si $n$ es el resultado de otro cálculo, ya que permite calcular $n$ bajo un módulo.

\subsubsection{Teoría de grupos}

$\phi(n)$ es el orden del grupo multiplicativo mod n $(\mathbf{Z} /n\mathbf{Z} )^\times$ , es decir el grupo de unidades (elementos con inversos multiplicativos). Los elementos con inversos multiplicativos son precisamente aquellos coprimos de $n$ .

El orden multiplicativo de un elemento $a$ mod $n$ , denotado $\operatorname{ord}_n(a)$ , es el $k>0$ más pequeño tal que $a^k \equiv 1 \pmod m$ . $\operatorname{ord}_n(a)$ es el tamaño del subgrupo generado por $a$ , por lo que según el teorema de Lagrange, el orden multiplicativo de cualquier $a$ debe dividir $\phi(n)$ . Si el orden multiplicativo de $a$ es $\phi(n)$ , el mayor posible, entonces $a$ es una raíz primitiva y el grupo es cíclico por definición.

\subsection{Generalización}

Existe una versión menos conocida de la última equivalencia, que permite calcular $x^n \bmod m$ de manera eficiente para $x$ y $m$ no coprimos . Para $x, m$ y $n \geq \log_2 m$ arbitrarios :
$$x^{n}\equiv x^{\phi(m)+[n \bmod \phi(m)]} \mod m$$

Prueba:

Sean $p_1, \dots, p_t$ divisores primos comunes de $x$ y $m$ , y $k_i$ sus exponentes en $m$ . Con ellos definimos $a = p_1^{k_1} \dots p_t^{k_t}$ , lo que hace que $\frac{m}{a}$ sea coprimo con $x$ . Y sea $k$ el número más pequeño tal que $a$ divida a $x^k$ . Suponiendo $n \ge k$ , podemos escribir:


\begin{align}x^n \bmod m &= \frac{x^k}{a}ax^{n-k}\bmod m \\
	&= \frac{x^k}{a}\left(ax^{n-k}\bmod m\right) \bmod m \\
	&= \frac{x^k}{a}\left(ax^{n-k}\bmod a \frac{m}{a}\right) \bmod m \\
	&=\frac{x^k}{a} a \left(x^{n-k} \bmod \frac{m}{a}\right)\bmod m \\
	&= x^k\left(x^{n-k} \bmod \frac{m}{a}\right)\bmod m
\end{align}

La equivalencia entre la tercera y cuarta línea se deriva del hecho de que $ab \bmod ac = a(b \bmod c)$ . De hecho, si $b = cd + r$ con $r < c$ , entonces $ab = acd + ar$ con $ar < ac$ .

Dado que $x$ y $\frac{m}{a}$ son coprimos, podemos aplicar el teorema de Euler y obtener la fórmula eficiente (ya que $k$ es muy pequeña; de hecho, $k \le \log_2 m$ ):
$$x^n \bmod m = x^k\left(x^{nk \bmod \phi(\frac{m}{a})} \bmod \frac{m}{a}\right)\bmod m .$$

Esta fórmula es difícil de aplicar, pero podemos usarla para analizar el comportamiento de $x^n \bmod m$ . Podemos ver que la secuencia de potencias $(x^1 \bmod m, x^2 \bmod m, x^3 \bmod m, \dots)$ entra en un ciclo de longitud $\phi\left(\frac{m }{a}\right)$ después de los primeros $k$ (o menos) elementos. $\phi\left(\frac{m}{a}\right)$ divide $\phi(m)$ (porque $a$ y $\frac{m}{a}$ son coprimos, tenemos $\phi( a) \cdot \phi\left(\frac{m}{a}\right) = \phi(m)$ ), por lo tanto también podemos decir que el período tiene longitud $\phi(m)$ . Y dado que $\phi(m) \ge \log_2 m \ge k$ , podemos concluir la fórmula deseada, mucho más simple:
$$ x^n \equiv x^{\phi(m)} x^{(n - \phi(m)) \bmod \phi(m)} \bmod m \equiv x^{\phi(m)+ [n \bmod \phi(m)]} \mod m.$$ 

