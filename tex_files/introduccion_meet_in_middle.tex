Digamos que tenemos $n$ números y queremos calcular cuántos subconjuntos tienen la suma de los números $x$.
Esto se puede hacer en tiempo O($2^n$) usando la tecnica de máscara bit la cual nos pemite repasar todos los subconjuntos, pero usar la técnica de \textbf{encuentro en el medio (\emph{meet in the middle})} solo toma tiempo O($2^{\frac{n}{2}}$). Esta es una mejora significativa porque el número del exponente se reduce a la mitad.