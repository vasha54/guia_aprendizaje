\subsection{Representación de grafos}
Existen diferentes formas de representar un grafo. La estructura de datos usada depende de las características del grafo y el algoritmo usado para manipularlo. Entre las estructuras más sencillas y usadas se encuentran las listas y las matrices, aunque frecuentemente se usa una combinación de ambas. Las listas son preferidas en grafos dispersos porque tienen un eficiente uso de la memoria. Por otro lado, las matrices proveen acceso rápido, pero pueden consumir grandes cantidades de memoria.

La estructura de datos usada depende de las características del grafo y el algoritmo usado para manipularlo. Entre las estructuras más sencillas y usadas se encuentran las listas y las matrices, aunque frecuentemente se usa una combinación de ambas. Las listas son preferidas en grafos dispersos porque tienen un eficiente uso de la memoria. Por otro lado, las matrices proveen acceso rápido, pero pueden consumir grandes cantidades de memoria.

\subsection{Grafo denso}
En teoría de grafos, la densidad de un grafo es una propiedad que determina la proporción de aristas que posee. Un grafo denso es un grafo en el que el número de aristas es cercano al número máximo de aristas posibles, es decir, a las que tendría si el grafo fuera completo. Al contrario, un grafo disperso es un grafo con un número de aristas muy bajo, es decir, cercano al que tendría si fuera un grafo vacío. 


\subsection{Cola con prioridad}
Una cola de prioridad es una estructura de datos en la que los elementos se atienden en el orden
indicado por una prioridad asociada a cada uno. Si varios elementos tienen la misma prioridad, se
atenderán de modo convencional según la posición que ocupen, puede llegar a ofrecer la extracción
constante de tiempo del elemento más grande (por defecto), a expensas de la inserción logarítmica, o
utilizando greater<int> causaría que el menor elemento aparezca como la parte superior con .top().
Trabajar con una cola de prioridad es similar a la gestión de un heap

\subsection{Árbol de expasión mínima}

Dado un grafo conexo y no dirigido, un árbol recubridor, árbol de cobertura o árbol de expansión de ese grafo es un subgrafo que tiene que ser un árbol y contener todos los vértices del grafo inicial. Cada arista tiene asignado un peso proporcional entre ellos, que es un número representativo de algún objeto, distancia, etc.; y se usa para asignar un peso total al árbol recubridor mínimo computando la suma de todos los pesos de las aristas del árbol en cuestión. Un árbol recubridor mínimo o un árbol de expansión mínimo es un árbol recubridor que pesa menos o igual que todos los otros árboles recubridores. Todo grafo tiene un bosque recubridor mínimo. 