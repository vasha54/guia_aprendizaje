\subsection{Eugène Charles Catalan}
Eugène Charles Catalan (30 de mayo de 1814-14 de febrero de 1894) fue un matemático francés y belga que trabajó en la teoría de números.

Trabajó en fracciones continuas, geometría descriptiva, teoría de números y combinatoria. Dio su nombre a una superficie única (superficie periódica mínima en el espacio  $R^3$) que descubrió en 1855. Anteriormente había enunciado la famosa conjetura de Catalan, que fue publicada en 1844 y probada finalmente en 2002 por el matemático rumano Preda Mihăilescu. Introdujo los números de Catalan para resolver un problema combinatorio. 

\subsection{Coeficientes binomiales}
En matemáticas, los coeficientes binomiales, números combinatorios o combinaciones son números estudiados en combinatoria que corresponden al número de formas en que se puede extraer subconjuntos a partir de un conjunto dado. Sin embargo, dependiendo del enfoque que tenga la exposición, se pueden usar otras definiciones equivalentes. 

\subsection{Programación Dinámica}
La programación dinámica es un método para reducir el tiempo de ejecución de un algoritmo
mediante la utilización de subproblemas superpuestos y subestructuras óptimas.

La teoría de programación dinámica se basa en una estructura de optimización, la cual consiste
en descomponer el problema en subproblemas (más manejables). Los cálculos se realizan entonces recursivamente donde la solución óptima de un subproblema se utiliza como dato de entrada
al siguiente problema. Por lo cual, se entiende que el problema es solucionado en su totalidad,
una vez se haya solucionado el último subproblema. Dentro de esta teoría, Bellman desarrolla el
Principio de Optimalidad, el cual es fundamental para la resolución adecuada de los cálculos recursivos. Lo cual quiere decir que las etapas futuras desarrollan una política óptima independiente
de las decisiones de las etapas predecesoras. Es por ello, que se define a la programación dinámica como una técnica matemática que ayuda a resolver decisiones secuenciales interrelacionadas,
combinándolas para obtener de la solución óptima.