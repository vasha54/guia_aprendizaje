\subsection{Programación dinámica}
En informática, la programación dinámica es un método para reducir el tiempo de ejecución de un algoritmo mediante la utilización de subproblemas superpuestos y subestructuras óptimas. 
Una subestructura óptima significa que se pueden usar soluciones óptimas de subproblemas para encontrar la solución óptima del problema en su conjunto.  Se pueden resolver problemas con subestructuras óptimas siguiendo estos tres pasos: 
\begin{enumerate}
	\item Dividir el problema en subproblemas más pequeños. 
	\item Resolver estos problemas de manera óptima usando este proceso de tres pasos recursivamente.
	\item Usar estas soluciones óptimas para construir una solución óptima al problema original. 
\end{enumerate}