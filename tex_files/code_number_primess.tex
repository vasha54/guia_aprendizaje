\subsection{C++}

La implementacion de saber si es primo chequeando todos los numeros en el rango de 2 hasta $\sqrt{N}$:

\begin{lstlisting}[language=C++]
bool isPrime(int x) {
   for (int d = 2; d * d <= x; d++) {
      if (x % d == 0)
         return false;
   }
   return true;
}
\end{lstlisting} 


La implementación del test primalidad Miller-Rabin en C++:

\begin{lstlisting}[language=C++]
	int powMod (int x, int k, int m) {
		if (k == 0) return 1;
		if (k% 2 == 0) return powMod (x*x%m, k/2, m);
		else return x * powMod (x, k-1, m)% m;
	}
	
	bool suspect (int a, int s, int d, int n) {
		int x = powMod (a, d, n);
		if (x == 1) return true;
		for (int r = 0; r<s; ++r){
			if (x == n - 1) return true;
			x = x*x%n;
		}
		return false;
	}
	
	// {2, 7, 61, - 1} para n <4759123141 (= 2 ^ 32)
	// {2, 3, 5, 7, 11, 13, 17, 19, 23, - 1} para n <10 ^ 16 (hasta el momento)
	bool isPrime (int n) {
		if (n<=1 || (n>2 && n%2==0)) return false;
		int test [] = {2, 3, 5, 7, 11, 13, 17, 19, 23, - 1};
		int d = n - 1, s = 0;
		while (d%2==0) ++s, d/= 2;
		
		for (int i = 0; test[i] <n && test [i]!=-1; ++i)
		if (!suspect(test[i],s,d,n)) return false;
		return true;
	}
	
	// ----- Otra implementacion
	
	using u64 = uint64_t;
	using u128 = __uint128_t;
	
	u64 binpower(u64 base, u64 e, u64 mod) {
		u64 result = 1;
		base %= mod;
		while (e) {
			if (e & 1) result = (u128)result * base % mod;
			base = (u128)base * base % mod;
			e >>= 1;
		}
		return result;
	}
	
	bool check_composite(u64 n, u64 a, u64 d, int s) {
		u64 x = binpower(a, d, n);
		if (x == 1 || x == n - 1) return false;
		for (int r = 1; r < s; r++) {
			x = (u128)x * x % n;
			if (x == n - 1) return false;
		}
		return true;
	};
	
	bool MillerRabin(u64 n) { 
		if (n < 4) return n == 2 || n == 3;
		int s = 0;
		u64 d = n - 1;
		while ((d & 1) == 0) {
			d >>= 1;
			s++;
		}
		
		for (int i = 0; i < iter; i++) {
			int a = 2 + rand() % (n - 3);
			if (check_composite(n, a, d, s)) return false;
		}
		return true;
	}
	
	// -----Otra implementacion
	
	bool MillerRabin(u64 n) { 
		if (n < 2) return false;
		int r = 0;
		u64 d = n - 1;
		while ((d & 1) == 0) {
			d >>= 1;
			r++;
		}
		
		for (int a : {2, 3, 5, 7, 11, 13, 17, 19, 23, 29, 31, 37}) {
			if (n == a) return true;
			if (check_composite(n, a, d, r)) return false;
		}
		return true;
	}
\end{lstlisting} 

Implementación de Test de Fermat en C++:

\begin{lstlisting}[language=C++]
	int binpow(int a, int b) {
		int res = 1;
		while (b > 0) {
			if (b & 1)res = res * a;
			a = a * a;
			b >>= 1;
		}
		return res;
	}
	
	bool probablyPrimeFermat(int n, int iter=5) {
		if (n <= 4) return n == 2 || n == 3;
		
		for (int i = 0; i &lt; iter; i++) {
			int a = 2 + rand() % (n - 3);
			if (binpow(a, n - 1, n) != 1) return false;
		}
		return true;
	} 
\end{lstlisting}

Veamos algunas de las posibles implementaciones de la criba de Eratóstenes

\begin{lstlisting}[language=C++]
	#define MAX_N 1000001
	#include <vector>
	
	vector<int> primes;
	bool mark[MAX_N];
	
	inline void sieve(int B){
		if (B > 1) primes.push_back(2);
		for (int i=3;i<=B;i+=2){
			if(!mark[i]){
				mark[i]=true;
				primes.push_back(i);
				if (i<=sqrt(B)+1)  
				for (int j=i*i;j<=B;j+=i)
				mark[j]=true;
			}
		}
	} 
	
	/* Otra implementacion pero devuelve un vector donde si 
	primes[i]!=0 entonces i es un numero primo es la siguiente*/
	
	vector <int> sieve_of_eratosthenes (int n) {
		vector <int> primes (n);
		for (int i = 2; i <n; ++i)
		primes [i] = i;
		for (int i = 2; i*i<n; ++ i)
		if (primes [i])
		for (int j = i*i; j <n; j+=i)
		primes [j] = 0;
		return primes;
	}
\end{lstlisting}

Ahora algunas implementaciones de la criba de Atkin:
\begin{lstlisting}[language=C++]
	void sieve_of_atkin (){
		int n;
		for (int z = 1; z <= 5; z+=4) {
			for (int y = z; y<= sqrt(N); y+=6) {
				for (int x = 1; x <= sqrt(N) && (n=4*x*x+y*y)<= N; ++x)
				isprime [n]=!isprime[n];
				for (int x = y + 1; x <= sqrt(N) && (n=3*x*x-y*y)<=N; x+=2)
				isprime[n]=!isprime[n];
			}
		}
		
		for (int z = 2; z<= 4; z+=2) {
			for (int y = z; y<=sqrt(N); y +=6) {
				for (int x = 1; x<=sqrt(N) && (n=3*x*x+y*y)<=N;x+=2)
				isprime[n]=!isprime[n];
				for (int x = y+1;x<=sqrt(N) && (n=3*x*x-y*y)<=N;x+=2)
				isprime[n]=!isprime[n];
			}
		}
		
		for (int y = 3; y <= sqrt(N); y+=6){
			for (int z = 1; z<=2; ++z){
				for (int x = z; x<=sqrt(N) && (n=4*x*x+y*y)<=N; x+= 3)
				isprime[n]=!isprime[n];
			}
		}
		
		for (int n=5; n<=sqrt(N); ++n)
		if (isprime[n])
		for (int k=n*n; k<=N; k+=n*n)
		isprime[k]=false;
		isprime[2]=isprime[3]=true;
	}
	
	/* Otra implementacion */
	
	void sieve_of_atkin (){
		int n;
		for (int x = 1; x<=sqrt(N); ++x) {
			for (int y = 1; y<=sqrt(N); ++y) {
				n= 4*x*x+y*y;
				if(n <=N && (n%12==1 || n%12==5))
				isprime[n]=!isprime[n];
				n=3*x*x+y*y;
				if(n<=N && n%12==7)
				isprime[n]=!isprime[n];
				n=3*x*x-y*y;
				if(x>y && n<=N && n%12==11)
				isprime[n]=!isprime[n];
			}
		}
		for (int n = 5; n<=sqrt(N); ++n)
		if (isprime[n])
		for (int k = n*n; k<=N; k+=n*n)
		isprime[k]=false;
		isprime[2]=isprime[3]=true;
	}
\end{lstlisting}

Implementación de la criba lineal
\begin{lstlisting}[language=C++]
	const int N = 10000000;
	vector<int> lp(N+1);
	vector<int> pr;
	
	for(int i=2; i <= N; ++i) {
		if(lp[i] == 0) {
			lp[i] = i;
			pr.push_back(i);
		}
		for(int j=0; j < (int)pr.size() && pr[j] <= lp[i] && i*pr[j] <= N; ++j) {
			lp[i * pr[j]] = pr[j];
		}
	}
\end{lstlisting}

\subsection{Java}
La implementacion de saber si es primo chequeando todos los numeros en el rango de 2 hasta $\sqrt{N}$:

\begin{lstlisting}[language=C++]
public static boolean isPrime(long n) {
   if (n <= 1) return false;
   for (long i = 2; i * i <= n; i++)
      if (n % i == 0) return false;
   return true;
}
\end{lstlisting} 



Otra implementaciones realizadas en Java que son optimizaciones de las cribas
\begin{lstlisting}[language=C++]
// Generates prime numbers up to n in O(n*log(log(n))) time
public static int[] generatePrimes(int n) {
   boolean[] prime = new boolean[n + 1];
   Arrays.fill(prime, 2, n + 1, true);
	
   for (int i = 2; i * i <= n; i++)
      if (prime[i])
         for (int j = i * i; j <= n; j += i) prime[j] = false;
   int[] primes = new int[n + 1];
   int cnt = 0;
   for (int i = 0; i < prime.length; i++)
      if (prime[i]) primes[cnt++] = i;
   return Arrays.copyOf(primes, cnt);
}

// Generates prime numbers up to n in O(n) time
public static int[] generatePrimesLinearTime(int n) {
   int[] lp = new int[n + 1];
   int[] primes = new int[n + 1];
   int cnt = 0;
   for (int i = 2; i <= n; ++i) {
      if (lp[i] == 0) {
         lp[i] = i;
         primes[cnt++] = i;
      }
      for (int j = 0; j < cnt && primes[j] <= lp[i] && i * primes[j] <= n; ++j) lp[i * primes[j]] = primes[j];
   }
   return Arrays.copyOf(primes, cnt);
}

public static boolean isPrime(long n) {
   if (n <= 1)return false;
   for (long i = 2; i * i <= n; i++)
      if (n % i == 0) return false;
   return true;
}
\end{lstlisting}


