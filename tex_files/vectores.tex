Es un modelo basado en un modelo de arreglo dinámico que permite el trabajo y gestión con los elementos almacenados en un arreglo dinámico. Pero a diferencia de los arreglos regulares, el almacenamiento en vectores se maneja automáticamente, lo que permite que se amplíe y se contrate según sea necesario. Para utilizar un vector en C++ se debe incluir en la cabecera del archivo el siguiente fragmento:

\begin{lstlisting}[language=Java]
#include <vector>
\end{lstlisting} 

Para declarar un vector solo basta con poner algo como lo que sigue:

\begin{lstlisting}[language=Java]
vector<T> nombre_del_vector;
\end{lstlisting}

Donde {\em T} debe ser uno de los tipos de datos definidos por el lenguaje de progrmación o por el propio programador. El vector es muy útil cuando se va acceder a los elementos conocidos su posición dentro de la estructura y se va añadir o eliminar elementos de última posición. No presenta igual desempeño cuando las inserciones y eliminaciones se producen en otras posiciones. Una forma muy eficiente de utilizar el vector es una vez creado inicializarlo con la cantidad máxima de elementos que puede alamcenar siempre y cuando se conozca este valor.

\begin{lstlisting}[language=C++]
#include <iostream>
#include <vector>
using namespace std;
	
int main (){
   unsigned int i;
   // variantes para constuir un vector:
   // un vector de enteros vacios
   vector<int> first;
		
   /*un vector con cuatro elementos, cada elementos con el valor 100*/
   vector<int> second (4,100);
		
   /*creando un vector iterando sobre otro*/                                           
   vector<int> third (second.begin(),second.end()); 
		
   /*creando un vector copia del tercer vector*/
   vector<int> fourth (third);
		
   /*creando un vector a partir de un vector*/
   int myints[] = {16,2,77,29};
   vector<int> fifth (myints, myints + sizeof(myints) / sizeof(int) );
		
   /*adicionando un elemento al vector*/
   first.push_back(2);
		
   /*limpiar un vector*/
   second.clear();
		
   /*saber si un vector no tiene elemento*/
   if(fourth.empty()) cout<<"Vector empty"<<endl;
   else cout<<"Vector not empty"<<endl;
   
   cout << "The contents of fifth are:";
   for (i=0; i < fifth.size(); i++)
      cout << " " << fifth[i];
   cout << endl;
   return 0;
}
\end{lstlisting}