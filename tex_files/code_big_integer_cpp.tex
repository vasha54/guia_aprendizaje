\begin{lstlisting}[language=C++]
#define fst first
#define snd second
#define all(c) ((c).begin()), ((c).end())

struct bigint : vector<long long> {
// Si B es chiquito ==> tratos mas grandes/operaciones mas lentas
// B = 100000000  (10^8) ==> se trata de 10^645636
// B = 1000000000 (10^9) ==> se trata de  10^7378
   static constexpr long long B = 1000000000, W = log10(B);
   char sign;
   bigint& trim() {
      while (!empty() && !back()) pop_back();
      if (empty()) sign = 1;
      return *this;
   }
	
   bigint(long long v = 0) : sign(1) {
      if (v < 0) { sign = -1; v = -v; }
      while (v > 0) { push_back(v % B); v /= B; }
   }
   
   bigint(string s) : sign(+1) {
      long long e = 0;
      for (; e < s.size() && s[e] <= '0' || s[e] >= '9'; ++e)
         if (s[e] == '-') sign = -sign;
      for (long long i = s.size()-1; i >= e; i -= W) {
         push_back(0);
         for (long long j = max(e, i - W + 1); j <= i; ++j)
            back() = 10 * back() + (s[j] - '0');
      }
      trim();
   }
   
   bigint operator-() const {
      bigint x = *this;
      if (!empty()) x.sign = -x.sign;
      return x;
   }
   
   operator vector<long long>(); // proteger de lanzamientos inesperados
   bigint &operator+=(bigint x);
};

// entrada y salida
ostream &operator<<(ostream &os, const bigint &x) {
   if (x.sign < 0) os << '-';
   os << (x.empty() ? 0 : x.back());
   for (int i = (int)x.size()-2; i >= 0; --i)
      os << setfill('0') << setw(bigint::W) << x[i];
   return os;
}
istream &operator>>(istream &is, bigint &x) {
   string s; is >> s; x = bigint(s); return is;
}

bigint operator-(bigint x, bigint y);
bigint operator+(bigint x, bigint y) {
   if (x.sign != y.sign) return x - (-y);
   if (x.size() < y.size()) swap(x, y);
   int c = 0;
   for (int i = 0; i < y.size(); ++i) {
      x[i] += y[i] + c;
      c = (x[i] >= bigint::B);
      if (c) x[i] -= bigint::B;
   }
   if (c) x.push_back(c);
   return x;
}

bigint operator-(bigint x, bigint y) {
   if (x.sign != y.sign) return x + (-y);
   int sign = x.sign; x.sign = y.sign = +1;
   if (x < y) {swap(x, y); sign = -sign;}
   int c = 0;
   for (int i = 0; i < x.size(); ++i) {
       x[i] -= (i < y.size() ? y[i] : 0) + c;
       c = (x[i] < 0);
       if (c) x[i] += bigint::B;
   }
   x.sign = sign; return x.trim();
}

bigint operator*(bigint x, int y) {
   if (y == 0) return bigint(0);
   if (y < 0) { x.sign = -x.sign; y = -y; }
   int c = 0;
   for (int i = 0; i < x.size() || c; ++i) {
      if (i == x.size()) x.push_back(0);
      x[i] = x[i] * y + c;
      c = x[i] / bigint::B;
      if (c) x[i] %= bigint::B;
   }
   return x;
}
bigint operator*(int x, bigint y) { return y * x; }

// multiplicacion utilizando el algoritmo de Karatsuba 
bigint operator*(bigint x, bigint y) {
   if (x.empty()||y.empty()) return bigint(0);
   function<void(long long*, int, long long*, int, long long*)>
      rec = [&](long long *x, int xn, long long *y, int yn, long long *z) {
         if (xn < yn) { auto tmp =xn; xn=yn; yn=tmp; auto tmp1 =x;x=y;y=tmp1;}
         fill(z, z+xn+yn, 0);
         if (yn <= 2) {
            for (int i = 0; i < xn; ++i)
               for (int j = 0; j < yn; ++j)
                  z[i+j] += x[i] * y[j];
         } else {
            int m = (xn+1)/2, n = min(m, yn);
            for (int i = 0; i+m < xn; ++i) x[i] += x[i+m];
            for (int j = 0; j+m < yn; ++j) y[j] += y[j+m];
            rec(x, m, y, n, z+m);
            for (int i = 0; i+m < xn; ++i) x[i] -= x[i+m];
            for (int j = 0; j+m < yn; ++j) y[j] -= y[j+m];
            long long p[2*m+2]; rec(x, m, y, n, p);
            for (int i = 0; i < m+n; ++i){ z[i] += p[i]; z[i+m] -= p[i]; }
            rec(x+m, xn-m, y+m, yn-n, p);
            for (int i = 0; i < xn+yn-m-n; ++i){ 
               z[i+m] -= p[i]; z[i+2*m] += p[i];
            }
         }
   };
   bigint z; z.resize(x.size()+y.size()+2);
   z.sign = x.sign * y.sign;
   rec(&x[0], x.size(), &y[0], y.size(), &z[0]);
   for (int i = 0; i+1 < z.size(); ++i) {
      z[i+1] += z[i] / bigint::B;
      z[i] %= bigint::B;
      if(z[i] < 0){ z[i+1] -= 1; z[i] += bigint::B;}
   }
   return z.trim();
}

bigint pow(bigint x, int b) {
   bigint y(1);
   for (; b > 0; b /= 2) {
      if (b % 2) y = y * x;
      x = x * x;
   }
   return y;
}

pair<bigint, int> divmod(bigint x, int y) {
   if (y < 0) { x.sign = -x.sign; y = -y; }
   int rem = 0;
   for (int i = x.size()-1; i >= 0; --i) {
      x[i] += rem * bigint::B; rem = x[i] % y;
      x[i] /= y;
   }
   return {x.trim(), rem};
}

bigint operator/(bigint x, int y) { return divmod(x, y).fst; }
bigint operator%(bigint x, int y) { return divmod(x, y).snd; }

pair<bigint, bigint> divmod(bigint x, bigint y) {
   auto norm = bigint::B / (y.back() + 1);
   int qsign = x.sign * y.sign, rsign = x.sign;
   x = x * norm; x.sign = +1; y = y * norm; y.sign = +1;
   bigint q, r;
   for (int i = (int)x.size()-1; i >= 0; --i) {
      r.insert(r.begin(), x[i]);
      long long d = ((r.size()   > y.size() ? bigint::B * r[y.size()] : 0) 
                  + (r.size()+1 > y.size() ? r[y.size()-1] : 0)) / y.back();
      r = r - y * d;
      while (r.sign < 0) { r = r + y; d -= 1; }
      q.push_back(d);
   }
   reverse(all(q));
   q.sign = qsign; r.sign = rsign;
   return make_pair(q.trim(), divmod(r.trim(), norm).fst);
}
bigint operator/(bigint x, bigint y) { return divmod(x, y).fst; }
bigint operator%(bigint x, bigint y) { return divmod(x, y).snd; }
\end{lstlisting}

Aunque la implementación presentada aqui es bastante larga, demasiada larga es porque hemos incluidos algunas otras operaciones no analizadas en la guía.