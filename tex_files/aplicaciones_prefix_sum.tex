Entre las aplicaciones de las suma de prefijo podemos citar:

\begin{itemize}
	\item \textbf{Encontrar si hay una subarray con 0 sumas:} dado un arreglo de números positivos y negativos, encuentrar si hay una subarrray (de tamaño al menos uno) con 0 suma.
	\item \textbf{Índice de equilibrio de un arreglo:} el índice de equilibrio de un arreglo es un índice tal que la suma de los elementos en los índices más bajos es igual a la suma de los elementos en los índices más altos.
	\item \textbf{Tamaño máximo del subarreglo, tal que todos los subarreglos de ese tamaño tengan una suma menor que $k$:} dado un arreglo de $n$ enteros positivos y un entero positivo $k$, la tarea es encontrar el tamaño máximo del subarreglo tal que todos los subarreglos de ese tamaño tengan la suma de elementos menores que $k$.
	\item \textbf{Encuentre los números primos que se pueden escribir como la suma de la mayoría de los números primos consecutivos:} dado una serie de límites. Para cada límite, encuentre el número primo que se puede escribir como la suma de la mayoría de los primos consecutivos menores o iguales que el límite.
	\item \textbf{Intervalo más largo con la misma suma en dos arreglos binarios:} dados dos arreglos binarias, $arr1[]$ y $arr2[]$ del mismo tamaño $n$. Encuentre la longitud del tramo común más largo $(i, j)$ donde $j \ge i$ tal que $arr1[i] + arr1[i+1] + \dots + arr1[j] = arr2[i] + arr2[i+1] + \dots + arr2[j]$.
	\item \textbf{Máxima suma de subarreglo módulo $m$:} dado un arreglo de $n$ elementos y un entero $m$. La tarea es encontrar el valor máximo de la suma de su subarreglo módulo $m$, es decir, encontrar la suma de cada subarreglo módulo $m$ e imprimir el valor máximo de esta operación de módulo.
	\item \textbf{Tamaño máximo del subarreglo, tal que todos los subarreglos de ese tamaño tengan una suma menor que k:} dado un arreglo de $n$ enteros positivos y un entero positivo $k$, la tarea es encontrar el tamaño máximo del subarreglo tal que todos los subarreglos de ese tamaño tengan la suma de elementos menores que $k$.
	\item \textbf{Entero máximo ocurrido en n rangos:} dados $n$ rangos de la forma $L$ y $R$, la tarea es encontrar el entero máximo que ocurre en todos los rangos. Si sale más de uno de estos enteros, imprima el más pequeño.
	\item  \textbf{Costo mínimo para adquirir todas las monedas con $k$ monedas extra permitidas con cada moneda:} se le da una lista de $N$ monedas de diferentes denominaciones. puede pagar una cantidad equivalente a cualquier moneda 1 y puede adquirir esa moneda. Además, una vez que hayamos pagado una moneda, podremos elegir como máximo $K$ monedas más y podremos adquirirlas gratis. La tarea es encontrar la cantidad mínima requerida para adquirir todas las $N$ monedas por un valor dado de $K$.
	\item \textbf{Generador de números aleatorios en forma de distribución de probabilidad arbitraria:} dados $n$ números, cada uno con cierta frecuencia de ocurrencia. Devuelve un número aleatorio con una probabilidad proporcional a su frecuencia de aparición.
\end{itemize}