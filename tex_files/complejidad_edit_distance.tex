El procedimiento descrito  va a permitir la creación de la tabla  que calcula el
número mínimo de operaciones básicas para la distancia de edición. La solución se encuentra en $[n,m]$, y la
 tabla se construye fila a fila (a partir de los valores que definen las condiciones
 iniciales) para poder ir reutilizando los valores calculados previamente. 
Como el algoritmo se limita a dos bucles anidados que sólo incluyen
operaciones constantes la complejidad de este algoritmo es de orden O($n\times m$) con similar complejidad espacial. Siendo $n$ y $m$ la longitud de las cadenas inciales. 