En matemáticas, la distancia euclidiana o euclídea, es la distancia ordinaria entre dos puntos de un espacio euclídeo, la cual se deduce a partir del teorema de Pitágoras. 

Esta función de distancia define la distancia entre dos puntos. La función de distancia habitual es la distancia euclidiana donde la distancia entre los puntos $(x_1, y_1)$ y $(x_2, y_2)$ es:

$$\sqrt{(x_2-x_1)^2 + (y_2-y_1)^2}$$

En general, la distancia euclidiana entre los puntos $P=(p_1,p_2,\dots,p_n)$ y 
$Q=(q_1,q_2,\dots ,q_n)$, del espacio euclídeo n-dimensional, se define como: 

$$d_E(P,Q)=\sqrt{(p_1-q_1)^2 + (p_2-q_2)^2 + \cdots + (p_n-q_n)^2} = \sqrt{\sum_{i=1}^n (p_i-q_i)^2}$$