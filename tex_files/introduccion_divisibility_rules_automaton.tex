Por muy trivial que parezca este asunto, ha sido bastante explotado en un buen número de problemas de concursos e inclusos en competencias. Y porque ?. Si para saber si un número {\em a} es divisible por otro número {\em b} solo basta con comprobar si el resto de la división es cero. Donde está entonces la dificultad del problema ?.

Resulta que el el 100\% de los problemas donde se necesita verificar si $a \hspace*{0.3em}mod \hspace*{0.3em} b = 0$ el número {\em a} tiene una cantidad de dígitos no puede ser almacenado por ningún de los tipos de datos númericos enteros conocidos. Entonces que hacer ?.