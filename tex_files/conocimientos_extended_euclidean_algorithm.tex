\subsection{Algoritmo de Euclides}
El algoritmo de Euclides es un método eficiente para encontrar el máximo común divisor (gcd) de dos números enteros. El algoritmo se basa en la observación de que el gcd de dos números $a$ y $b$ es igual al gcd de $b$ y el residuo de dividir $a$ por $b$. 

\subsection{Identidad de Bézout}
La identidad de Bézout es un resultado importante en teoría de números que establece lo siguiente: si $a$ y $b$ son dos enteros no nulos, entonces existen enteros $x$ e $y$ tales que $ax + by = \gcd(a, b)$, donde $\gcd(a, b)$ es el máximo común divisor de $a$ y $b$. Esta identidad es conocida como el teorema de Bézout en honor al matemático francés Étienne Bézout.

\subsection{Ecuaciones lineales diofánticas}
Una ecuación lineal diofántica es una ecuación de la forma $ax + by = c$, donde $a$, $b$, $c$ son enteros y queremos encontrar soluciones enteras para $x$ e $y$. Estas ecuaciones llevan el nombre del matemático griego Diofanto de Alejandría, quien estudió este tipo de ecuaciones en su obra.