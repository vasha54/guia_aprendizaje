Podemos utilizar recursividad para reemplazar cualquier tipo de bucle. A pesar de ello en la práctica no se utiliza demasiado, debido a que un error puede ser trágico en la memoria, así como tener una lista con millones de datos, puede hacer que utiliza mucha memoria. Aun así, la gran mayoría de las veces, utilizamos recursividad para algoritmos de búsqueda u ordenación.

La razón fundamental es que existen numerosos
problemas complejos que poseen naturaleza
recursiva, por lo cual son más fáciles de
implementar con este tipo de algoritmos. Sin embargo, en condiciones críticas de tiempo y
de memoria, la solución a elegir debe ser
normalmente de forma iterativa siempre que sea posible.