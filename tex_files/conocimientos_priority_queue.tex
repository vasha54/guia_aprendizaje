\subsection{Estructuras Dinámicas No Lineales}

Estas estructuras se caracterizan por que el acceso y la modificación ya no son constantes, es
necesario especificar la complejidad de cada función de ahora en adelante.
Entre las estructuras dinámicas no lineales están:

\begin{itemize}
	\item Árboles
	\item Grafos
	\item Cola de Prioridad
	\item Conjunto
	\item Diccionario
\end{itemize}

\subsection{Mónticulo (\emph{Heap})}
En computación, un montículo (o heap en inglés) es una estructura de datos del tipo árbol con información perteneciente a un conjunto ordenado. Los montículos máximos tienen la característica de que cada nodo padre tiene un valor mayor que el de cualquiera de sus nodos hijos, mientras que en los montículos mínimos, el valor del nodo padre es siempre menor al de sus nodos hijos. 

Un árbol cumple la condición de montículo si satisface dicha condición y además es un árbol binario casi completo. Un árbol binario es completo cuando todos los niveles están llenos, con la excepción del último, que se llena desde la izquierda hacia la derecha. 

En un montículo de prioridad, el mayor elemento (o el menor, dependiendo de la relación de orden escogida) está siempre en el nodo raíz. Por esta razón, los montículos son útiles para implementar colas de prioridad. Una ventaja que poseen los montículos es que, por ser árboles completos, se pueden implementar usando arreglos (arrays), lo cual simplifica su codificación y libera al programador del uso de punteros. 