Se llama {\bf combinación} de $n$ objetos tomados $p$ a $p$, a todo conjunto de $p$ objetos
elegidos entre ellos de tal modo que dos conjuntos se diferencien al menos en un
objeto.

Denotaremos por $C_{n,p}$  al número de combinaciones de $n$ objetos tomados $p$ a $p$.

Las combinaciones de $n$ objetos tomados $p$ a $p$ son los distintos conjuntos que pueden
formarse con $p$ objetos elegidos entre $n$ dados, de modo que un conjunto se diferencie
de otro al menos en uno de los elementos.

En las variaciones, \emph{abc} y \emph{bca} son variaciones distintas, pero es la misma
combinación, sólo un conjunto que contenga un nuevo elemento se considera una nueva
combinación. ej: \emph{abd}

Si imaginamos formadas $C_{n,p}$ combinaciones de orden $p$ que se pueden formar con $n$
objetos, ejemplo, las combinaciones ternarias de las cuatro letras a, b, c, d

abc abd acd bcd

En cada combinación permutamos las letras de todas las maneras posibles y obtenemos
el cuadro:

abc abd acd bcd

acb adb adc bdc

bac bad cad cbd

bca bda dac dbc

cba dba dca dcb

El cual contiene las variaciones ternarias de las cuatro letras a, b, c, d pues las
que proceden de la misma combinación difieren en el orden de las letras y las que
proceden de combinaciones distintas difieren al menos en una letra. Como cada
combinación de orden p da lugar a $P_{p}$ combinaciones distintas, entre los números
$C_{n,p}$, $P_{p}$ y $V_{p,n}$ existe la relación:

$C_{n,p}*P_{p}=V_{p,n}$ 

de donde 

$C_{n,p}=\frac{V_{p,n}}{P_{p}}$

sustituyendo $P_{p}$ y $V_{p,n}$

$C_{n,p}=\frac{n*(n-1)*(n-2)* ... *(n-p+1)}{p!}$

Con el objetivo de completar n! en el numerador, multiplicamos el numerador y el
denominador por $(n-p)!$

$C_{n,p}=\frac{n*(n-1)*(n-2)* ... *(n-p+1)*(n-p)!}{p!*(n-p)!}$

$C_{n,p}=\frac{n!}{p!*(n-p)!}$

Las expresiones de la forma $C_{n,p}$ reciben el nombre de números combinatorios.


