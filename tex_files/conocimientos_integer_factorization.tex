\subsection{Número primo}
En matemáticas, un número primo es un número natural mayor que 1 que tiene únicamente
dos divisores positivos distintos: él mismo y el 1. Por el contrario, los números compuestos son
los números naturales que tienen algún divisor natural aparte de sí mismos y del 1, y, por lo tanto,
pueden factorizarse. El número 1, por convenio, no se considera ni primo ni compuesto.

\subsection{Criba de Eratóstenes}
La Criba de Eratóstenes es un algoritmo utilizado para encontrar todos los números primos hasta un cierto límite dado. Fue desarrollado por el matemático griego Eratóstenes en el siglo III a.C. y es uno de los métodos más antiguos y eficientes para encontrar números primos.

\subsection{Factorial}
El factorial de un entero positivo N , el factorial de N o N factorial se define en principio como
el producto de todos los números enteros positivos desde 1 (es decir, los números naturales) hasta
N.

\subsection{Pierre Fermat}
Pierre de Fermat fue un matemático francés del siglo XVII conocido por sus contribuciones
a la teoría de números, geometría analítica y cálculo. Una de las contribuciones más famosas de
Fermat es el Teorema de Fermat, que enunció en una nota al margen de su ejemplar de la obra
de Diofanto. El teorema establece que no existen enteros positivos $a$, $b$, $c$ y $n$ mayores que 2 que satisfagan la ecuación $a^n + b^n = c^n$ . Este teorema fue uno de los problemas abiertos más famosos en matemáticas y se conoció como el Último Teorema de Fermat. Fue finalmente demostrado por Andrew Wiles en 1994 utilizando técnicas avanzadas de matemáticas modernas.

\subsection{John Pollard}
John Pollard es un matemático británico conocido por sus contribuciones en el campo de la teoría de números y la criptografía. Nació en 1926 y ha realizado importantes investigaciones en áreas como los números primos, los algoritmos de factorización y la seguridad en sistemas criptográficos.


\subsection{Multiplicación de Montgomery}
La multiplicación de Montgomery es un método eficiente para realizar operaciones de multiplicación en aritmética modular. Fue desarrollada por Peter L. Montgomery en la década de 1980 y se utiliza principalmente en criptografía asimétrica, como en algoritmos de cifrado de clave pública como RSA.

\subsection{Algoritmo de Euclides}
El algoritmo de Euclides es un método eficiente para encontrar el máximo común divisor (MCD) de dos números enteros. Fue desarrollado por el matemático griego Euclides en el siglo III a.C. y sigue siendo uno de los algoritmos más utilizados en matemáticas y computación.