Este es de los problemas de la programación dinámica, uno de los denomindados como clásicos. Para la solución de las dos variantes se llega a un enfoque \textbf{\emph{Bottom Up + Memorization}} (implementaciones) aunque en el proceso de análisis se trabajo con el enfoque \textbf{\emph{Top Down + (Memorization)}}. 

Aunque los problemas de tipo programación dinámica son muy popular con una alta frecuencia de aparición en concursos de programación
recientes, los problemas clásicos de programación dinámica en su forma pura (como el presentado en esta guía) por lo general ya no aparecen en los IOI o ICPC modernos. A pesar de esto es necesario su estudio ya
que nos permite entender la programación dinámica y como poder resolver aquellos problema de programación dinámica clasificados como no-clásicos e incluso nos permite desarrollar nuestras habilidades de programación
dinámica en el proceso.