\subsection{C++}

\subsubsection{Algoritmo trivial}
\begin{lstlisting}[language=C++]
vector<int> manacher_odd(string s) {
  int n = s.size();
  s = "$" + s + "^";
  vector<int> p(n + 2);
  for(int i = 1; i <= n; i++) {
     while(s[i-p[i]]==s[i+p[i]])p[i]++;
  }
  return vector<int>(begin(p) + 1, end(p) - 1);
}
\end{lstlisting} 

Los caracteres terminales \$ y $\wedge$ se usaron para evitar tratar los extremos de la cadena por separado.

\subsubsection{Algoritmo Manacher}

\begin{lstlisting}[language=C++]
int manacher(string _text){
   int n=_text.size();
   int rad[2*n];
   int i,j,k;
   for(i=0,j=0;i<2*n;i+=k, j=max(j-k,0)){
      while(i-j>=0 && i+j+1<2*n && _text[(i-j)/2]==_text[(i+j+1)/2])
        ++j;
      rad[i]=j;
      for(k=1; i-k>=0 && rad[i]-k>=0 && rad[i-k]!=rad[i]-k ;++k)
         rad[i+k]=min(rad[i-k],rad[i]-k);
    }
    return *max_element(rad,rad+(2*n));
}
	
\end{lstlisting}

\subsection{Java}
\subsubsection{Algoritmo Manacher}

\begin{lstlisting}[language=C++]
public int manacher(String _text){
  int n=_text.length();
  int [] rad=new int [2*n];
  int i,j,k;
  for(i=0,j=0;i<2*n;i+=k, j=Math.max(j-k,0)){
     while(i-j>=0 && i+j+1<2*n && _text.charAt((i-j)/2)==_text.charAt((i+j+1)/2))
       ++j;
     rad[i]=j;
     for(k=1; i-k>=0 && rad[i]-k>=0 && rad[i-k]!=rad[i]-k ;++k)
     rad[i+k]=Math.min(rad[i-k],rad[i]-k);
   }
   return Arrays.stream(rad).max().getAsInt(); 
}
	
\end{lstlisting}