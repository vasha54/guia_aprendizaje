\subsection{Función continua}
Una función se dice continua en un punto si el límite de la función en ese punto es igual al valor de la función en ese punto. En otras palabras, una función es continua si no hay saltos, huecos ni discontinuidades en su gráfico.

Una función se considera continua en un intervalo si es continua en cada punto dentro de ese intervalo. La continuidad de una función es un concepto fundamental en el análisis matemático y es importante en muchas áreas de las matemáticas y la física.

\subsection{Función diferenciable}
Una función se dice diferenciable en un punto si su derivada existe en ese punto. La derivada de una función en un punto representa la tasa de cambio instantánea de la función en ese punto. En otras palabras, la función es diferenciable en un punto si tiene una recta tangente bien definida en ese punto.

Una función se considera diferenciable en un intervalo si es diferenciable en cada punto dentro de ese intervalo. La diferenciabilidad de una función es un concepto fundamental en el cálculo y es esencial para comprender el comportamiento local de las funciones.

Las funciones diferenciables tienen propiedades interesantes y útiles, como la regla del producto, la regla de la cadena y la regla de derivación de funciones compuestas, que permiten calcular derivadas de funciones más complejas. La diferenciabilidad es un concepto fundamental en el análisis matemático y es utilizado en diversas áreas, como la física, la ingeniería y la economía.