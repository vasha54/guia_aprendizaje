\subsection{Recorrido en Profundidad (\emph{DFS})}
Búsqueda en profundidad. Una Búsqueda en profundidad (en inglés \emph{DFS} o \emph{Depth First Search}) es un algoritmo que permite recorrer todos los nodos de un grafo o árbol (teoría de grafos) de manera ordenada, pero no uniforme. Su funcionamiento consiste en ir expandiendo todos y cada uno de los nodos que va localizando, de forma recurrente, en un camino concreto. Cuando ya no quedan más nodos que visitar en dicho camino, regresa (Backtracking), de modo que repite el mismo proceso con cada uno de los hermanos del nodo ya procesado. 

\subsection{Programación Dinámica}
La programación dinámica es un método para reducir el tiempo de ejecución de un algoritmo mediante la utilización de subproblemas superpuestos y subestructuras óptimas. 