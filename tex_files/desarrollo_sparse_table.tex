El algoritmo ingenuo recorre todos los elementos en $[i, j]$, y requiere O($N$) en el peor caso. Ahora si nos hicieran $M$ preguntas tendríamos un algoritmo con una complejidad de O($MN$) lo cual no sería factible si la cantidad de preguntas ronda las $10^5$. Ahora hagamos un breve análisis:

Si los elementos del arreglo no pueden cambiar de valor, podemos preprocesar los datos para acelerar el cálculo de las respuestas. Hay $N^2$ preguntas posibles, y podemos precalcular todas sus respuestas en  $N^2$  usando programación dinámica. No hace falta guardar todas las respuestas: basta con
precalcular una cantidad suficiente como para poder responder cualquier pregunta.

Todo número entero no negativo puede ser representado únicamente como una suma de decrecientes potencias de 2. Esto es precisamente la representación binaria de un número. Ej. $13 = (1101)_2 = 8 + 4 + 1$. Para un número $x$ habrá cuando más $\lceil \log_2 x \rceil$ sumandos.

Por este mismo razonamiento cualquier intervalo puede ser únicamente representado como una unión de intervalos de longitud igual a decrecientes potencias de 2. Por ejemplo $[2, 14] = [2, 9] \cup [10, 13] \cup [14, 14]$ donde la longitud del intervalo completo es $13$ y donde los intervalos individuales tienen de longitud $8$, $4$ y $1$ respectivamente, coincidiendo con la representación binaria de $13$ que es la longitud del intervalo completo. Aquí también el intervalo principal podrá ser representado como máximo por  $\lceil \log_2(\text{longitud del intervalo}) \rceil$ intervalos individuales.

La idea detrás de la Tabla dispersa es precalcular todas las respuestas para consultas de intervalos con longitud igual a una potencia de 2. Luego para responder a la consulta de un intervalo de longitud cualquiera se divide éste en la unión de intervalos de longitud igual a potencias de 2, donde conociendo las respuestas para estos últimos (pues lo precalculamos inicialmente), se combinan dichas respuestas y se obtiene el valor correspondiente al intervalo principal.

\subsection{Precálculo}
Usaremos una matriz bidimensional para almacenar las respuestas a las consultas precalculadas $\text{st}[i][j]$ almacenará la respuesta para el rango  $[j, j + 2^i-1]$  de longitud $2^i$. El tamaño de la matriz bidimensional será $(K + 1) \times MAXN$ , dónde $MAXN$ es la mayor longitud de matriz posible. K tiene que satisfacer $\text{K} \ge \lfloor \log_2 \text{MAXN} \rfloor$ porque  $2^{\lfloor \log_2 \text{MAXN} \rfloor}$ es la potencia más grande de dos rangos, que tenemos que soportar. Para arreglos con una longitud razonable ( $\le 10^7$  elementos), $K = 25$ es un buen valor. El MAXN es la segunda dimensión para permitir accesos consecutivos a la memoria (compatibles con la caché). 

Porque el rango  $[j, j + 2^i - 1]$ de longitud $2^i$ se divide muy bien en los rangos  $[j, j + 2^{i - 1} - 1]$ y $[j + 2^{i - 1}, j + 2^i - 1]$, ambos de longitud $2^{i - 1}$, podemos generar la tabla de manera eficiente usando programación dinámica. 

\subsection{Consultas de suma en rangos}
Para este tipo de consultas, nosotros necesitamos hallar la suma de todos los valores en un rango. La operación asociativa empleada sería la suma. Para este tipo de consultas, queremos encontrar la suma de todos los valores en un rango. Por lo tanto la definición natural de la función $f$ es $f(x, y) = x + y$.

Para responder a la consulta de suma para el rango $[L, R]$, iteramos sobre todas las potencias de dos, comenzando por la más grande. Tan pronto
como una potencia de dos $2^i$ es menor o igual a la longitud del rango ($I= R - L + 1$) , procesamos la primera parte del rango $[L, L + 2^i - 1]$, y y
continuar con el rango restante $[L + 2^i, R]$. Nosotros simplemente hallamos la longitud del intervalo $I= R - L + 1$ y descomponemos dicho intervalo en intervalos de longitud igual a una potencia de 2, que como vimos al inicio coincide con los bits activos de la representación binaria del número $I$. 

Donde $I$ en su representación binaria tenga el bit $k$ activo, el subintervalo a tener en cuenta es $st[k][L]$. Luego aumentamos  el  extremo  derecho de  $L$  en   $2^k$  para continuar el procesamiento ahora en el intervalo $[L + 2^k, R]$. Continuar el procesamiento significa hallar los restantes bits activos de $I$ y desplazar el extremo derecho del intervalo en $2^k$ posiciones, siendo $k$ el bit activo que se encontró en $I$. El resultado a devolver en el método, la variable $r$, se inicializa con el elemento neutro de la operación asociativa empleada, pues cada vez que se encuentre un bit activo en $I$, $r$ se operará con $st[k][L]$. 
Para chequear si $I$ tiene el bit $k$ activo se emplea la operación de bits $I^ \& (1<<k)$. Para desplazar el extremo izquierdo del intervalo en $2^k$ posiciones, se emplea la operación de bits $L += (1<<k)$.



\subsection{Consultas de mínimo en rangos}
Al calcular el mínimo de un rango, no importa si procesamos un valor en el rango una o dos veces. Por lo tanto, en lugar de dividir un rango en varios rangos, también podemos dividir el rango en solo dos rangos superpuestos con una
potencia de dos longitudes. Por ejemplo, podemos dividir el rango $[1, 6]$ en los rangos $[1, 4]$ y $[3, 6]$. El rango mínimo de $[1, 6]$ es claramente el mismo que el mínimo del rango mínimo de $[1, 4]$ y el rango mínimo de $[3, 6]$. 
Entonces podemos calcular el mínimo del rango $[L, R]$ con:

$$\min(\text{st}[i][L], \text{st}[i][R - 2^i + 1]) \quad \text{ donde } i = \log_2(R - L + 1)$$

Esto requiere que seamos capaces de calcular $\log_2(R - L + 1)$ rápido. Puedes lograrlo precalculando todos los logaritmos o alternativamente, se puede calcular sobre la marcha en espacio y tiempo constantes.