El ordenamiento por cuentas (counting sort en inglés) es un algoritmo de ordenamiento en el que se cuenta el número de elementos de cada clase para luego ordenarlos. Sólo puede ser utilizado por tanto para ordenar elementos que sean contables (como los números enteros en un determinado intervalo, pero no los números reales, por ejemplo).

El primer paso consiste en averiguar cuál es el intervalo dentro del que están los datos a ordenar (valores mínimo y máximo). Después se crea un vector de números enteros con tantos elementos como valores haya en el intervalo [mínimo, máximo], y a cada elemento se le da el valor 0 (0 apariciones). Tras esto se recorren todos los elementos a ordenar y se cuenta el número de apariciones de cada elemento (usando el vector que hemos creado). Por último, basta con recorrer este vector para tener todos los elementos ordenados.