En los ejercicios y problemas de concursos para resolverlos utilizando el algoritmo BFS hay que partir de las aplicaciones o usos que se le puede dar a este en diferentes situaciones donde la utilización de este algoritmo es parte de la solución global al problema.

\begin{itemize}
	\item Encuentre la ruta más corta desde un vertices a otros vértices en un grafo no ponderado.
	\item Encuentra todas las componentes conexa en un grafo no dirigido en tiempo O(n+m): Para hacer esto, solo ejecutamos BFS comenzando desde cada vértice, excepto los vértices que ya han sido visitados en ejecuciones anteriores. Por lo tanto, realizamos BFS normal desde cada uno de los vértices, pero no reiniciamos el arreglo de visitados utilizados cada vez que obtenemos una nueva componente conectada, y el tiempo total de ejecución seguirá siendo O (n + m) (realizando múltiples BFS en el gráfico sin poner a cero el arreglo $used[]$ se llama una serie de búsquedas primero en amplitud).
	\item Encontrar una solución a un problema o un juego con el menor número de movimientos, si cada estado del juego se puede representar por un vértice del grafo, y las transiciones de un estado al otro son las aristas del grafo.
	\item Encontrar la ruta más corta en un gráfico con pesos 0 o 1: Esto requiere solo una pequeña modificación a la búsqueda normal en anchura: si la arista actual de peso cero y la distancia al vértice es más corta que la distancia encontrada actual, agregue esto vértice no hacia atrás, sino hacia el frente de la cola.
	\item Encontrar el ciclo más corto en un grafo no ponderado dirigido: Inicie una búsqueda en anchura desde cada vértice. Tan pronto como intentamos volver del vértice actual al vértice de origen, hemos encontrado el ciclo más corto que contiene el vértice de origen. En este punto, podemos detener el BFS y comenzar un nuevo BFS desde el siguiente vértice. De todos esos ciclos (como máximo uno de cada BFS) elija el más corto.
	\item Encuentre todos las aristas que se encuentran en cualquier camino más corto entre un par de vértices dados (a, b). Para hacer esto, ejecute dos búsquedas primero en amplitud: una desde a y otra desde b. Sea $d_{a}[]$ la matriz que contiene las distancias más cortas obtenidas del primer BFS (de a) y $d_{b}[]$ sea la matriz que contiene las distancias más cortas obtenidas del segundo BFS de b. Ahora, para cada arista $(u,v)$ es fácil verificar si esa arista se encuentra en algún camino más corto entre a y b: el criterio es la condición $d_{a}[u]+1+d_{b}[v]=d_{a}[b]$.
	\item Encuentre la ruta más corta de longitud uniforme desde un vértice de origen s hasta un vértice de destino t en un grafo no ponderado: para esto, debemos construir un grafo auxiliar, cuyos vértices son el estado (v, c), donde v - el nodo actual, c=0 o c=1 - la paridad actual. Cualquier borde (a,b) del grafo original en esta nueva columna se convertirá en dos aristas ((u,0),(v,1)) y ((u,1),(v,0)). Después de eso, ejecutamos un BFS para encontrar la ruta más corta desde el vértice inicial (s,0) hasta el vértice final (t,0).
\end{itemize}