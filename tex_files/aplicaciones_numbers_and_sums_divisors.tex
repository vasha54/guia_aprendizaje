Entre las aplicaciones que tiene el poder determinar la cantidad y suma de los divisores de un número $n$ están:

\begin{enumerate}
	\item \textbf{Criptografía:} En criptografía, el cálculo de la cantidad o suma de divisores  de un número puede ser utilizado en la generación de claves criptográficas seguras.
	
	\item \textbf{Teoría de números:} El cálculo de la cantidad de divisores de un número es fundamental en la teoría de números para estudiar propiedades de los números enteros, como los números primos y compuestos. Calcular la suma de los divisores de un número es útil en la teoría de números para estudiar propiedades de los números enteros, como los números perfectos, abundantes y deficientes.
	
	\item \textbf{Optimización de algoritmos:} En informática, el cálculo de la cantidad o suma de divisores de un número puede ser utilizado para optimizar algoritmos en diferentes áreas, como la programación dinámica y la optimización combinatoria.
	
	\item \textbf{Matemáticas recreativas:} El cálculo de la cantidad o suma de divisores de un número también puede ser utilizado en problemas matemáticos recreativos y desafíos matemáticos para ejercitar el razonamiento lógico y matemático. 
	
	\item  \textbf{Ingeniería eléctrica:} En ingeniería eléctrica, el cálculo de la cantidad o suma de divisores de un número puede ser utilizado en el diseño y análisis de circuitos eléctricos para determinar la eficiencia y la estabilidad del sistema.
	
\end{enumerate}
