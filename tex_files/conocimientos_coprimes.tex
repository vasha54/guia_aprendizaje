\subsection{Numero primo}
Número primo es cualquier número natural mayor que 1 cuyo únicos divisores posibles son el mismo número primo y el factor natural 1 .A diferencia de los números primos , los números compuestos son naturales que pueden factorizarse. Ejemplo de números primos son el 2, 3, 5, 7, 11, 13.

\subsection{Máximo común divisor}
En matemática, se define el máximo común divisor (MCD) de dos o más números enteros al mayor número entero que los divide sin dejar resto.

\subsection{Teorema chino del resto}
El teorema chino del resto es un resultado sobre congruencias en teoría de números y sus generalizaciones en álgebra abstracta. Fue publicado por primera vez en el siglo III por el matemático chino Sun Tzu. 

\subsection{Criba de Eratóstenes}
Es un algoritmo para encontrar todos los números primos en un segmento $[1;n]$ usando $O(n \log \log n)$ operaciones.

