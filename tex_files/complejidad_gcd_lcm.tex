El tiempo de ejecución del algoritmo de Euclides del GCD se estima mediante el teorema de Lamé, que establece una conexión sorprendente entre el algoritmo de Euclides y la sucesión de Fibonacci:

Si $a > b \geq 1$ y $b < F_n$ para algunos $n$, el algoritmo de Euclides realiza como máximo $n-2$ llamadas recursivas.

Además, es posible demostrar que el límite superior de este teorema es óptimo. Cuando $a = F_n$ y $b = F_{n-1}$, $gcd(a, b)$ funcionará exactamente $n-2$ llamadas recursivas. En otras palabras, los números de Fibonacci consecutivos son la entrada del peor de los casos para el algoritmo de Euclides.

Dado que los números de Fibonacci crecen exponencialmente, obtenemos que el algoritmo de Euclides funciona en $O(\log \min(a, b))$.

Otra forma de estimar la complejidad es notar que $a \bmod b$ para el caso $a \geq b$ Por lo menos $2$ veces menor que $a$, por lo que el número mayor se reduce al menos a la mitad en cada iteración del algoritmo.