Existen muchas razones por las cuales un cubierta convexa de un conjunto de puntos es una estructura geométrica importante:

\begin{itemize}
	\item Es una de las aproximaciones de forma de un conjunto de puntos más simples (otras incluyen rectángulos, círculos, etc.)
	\item Puede ser usada para aproximar formas más complejas (cubiertas convexas de polígonos o poliedros)
	\item Algunos algoritmos calculan la cubierta convexa como una etapa inicial (preprocesamiento) de su ejecución (filtrar puntos irrelevantes)
\end{itemize}

Por ejemplo, el diámetro de un conjunto de puntos es la máxima distancia entre cualesquiera dos puntos del conjunto. Puede demostrarse que el par de puntos que determina el diámetro son ambos vértices de la cubierta convexa. También se puede observar que las mínimas formas convexas envolventes (rectángulo, círculo, etc.) depende sólo de los puntos
de la cubierta convexa.