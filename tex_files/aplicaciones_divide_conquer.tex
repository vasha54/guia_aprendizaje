La uso de esta técnica para el diseño de algoritmo ha posibilitado que podamos contar con algoritmos ya bien definidos que se apoyan en esta técnica o basados en esta técnica para resolver problemas concretos o se pueden implementar algoritmos capaces de solucionar determinados problemas cuyas soluciones triviales presentan mayor complejidad que los algoritmos diseñados bajo esta técnica. A continuación una lista de estos ejemplos:

\begin{enumerate}
	\item Búsqueda binaria, binaria no centrada y ternaria son ejemplos claros de la técnica Divide y
	Vencerás. El problema de partida es decidir si existe un elemento dado x en un
	vector de enteros ordenado. El hecho de que esté ordenado va a permitir utilizar
	esta técnica
	\item Para mulplicar  $u$ y $v$ que son dos números naturales de $n$ bits donde, por simplicidad, $n$ es una
potencia de 2. 
	\item Buscar la moda en un arreglo.
	\item Buscar la mediana de dos arreglos
	\item La multiplicación de matrices cuadradas
	\item Buscar la subsecuencia de suma máxima de elementos concecutivos en un arreglo.
	\item Organizar el pareo de un torneo con $n$ jugadores en donde cada jugador
	ha de jugar exactamente una vez contra cada uno de sus posibles $n-1$ competidores,
	y además ha de jugar un partido cada día,teniendo a lo sumo un día de descanso en
	todo el torneo.
	\item El elemento en su posición. Sea $a[1 \dots n]$ un arreglo ordenado de enteros todos distintos. Nuestro problema es implementar un algoritmo de complejidad O(logn) en el peor caso capaz de
encontrar un índice i tal que $1 \le i \le n$ y $a[i] = i$, suponiendo que tal índice exista.
	\item El elemento mayoritario de un arreglo. Sea $a[1 \dots n]$ un arreglo de enteros. Un elemento $x$ se denomina elemento mayoritario
de a si $x$ aparece en el vector más de $n/2$ veces,
\end{enumerate}  









