\subsection{Representación de grafos}
Existen diferentes formas de representar un grafo. La estructura de datos usada depende de las características del grafo y el algoritmo usado para manipularlo. Entre las estructuras más sencillas y usadas se encuentran las listas y las matrices, aunque frecuentemente se usa una combinación de ambas. Las listas son preferidas en grafos dispersos porque tienen un eficiente uso de la memoria. Por otro lado, las matrices proveen acceso rápido, pero pueden consumir grandes cantidades de memoria.

La estructura de datos usada depende de las características del grafo y el algoritmo usado para manipularlo. Entre las estructuras más sencillas y usadas se encuentran las listas y las matrices, aunque frecuentemente se usa una combinación de ambas. Las listas son preferidas en grafos dispersos porque tienen un eficiente uso de la memoria. Por otro lado, las matrices proveen acceso rápido, pero pueden consumir grandes cantidades de memoria.

\subsection{Cola con prioridad}
Una cola de prioridad es una estructura de datos en la que los elementos se atienden en el orden
 indicado por una prioridad asociada a cada uno. Si varios elementos tienen la misma prioridad, se
atenderán de modo convencional según la posición que ocupen, puede llegar a ofrecer la extracción
constante de tiempo del elemento más grande (por defecto), a expensas de la inserción logarítmica, o
utilizando greater<int> causaría que el menor elemento aparezca como la parte superior con .top().
Trabajar con una cola de prioridad es similar a la gestión de un heap



