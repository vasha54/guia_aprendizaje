\subsection{Programación Dinámica}
Es un método para reducir el tiempo de ejecución de un algoritmo mediante la utilización de subproblemas superpuestos y subestructuras óptimas. 

La programación dinámica toma normalmente uno de los dos siguientes enfoques:

\begin{enumerate}
	\item \textbf{Top-down:} El problema se divide en subproblemas, y estos se resuelven recordando las soluciones por si fueran necesarias nuevamente. Es una combinación de memoización y recursión.
	\item \textbf{Bottom-up:} Todos los problemas que puedan ser necesarios se resuelven de antemano y después se usan para resolver las soluciones a problemas mayores. Este enfoque es ligeramente mejor en consumo de espacio y llamadas a funciones, pero a veces resulta poco intuitivo encontrar todos los subproblemas necesarios para resolver un problema dado.
\end{enumerate}




