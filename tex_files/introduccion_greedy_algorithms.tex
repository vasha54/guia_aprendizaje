En ciencias de la computación, un algoritmo voraz (también conocido como goloso, ávido, devorador o greedy) es una estrategia de búsqueda por la cual se sigue una heurística consistente en elegir la opción óptima en cada paso local con la esperanza de llegar a una solución general óptima. Este esquema algorítmico es el que menos dificultades plantea a la hora de diseñar y comprobar su funcionamiento. Normalmente se aplica a los problemas de optimización.