\subsection{Función discreta}
Una función discreta $f$, es una función matemática cuyo dominio de definición es un conjunto numerable (o discreto).

\subsection{Método de Newton}

\subsection{Sucesión numérica}
Una sucesión numérica es un conjunto ordenado de objetos de números. Cada uno de ellos es denominado \textbf{término} (también elemento o miembro) de la sucesión y al número de elementos ordenados (posiblemente infinitos) se le denomina la longitud de la sucesión. 
A diferencia de un conjunto, el orden en que aparecen los términos sí es relevante y un mismo término puede aparecer en más de una posición. De manera formal, una sucesión puede definirse como una función sobre el conjunto de los números naturales (o un subconjunto del mismo) y es por tanto una función discreta.

Se puede usar la notación $a_n$  para indicar una sucesión, en donde $a_n$ hace referencia al elemento de la sucesión en la posición $n$ y de denomina  \textbf{término general}. El conocimiento de la expresión del término general permite calcular cada término de la sucesión atendiendo al lugar que ocupa.