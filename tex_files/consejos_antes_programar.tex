Con la práctica, se puede observar que se puede confundir a otros programadores con el código que
 se haga. Antes de siquiera hacer una línea de código, si se trabaja con otros programadores, ha de
tenerse en cuenta que todos deben escribir de una forma similar el código, para que de forma global
 puedan corregir el código en el caso de que hubieran errores o rastrearlos en el caso de haberlos.
También es muy recomendable hacer uso de comentarios (comenta todo lo que puedas, hay veces
 que lo que parece obvio para ti, no lo es para los demás) y tratar de hacer un código limpio y comprensible, especificando detalles y haciendo tabulaciones, aunque te tome un poco más de
tiempo, es posible que más adelante lo agradezcas tú mismo.