Una variante de este algoritmo es desarrollarlo sobre una matriz que en disimiles situaciones actua como un mapa de donde una celda se puede a unas de sus vecinas acorde a al concepto de vecinas que plantee el problema. Para este tipo de sistuaciones se trabaja con una matriz de visitados para saber cual celda de la matriz ha sido visitado y cual no. 

Además se utiliza una estructura para representar la celda de la matriz de la cual siempre en todas las ocasiones se almacena la fila y columna de dicha celda. El otro elemento en este tipo de situaciones son las llamadas matrices direccionales que ayudan de una forma más practica dada una celda hallar todas las celdas vecinas a esta dependiendo del problema. En la implementación del guía se pone un ejemplo de este caso muy peculiar pero super utilizado para resolver problemas.