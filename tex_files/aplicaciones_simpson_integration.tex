La fórmula de Simpson es un método numérico utilizado para aproximar el valor de una integral definida. Algunas de las aplicaciones de la integración mediante la fórmula de Simpson incluyen:

\begin{enumerate}
	\item \textbf{Cálculo de áreas:} La fórmula de Simpson se puede utilizar para aproximar el área bajo una curva, lo que es útil en campos como la geometría y la física para determinar áreas de regiones irregulares.
	
	\item \textbf{Cálculo de volúmenes:} En el caso de funciones tridimensionales, la fórmula de Simpson se puede aplicar para aproximar el volumen de sólidos obtenidos al rotar una curva alrededor de un eje.
	
	\item \textbf{Análisis de datos:} En estadística, la integración mediante la fórmula de Simpson se puede utilizar para calcular la probabilidad bajo una distribución de probabilidad continua.
	
	\item \textbf{Resolución de ecuaciones diferenciales:} En algunos casos, las ecuaciones diferenciales se pueden resolver mediante métodos numéricos que involucran la integración, como la fórmula de Simpson.
	
	\item \textbf{Modelado matemático:} La fórmula de Simpson también se utiliza en la construcción de modelos matemáticos en diversas disciplinas, como la ingeniería, la economía y la biología, para analizar fenómenos y predecir comportamientos.
	 
\end{enumerate}

