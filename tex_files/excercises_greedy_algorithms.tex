En la programación competitiva es común encontrar ejercicios o problemas que la solución radica en aplicar este enfoque algorítmico, por lo cual es posible decir que existe un grupo de situaciones que se conoce que su solución parte de este enfoque. Por lo cual nos vamos a encontrar:

\begin{itemize}
	\item Algoritmos greedy estándar.
	\item Algoritmos greedy en arreglos.
	\item Algoritmos greedy en grafos.
	\item Algoritmos greedy aproximado para NP completo
	\item Algoritmos greedy para casos especiales de la programación dinámica 
\end{itemize} 

A continuación una lista de ejercicios que se pueden resolver utilizando este enfoque algorítmico:

\begin{itemize}
	\item \href{https://dmoj.uclv.edu.cu/problem/pallargo}{DMOJ - Palíndromo}
	\item \href{https://dmoj.uclv.edu.cu/problem/m235}{DMOJ - Múltiplo de 2, 3 y 5}
	\item \href{https://dmoj.uclv.edu.cu/problem/maxmin}{DMOJ - Arreglos Injustos}
	\item \href{https://dmoj.uclv.edu.cu/problem/aisalador}{DMOJ - Aislador}
\end{itemize}