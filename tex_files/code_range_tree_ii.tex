\subsection{C++}

\subsubsection{Hallar el máximo y el número de veces que aparece}

\begin{lstlisting}[language=C++]
pair<int, int> t[4*MAXN];

pair<int, int> combine(pair<int, int> a, pair<int, int> b) {
   if (a.first > b.first) return a;
   if (b.first > a.first) return b;
   return make_pair(a.first, a.second + b.second);
}

void build(int a[], int v, int tl, int tr) {
   if (tl == tr) { t[v] = make_pair(a[tl], 1);} 
   else {
      int tm = (tl + tr) / 2;
      build(a, v*2, tl, tm); build(a, v*2+1, tm+1, tr);
      t[v] = combine(t[v*2], t[v*2+1]);
   }
}

pair<int, int> get_max(int v, int tl, int tr, int l, int r) {
   if (l > r) return make_pair(-INF, 0);
   if (l == tl && r == tr) return t[v];
   int tm = (tl + tr) / 2;
   return combine(get_max(v*2, tl, tm, l, min(r, tm)), 
   get_max(v*2+1, tm+1, tr, max(l, tm+1), r));
}

void update(int v, int tl, int tr, int pos, int new_val) {
   if (tl == tr) { t[v] = make_pair(new_val, 1);} 
   else {
      int tm = (tl + tr) / 2;
      if (pos <= tm) update(v*2, tl, tm, pos, new_val);
      else update(v*2+1, tm+1, tr, pos, new_val);
      t[v] = combine(t[v*2], t[v*2+1]);
   }
}
\end{lstlisting}

\subsubsection{Contando el número de ceros, buscando el $k$-th cero}
\begin{lstlisting}[language=C++]
int find_kth(int v, int tl, int tr, int k) {
   if (k > t[v]) return -1;
   if (tl == tr) return tl;
   int tm = (tl + tr) / 2; 
   if (t[v*2] >= k) return find_kth(v*2, tl, tm, k);
   else return find_kth(v*2+1, tm+1, tr, k - t[v*2]);
}
\end{lstlisting}

\subsubsection{Buscando el primer elemento mayor que una cantidad dada}

\begin{lstlisting}[language=C++]
int get_first(int v, int lv, int rv, int l, int r, int x) {
   if(lv > r || rv < l) return -1;
   if(l <= lv && rv <= r) {
      if(t[v] <= x) return -1;
      while(lv != rv) {
         int mid = lv + (rv-lv)/2;
         if(t[2*v] > x) { v = 2*v; rv = mid;}
         else { v = 2*v+1; lv = mid+1;}
      }
      return lv;
    }
    int mid = lv + (rv-lv)/2;
    int rs = get_first(2*v, lv, mid, l, r, x);
    if(rs != -1) return rs;
    return get_first(2*v+1, mid+1, rv, l ,r, x);
}
\end{lstlisting}

\subsubsection{Encontrar subrangos con la suma máxima}
\begin{lstlisting}[language=C++]
struct data { int sum, pref, suff, ans; };

data combine(data l, data r) {
   data res;
   res.sum = l.sum + r.sum;
   res.pref = max(l.pref, l.sum + r.pref);
   res.suff = max(r.suff, r.sum + l.suff);
   res.ans = max(max(l.ans, r.ans), l.suff + r.pref);
   return res;
}

data make_data(int val) {
   data res; res.sum = val;
   res.pref = res.suff = res.ans = max(0, val);
   return res;
}

void build(int a[], int v, int tl, int tr) {
   if (tl == tr) { t[v] = make_data(a[tl]);} 
   else {
      int tm = (tl + tr) / 2;
      build(a, v*2, tl, tm); build(a, v*2+1, tm+1, tr);
      t[v] = combine(t[v*2], t[v*2+1]);
   }
}

void update(int v, int tl, int tr, int pos, int new_val) {
   if (tl == tr) { t[v] = make_data(new_val);} 
   else { 
      int tm = (tl + tr) / 2;
      if (pos <= tm) update(v*2, tl, tm, pos, new_val);
      else update(v*2+1, tm+1, tr, pos, new_val); 
      t[v] = combine(t[v*2], t[v*2+1]);
   }
}

data query(int v, int tl, int tr, int l, int r) {
   if (l > r) return make_data(0);
   if (l == tl && r == tr) return t[v];
   int tm = (tl + tr) / 2;
   return combine(query(v*2,tl,tm,l,min(r,tm)),
                   query(v*2+1,tm+1,tr,max(l,tm+1),r));
}

\end{lstlisting}

\subsubsection{Encuentra el número más pequeño mayor o igual a un número específico. Sin consultas de modificación.}

\begin{lstlisting}[language=C++]
vector<int> t[4*MAXN];
void build(int a[], int v, int tl, int tr) {
   if (tl == tr) { t[v] = vector<int>(1, a[tl]);} 
   else {
      int tm = (tl + tr) / 2;
      build(a, v*2, tl, tm); build(a, v*2+1, tm+1, tr);
      merge(t[v*2].begin(),t[v*2].end(),t[v*2+1].begin(),t[v*2+1].end(),back_inserter(t[v]));
   }
}

int query(int v, int tl, int tr, int l, int r, int x) {
   if (l > r) return INF;
   if (l == tl && r == tr) {
      vector<int>::iterator pos = lower_bound(t[v].begin(), t[v].end(), x);
      if (pos != t[v].end()) return *pos;
      return INF;
   }
   int tm = (tl + tr) / 2;
   return min(query(v*2,tl,tm,l,min(r,tm),x),
              query(v*2+1,tm+1,tr,max(l,tm+1),r,x));
}
\end{lstlisting}

La constante INF es igual a un número grande que es mayor que todos los números en el arreglo. Su uso significa que no hay número mayor o igual que x en el segmento Tiene el signifcado de \emph{no hay respuesta en el intervalo dado}.

\subsubsection{Encuentra el número más pequeño mayor o igual a un número específico. Con consultas de modificación.}

\begin{lstlisting}[language=C++]
void update(int v, int tl, int tr, int pos, int new_val) {
   t[v].erase(t[v].find(a[pos]));t[v].insert(new_val);
   if (tl != tr) {
      int tm = (tl + tr) / 2;
      if (pos <= tm) update(v*2, tl, tm, pos, new_val);
      else update(v*2+1, tm+1, tr, pos, new_val);
	} else { a[pos] = new_val; }
}
\end{lstlisting}

%\subsection{Java}

%\subsubsection{Hallar el máximo y el número de veces que aparece}

%\subsubsection{Buscando el primer elemento mayor que una cantidad dada}
%\begin{lstlisting}[language=C++]
%public int get_first(int indexTree, int x, int l, int r, int left, int rigth) {
%   if (left > r || rigth < l) return -1;
%   if (l <= left && rigth <= r) {
%      if (rt[indexTree] <= x) return -1;
%      while (left != rigth) {
%         int MID = (left + rigth) / 2;
%         if(rt[2*indexTree]> x){indexTree=2*indexTree;rigth = MID;} 
%         else { indexTree=2*indexTree+1;left=MID+1;}
%      }
%      return left;
%   }
%   int MID = (left + rigth) / 2;
%   int rs = get_first(2 * indexTree, x, l, r, left, MID);
%   if (rs != -1) return rs;
%   return get_first(2 * indexTree + 1,x, l, r, MID + 1, rigth);
%}
%\end{lstlisting}

%\subsubsection{Contando el número de ceros, buscando el $k$-th cero}

%\subsubsection{Encontrar subsubrangos con la suma máxima}

%\subsubsection{Encuentra el número más pequeño mayor o igual a un número específico. Sin consultas de modificación.}

%\subsubsection{Encuentra el número más pequeño mayor o igual a un número específico. Con consultas de modificación.}