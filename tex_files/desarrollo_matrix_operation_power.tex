La potencia de $A^k$ de una matrix $A$ es definida si $A$ es una matriz cuadrada. La definición esta basada en la multiplicación de matrices:

$$
A^k = \underset{\text{k veces}}{\underbrace{A \cdot A \cdot A \dots A}}
$$

Por ejemplo:

$$
\begin{bmatrix}
	2 & 5  \\
	1 & 4 
\end{bmatrix}^3  =
\begin{bmatrix}
	2 & 5  \\
	1 & 4 
\end{bmatrix} \cdot
\begin{bmatrix}
	2 & 5  \\
	1 & 4 
\end{bmatrix} \cdot
\begin{bmatrix}
	2 & 5  \\
	1 & 4 
\end{bmatrix}
=
\begin{bmatrix}
	48 & 165  \\
	33 & 114
\end{bmatrix}$$ 

En correspondencia $A^0$ es una matriz identidad. Por ejemplo:

$$\begin{bmatrix}
	2 & 5  \\
	1 & 4 
\end{bmatrix}^0  =
\begin{bmatrix}
	1 & 0  \\
	0 & 1 
\end{bmatrix}$$



La matrix $A^k$ puede ser eficientemente calculada en O($n^3\log k$) usando el algoritmo de exponenciación binaria. Por Ejemplo

$$\begin{bmatrix}
	2 & 5  \\
	1 & 4 
\end{bmatrix}^8  =
\begin{bmatrix}
	2 & 5  \\
	1 & 4
\end{bmatrix}^4 \cdot
\begin{bmatrix}
	2 & 5  \\
	1 & 4
\end{bmatrix}^4$$

