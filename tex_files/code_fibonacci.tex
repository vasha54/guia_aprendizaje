\subsection{C++}

\subsubsection{Variante recursiva}

\begin{lstlisting}[language=C++]
int fibonacci(int n){
   if(n<2) return n;
   else return fibonacci(n-1)+fibonacci(n-2);	
}
\end{lstlisting}

\subsubsection{Variante iterativa}

\begin{lstlisting}[language=C++]
	
int fibonacci[1000];

// Complejidad temporal y espacial O(n)
void buildFibonacci(){ 
   fibonacci[0]=0; fibonacci[1]=1;
   for(int i=2;i<1000;i++) fibonacci[i]=fibonacci[i-1]+fibonacci[i-2];	
}

// Complejidad temporal y espacial O(n) y O(1) respectivamente
int fibonacci(int n){ 
   int a=0, b=1,c=n;
   for(int i=2;i<=n;i++){ c=a+b; a=b; b=c; }
   return c;
}
\end{lstlisting}

\subsubsection{Variante matricial}

\begin{lstlisting}[language=C++]
int fibonacci(int n){
   int h,i,j,k,aux; h=i=1; j=k=0;
   while(n>0){
      if(n%2!=0){
         aux=h*j; j=h*i+j*k+aux; i=i*k+aux; 
      }
      aux=h*h; h=2*h*k+aux; k=k*k+aux; n=n/2;
    }
    return j;
}
\end{lstlisting}

\subsubsection{Función}

\begin{lstlisting}[language=C++]
double fibonacci(int n){
   double root=sqrt(5);
   double fib=((1/root)*(pow((1+root)/2,n))- (1/root)*(pow((1-root)/2,n)));
   return fib;
}
\end{lstlisting}

\subsection{Java}

\subsubsection{Variante recursiva}

\begin{lstlisting}[language=Java]
public int fibonacci(int n){
   if(n<2) return n;
   else return fibonacci(n-1)+fibonacci(n-2);	
}
\end{lstlisting}

\subsubsection{Variante iterativa}

\begin{lstlisting}[language=Java]
	
int [] fibonacci = new int[1000];
	
// Complejidad temporal y espacial O(n)
public void buildFibonacci(){ 
   fibonacci[0]=0; fibonacci[1]=1;
   for(int i=2;i<1000;i++) fibonacci[i]=fibonacci[i-1]+fibonacci[i-2];	
}
	
// Complejidad temporal y espacial O(n) y O(1) respectivamente
public int fibonacci(int n){ 
   int a=0, b=1,c=n;
   for(int i=2;i<n;i++){ c=a+b; a=b; b=c; }
   return c;
}
\end{lstlisting}

\subsubsection{Variante matricial}

\begin{lstlisting}[language=C++]
public int fibonacci(int n){
   int h,i,j,k,aux; h=i=1; j=k=0;
   while(n>0){
      if(n%2!=0){
         aux=h*j; j=h*i+j*k+aux;i=i*k+aux; 
      }
      aux=h*h; h=2*h*k+aux; k=k*k+aux; n=n/2;
   }
   return j;
}
\end{lstlisting}

\subsubsection{Función}

\begin{lstlisting}[language=C++]
public double fibonacci(int n){
   double root=Math.sqrt(5);
   double fib=((1/root)*(Math.pow((1+root)/2,n))- (1/root)*(Math.pow((1-root)/2,n)));
   return fib;
}
\end{lstlisting}