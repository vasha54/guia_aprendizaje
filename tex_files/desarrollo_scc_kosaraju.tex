El algoritmo descrito fue sugerido de forma independiente por Kosaraju y Sharir en 1979. Este es un algoritmo fácil de implementar basado en dos series de búsqueda en profundidad.

En el primer paso del algoritmo estamos haciendo una secuencia de primeras búsquedas en profundidad, visitando todo el grafo. Comenzamos en cada vértice del grafo y ejecutamos una búsqueda en profundidad desde cada vértice no visitado. Para cada vértice estamos haciendo un seguimiento del tiempo de salida $tout[v]$. Estos tiempos de salida tienen un papel clave en un algoritmo y este papel se expresa en el siguiente teorema.

Primero, hagamos anotaciones: definamos el tiempo de salida $tout[C]$ de la componente fuertemente conexa $C$ como máximo de valores $tout[v]$ Por todos $v \in C$. Además, durante la demostración del teorema mencionaremos los tiempos de entrada $in[v]$ en cada vértice y de la misma manera considerar $tin[C]$ para cada componente fuertemente conectado $C$ como mínimo de valores $tin[v]$ Por todos $v \in C$.

\textbf{teorema:} Dejar $C$ y $C'$ son dos componentes diferentes fuertemente conexa y hay una arista $(C, C')$ en un grafo de condensación entre estos dos vértices. Entonces $out[C] > out[C']$.

Hay dos casos diferentes principales en la prueba, dependiendo de qué componente visitará primero la 
búsqueda en profundidad primero, es decir, dependiendo de la diferencia entre $tin[C]$ y $tin[C']$ :

\begin{itemize}
	\item El componente $C$ se alcanzó primero. Significa que la búsqueda en profundidad primero viene 
	en algún vértice $v$ de componente $C$ en algún momento, pero todos los demás vértices de los 
	componentes $C$ y $C'$ no fueron visitados todavía. Por condición hay una arista $(C, C')$ en un 
	grafo de condensación, por lo que no solo el componente completo $C$ es accesible desde $v$ pero 
	todo el componente $C'$ es accesible también. Significa que la primera búsqueda en profundidad se 
	ejecuta desde el vértice $v$ visitará todos los vértices de los componentes $C$ y $C'$, así serán 
	descendientes para $v$ en un primer árbol de búsqueda en profundidad, es decir, para cada vértice $u \in C \cup C', u \ne v$ tenemos eso $tout[v] > tout[u]$, como decíamos.
	
	\item Suponga que ese componente $C'$ fue visitado primero. De manera similar, la primera búsqueda 
	en profundidad llega en algún vértice $v$ de componente $C'$ en algún momento, pero todos los demás 
	vértices de los componentes $C$ y $C'$ no fueron visitados todavía. Pero por condición hay una 
	ventaja. $(C, C')$ en el grafo de condensación, por lo tanto, debido a la propiedad acíclica del 
	grafo de condensación, no hay un camino de regreso desde $C'$ a $C$ , es decir, búsqueda en 
	profundidad desde el vértice $v$ no llegará a los vértices de $C$. Significa que los vértices de $C$ será visitado por profundidad primera búsqueda más tarde, por lo que $tout[C] > tout[C']$. Esto completa la prueba.
\end{itemize}

El teorema probado es la base del algoritmo para encontrar componentes fuertemente conectados. De ello 
se deduce que cualquier arista $(C, C')$ en el gráfico de condensación proviene de un componente con un 
valor mayor de $tout$ al componente con un valor menor.

Si ordenamos todos los vértices $v \in V$ en orden decreciente de su tiempo de salida $tout[v]$ entonces el primer vértice $u$ va a ser un vértice que pertenece al componente \emph{raiz} fuertemente 
conectado, es decir, un vértice que no tiene aristas entrantes en el gráfico de condensación. Ahora 
queremos ejecutar dicha búsqueda desde este vértice $u$ de modo que visitará todos los vértices en 
este componente fuertemente conectado, pero no en otros; al hacerlo, podemos seleccionar gradualmente 
todos los componentes fuertemente conectados: eliminemos todos los vértices correspondientes al primer 
componente seleccionado, y luego encontremos un vértice con el mayor valor de $tout$ y ejecutar esta 
búsqueda desde allí, y así sucesivamente.

Consideremos el grafo transpuesto $G^T$ , es decir, grafo recibido de $G$ invirtiendo la dirección 
de cada arista. Obviamente, este grafo tendrá las mismas componentes fuertemente conexa que el
grafo inicial. Además, el grafo de condensación $G^{SCC}$ también se transpondrá. Significa que no 
habrá aristas desde nuestro componente \emph{raíz} a otros componentes.

Por lo tanto, para visitar todo el componente \emph{raíz} fuertemente conexo, que contiene el vértice $v$, es suficiente para ejecutar la búsqueda desde el vértice $v$ en grafo $G^T$. Esta búsqueda visitará todos los vértices de esta componente fuertemente conexa y solo ellos. Como se mencionó 
anteriormente, podemos eliminar estos vértices del grafo y encontrar el siguiente vértice con un 
valor máximo de $tout[v]$ y ejecutar la búsqueda en el grafo transpuesto desde él, y así 
sucesivamente.

Por lo tanto, construimos el siguiente algoritmo para seleccionar componentes fuertemente conexa:

\begin{enumerate}
	\item Ejecutar secuencia de profundidad primera búsqueda de grafo $G$ que devolverá vértices con 
	el aumento del tiempo de salida $tout$, es decir, alguna lista $order$.
	
	\item Construir grafo transpuesto $G^T$ . Ejecute una serie de primeras búsquedas en profundidad 
	(amplitud) en el orden determinado por la lista $order$ (para ser exactos en orden inverso, es 
	decir, en orden decreciente de tiempos de salida). Cada conjunto de vértices, alcanzado después de 
	la siguiente búsqueda, será el próximo componente fuertemente conectado.
\end{enumerate}


Finalmente, es apropiado mencionar aquí la ordenación topológica. En primer lugar, el paso 1 del 
algoritmo representa un tipo de grafo topológico inverso $G$ (en realidad, esto es exactamente lo que 
significa ordenar los vértices por tiempo de salida). En segundo lugar, el esquema del algoritmo genera 
componentes fuertemente conectados por orden decreciente de sus tiempos de salida, por lo que genera 
componentes (vértices del grafo de condensación) en orden de clasificación topológico.
