\subsection{Estructura de Datos}
Una estructura de datos es una forma de organizar un conjunto de datos elementales con el objetivo
de facilitar su manipulación. Un dato elemental es la mínima información que se tiene en un sistema.
Una estructura de datos define la organización e interrelación de estos y un conjunto de operaciones
que se pueden realizar sobre ellos. Cada estructura ofrece ventajas y desventajas en relación a la
simplicidad y eficiencia para la realización de cada operación. De está forma, la elección de la
estructura de datos apropiada para cada problema depende de factores como la frecuencia y el orden
en que se realiza cada operación sobre los datos.

\subsection{Estructuras Dinámicas Lineales}
Las estructuras dinámicas son aquellas que el tamaño no está definido y puede cambiar en la
ejecución del algoritmo. Estas estructuras suelen ser eficaces y efectivas en la solución de problemas
complejos. Su tamaño puede reducir o incrementar en la ejecución del algoritmo.

Entre las estructuras dinámicas lineales están:
\begin{itemize}
	\item Vector
	\item Pila
	\item Cola
\end{itemize}