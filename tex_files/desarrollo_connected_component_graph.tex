Para resolver el problema, podemos usar la búsqueda primero en profundidad o la búsqueda primero en  amplitud. De hecho, haremos una serie de rondas de DFS: la primera ronda comenzará desde el primer nodo  y todos los nodos en el primer componente conectado serán atravesados (encontrados). Luego encontramos  el primer nodo no visitado de los nodos restantes y ejecutamos la primera búsqueda en  profundidad en él, encontrando así un segundo componente conectado. Y así sucesivamente, hasta que se  visiten todos los nodos.

Las funciones profundamente recursivas son en general malas. Cada llamada recursiva requerirá un poco de memoria en la pila y, por defecto, los programas solo tienen una cantidad limitada de espacio en la pila. Entonces, cuando realiza un DFS recursivo sobre un gráfico conectado con millones de nodos, es posible que se encuentre con desbordamientos de pila. Siempre es posible convertir un programa recursivo en un programa iterativo, manteniendo manualmente una estructura de datos de pila. Dado que esta estructura de datos se asigna en el montón, no se producirá ningún desbordamiento de pila.
