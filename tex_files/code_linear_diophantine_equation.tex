\subsection{C++}

\subsubsection{Encontrar una solución}
\begin{lstlisting}[language=C++]
int gcd(int a, int b, int& x, int& y) {
   if (b == 0){x = 1; y = 0; return a;}
   int x1, y1;
   int d = gcd(b,a%b,x1,y1);
   x = y1; y = x1 - y1 * (a / b);
   return d;
}

bool find_any_solution(int a, int b, int c, int &x0, int &y0, int &g) {
   g = gcd(abs(a), abs(b), x0, y0);
   if (c % g){return false;}
   x0 *= c / g;
   y0 *= c / g;
   if (a < 0) x0 = -x0;
   if (b < 0) y0 = -y0;
   return true;
}
\end{lstlisting}

\subsubsection{Encontrar el número de soluciones y las soluciones en un intervalo dado}
\begin{lstlisting}[language=C++]
void shift_solution(int & x, int & y, int a, int b, int cnt) {
   x += cnt * b; y -= cnt * a;
}

int find_all_solutions(int a, int b, int c, int minx, int maxx, int miny, int maxy) {
   int x, y, g;
   if (!find_any_solution(a, b, c, x, y, g)) return 0;
   a /= g; b /= g;
   int sign_a = a > 0 ? +1 : -1;
   int sign_b = b > 0 ? +1 : -1;
   shift_solution(x, y, a, b, (minx - x) / b);
   if (x < minx) shift_solution(x, y, a, b, sign_b);
   if (x > maxx) return 0;
   int lx1 = x;
   shift_solution(x, y, a, b, (maxx - x) / b);
   if (x > maxx) shift_solution(x, y, a, b, -sign_b);
   int rx1 = x;
   shift_solution(x, y, a, b, -(miny - y) / a);
   if (y < miny) shift_solution(x, y, a, b, -sign_a);
   if (y > maxy) return 0;
   int lx2 = x;
   shift_solution(x, y, a, b, -(maxy - y) / a);
   if (y > maxy) shift_solution(x, y, a, b, sign_a);
   int rx2 = x;
   if (lx2 > rx2) swap(lx2, rx2);
   int lx = max(lx1, lx2);
   int rx = min(rx1, rx2);
   if (lx > rx) return 0;
   return (rx - lx) / abs(b) + 1;
}
\end{lstlisting}


\subsection{Java}

\subsubsection{Encontrar una solución}
\begin{lstlisting}[language=Java]
// retorna { gcd(a,b), x, y } tal que gcd(a,b) = a*x + b*y
public static long[] gcd_recursive(long a, long b) {
   if (b == 0) return a > 0 ? new long[]{a, 1, 0} : new long[]{-a, -1, 0};
   long[] r = gcd_recursive(b, a % b);
   return new long[]{r[0], r[2], r[1] - a / b * r[2]};
}

//En el ArrayList answer retorna {gcd(a,b),x0,y0}
public static boolean find_any_solution(long a, long b, long c,ArrayList<Long> answer) {
   long [] r = gcd_recursive(Math.abs(a), Math.abs(b));
   long g= r[0];
   long x0=r[1];
   long y0=r[2];
   if (c % g!=0) { return false;}
   x0 *= c / g;
   y0 *= c / g;
   if (a < 0) x0 = -x0;
   if (b < 0) y0 = -y0;
   answer = new ArrayList<>();
   answer.add(g); answer.add(x0);answer.add(y0);
   return true;
}
\end{lstlisting}


\subsubsection{Encontrar el número de soluciones y las soluciones en un intervalo dado}
\begin{lstlisting}[language=Java]
public static long [] shift_solution(long x, long y, long a, long b, long cnt) {
   x += cnt * b;
   y -= cnt * a;
   return new long[]{x,y};
}

public static long find_all_solutions(long a, long b, long c, long minx, long maxx, long miny, long maxy) {
   long x, y, g;
   ArrayList<Long> t = new ArrayList<>();
   if (!find_any_solution(a, b, c, t)) return 0L;
   g = t.get(0); x = t.get(1); y = t.get(1);
   a /= g;
   b /= g;
   long sign_a = a > 0 ? +1 : -1;
   long sign_b = b > 0 ? +1 : -1;
   long [] r =shift_solution(x, y, a, b, (minx - x) / b);
   x = r[0]; y = r[1];
   if (x < minx){
      r= shift_solution(x, y, a, b, sign_b);
      x = r[0]; y = r[1];
   }
   if (x > maxx) return 0;
   long lx1 = x;
   r = shift_solution(x, y, a, b, (maxx - x) / b);
   x = r[0]; y = r[1];
   if (x > maxx){
      r=  shift_solution(x, y, a, b, -sign_b);
      x = r[0]; y = r[1];
   }
   long rx1 = x;
   r = shift_solution(x, y, a, b, -(miny - y) / a);
   x = r[0]; y = r[1];
   if (y < miny){
      r=  shift_solution(x, y, a, b, -sign_a);
      x = r[0]; y = r[1];
   }
   if (y > maxy) return 0;
   long lx2 = x;
   r = shift_solution(x, y, a, b, -(maxy - y) / a);
   x = r[0]; y = r[1];
   if (y > maxy){
      r=  shift_solution(x, y, a, b, sign_a);
      x = r[0]; y = r[1];
   }
   long rx2 = x;
   if (lx2 > rx2){
      long tmp = lx2; lx2 = rx2; rx2 = tmp;
   }
   long lx = Math.max(lx1, lx2);
   long rx = Math.min(rx1, rx2);
   if (lx > rx) return 0;
   return (rx - lx) / Math.abs(b) + 1;
}
\end{lstlisting}
