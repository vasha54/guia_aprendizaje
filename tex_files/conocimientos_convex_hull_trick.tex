\subsection{Producto punto}

En matemáticas, el producto escalar, también conocido como producto interno o producto punto, es una operación algebraica que toma dos vectores y retorna un escalar, y que satisface ciertas condiciones. El producto escalar se define como la suma de los productos componente por componente de los dos vectores.



\subsection{Envoltura Convexa (\emph{Convex Hull})}

En matemáticas se define la envolvente convexa, envoltura convexa o cápsula convexa de un conjunto de puntos $X$ de dimensión $n$ como la intersección de todos los conjuntos convexos que contienen a $X$

En el caso particular de puntos en un plano, si no todos los puntos están alineados, entonces su envolvente convexa corresponde a un polígono convexo cuyos vértices son algunos de los puntos del conjunto inicial de puntos.

Una forma intuitiva de ver la envolvente convexa de un conjunto de puntos en el plano, es imaginar una banda elástica estirada que los encierra a todos. Cuando se libere la banda elástica tomará la forma de la envolvente convexa.

\subsection{Vector normal}
En geometría un vector normal a una cantidad geometrica (línea, curva, superficie, etc) es un vector de un espacio de producto escalar que contiene tanto la entidad geométrica como al vector normal, que tiene la propiedad de ser ortogonal a todos los vectores tangentes a la entidad geométrica  