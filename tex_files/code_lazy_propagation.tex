Vamos a utilizar algunas implementaciones de soluciones para comentar las adecuaciones.

\subsection{C++}

\begin{lstlisting}[language=C++]
#include <iostream>
#include <algorithm>
#include <math.h>
#define MID (left+right)/2
#define MOD  1000000007
#define MAX 1000010
#define MAXTREE (MAX << 2)

using namespace std;

struct Node{
   int flowers;
   bool lazy; /* va indicar si determinado nodo en 
	el arbol tiene o no propagacion el tipo de dato 
	puede variar segun la necesidad o el enfoque 
	que se de la propagacion*/
};

Node tree[MAXTREE];

void buildRangeTree(int treeIndex, int left, int right){
	
  tree[treeIndex].lazy=false; /*inicialmente cada 
                              nodo no tiene propagacion*/
  tree[treeIndex].flowers=0;
	
  if (left == right) {
     tree[treeIndex].flowers=0;return;
  }
	
  buildRangeTree(2*treeIndex,left,MID);
  buildRangeTree(2*treeIndex+1,MID+1,right);
}

/*Aqui cambia la actualizacion antes era una posicion 
ahora es un rango (i,j)*/
void updateLazyRangeTree(int treeIndex, int left, int right, int i, int j){
   if (left > right || left > j || right < i) return;

   /*Si el rango que representa ese nodo o vertice 
   en el arbol esta contenido en el rango de la actualizacion(i,j)  
   detengo la actualizacion y lo marco como que existe una propagacion 
   pendiente
   */	
   if (i <= left && right <= j){  
     tree[treeIndex].lazy=true; tree[treeIndex].flowers++; return;
   }

   /*El rango de actualizacion(i,j) es un subrango del rango que 
   representa el nodo o vertice del arbol donde estoy parado que 
   tiene propagaciones pendientes por tanto propago primero 
   dichas actualizaciones pendientes y luego sigo el proceso de
   actualizacion en rango. Una vez propagado la actualizaciones
   pendientes ya el nodo o vertice del arbol deja de tener propagaciones
   pendientes. Funcion push*/	
   if(tree[treeIndex].lazy==true){
      tree[2*treeIndex].lazy=tree[treeIndex].lazy;
      tree[2*treeIndex+1].lazy=tree[treeIndex].lazy;
      tree[2*treeIndex].flowers+=tree[treeIndex].flowers;
      tree[2*treeIndex+1].flowers+=tree[treeIndex].flowers;
      tree[treeIndex].lazy = false;
      tree[treeIndex].flowers = 0;
   }
	
   updateLazyRangeTree(2 * treeIndex , left, MID, i, j);
   updateLazyRangeTree(2 * treeIndex + 1, MID + 1, right, i, j);
}

int queryLazyRangeTree(int treeIndex, int left, int right, int pos){
	
   if(left==right){ 
      int answer = tree[treeIndex].flowers; tree[treeIndex].flowers=0;
      return answer;
   }
	
   /*El rango de consulta(left,rigth) es un subrango del rango que 
   representa el nodo o vertice del arbol donde estoy parado que 
   tiene propagaciones pendientes por tanto propago primero 
   dichas actualizaciones pendientes y luego sigo el proceso de
   consulta en rango. Una vez propagado la actualizaciones
   pendientes ya el nodo o vertice del arbol deja de tener propagaciones
   pendientes. Funcion push*/	
   if (tree[treeIndex].lazy == true){
      tree[2*treeIndex].lazy=tree[treeIndex].lazy;
      tree[2*treeIndex+1].lazy=tree[treeIndex].lazy;
      tree[2*treeIndex].flowers+=tree[treeIndex].flowers;
      tree[2*treeIndex+1].flowers+=tree[treeIndex].flowers;
      tree[treeIndex].lazy = false;
      tree[treeIndex].flowers = 0;
   }
   if (pos > MID)
      return queryLazyRangeTree(2*treeIndex+1,MID+1,right,pos);
   if (pos <= MID)
      return queryLazyRangeTree(2*treeIndex,left,MID,pos);
}

\end{lstlisting}


\subsection{Java}

\begin{lstlisting}[language=Java]
private class RangeTreeLP {
   private int[] tree;
   private boolean[] lz;
	
   public RangeTreeLP(int n) {
      tree = new int[n << 2]; lz = new boolean[n << 2];
   }
	
   public void buildRangeTree(int treeIndex, int left, int right) {
      /*
      * inicialmente cada nodo no tiene propagacion
      */
      lz[treeIndex] = false; tree[treeIndex] = 0;
      
      if (left == right) { tree[treeIndex] = 0; return; }
      
      int MID = (left + right) / 2;
      buildRangeTree(2 * treeIndex, left, MID);
      buildRangeTree(2 * treeIndex + 1, MID + 1, right);
   }
	
   /*
   * Aqui cambia la actualizacion antes era una posicion 
   * ahora es un rango (i,j)
   */
   public void updateLazyRangeTree(int treeIndex,int left,int right,int i,int j){
      if (left > right || left > j || right < i) return;
		
      /*
      * Si el rango que representa ese nodo o vertice en el arbol esta 
      * contenido en el rango de la actualizacion(i,j) detengo la 
      * actualizacion y lo marco como que existe una propagacion pendiente
      */
      if (i <= left && right <= j) {
         lz[treeIndex] = true; tree[treeIndex]++; return;
      }
		
      /*
      * El rango de actualizacion(i,j) es un subrango del rango que 
      * representa el nodo o vertice del arbol donde estoy parado que 
      * tiene propagaciones pendientes por tanto propago primero dichas 
      * actualizaciones pendientes y luego sigo el proceso de 
      * actualizacion en rango. Una vez propagado la actualizaciones 
      * pendientes ya el nodo o vertice del arbol deja de tener propagaciones 
      * pendientes. Funcion push
      */
      if (lz[treeIndex] == true) {
         lz[2 * treeIndex] = lz[treeIndex]; 
         lz[2 * treeIndex + 1] = lz[treeIndex];
         tree[2 * treeIndex] += tree[treeIndex]; 
         tree[2 * treeIndex + 1] += tree[treeIndex];
         lz[treeIndex] = false; tree[treeIndex] = 0;
      }
      int MID = (left + right) / 2;
      
      updateLazyRangeTree(2 * treeIndex, left, MID, i, j);
      updateLazyRangeTree(2 * treeIndex + 1, MID + 1, right, i, j);
   }
	
   public int queryLazyRangeTree(int treeIndex,int left,int right,int pos){
      if (left == right){
         int answer = tree[treeIndex];
         tree[treeIndex] = 0; return answer;
      }
      
      /*
      * El rango de consulta(left,rigth) es un subrango del rango 
      * que representa el nodo o vertice del arbol donde estoy 
      * parado que tiene propagaciones pendientes por tanto propago 
      * primero dichas actualizaciones pendientes y luego sigo el 
      * proceso de consulta en rango. Una vez propagado la 
      * actualizaciones pendientes ya el nodo o vertice del arbol 
      * deja de tener propagaciones pendientes. Funcion push
      */
      if(lz[treeIndex] == true){
         lz[2 * treeIndex] = lz[treeIndex];
         lz[2 * treeIndex + 1] = lz[treeIndex];
         tree[2 * treeIndex] += tree[treeIndex];
         tree[2 * treeIndex + 1] += tree[treeIndex];
         lz[treeIndex] = false;
         tree[treeIndex] = 0;
      }
		
      int MID = (left + right) / 2;
      if (pos > MID)
         return queryLazyRangeTree(2 * treeIndex + 1, MID + 1, right, pos);
      if (pos <= MID)
         return queryLazyRangeTree(2 * treeIndex, left, MID, pos);
  }
}
\end{lstlisting} 