El ordenamiento por inserción (insertion sort en inglés) es una manera muy natural de ordenar para un ser humano, y puede usarse fácilmente para ordenar un mazo de cartas numeradas en forma arbitraria. Inicialmente se tiene un solo elemento, que obviamente es un conjunto ordenado. Después, cuando hay k elementos ordenados de menor a mayor, se toma el elemento k+1 y se compara con todos los elementos ya ordenados, deteniéndose cuando se encuentra un elemento menor (todos los elementos mayores han sido desplazados una posición a la derecha) o cuando ya no se encuentran elementos (todos los elementos fueron desplazados y este es el más pequeño). En este punto se inserta el elemento k+1 debiendo desplazarse los demás elementos.