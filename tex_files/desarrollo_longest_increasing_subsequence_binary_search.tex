Este enfoque de solución no se apoya en los enfoques anteriores para dar una solución como si sucedió con los enfoques anteriores. Vamos a partir de la idea que en la LIS de una colección puede ser toda la colección en sí, por ejemplo una colección donde todos los elementos son distintos y casualmente están ordenados de forma creciente. Basados en esta idea vamos a construir una colección de depositos $S$ donde en $S[i]$ se almacenará el entero más pequeño que termina una secuencia creciente de longitud $i$. 

La idea principal del enfoque es simular el proceso de encontrar una subsecuencia manteniendo una lista de \emph{depósitos} donde cada depósito representa una subsecuencia válida. Inicialmente, comenzamos con una lista vacía y recorremos los números del vector de entrada de izquierda a derecha.

Para cada número en la colección, realizamos los siguientes pasos:

\begin{itemize}
	\item Si el número es mayor que el último elemento del último depósito (es decir, el elemento más grande en la subsecuencia actual), agregamos el número al final de la lista. Esto indica que hemos encontrado una subsecuencia más larga.
	\item De lo contrario, realizamos una búsqueda binaria en la lista de depósitos para encontrar el elemento más pequeño que sea mayor o igual al número actual. Este paso nos ayuda a mantener la propiedad de aumentar los elementos en los depósitos.
	\item Una vez que encontramos la posición a actualizar, reemplazamos ese elemento con el número actual. Esto mantiene los despositos ordenados y garantiza que tengamos potencial para una subsecuencia más larga en el futuro.
\end{itemize}

Ilustremos esto con la ayuda de un ejemplo. Los siguientes son los pasos seguidos por el algoritmo para un arreglo de enteros $A=\{2, 6, 3, 4, 1, 2, 9, 5, 8\}$:

\begin{enumerate}
	\item Inicializamos la colección de depósito S vacio. $S= \{\}$
	\item Insertando 2  S = \{2\} - Nuevo LIS más grande.
	\item Insertando 6  S = \{2, 6\} - Nuevo LIS más grande.
	\item Insertanto 3  S = \{2, 3\} - Remplaza 6 con 3.
	\item Insertanto 4  S = \{2, 3, 4\} - Nuevo LIS más grande.
	\item Insertanto 1  S = \{1, 3, 4\} - Remplaza 2 con 1
	\item Insertanto 2  S = \{1, 2, 4\} - Remplaza 3 con 2
	\item Insertanto 9  S = \{1, 2, 4, 9\} - Nuevo LIS más grande.
	\item Insertanto 5  S = \{1, 2, 4, 5\} - Remplaza 9 con 5
	\item Insertanto 8  S = \{1, 2, 4, 5, 8\} - Nuevo LIS más grande.
\end{enumerate}

Entonces, la longitud del LIS es 5 (el tamaño de S). Tenga en cuenta que aquí $S[i]$ se define como el entero más pequeño que termina una secuencia creciente de longitud $i$. Por lo tanto, $S$ no representa una secuencia real, pero el tamaño de $S$ representa la longitud LIS.