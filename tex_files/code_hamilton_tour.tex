\subsection{C++}

\subsubsection{Verificar si existe ciclo (Algoritmo retroceso)}
\begin{lstlisting}[language=C++]
#define V 5 //cantidad de vertices

void printSolution(int path[]); 

bool isSafe(int v, bool graph[V][V], int path[], int pos) { 
   if (graph [path[pos - 1]][ v ] == 0) return false; 
   for (int i = 0; i < pos; i++) if (path[i] == v) return false; 
   return true; 
} 

bool hamCycleUtil(bool graph[V][V], int path[], int pos){ 
   if (pos == V) { 
      if (graph[path[pos - 1]][path[0]] == 1) return true; 
      else return false; 
   } 
	
   for (int v = 1; v < V; v++){ 
      if (isSafe(v, graph, path, pos)){ 
         path[pos] = v; 
         if (hamCycleUtil (graph, path, pos + 1) == true) return true; 
         path[pos] = -1; 
      } 
   } 
   return false; 
} 

//Se invoca este metodo pasando la matriz de adyacencia del grafo
bool hamCycle(bool graph[V][V]){ 
   int *path = new int[V]; 
   for (int i = 0; i < V; i++) path[i] = -1; 
   
   path[0] = 0; 
   if (hamCycleUtil(graph, path, 1) == false ){ 
      cout << "\nNo existe una solucion"; return false; 
   } 
   printSolution(path); return true; 
} 


void printSolution(int path[]) { 
   cout << "Existe solucion: El siguiente es un ciclo de Hamilton \n"; 
   for (int i = 0; i < V; i++) cout << path[i] << " "; 
   cout << path[0] << " ";  cout << endl;
} 
\end{lstlisting}

\subsubsection{Imprimir todos los posibles ciclos de Hamilton grafo no dirigido (Algoritmo retroceso)}
\begin{lstlisting}[language=C++]
#define V 6 //cantidad de vertices
bool hasCycle;

bool isSafe(int v, int graph[][V], vector<int> path, int pos){
   if (graph[path[pos - 1]][v] == 0) return false;
   for (int i = 0; i < pos; i++) if (path[i] == v) return false;
   return true;
}

void FindHamCycle(int graph[][V], int pos, vector<int> path, bool visited[], int N) {
   if (pos == N) {
      if (graph[path[path.size() - 1]][path[0]] != 0) {
         path.push_back(0);
         for (int i = 0; i < path.size(); i++) cout << path[i] << " ";
         cout << endl;
         path.pop_back();
         hasCycle = true;
      }
      return;
   }
   for (int v = 0; v < N; v++) {
      if (isSafe(v, graph, path, pos) && !visited[v]) {
         path.push_back(v); visited[v] = true;
         FindHamCycle(graph, pos + 1, path, visited, N);
         visited[v] = false; path.pop_back();
      }
   }
}

void hamCycle(int graph[][V], int N){
   hasCycle = false;
   vector<int> path;
   path.push_back(0);
   bool visited[N];
   for (int i = 0; i < N; i++) visited[i] = false;
   visited[0] = true;
   FindHamCycle(graph, 1, path, visited, N);
   if (!hasCycle) {
     cout << "No existe ciclo hamiltoniano " << endl; return;
   }
}
\end{lstlisting}

\subsubsection{Verificar si existe camino (Programación Dinámica)}
\begin{lstlisting}[language=C++]
bool Hamiltonian_path( vector<vector<int> >& adj, int N){
   int dp[N][(1 << N)];
   memset(dp, 0, sizeof dp);
   
   for (int i = 0; i < N; i++) dp[i][(1 << i)] = true;
	
   for (int i = 0; i < (1 << N); i++) {
      for (int j = 0; j < N; j++) {
         if (i & (1 << j)) {
            for (int k = 0; k < N; k++) {
                if (i&(1<<k) && adj[k][j] && j!=k && dp[k][i ^ (1 << j)]){
                	dp[j][i] = true; break;
                }
            }
         }
      }
   }
	
   for (int i = 0; i < N; i++) {
      if (dp[i][(1 << N) - 1]) return true;
   }
   return false;
}
\end{lstlisting}

\subsubsection{Calcular todos los caminos de hamilton del nodo 1 al nodo N en grafo dirigido}
\begin{lstlisting}[language=C++]
int findAllPaths( int N, vector<vector<int> >& graph){
   int dp[N][(1 << N)];
   memset(dp, 0, sizeof dp);
   dp[0][1] = 1;
   for (int i = 2; i < (1 << N); i++) {
      // Si el primer vertice esta ausente
      if ((i & (1 << 0)) == 0) continue;
      
      // Considere solo los subconjuntos completos
      if ((i & (1 << (N - 1))) && i != ((1 << N) - 1)) continue;
		
      // Elige la ciudad final
      for (int end = 0; end < N; end++) {
        // Si esta ciudad no esta en el subconjunto
        if (i & (1 << end) == 0) continue;
        // Conjunto sin la ciudad final
        int prev = i - (1 << end);
        // Consultar por las ciudades adyacentes.
        for (int it : graph[end]) {
            if ((i & (1 << it))) {
               dp[end][i] += dp[it][prev];
            }
        }
      }
   }
   return dp[N - 1][(1 << N) - 1];
}
\end{lstlisting}

\subsection{Java}

\subsubsection{Verificar si existe ciclo (Algoritmo retroceso)}
\begin{lstlisting}[language=Java]
class HamiltonianCycle {
   final int V = 5; //cantidad de vertices del grafo
   int path[]; // secuencia que indica un posible ciclo
	
   boolean isSafe(int v, int graph[][], int path[], int pos){
      if (graph[path[pos - 1]][v] == 0) return false;
      for (int i = 0; i < pos; i++) if (path[i] == v) return false;
      return true;
   }
   
   boolean hamCycleUtil(int graph[][], int path[], int pos){
      if (pos == V){
         if (graph[path[pos - 1]][path[0]] == 1) return true;
         else return false;
      }
		
      for (int v = 1; v < V; v++) {
         if (isSafe(v, graph, path, pos)) {
            path[pos] = v;
            if (hamCycleUtil(graph, path, pos + 1) == true) return true;
            path[pos] = -1;
         }
       }
       return false;
   }
	
   int hamCycle(int graph[][]) {
      path = new int[V];
      for (int i = 0; i < V; i++) path[i] = -1;

      path[0] = 0;
      if (hamCycleUtil(graph, path, 1) == false) {
         System.out.println("\nNo existe ciclo"); return 0;
      }
      printSolution(path); return 1;
   }
	
   void printSolution(int path[]){
      System.out.println("Existe solucion : El siguiente" +" es un ciclo de Hamilton");
      for (int i = 0; i < V; i++) System.out.print(" " + path[i] + " ");
      System.out.println(" " + path[0] + " ");
   }
}
\end{lstlisting}

\subsubsection{Imprimir todos los posibles ciclos de Hamilton grafo no dirigido (Algoritmo retroceso)}
\begin{lstlisting}[language=Java]
import java.util.ArrayList;
public class Main {
	
   public boolean isSafe(int v,int graph[][],ArrayList<Integer> path,int pos){
      if (graph[path.get(pos - 1)][v] == 0) return false;
      for (int i = 0; i < pos; i++) if (path.get(i) == v) return false;
      return true;
   }
	
   public boolean hasCycle;
	
   void hamCycle(int graph[][]){
      hasCycle = false;
		
      ArrayList<Integer> path = new ArrayList<>();
      path.add(0);
		
      boolean[] visited = new boolean[graph.length];
		
      for (int i = 0; i < visited.length; i++) visited[i] = false;
      visited[0] = true;
		
      FindHamCycle(graph, 1, path, visited);
		
      if (!hasCycle) {
         System.out.println("No Hamiltonian Cycle" + "possible "); return;
      }
   }
	
   void FindHamCycle(int graph[][], int pos, ArrayList<Integer> path, boolean[] visited){
      if (pos == graph.length) {
         if (graph[path.get(path.size() - 1)][path.get(0)] != 0) {
            path.add(0);
            for (int i = 0; i < path.size(); i++) {
            	System.out.print(path.get(i) + " ");
            }
            System.out.println();
            path.remove(path.size() - 1);
            hasCycle = true;
         }
         return;
      }
		
      for (int v = 0; v < graph.length; v++) {
         if (isSafe(v, graph, path, pos) && !visited[v]) {
            path.add(v); visited[v] = true;
            FindHamCycle( graph, pos + 1,path, visited);
            visited[v] = false; path.remove(path.size() - 1);
         }
      }
   }
}

\end{lstlisting}

\subsubsection{Verificar si existe camino (Programación Dinámica)}
\begin{lstlisting}[language=Java]
class Main{
   static boolean Hamiltonian_path(int adj[][], int N) {
      boolean dp[][] = new boolean[N][(1 << N)];
      for(int i = 0; i < N; i++) dp[i][(1 << i)] = true;
      
      for(int i = 0; i < (1 << N); i++){
          for(int j = 0; j < N; j++){
              if ((i & (1 << j)) != 0){
              	for(int k = 0; k < N; k++){
                   if ((i & (1 << k)) != 0 &&  adj[k][j] == 1 && j != k && dp[k][i ^ (1 << j)]) {
                      dp[j][i] = true; break;
                   }
                }
              }
          }
       }
			
       for(int i = 0; i < N; i++) {
          if (dp[i][(1 << N) - 1]) return true;
       }
	   return false;
   }
}
\end{lstlisting}

\subsubsection{Calcular todos los caminos de hamilton del nodo 1 al nodo N en grafo dirigido}
\begin{lstlisting}[language=Java]
class Main {
   static int findAllPaths(int N, List<List<Integer>> graph){
      int dp[][] = new int[N][(1<<N)];
      for(int i=0;i<N;i++){
         for(int j=0;j<(1<<N);j++){
            dp[i][j]=0;
         }
      }
		
      // Inicializar para el primer vertice
      dp[0][1] = 1;
      
      // Iterar sobre todas las mascaras.
      for (int i = 2; i < (1 << N); i++) {
         // Si el primer vertice esta ausente
         if ((i & (1 << 0)) == 0) continue;
         // Considere solo los subconjuntos completos
         if ((i & (1 << (N - 1)))==1 && (i != ((1 << N) - 1))) continue;
         // Elige la ciudad final
         for (int end = 0; end < N; end++) {
            // Si esta ciudad no esta en el subconjunto
            if ((i & (1 << end)) == 0) continue;
            // Conjunto sin la ciudad final
            int prev = i - (1 << end);
            // Consultar por las ciudades adyacentes.
            for (int it : graph.get(end)) {
               if ((i & (1 << it))!=0)
                  dp[end][i] += dp[it][prev];
            }
          }
       }
       return dp[N - 1][(1 << N) - 1];
   }
}
\end{lstlisting}