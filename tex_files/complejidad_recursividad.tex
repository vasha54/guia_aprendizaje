La complejidad de algoritmos recursivos involucra la solución de una ecuación diferencial. El método mas simple es adivinar una solución y verificar si esta bien la adivinanza.

\textbf{Recurrencia matemática:} Define una función en f términos de ella misma. En el ejemplo del factorial de un numero natural, tenemos: $n! = n(n-1)$.

En el ejemplo de los números Fibonacci

$F_{n} = F_{n-1} + F_{n-2}  ;   n >= 2$  

$F_{0} = 0$

$F_{1} = 1$


(Donde n, 1, 0 son las bases de F).

Algoritmo factorial recursivo

Dado n
Factorial n*F(n-1) devolver factorial


Procedimiento F(k)
Si k=0 o 1 entonces devolver 1 caso contrario devolver k*F(k-1)


La complejidad de un algoritmo recursivo se expresa a través de una ecuación de recurrencia:


Sea T(n) = tiempo para calcular n!, entonces:


$T(n) = T(n-1) + 1 ;  si n >= 1$

$T(1) = 1$


Si resolvemos de una manera mas sencilla esta ecuación tenemos:


$T(n) = T(n-1) + 1$

$T(n) = (T(n-2)+1)+1   =   T(n-2) + 2$

$T(n) = ((T(n-3)+1)+1)+1   =   T(n-3) + 3$

$...$

$T(n) = T(n-k) + k$

$...$

$T(n) = T(1) + n-1$

$T(n) = T(0) + n = n$


Luego la complejidad del algoritmo factorial recursivo es $~O(n)$.

\subsection{Reducción por sustracción}

Si el tamaño  $n$ del problema decrece en una cantidad constante $b$ en cada llamada, se realizan a  llamadas recursivas y las operaciones correspondientes a la parte no recursiva del algoritmo toman un tiempo $O(n^k)$ entonces:

$T(n) = a T(n-b) + O(n^k)  ;  si \quad  n >= b$


La solución de esta ecuación de recurrencia es de la forma: 


\begin{math}
	T(n)=\left\{
	\begin{array}{ll}
		n^k, & si \quad a < 1\\
		n^k+1, & si \quad a = 1\\
		a^n/b, & si \quad a > 1
	\end{array}
	\right.
\end{math}



\subsection{Reducción por división}

Si el algoritmo con un problema de tamaño $n$ realiza a llamadas recursivas con subproblemas de tamaño $n/b$ y las operaciones correspondientes a la parte no recursiva, que corresponde a descomponer el problema en los subproblemas y luego combinarlas soluciones de estos, toman un tiempo O($n^k$) entonces: 

$T(n) = aT(n/b) + O(n^k)   ;   si \quad n >= b$

La solución de esta ecuación de recurrencia es de la forma:

\begin{math}
	T(n)=\left\{
	\begin{array}{ll}
		n^k, & si \quad  a < b^k\\
	    n^k \log n , & si \quad a = b^k\\
	   n^{\log_{b} a}, & si \quad a > b^k
	\end{array}
	\right.
\end{math}

\subsection{Complejidad para los algoritmos recursivos del tipo divide y venceras}
Estos algoritmos recursivos tienen asociada una ecuación de recurrencia tipo, cuya resolución
es lo que se conoce como el \emph{Teorema Maestro}.

A continuación se explica cómo interpretar dicho teorema maestro.

Aplicar el teorema maestro implica identificar diversos parámetros de un algoritmo recursivo
tipo divide y vencerás. Dichos parámetros son:

\begin{itemize}
	\item \textbf{a}: Número de llamadas recursivas que se realizan en el caso recursivo
	\item \textbf{b}: Factor por el cual se divide el tamaño del problema en cada llamada recursiva
	\item \textbf{k}: Mayor exponente de las polinomios asociados a las complejidades del caso base y de la
	parte no recursiva de la rama recursiva del algoritmo, asumiendo que en ambos casos
	se trata de complejidades polinómicas
\end{itemize}


Por tanto, el procedimiento para aplicar el teorema maestro es tal como sigue:

\begin{enumerate}
	\item Identificar cuáles son las ramas recursivas del algoritmo y cuáles corresponden al caso
 base.
    \item Identificar el número de llamadas recursivas (parámetro a) que se realizan en el caso
recursivo.
    \item Identificar la razón (parámetro b) por la cual se divide el tamaño del problema en cada
 llamada recursiva.
    \item Calcular la complejidad del caso base, como si se tratase un algoritmo iterativo normal. La
complejidad resultante, para aplicar el teorema maestro, debe ser polinómica. Sea $k$ el
grado de dicho polinomio. Si el algoritmo tiene complejidad constante, es decir, O(1), k=0, ya que $n^{0} = 1$.
    \item El teorema maestro asume que las complejidades del caso base y de la parte iterativa del
caso recursivo son polinómicas y poseen el mismo grado. Por tanto, tomaremos como
dicho grado al mayor de los grados de los polinomios asociados a las complejidades del
caso base y de la parte iterativa del caso recursivo. Es decir, $k = max(caso base^k,caso recurrente^k)$. Esto es posible
gracias a que un algoritmo de complejidad O$(n^a)$ se puede siempre convertir en otro
    equivalente de complejidad O$(n^b)$, siempre que $b > a$. Pensad que bastaría con anidar el algoritmo de complejidad O$(n^a )$ en tantos bucles sin sentido como hiciese falta para incrementarle la complejidad hasta que llegase a $b$.
    \item A continuación se calcula la relación entre $a$ y $b^k$ y se aplica el teorema maestro, sustituyendo los valores de $k$ y $a$ tal como corresponda.
\end{enumerate}


