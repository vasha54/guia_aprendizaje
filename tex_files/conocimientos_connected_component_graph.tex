\subsection{Componente conexa}
En teoría de grafos, una componente conexa es un subgrafo inducido de un grafo en que cualquiera dos vértices que integran este subgrafo están conectados mediante un camino. Un vértice aislado, el grafo trivial o un grafo conexo son en sí mismos componentes. Es importante que el termino de componente conexa solo es aplicable a un \textbf{grafo no dirigido}.