\subsection{C++}
\begin{lstlisting}[language=C++]
vector<int> findPathEuler(vector<vector<int> > matrix_ady, int n) {
   vector<int> deg(n); vector<int> path; path.assign(1,-1);
   for (int i = 0; i < n; ++i) {
      for (int j = 0; j < n; ++j)
         deg[i] += matrix_ady[i][j];
   }
   int first = 0;
   while (first < n && !deg[first]) ++first;
   
   if(first == n){return path;}
	
   int v1 = -1, v2 = -1;
   bool bad = false;
   for (int i = 0; i < n; ++i) {
      if (deg[i] & 1) {
         if (v1 == -1) v1 = i;
         else if (v2 == -1) v2 = i;
         else bad = true;
      }
   }
   
   if (v1 != -1){
      ++matrix_ady[v1][v2]; ++matrix_ady[v2][v1];
   }	
   stack<int> st; st.push(first);
   vector<int> res;
   while (!st.empty()) {
      int v = st.top(); int i;
      for (i = 0; i < n; ++i)
         if (matrix_ady[v][i]) break;
      if (i == n) { res.push_back(v); st.pop();} 
      else { --matrix_ady[v][i];
             --matrix_ady[i][v]; st.push(i); }
   }
   if (v1 != -1) {
      for (size_t i = 0; i + 1 < res.size(); ++i) {
         if ((res[i] == v1 && res[i + 1] == v2)
             || (res[i] == v2 && res[i + 1] == v1)) {
            vector<int> res2;
            for (size_t j = i + 1; j < res.size(); ++j)
               res2.push_back(res[j]);
            for (size_t j = 1; j <= i; ++j)
               res2.push_back(res[j]);
            res = res2;
            break;
         }
      }
   }
	
   for (int i = 0; i < n; ++i) {
      for (int j = 0; j < n; ++j) {
          if (matrix_ady[i][j]) bad = true;
      }
   }
	
   if (bad){ return path; } 
   else {
      path.clear();
      for (int x : res)
         path.push_back(x);
   }
}
\end{lstlisting}

Esta implementación esta pensada para multigrafo no dirigido con bucles

La siguiente implementación  halla el ciclo de Euler en un grafo dirigido asumiendo que todos los nodos sus grados de entrada y salida son iguales y el grafo es conexo. Al método se le pasa el grafo y el nodo por el cual quiere comenzar el ciclo.

\begin{lstlisting}[language=C++]
vector<int> eulerCycle(vector<vector<int> > graph, int v) {
   vector<int> curEdge (graph.size(),0);
   vector<int> res; stack<int> st ;
   st.push(v);
   while (!st.empty()) {
      v = st.pop();
      while (curEdge[v] < graph[v].size()) {
         st.push(v); v = graph[v][curEdge[v]++];
      }
      res.push_back(v);
   }
   reverse(res.begin(),res.end());
   return res;
}
\end{lstlisting}

Implementación del algoritmo de Hierholzer que realiza DFS desde el vértice inicial y permite determinar si hay camino (\emph{eulerian\_path}) o ciclo (\emph{eulerian\_tour}) de euler para grafos no dirigidos.

\begin{lstlisting}[language=C++]
#define fst first
#define snd second
#define all(c) ((c).begin()), ((c).end())

struct graph {
   int n;
   struct edge { int src, dst, rev; };
   vector<vector<edge>> adj;
   graph() : n(0) { }
   //graph(int n) : n(n), adj(n) { }
	
   void add_edge(int src, int dst) {
      n = max(n, max(src, dst)+1);
      if (adj.size() < n) adj.resize(n);
      adj[src].push_back({src, dst, (int)adj[dst].size()});
      adj[dst].push_back({dst, src, (int)adj[src].size()-1});
   }

   vector<int> path;
   void visit(int u) {
      while (!adj[u].empty()) {
         auto e = adj[u].back(); adj[u].pop_back();
         if (e.src >= 0){
            adj[e.dst][e.rev].src = -1;
            visit(e.dst);
         }
      }
      path.push_back(u);
   }
   vector<int> eulerian_path() {
      int m = 0, s = -1;
      for (int u = 0; u < n; ++u) {
         m += adj[u].size();
         if (adj[u].size() % 2 == 1) s = u;
      }
      path.clear(); if (s >= 0) visit(s);
      if (path.size() != m/2 + 1) return {};
      return path;
   }
   vector<int> eulerian_tour() {
      int m = 0, s = 0;
      for (int u = 0; u < n; ++u) {
         m += adj[u].size();
         if (adj[u].size() > 0) s = u;
      }
      path.clear(); visit(s);
      if (path.size() != m/2 + 1 || path[0] != path.back()) return {};
      return path;
   }
};
\end{lstlisting}

\subsection{Java}
\begin{lstlisting}[language=Java]
public List<Integer> findPathEuler(int[][] matrix_ady, int n) {
   int[] deg = new int[n]; Arrays.fill(deg, 0);
   List<Integer> path = new ArrayList<Integer>();
   path.add(-1);
   for (int i = 0; i < n; ++i) {
      for (int j = 0; j < n; ++j)
         deg[i] += matrix_ady[i][j];
   }
   int first = 0;
   while (first < n && deg[first]==0) ++first;
   if (first == n) return path;
	
   int v1 = -1, v2 = -1; boolean bad = false;
   for (int i = 0; i < n; ++i) {
      if ((deg[i] & 1) == 1) {
         if (v1 == -1) v1 = i;
         else if (v2 == -1) v2 = i;
         else bad = true;
      }
   }
	
   if (v1 != -1) {
      ++matrix_ady[v1][v2];
      ++matrix_ady[v2][v1];
   }
	
   Stack<Integer> st = new Stack<Integer>();
   st.push(first);
   List<Integer> res = new ArrayList<Integer>();
   while (!st.empty()) {
      int v = st.peek(); int i;
      for (i = 0; i < n; ++i)
         if(matrix_ady[v][i] != 0) break;
      if(i == n){res.add(v);st.pop();}
      else {
          --matrix_ady[v][i]; --matrix_ady[i][v];
          st.push(i);
      }
   }
   if(v1 != -1) {
      for(int i = 0; i + 1 < res.size(); ++i) {
         if((res.get(i) == v1 && res.get(i + 1) == v2) || (res.get(i) == v2 && res.get(i + 1) == v1)) {
            List<Integer> res2 = new ArrayList<Integer>();
            for(int j = i + 1; j < res.size(); ++j)
               res2.add(res.get(j));
            for (int j = 1; j <= i; ++j)
               res2.add(res.get(j));
            res.clear();
            res.addAll(res2);
            break;
         }
      }
   }
	
   for (int i = 0; i < n; ++i) {
      for (int j = 0; j < n; ++j) {
         if (matrix_ady[i][j] != 0) bad = true;
      }
   }
	
   if (bad) { return path;}
   else {
      path.clear();
      for (Integer x : res) path.add(x);
   }
   return path;
}
\end{lstlisting}
Esta implementación esta pensada para multigrafo no dirigido con bucles


La siguiente implementación  halla el ciclo de Euler en un grafo dirigido asumiendo que todos los nodos sus grados de entrada y salida son iguales y el grafo es conexo. Al método se le pasa el grafo y el nodo por el cual quiere comenzar el ciclo.

\begin{lstlisting}[language=Java]
public static List<Integer> eulerCycle(List<Integer>[] graph, int v) {
   int[] curEdge = new int[graph.length];
   List<Integer> res = new ArrayList<>();
   Stack<Integer> stack = new Stack<>();
   stack.add(v);
   while (!stack.isEmpty()) {
      v = stack.pop();
      while (curEdge[v] < graph[v].size()) {
         stack.push(v);
         v = graph[v].get(curEdge[v]++);
      }
      res.add(v);
    }
    Collections.reverse(res);
    return res;
}
\end{lstlisting}