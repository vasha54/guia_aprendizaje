\subsection{Integral}
Una integral en matemáticas es una operación matemática que se utiliza para calcular el área bajo una curva o la acumulación de cierta cantidad a lo largo de un intervalo. En términos más simples, la integral nos permite encontrar el área bajo una curva en un gráfico, lo cual es útil en diversas aplicaciones matemáticas y científicas.

Existen dos tipos principales de integrales: la integral definida y la integral indefinida. La integral definida se utiliza para calcular el área bajo una curva entre dos puntos específicos en un eje, mientras que la integral indefinida es una función que representa todas las posibles antiderivadas de una función dada.


\subsection{Thomas Simpson}
Thomas Simpson fue un matemático británico del siglo XVIII conocido por sus contribuciones en el campo del cálculo numérico y la interpolación. Uno de sus logros más destacados es el desarrollo de la regla de Simpson, un método numérico más preciso que la regla del trapecio para aproximar el valor de una integral definida.

\subsection{Método númericos}
Los métodos numéricos son técnicas matemáticas que se utilizan para resolver problemas numéricos o computacionales en diversas áreas, como la ingeniería, la física, la economía, la informática, entre otras. Estos métodos permiten aproximar soluciones a problemas matemáticos que no pueden resolverse de forma exacta o analítica.

\subsection{Regla del trapecio}
El método de la regla del trapecio es un método numérico utilizado para aproximar el valor de una integral definida. Consiste en dividir el intervalo de integración en segmentos pequeños y aproximar el área bajo la curva mediante la suma de áreas de trapecios.

\subsection{Interpolación polinomial Lagrange}
La interpolación polinomial de Lagrange es un método utilizado en matemáticas y computación para encontrar un polinomio que pase por un conjunto dado de puntos en un plano cartesiano. Este método se basa en la idea de construir un polinomio de grado n-1 que pase por n puntos dados, donde n es el número de puntos de interpolación.

