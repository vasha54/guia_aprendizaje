\subsection{Punto}
El punto es la unidad más simple, irreductiblemente mínima, de la comunicación visual es una figura 
geométrica sin dimensión, tampoco tiene longitud, área, volumen, ni otro ángulo dimensional. No es un 
objeto físico. Describe una posición en el plano, determinada respecto de un sistema de coordenadas 
preestablecidas. 


\subsection{Segmento}
Es un fragmento de la recta que está comprendido entre dos puntos, llamados puntos extremos o finales. Así, dado dos puntos A y B, se llama segmento AB a la intersección de la semirrecta de origen A que contiene al punto B con la semirrecta de origen B que contiene al punto A. Los puntos A y B son extremos del segmento y los puntos sobre la recta a la que pertenece el segmento.

%\subsection{Línea}
%Funciona como una sucesión continua de puntos trazados, como por ejemplo un trazo o un guion.  En geometría euclidiana, la recta o la línea recta se extiende en una misma dirección por tanto tiene una sola dimensión y contiene infinitos puntos; se puede considerar que está compuesta de infinitos segmentos. Dicha recta también se puede describir como una sucesión continua e indefinida de puntos extendidos en una sola dimensión, es decir, no posee principio ni fin. 

%\subsection{Plano}
%Es un objeto ideal que solo posee dos dimensiones, y contiene infinitos puntos y rectas; es un concepto fundamental de la geometría junto con el punto y la recta. Un plano queda definido por los siguientes elementos geométricos:

%\begin{itemize}
%	\item Tres puntos no alineados.
%	\item Una recta y un punto exterior a ella.
%	\item Dos rectas paralelas o dos rectas que se cortan.
%\end{itemize}

\subsection{Estructura}
Las estructuras (también llamadas \textbf{struct}) son una forma de agrupar varias variables relacionadas en un solo lugar. Cada variable en la estructura se conoce como un miembro de la estructura.






