Al igual que la cola está estructura puede ser útil en ejercicios de simulación donde el orden y además una determinada prioridad sea importante. Uno de las aplicaciones de esta estructura es en el algoritmo de Dijkstra donde la utilización de esta estructura reduce considerablemente la complejidad del algoritmo.

En la gestión de procesos en un sistema operativo. Los procesos no se ejecutan uno
tras otro en base a su orden de llegada. Algunos procesos deben tener prioridad (por su mayor importancia, por su menor duración, etc.) sobre otros.

Implementación de algoritmos voraces, los cuales proporcionan soluciones globales
a problemas basandose en decisiones tomadas sólo con información local. La
determinación de la mejor opción local suele basarse en una cola de prioridad.

