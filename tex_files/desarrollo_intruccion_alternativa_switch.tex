Se trata de una alternativa a la bifurcación if elseif else cuando se compara la misma expresión con
distintos valores. Su forma general es la siguiente:

\begin{lstlisting}[language=C++]
switch (expression) {
   case value1: statements1;
   case value2: statements2;
   case value3: statements3;
   case value4: statements4;
   case value5: statements5;
   case value6: statements6;
   default: statements7;
}
\end{lstlisting}

Las características más relevantes de switch son las siguientes:

\begin{enumerate}
	\item Cada sentencia \textbf{case} se corresponde con un único valor de \textbf{expression}. No se pueden establecer rangos o condiciones sino que se debe comparar con valores concretos.
	\item Los valores no comprendidos en ninguna sentencia \textbf{case} se pueden gestionar en \textbf{default}, que es opcional.
	\item En ausencia de \textbf{break}, cuando se ejecuta una sentencia \textbf{case} se ejecutan también todas las \textbf{case} que van a continuación, hasta que se llega a un \textbf{break} o hasta que se termina el \textbf{switch}.
\end{enumerate}