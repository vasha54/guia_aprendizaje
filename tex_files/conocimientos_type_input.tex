\subsection{Lectura de datos}

Para la lectura  se puede realizar por el método \textbf{cin} primitivo
de C++ o por \textbf{scanf} primitivo de C, en el programa se pueden usar ambos métodos si es necesario pero no es recomendable.

\subsection{Esctructura de repetición (bucles o ciclos)}

Un \textbf{bucle} se utiliza para realizar un proceso repetidas veces. Se denomina también \textbf{lazo} o \textbf{loop}. El
código incluido entre las llaves \{\} (opcionales si el proceso repetitivo consta de una sola línea), se
ejecutará mientras se cumpla unas determinadas condiciones. Hay que prestar especial atención a
los bucles infinitos, hecho que ocurre cuando la condición de finalizar el bucle
(booleanExpression) no se llega a cumplir nunca. Se trata de un fallo muy típico, habitual sobre
todo entre programadores poco experimentados. En el caso de los lenguajes de programación de C++ y Java se cuenta con varias instrucciones de tipo bucle o ciclo como son el \textbf{for},\textbf{while} y \textbf{do ... while}.