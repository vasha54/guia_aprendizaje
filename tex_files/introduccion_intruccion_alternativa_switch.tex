En el diseño e implementación de un algoritmo el mismo no siempre va ser de forma lineal (que se ejecuten cada instrucción una detrás de la otra) sino que llegado determinado paso del algoritmo el mismo debe ser capaz de decidir que cojunto de instrucciones se debe realizar  o que conjunto de instrucciones se debe omitir. Para poder realizar es necesario el uso de estructura de control las cuales permiten modificar el flujo de ejecución de las instrucciones de un algoritmo.

Vamos analizar dentros de la estructura de control las sentencias de decisión que
realizan una pregunta la cual retorna verdadero o falso (evalúa una condición) y selecciona la
siguiente instrucción a ejecutar dependiendo la respuesta o resultado. Especificamente veremos la instrucción \textbf{switch}