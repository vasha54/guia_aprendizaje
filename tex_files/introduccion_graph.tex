La teoría de grafo es un campo de estudio de las matemáticas y las ciencias de la computación, que estudia las propiedades de los grafos estructuras que constan de dos partes, el conjunto de vértices, nodos o puntos; y el conjunto de aristas, líneas o lados (edges en inglés) que pueden ser orientados o no. Por lo tanto también está conocido como análisis de redes.

La teoría de grafos es una rama de las matemáticas discretas y de las matemáticas aplicadas, y es un tratado que usa diferentes conceptos de diversas áreas como combinatoria, álgebra, probabilidad, geometría de polígonos, aritmética y topología.
