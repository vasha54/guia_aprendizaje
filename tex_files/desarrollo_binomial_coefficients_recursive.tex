Los coeficientes binomiales se pueden calcular de manera recursiva de la siguiente manera:

$$\binom n k = \binom {n-1} {k-1} + \binom {n-1} k$$

que está asociado con el famoso Triángulo de Pascal. Es fácil deducir esto utilizando la fórmula analítica. La idea es agregar un elemento $x$ en el conjunto. Si se incluye $x$ en el subconjunto, tenemos que elegir elementos $k-1$ de los elementos $n-1$, y si $x$ no está incluido en el subconjunto, tenemos que elegir $k$ elementos de los elementos $n-1$.

Tenga en cuenta que para $n < k$ se supone que el valor de $\binom n k$ es cero.

La idea es fijar un elemento $x$ en el conjunto. Si $x$ está incluido en el subconjunto, tenemos que elegir $k-1$ elementos de $n-1$ elementos, y si $x$ no está incluido en el subconjunto, tenemos que elegir $k$ elementos de $n-1$ elementos.

Los casos base para la recursividad son:
$$\binom n 0 = \binom n n = 1$$

porque siempre hay exactamente una manera de construir un subconjunto vacío y un subconjunto que contenga todos los elementos.

