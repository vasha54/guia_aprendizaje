La factorización de enteros  tiene diversas aplicaciones en diferentes campos. Algunas de las aplicaciones más comunes de la factorización de enteros son:

\begin{enumerate}
	\item \textbf{Seguridad en criptografía:} La factorización de enteros es fundamental en la criptografía asimétrica, específicamente en el algoritmo RSA (Rivest-Shamir-Adleman). En RSA, la seguridad del sistema se basa en la dificultad computacional de factorizar grandes números enteros en sus factores primos. Si un atacante logra factorizar el número utilizado en la clave pública, podría comprometer la seguridad del sistema.
	 
	\item \textbf{Algoritmos de optimización:} En algunos problemas de optimización combinatoria, como el problema del viajante o el problema del corte máximo, la factorización de enteros puede utilizarse para encontrar soluciones óptimas o aproximadas.
	 
	\item \textbf{Matemáticas puras:} La factorización de enteros es un problema matemático interesante por sí mismo y ha sido objeto de estudio durante siglos. La teoría de números se ocupa de propiedades relacionadas con los números enteros, incluida la factorización.
	 
	\item \textbf{Compresión de datos:} En la compresión de datos, a veces se utilizan técnicas que involucran la factorización de enteros para reducir el tamaño de los datos sin perder información importante.
	
	\item \textbf{Resolución de problemas prácticos:} En situaciones cotidianas, la factorización de enteros puede ser útil para descomponer números grandes en sus factores primos con el fin de simplificar cálculos o entender mejor la estructura de los números.
	 
\end{enumerate}

Estas son solo algunas de las aplicaciones de la factorización de enteros. Es una herramienta poderosa y versátil que se utiliza en una variedad de contextos, desde la seguridad informática hasta las matemáticas puras.