Durante el desarrollo de una posible solución a un ejercicio de concurso por parte del concursante se puede dividir en tres posibles etapas:

\begin{itemize}
	\item \textbf{Lectura y compresión del problema:} El concursante lee y comprende de que se trata el problema. Es capaz de delimitar cual es la información descartable y cual es la importante del la cual extrae y sintetiza los datos con que cuenta.
	\item \textbf{Análisis y diseño de una solución lógica-matemática:} Elaboración de un algoritmo que permita resolver cualquier instancia de dicho problema de forma correcta sin importar los valores que pueden tomar datos para dicha instancia del problema. Para esto el concursante se apoya en algoritmos ya conocidos los cuales puede usar propiamente como fueron defindos por sus autores o con modificaciones realizadas por el concursante o un algoritmo elaborado por el propio concursante. 
	\item \textbf{Implementación del algoritmo solución:} El concursante con apoyo de un lenguaje comprensible para la computadora escribe un programa que realice el algoritmo que da solución a cualquier instancia del problema.
\end{itemize}

En el paso 3 se dice \emph{un lenguaje comprensible para la computadora} que no son más que los llamados lenguajes de programación. En los concursos de programación se permiten varios lenguajes de programación y los problemas deben tener una solución factible en cada uno de los lenguajes permitidos. Lo que sucede que esa llamada \emph{solución factible} en cada uno de los lenguajes de programación no es equitativa producto que muchas veces dependerá de las restricciones del propio lenguaje y el conocimiento del concursante en dicho lenguajes. Por lo anterior sucede que en los concursos de programación existan preferencias por determinados lenguajes de programación. Entre dichos lenguajes de programación de encuentra \textbf{Java}.    