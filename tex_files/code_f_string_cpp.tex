\subsection{Operaciones de entrada}
\begin{lstlisting}[language=C++]
#include <iostream>
#include <bits/stdc++.h>
#include <string> // para la clase string
using namespace std;

int main(){
  // Declarando string
  string str;
  //Declarando un arreglo de char
  char str2 [12];
  // Entrada clasica del string por consola
  cin>>str;
  // Tomando el string de entrada por consola usando getline()
  getline(cin, str);
  // Otra variante de capturar el string por consola  con el
  // uso del getline() con un arreglo de char y la cantidad de caracteres
  // a leer
  cin.getline(str2,12);
  //Insertando un caracter al final del string
  str.push_back('s');
  //Eliminando el ultimo caracter
  str.pop_back();
  return 0;
}
\end{lstlisting}
\subsection{Operaciones de capacidad}
\begin{lstlisting}[language=C++]
#include <iostream>
#include <bits/stdc++.h>
#include <string> // para la clase string
using namespace std;

int main(){
   // Declarando string con valor
   string str = "C++ Programming";
   //Invocando a la funcion length()
   cout << "La longitud de la cadena es: " << str.length() << endl;
   //Invocando a la funcion capacity()
   cout << "La capacidad de la cadena es: " << str.capacity() << endl;
   cout << "La cadena original es: " << str << endl;
   str.resize(10);
   cout << "La cadena despues de usar resize es: " << str << endl;
   str.resize(17);
   cout << "La capacidad de la cadena antes de usar shrink_to_fit es: "<<str.capacity() << endl;
   str.shrink_to_fit();
   cout << "La capacidad de la cadena despues de usar shrink_to_fit es: " << str.capacity() << endl;
   if(str.empty()==true)
      cout<<"Cadena vacia"<<endl;
   else
      cout<<"Cadena no vacia"<<endl;
	return 0;
}	
\end{lstlisting}
\subsection{Operaciones de iteraciones}
\begin{lstlisting}[language=C++]
#include <iostream>
#include <bits/stdc++.h>
#include <string> // para la clase string
using namespace std;

int main(){
   //Incializando cadena 
   string str = "Esto es una cadena de caracteres"; 
   //Declarando iterador 
   string::iterator it; 
   //Declarando iterator reverso  
   string::reverse_iterator it1; 
   //Visualizando la cadena
   cout<<"La cadena usando los iteradores : "; 
   for(it=str.begin();it!=str.end(); it++) 
      cout<<*it; 
   cout<<endl; 
   //Visualizando el inverso de la cadena 
   cout<<""La cadena usando los iteradores inversos es: "; 
   for(it1=str.rbegin();it1!=str.rend(); it1++) 
      cout<<*it1; 
   cout<<endl; 
   return 0;
}	
\end{lstlisting}
\subsection{Operaciones de manipulación}
\begin{lstlisting}[language=C++]
#include <iostream>
#include <bits/stdc++.h>
#include <string> // para la clase string
using namespace std;

int main(){
   string str = "Hello, World!";
   str.replace(7,5,"Universe");  //Remplaza la subcadena "World" por "Universe"
   cout << str << endl;
	
   // Incializa la primera cadena
   string str1 = "Una cadena de caracteres";
	
   // Incializa la segunda cadena
   string str2 = "Otra cadena de caracteres";
	
   // Visualizando las cadenas antes de intercambiar
   cout << "La primera cadena antes de intercambiar: ";
   cout << str1 << endl;
   cout << "La segunda cadena antes de intercambiar : ";
   cout << str2 << endl;
	
   // using swap() to swap string content
   str1.swap(str2);
	
   // Visualizando las cadenas despues de intercambiar
   cout << "La primera cadena despues de intercambiar: ";
   cout << str1 << endl;
   cout << "La segunda cadena despues de intercambiar: ";
   cout << str2 << endl;
	
   string text2 = "Yo tengo un gato.";
   text2.insert(16, " negro"); // Inserta " negro" en la posicion 16
   cout<<text2<<endl;
	
   string text3 = "Esto es un ejemplo.";
   text3.erase(7, 3); // Borra "un "
   cout<<text3<<endl;
	
   return 0;
}	
\end{lstlisting}
\subsection{Operaciones de generación}
\begin{lstlisting}[language=C++]
#include <iostream>
#include <bits/stdc++.h>
#include <string> // para la clase string
using namespace std;

int main(){
  string str = "Hola";
  cout << setw(10) << setfill(' ') << str << endl;
	
  string str2 = "Hola, Mundo";
  string substr = str2.substr(6, 5);  //Extrae "Mundo" de la cadena original
  cout << "La subcadena es: " << substr << endl;
	
  char source[] = "Hola, Mundo"; //arreglo de caracteres de origen
  char destination[20]; // arreglo de caracteres de destino
  strcpy(destination, source); // Copia
  cout << "Cadena original: " << source <<endl;
  cout << "Cadena copiada: " << destination <<endl;
	
  // Inicializando 1era cadena
  string str3 = "Esto es una cadena sin sentido niguno";
	
  // Declarando arreglo de caracteres
  char ch[80];
	
  // usando copy() para copiar el contenido str3
  // en ch comenzando en la posicion 0 los siguientes
  // 13 de caracteres.
  str3.copy(ch, 13, 0);
	
  // Visualizando el arreglo de cadena
  cout << "La nueva cadena de caracteres copiada en el arreglo es: ";
  cout << ch << endl;
  return 0;
}
\end{lstlisting}
\subsection{Operaciones de concatenación}
\begin{lstlisting}[language=C++]
#include <iostream>
#include <bits/stdc++.h>
#include <string> // para la clase string
using namespace std;

int main(){
   string str1 = "Hello";
   string str2 = " World!";
   string result = str1 + str2;
   cout << result << endl;
   
   string base = "Hello";
   base.append(" World!"); // Adiciona la cadena
   cout<<<base<<endl;
	
   char str3[50] = "Hello ";
   char str4[] = "Wordl!!!.";
   strcat(str3, str4);
   cout << str3 << endl;
   return 0;
}
\end{lstlisting}
\subsection{Operaciones de comparación}
\begin{lstlisting}[language=C++]
#include <iostream>
#include <bits/stdc++.h>
#include <string> // para la clase string
using namespace std;

int main(){
   string str1 = "manzana";
   string str2 = "platano";
   if (str1 == str2) { cout << "Las cadenas son iguales" << endl;}
   else{ cout << "Las cadenas son iguales" << endl;}
	
   int result = str1.compare(str2);
   if (result == 0) { cout << "Las cadenas son iguales" << endl;}
   else if (result < 0){ cout << "str1 es lexicograficamente menor que str2" << endl; }
   else { cout << "str1 es lexicograficamente menor que str2" << endl; }
   return 0;
}
	
\end{lstlisting}
\subsection{Operaciones de búsquedas}
\begin{lstlisting}[language=C++]
#include <iostream>
#include <bits/stdc++.h>
#include <string> // para la clase string
using namespace std;

int main(){
   string searchIn = "C++ Programming"; 
   size_t position = searchIn.find("Programming"); 
   if (position != string::npos) { 
      cout << "Encontrado en la posicion:" << position << endl; 
   }else { 
      cout << "No encontrado" << endl; 
   } 
   return 0;
}
	
\end{lstlisting}
\subsection{Operaciones de conversión}
\begin{lstlisting}[language=C++]
#include <iostream>
#include <bits/stdc++.h>
#include <string> // para la clase string
using namespace std;

int main(){
   string str = "123";
   int num = stoi(str);
   cout << num << endl;
	
   string str2 = "3.14";
   double num2 = stod(str2);
   cout << num2 << endl;
	
   int num3 = 42;
   string str3 = to_string(num3);
   cout << str3 << endl;
	
   string str4 = "C++";
   const char* cstr = str4.c_str();
   cout << cstr << endl;
   return 0;
}
	
\end{lstlisting}