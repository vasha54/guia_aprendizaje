\subsection{Máscara de bits}
Las máscaras de bits son usadas usualmente para operaciones bitwise para acceder o establecer secciones individuales de las estructuras de datos estilo campo de bits. Por otro lado, los campos de bits se utilizan para almacenar datos de manera eficiente y reducir la huella de memoria. Además, las operaciones a nivel de bits son relativamente más rápidas de ejecutar en el hardware que las operaciones aritméticas comunes. 

\subsection{Programación Dinámica}
La programación dinámica es una técnica de optimización que se utiliza para resolver problemas complejos dividiéndolos en subproblemas más pequeños y resolviéndolos de manera recursiva. La idea principal detrás de la programación dinámica es almacenar los resultados de los subproblemas resueltos para evitar repetir los cálculos y mejorar la eficiencia del algoritmo.