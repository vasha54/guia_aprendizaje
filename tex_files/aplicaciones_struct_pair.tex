Uno de los usos más comunes del \emph{pair} es con los contenedores de biblioteca de plantillas estándar (STL). std::map y std::multimap almacena sus elementos en forma de parejas. De hecho, un mapa STD es esencialmente una colección de \emph{pair}, donde el primer elemento de cada \emph{pair} es una clave y el segundo elemento es el valor correspondiente.

El \emph{pair}, en su simplicidad, ofrece mucha versatilidad y funcionalidad que pueden simplificar su experiencia de codificación de muchas maneras diferentes. Es una herramienta que merece un lugar en cada kit de herramientas del programador de C++.

Comprender cómo y cuándo usar \emph{pair} puede ser fundamental para escribir código C++ limpio y eficiente. Es un testimonio del hecho de que a veces, las herramientas más simples pueden ser las más poderosas.

Las estructuras constituyen uno de los aspectos más potentes del lenguaje C. En esta
sección se ha tratado sólo de hacer una breve presentación de sus posibilidades. C++
generaliza este concepto incluyendo funciones miembro además de variables miembro,
llamándolo clase, y convirtiéndolo en la base de la programación orientada a objetos.

Una de las características más interesantes de las estructuras es su enorme potencial para facilitar el trabajo del programador, al permitir referirse a un grupo grande de variables mediante un único nombre. Esto es especialmente útil cuando es necesario pasar un número grande de argumentos a una función. De la forma clásica, si tuviéramos
veinte variables, habría que pasar las veinte variables como veinte argumentos distintos a una función, lo cual en más de una ocasión, provocaría errores. Gracias al uso de
estructuras, en cambio, bastará con pasar a la función el nombre de la variable estructura que agrupa todos los argumentos necesarios para la función.