\subsection{Triángulo de Pascal}
En las matemáticas, el triángulo de Pascal es una representación de los coeficientes binomiales ordenados en forma de triángulo. Es llamado así en honor al filósofo y matemático francés Blaise Pascal, quien introdujo esta notación en 1654, en su Tratado del triángulo aritmético. Si bien las propiedades y aplicaciones del triángulo las conocieron matemáticos indios, chinos, persas, alemanes e italianos antes del triángulo de Pascal, fue Pascal quien desarrolló muchas de sus aplicaciones y el primero en organizar la información de manera conjunta.
$$
\begin{tabular}{rccccccccc}
	$n=0$:&    &    &    &    &  1\\\noalign{\smallskip\smallskip}
	$n=1$:&    &    &    &  1 &    &  1\\\noalign{\smallskip\smallskip}
	$n=2$:&    &    &  1 &    &  2 &    &  1\\\noalign{\smallskip\smallskip}
	$n=3$:&    &  1 &    &  3 &    &  3 &    &  1\\\noalign{\smallskip\smallskip}
	$n=4$:&  1 &    &  4 &    &  6 &    &  4 &    &  1\\\noalign{\smallskip\smallskip}
\end{tabular}
$$

\subsection{Aritmética larga}
La aritmética de precisión arbitraria, también conocida como \emph{bignum} o simplemente \emph{aritmética larga}, es un conjunto de estructuras de datos y algoritmos que permiten procesar números mucho mayores de los que pueden caber en tipos de datos estándar. A continuación se presentan varios tipos de aritmética de precisión arbitraria.

\subsection{Inverso multiplicativo}

Un inverso multiplicativo modular de un número entero $a$ es un numero entero $x$ tal que $a \cdot x$ es congruente con $1$ modular algún módulo $m$.
Para escribirlo de forma formal: queremos encontrar un número entero $x$ de modo que:

$$a \cdot x \equiv 1 \mod m.$$
