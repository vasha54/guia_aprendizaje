En la programación competitiva la utilización del árbol se ve sobre todo en ejercicios cuya área de conocimientos son estructuras de datos y teoría de grafos. En la primera la idea conceptual del árbol permite conceptualizar e implementar estructuras basadas en este modelo con ciertas restricciones de acuerdo al problema que permiten manejar de forma optima de acuerdo a las restricciones una coleción de datos. En el segundo caso el árbol es visto como un caso especial del grafo.