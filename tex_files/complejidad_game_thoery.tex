No se preocupe si ve un problema de teoría de juego durante un concurso: puede ser similar a uno de los juegos descritos anteriormente o puede reducirse a uno de ellos. Si no, solo piénselo en ejemplos concretos. Una vez que lo descubres, la parte de codificación suele ser muy simple, directa y con una complejidad bastante asequible para la restricciones.

En el caso de la implementación de ganar o perder su complejidad es $O(nk)$ donde $n$ es la cantidad inicial y $k$ la cantidad de posibles movimientos. En el caso de la implementación del Nim su complejidad es O($n$) donde $n$ es la cantidad de pilas iniciales del juego.