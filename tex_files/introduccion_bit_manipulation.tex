El sistema de numeracion decimal (Base 10) esta compuesto por 10 digitos: 0, 1, 2, 3, 4, 5, 6, 7, 8, 9
 y todos los numeros estan compuestos por la combinacion de estos, por otro lado el Sistema Binario
(Base 2) esta compuesto unicamente por dos digitos: 0 y 1, Bit es el acronimo de Binary digit (digito
 binario), entonces un bit representa uno de estos valores 0 o 1, se puede interpretar como un foco
que tiene 2 estados: Encendido(1) y Apagado(0). Existen varios metodos par convertir numeros de
una base a otra.

Los datos que usamos en nuestros programas, internamente, estan representados en Binario con
una cadena de bits. Por ejemplo un \textbf{int} tiene 32 bits. Entonces muchas veces se requiere hacer
operaciones de bits, ya sea por que estas se ejecutaran mas rápido que otras mas complejas como la
multiplicacion, o por que se quiere modificarlos.

En esta guía trataremos las operaciones de bits que nos seran de mucha ayuda para la resolución
de problemas de algoritmia