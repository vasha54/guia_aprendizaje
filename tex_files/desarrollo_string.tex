Al considerar las cadenas como un tipo de datos, hay que definir cuáles son las operaciones que es posible hacer con ellas. En principio, podrían ser muchas y llegar a ser muy sofisticadas. Las siguientes son algunas de ellas:

\begin{enumerate}
	\item Asignación: Consiste en asignar una cadena a otra.
	\item Concatenación: Consiste en unir dos cadenas o más (o una cadena con un carácter) para formar una cadena de mayor tamaño.
	\item Búsqueda: Consiste en localizar dentro de una cadena una subcadena más pequeña o un carácter.
	\item Extracción: Se trata de sacar fuera de una cadena una porción de la misma según su posición dentro de ella.
	\item Comparación: Se utiliza para comparar dos cadenas.
\end{enumerate}

Tanto en C++ como Java las cadenas de caracteres se trabaja como si fueran objetos de clases y como clases al fin contienen un grupo de métodos que te permiten operar y manipular con la cadena de caracteres almacenadas en la variable string. La diferencia de la cadena de caracteres es que en C++ se utiliza el tipo de dato \emph{string} mientras en Java es \emph{String}.