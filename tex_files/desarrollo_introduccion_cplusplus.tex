C++ es un lenguaje de programación orientado a objetos que toma la base del lenguaje C y le agrega
 la capacidad de abstraer tipos como en Smalltalk.

La intención de su creación en los principios de la década de los ochenta por Bjarne Stroustrup fue el extender al exitoso lenguaje de programación C con mecanismos
que permitieran la manipulación de objetos. En ese sentido, desde el punto de vista de los lenguajes
orientados a objetos, el C++ es un lenguaje híbrido.

Posteriormente se añadieron facilidades de programación genérica, que se sumó a los otros dos
paradigmas que ya estaban admitidos (programación estructurada y la programación orientada a
objetos). Por esto se suele decir que el C++ es un lenguaje de programación multiparadigma.

Las principales herramientas necesarias para escribir un programa en C++ son las siguientes:

\begin{itemize}
	\item Un equipo ejecutando un sistema operativo (Una computadora con Windows, Mac o alguna distribución de Linux).
	\item Un compilador de C++ el mismo puede variar dependiendo del sistema operativo.
	\item Un entorno de desarrollo (IDE) de igual manera puede variar dependiendo del sistema operativo. A continuación de listan algunos de los mas usados por los concursantes.
	\begin{itemize}
		\item Code::Blocks 
		\item Eclipse
		\item CLion
	\end{itemize}
	\item Tiempo para practicar 
	\item Paciencia, pero mucha paciencia y perseverancia.
\end{itemize}

Adicionalmente deberías concocer

\begin{itemize}
	\item Inglés (Recomendado). Casi toda la documentación y los problemas de la competencias internacionales se escriben en este idioma. Por tanto al menos tener la habilidad de lectura y compresión en este idioma es fundamental.
	\item Estar familiarizado con C u otro lenguaje derivado (PHP, Python, etc).
\end{itemize}

Es recomendable tener conocimientos de C, debido a que C++ es una mejora de C, tener los
conocimientos sobre este te permitirá avanzar más rápido y comprender aún mas. También, hay
que recordar que C++, admite C, por lo que se puede programar (reutilizar), funciones de C que se
puedan usar en C++.

Aunque no es obligación aprender C, es recomendable tener nociones sobre la programación
orientada a objetos en el caso de no tener conocimientos previos de programación estructurada.
Asimismo, muchos programadores recomiendan no saber C para saber C++, por ser el primero de
ellos un lenguaje imperativo o procedimental y el segundo un lenguaje de programación orientado a
objetos.
