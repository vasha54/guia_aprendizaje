Esta estructura presenta un grupo de aplicaciones en la resolución de problemas entre los cuales podemos mencionar:

\begin{enumerate}
	\item \textbf{Encontrar el cambio cíclico más pequeño:} El algoritmo anterior ordena todos los cambios cíclicos (sin agregar un carácter a la cadena) y, por lo tanto, $p[0]$ da la posición del cambio cíclico más pequeño.
	
	\item \textbf{Encontrar una subcadena en una cadena:} La tarea es encontrar una cuerda $s$ dentro de algún texto $t$ en línea-conocemos el texto $t$ de
	antemano, pero no la cuerda $s$. Podemos crear la matriz de sufijos para el texto $t$ en O$(|t|\log|t|)$ tiempo. Ahora podemos buscar la subcadena $s$
	de la siguiente manera. La ocurrencia de $s$ debe ser un prefijo de algún sufijo de $t$. Como ordenamos todos los sufijos, podemos realizar una búsqueda 
	binaria de $s$ en $p$. Comparando el sufijo actual y la subcadena $s$ dentro de la búsqueda binaria se puede hacer en O$(|s|)$ tiempo, por lo tanto la
	complejidad para encontrar la subcadena es O$(|s|\log|t|)$. Observe también que si la subcadena aparece varias veces en $t$, entonces todas las 
	ocurrencias estarán una al lado de la otra en $p$. Por lo tanto, el número de apariciones se puede encontrar con una segunda búsqueda binaria y todas las 
	apariciones se pueden imprimir fácilmente.
	
	\item \textbf{Comparando dos subcadenas de una cadena:} Queremos poder comparar dos subcadenas de la misma longitud de una cadena determinada $s$ en $O(1)$ 
	tiempo, es decir, comprobar si la primera subcadena es más pequeña que la segunda. Para esto construimos la matriz de sufijos en O$(|s|\log|s|)$ tiempo y
	almacenar todos los resultados intermedios de las clases de equivalencia $c[]$.
	
	Usando esta información podemos comparar dos subcadenas cualesquiera cuya longitud sea igual a una potencia de dos en O(1): para esto es suficiente comparar las clases de equivalencia de ambas subcadenas. Ahora queremos generalizar este método a subcadenas de longitud arbitraria.
	
	Comparemos dos subcadenas de longitud $l$ con los índices iniciales $i$ y $j$. Encontramos la longitud más grande de un bloque que se coloca dentro de una 
	subcadena de esta longitud: la mayor $k$ tal que $2^k\le l$. Luego, la comparación de las dos subcadenas se puede reemplazar comparando dos bloques de longitud superpuestos $2^k$: primero necesitas comparar los dos bloques comenzando en $i$ y $j$, y si son iguales, compare los dos bloques que terminan en
	posiciones $i+l-1$ y $j+l-1$:
		
	$$\dots \overbrace{\underbrace{s_i \dots s_{i+l-2^k} \dots s_{i+2^k-1}}_{2^k} \dots s_{i+l-1}}^{\text{first}} \dots \overbrace{\underbrace{s_j \dots s_{j+l-2^k} \dots s_{j+2^k-1}}_{2^k} \dots s_{j+l-1}}^{\text{second}} \dots$$
	
	$$\dots \overbrace{s_i \dots \underbrace{s_{i+l-2^k} \dots s_{i+2^k-1} \dots s_{i+l-1}}_{2^k}}^{\text{first}} \dots \overbrace{s_j \dots \underbrace{s_{j+l-2^k} \dots s_{j+2^k-1} \dots s_{j+l-1}}_{2^k}}^{\text{second}} \dots$$
	
	Aquí está la implementación de la comparación. Tenga en cuenta que se supone que la función se llama con el valor ya calculado $k$. $k$ se puede calcular
	con $\lfloor \log l \rfloor$, pero es más eficiente precalcular todos $k$ valores para cada $l$.

\begin{lstlisting}[language=C++]	
int compare(int i, int j, int l, int k) {
   pair<int, int> a = {c[k][i], c[k][(i+l-(1 << k))%n]};
   pair<int, int> b = {c[k][j], c[k][(j+l-(1 << k))%n]};
   return a == b ? 0 : a < b ? -1 : 1;
}
\end{lstlisting}	

	\item \textbf{Prefijo común más largo de dos subcadenas con memoria adicional:} Para una cadena dada $s$ queremos calcular el prefijo común más largo (LCP) 
	de dos sufijos arbitrarios con posición $i$ y $j$.
	
	El método descrito aquí utiliza O$(|s|\log|s|)$ memoria adicional. En la siguiente sección se describe un enfoque completamente diferente que solo 
	utilizará una cantidad lineal de memoria.
	
	Construimos la matriz de sufijos en O$(|s|\log|s|)$ tiempo, y recordar los resultados intermedios de las matrices $c[]$ de cada iteración.
	
	Calculemos el LCP para dos sufijos comenzando en $i$ y $j$. Podemos comparar dos subcadenas cualesquiera con una longitud igual a una potencia de dos en $O(1)$. Para hacer esto, comparamos las cadena por potencia de dos (de mayor a menor potencia) y si las subcadenas de esta longitud son iguales, entonces 
	sumamos la longitud igual a la respuesta y continuamos verificando el LCP a la derecha de la parte igual, es decir $i$ y $j$ se suman por la potencia  actual de dos.
	
\begin{lstlisting}[language=C++]	
int lcp(int i, int j) {
   int ans = 0;
   for (int k = log_n; k >= 0; k--) {
      if (c[k][i % n] == c[k][j % n]) {
         ans += 1 << k;
         i += 1 << k;
         j += 1 << k;
      }
   }
   return ans;
}
\end{lstlisting}


Aquí $log\_n$ denota una constante que es igual al logaritmo de $n$ en base $2$ redondeado hacia abajo.
	
	\item \textbf{Prefijo común más largo de dos subcadenas sin memoria adicional:} Tenemos la misma tarea que en el apartado anterior. Hemos calculado el prefijo común más largo (LCP) para dos sufijos de una cadena $s$.

A diferencia del método anterior, este solo utilizará O$(|s|)$ memoria. El resultado del preprocesamiento será una matriz (que en sí misma es una fuente
importante de información sobre la cadena y, por lo tanto, también se utiliza para resolver otras tareas). Las consultas LCP se pueden responder realizando 
consultas RMQ (consultas de rango mínimo) en esta matriz, por lo que para diferentes implementaciones es posible lograr un tiempo de consulta logarítmico e 
incluso constante.

La base de este algoritmo es la siguiente idea: calcularemos el prefijo común más largo para cada par de sufijos adyacentes en el orden ordenado . En otras 
palabras construimos una matriz $\text{lcp}[0 \dots n-2]$, dónde $\text{lcp}[i]$ es igual a la longitud del prefijo común más largo de los sufijos que 
comienzan en $p[i]$ y $p[i+1]$. Esta matriz nos dará una respuesta para dos sufijos adyacentes de la cadena. Entonces, a partir de esta matriz se puede
obtener la respuesta para dos sufijos arbitrarios, no necesariamente vecinos. De hecho, sea la solicitud calcular el LCP de los sufijos $p[i]$ y $p[j]$. 
Entonces la respuesta a esta consulta será $\min(lcp[i],~lcp[i+1],~\dots,~lcp[j-1])$.

Por lo tanto, si tenemos tal matriz $\text{lcp}$, entonces el problema se reduce al RMQ, que tiene una gran cantidad de soluciones diferentes con diferentes
complejidades.

Entonces la tarea principal es construir esta matriz $\text{lcp}$. Usaremos el algoritmo de Kasai, que puede calcular esta matriz en O$(n)$ tiempo.

Veamos dos sufijos adyacentes en el orden de clasificación (orden de la matriz de sufijos). Que sus posiciones iniciales sean $i$ y $j$ y ellos $\text{lcp}$
igual a $k>0$. Si eliminamos la primera letra de ambos sufijos, es decir, tomamos los sufijos $i+1$ y $j+1$ - entonces debería ser obvio que el $\text{lcp}$ de
estos dos es $k-1$. Sin embargo, no podemos usar este valor y escribirlo en el $\text{lcp}$ matriz, porque es posible que estos dos sufijos no estén uno al
lado del otro en el orden de clasificación el sufijo $i+1$ por supuesto será más pequeño que el sufijo $j+1$, pero puede haber algunos sufijos entre ellos. 
Sin embargo, como sabemos que el LCP entre dos sufijos es el valor mínimo de todas las transiciones, también sabemos que el LCP entre dos pares cualesquiera en
ese intervalo tiene que ser al menos $k-1$, especialmente también entre $i+1$ y el siguiente sufijo. Y posiblemente pueda ser más grande.
	
Ahora ya podemos implementar el algoritmo. Repetiremos los sufijos en orden de longitud. De esta manera podemos reutilizar el último valor $k$, desde qu 
pasó del sufijo $i$ al sufijo $i+1$. Es exactamente lo mismo que quitar la primera letra. Necesitaremos una matriz adicional $\text{rank}$, que nos dará la 
posición de un sufijo en la lista ordenada de sufijos.
\begin{lstlisting}[language=C++]	
vector<int> lcp_construction(string const& s, vector<int> const& p) {
   int n = s.size();
   vector<int> rank(n, 0);
   for (int i = 0; i < n; i++) rank[p[i]] = i;
		
   int k = 0;
   vector<int> lcp(n-1, 0);
   for (int i = 0; i < n; i++) {
      if (rank[i] == n - 1) {
         k = 0; continue;
      }
      int j = p[rank[i] + 1];
      while (i + k < n && j + k < n && s[i+k] == s[j+k]) k++;
      lcp[rank[i]] = k;
      if (k) k--;
   }
   return lcp;
}
\end{lstlisting}	


Es fácil ver que disminuimos $k$ a lo sumo O$(n)$ veces (cada iteración como máximo una vez, excepto $\text{rank}[i]==n-1$, donde lo reiniciamos directamente 
a $0$), y el LCP entre dos cadenas es como máximo $n-1$, también aumentaremos $k$ solo O$(n)$ veces. Por lo tanto, el algoritmo se ejecuta en O$(n)$ tiempo.
	
	\item \textbf{Número de subcadenas diferentes:} Preprocesamos la cuerda $s$ calculando la matriz de sufijos y la matriz LCP. Usando esta información
	podemos calcular el número de subcadenas diferentes en la cadena.
	
	Para hacer esto, pensaremos qué \textbf{nuevas} subcadenas comienzan en la posición $p[0]$, entonces en $p[1]$, etc. De hecho, tomamos los sufijos en orden y 
	vemos qué prefijos dan nuevas subcadenas. Por tanto, no pasaremos por alto ninguno por accidente.
	
	Debido a que los sufijos están ordenados, está claro que el sufijo actual $p[i]$ dará nuevas subcadenas para todos sus prefijos, excepto los prefijos que
	coincidan con el sufijo $p[i-1]$. Así, todos sus prefijos excepto el primero $\text{lcp}[i-1]$ uno. Dado que la longitud del sufijo actual es $n-p[i]$,
	$n-p[i]-\text{lcp}[i-1]$ Los nuevos prefijos comienzan en $p[i]$. Sumando todos los sufijos, obtenemos la respuesta final:
	
	$$\sum_{i=0}^{n-1} (n - p[i]) - \sum_{i=0}^{n-2} \text{lcp}[i] = \frac{n^2 + n}{2} - \sum_{i=0}^{n-2} \text{lcp}[i]$$
\end{enumerate}

Como estructura de datos es ampliamente utilizada en áreas como la compresión de datos, la bioinformática y, en general, en cualquier área que trate con cadenas y problemas de coincidencia de cadenas.
	