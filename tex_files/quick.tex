El ordenamiento rápido (quicksort en inglés) es un algoritmo creado por el científico británico en computación C. A. R. Hoare, basado en la técnica de divide y vencerás, que permite, en promedio, ordenar n elementos en un tiempo proporcional a $N \log N$.

El algoritmo trabaja de la siguiente forma:
\begin{itemize}
	\item Elegir un elemento de la lista de elementos a ordenar, al que llamaremos pivote. 
	\item Resituar los demás elementos de la lista a cada lado del pivote, de manera que a un lado queden todos los menores que él, y al otro los mayores. Los elementos iguales al pivote pueden ser colocados tanto a su derecha como a su izquierda, dependiendo de la implementación deseada. En este momento, el pivote ocupa exactamente el lugar que le corresponderá en la lista ordenada.
	\item La lista queda separada en dos sublistas, una formada por los elementos a la izquierda del pivote, y otra por los elementos a su derecha. 
	\item Repetir este proceso de forma recursiva para cada sublista mientras éstas contengan más de un elemento. Una vez terminado este proceso todos los elementos estarán ordenados. 
\end{itemize}

Como se puede suponer, la eficiencia del algoritmo depende de la posición en la que termine el pivote elegido.
