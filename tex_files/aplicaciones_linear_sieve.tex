La cualidad redentora es que este algoritmo calcula una matriz $lp[]$, que nos permite encontrar la factorización de cualquier número en el rango $[2;n]$ en el momento del orden de tamaño de esta factorización. Además, usar solo una matriz extra nos permitirá evitar divisiones cuando busquemos factorización.

Conocer las factorizaciones de todos los números es muy útil para algunas tareas, y este algoritmo es uno de los pocos que permite encontrarlos en tiempo lineal.

La debilidad del algoritmo dado es que usa más memoria que la criba clásica de Eratóstenes: requiere una matriz de números, mientras que para criba clásico de Eratóstenes basta con tener
bits de memoria (que es 32 veces menos).

Por lo tanto, tiene sentido usar el algoritmo descrito solo hasta que para números de orden
y no mayor $10^{7}$.
