La aritmética modular se utiliza sobre todo en ejercicios donde los valores crecen muy rápidos y pueden provocar desbordamiento producto a que los valores no pueden ser almacenados en por los tipos de datos definidos en los lenguajes de programación.

Cuando un problema te pide modular el resultado final es recomendable modular desde las operaciones intermedias aplicando lo visto hasta ahora para cada operación aritmética y de esta forma se evita posibles desbordamiento de datos.  

Las aplicaciones de la aritmética modular son diversas y se encuentran en diferentes áreas, entre las cuales se destacan:

\begin{enumerate}
	\item \textbf{Criptografía:} La aritmética modular es fundamental en la criptografía, ya que se utiliza para implementar algoritmos de cifrado y descifrado como el cifrado RSA, el algoritmo de intercambio de claves Diffie-Hellman, entre otros. La seguridad de estos algoritmos se basa en la dificultad computacional de resolver ciertos problemas relacionados con la aritmética modular.
	
	\item \textbf{Teoría de números:} La aritmética modular es una herramienta importante en la teoría de números para el estudio de propiedades de los números enteros, como la congruencia, los residuos cuadráticos, el teorema chino del resto, entre otros.
	
	\item \textbf{Computación y programación:} En informática, la aritmética modular se utiliza en la implementación de algoritmos eficientes para realizar operaciones aritméticas en números grandes. Por ejemplo, en la programación competitiva y en la optimización de cálculos numéricos.
	
	\item \textbf{Matemáticas aplicadas:} La aritmética modular tiene aplicaciones en diversas áreas de las matemáticas aplicadas, como la teoría de grafos, la teoría de códigos correctores de errores, la teoría de autómatas finitos, entre otras.
	
\end{enumerate}

