\subsection{C++}
\begin{lstlisting}[language=C++]
struct edge{
   int a, b, cost;
};

int n, m, v;
vector<edge> e;
/*debe seleccionarse de tal manera que sea mayor que todas las longitudes de ruta posibles.*/
const int INF = 1000000000; 

void bellmanFord(){
   vector<int> d (n, INF);
   d[v] = 0;
   for (;;){
      bool any = false;
      for (int j=0; j<m; ++j)
         if (d[e[j].a] < INF)
            if (d[e[j].b] > d[e[j].a] + e[j].cost){
               d[e[j].b] = d[e[j].a] + e[j].cost;
               any = true;
            }
      if (!any) break;
    }
}
\end{lstlisting}

Si queremos recuperar el camino la modificación al algoritmo sería la siguiente

\begin{lstlisting}[language=C++]
void bellmanFord(){
   vector<int> d (n, INF);
   d[v] = 0;
   vector<int> p (n, -1);
   for (;;){
      bool any = false;
      for (int j = 0; j < m; ++j)
         if (d[e[j].a] < INF)
            if (d[e[j].b] > d[e[j].a] + e[j].cost){
               d[e[j].b] = d[e[j].a] + e[j].cost;
               p[e[j].b] = e[j].a;
               any = true;
            }
       if (!any)  break;
    }
	if (d[t] == INF)
	   cout << "No hay camino de " << v << " a " << t << ".";
	else{
       vector<int> path;
       for (int cur = t; cur != -1; cur = p[cur])
          path.push_back (cur);
       reverse (path.begin(), path.end());
	   cout << "Camino de " << v << " a " << t << ": ";
       for (size_t i=0; i<path.size(); ++i)
           cout << path[i] << ' ';
     }
}
\end{lstlisting}	

Para detectar los ciclos negativos

\begin{lstlisting}[language=C++]
void bellmanFord(){
   vector<int> d (n, INF);
   d[v] = 0;
   vector<int> p (n, - 1);
   int x;
   
   for (int i=0; i<n; ++i) {
      x = -1;
      for (int j=0; j<m; ++j)
         if (d[e[j].a] < INF)
            if (d[e[j].b] > d[e[j].a] + e[j].cost){
               d[e[j].b] = max (-INF, d[e[j].a] + e[j].cost);
               p[e[j].b] = e[j].a;
               x = e[j].b;
            }
   }
   if (x == -1)
      cout << "No existe ciclos negativos" << v;
   else{
      int y = x;
      for (int i=0; i<n; ++i)
         y = p[y];
      vector<int> path;
      for (int cur=y; ; cur=p[cur]){
         path.push_back (cur);
         if (cur == y && path.size() > 1) break;
      }
      reverse (path.begin(), path.end());
	  cout << "Ciclo negativo: ";
      for (size_t i=0; i<path.size(); ++i)
         cout << path[i] << ' ';
   }
}
\end{lstlisting}	

\subsection{Java}
\begin{lstlisting}[language=Java]
	
private class BellmanFord {
   final int  INF = Integer.MAX_VALUE / 2;
   
   public static class Edge {
      int v, cost;
      
      public Edge(int v, int cost) {
         this.v = v; this.cost = cost;
      }
   }
	
   public boolean bellmanFord(List<Edge>[] graph, int s, int[] dist, int[] pred) {
      Arrays.fill(pred, -1);
      Arrays.fill(dist, INF);
      dist[s] = 0;
      int n = graph.length;
      for (int step = 0; step < n; step++) {
         boolean updated = false;
         for (int u = 0; u < n; u++) {
            if (dist[u] == INF) continue;
            for (Edge e : graph[u]) {
               if (dist[e.v] > dist[u] + e.cost) {
                  dist[e.v] = dist[u] + e.cost;
                  dist[e.v] = Math.max(dist[e.v], -INF);
                  pred[e.v] = u;
                  updated = true;
               }
             }
          }
          if (!updated) return true;
       }
       // existe un ciclo negativo
       return false;
   }
	
   public int[] findNegativeCycle(List<Edge>[] graph) {
      int n = graph.length;
      int[] pred = new int[n];
      Arrays.fill(pred, -1);
      int[] dist = new int[n];
      int last = -1;
      for(int step=0;step<n;step++){
         last = -1;
         for(int u=0;u<n;u++){
            if(dist[u] == INF) continue;
            for (Edge e : graph[u]) {
               if(dist[e.v] > dist[u] + e.cost){
                  dist[e.v] = dist[u] + e.cost;
                  dist[e.v] = Math.max(dist[e.v], -INF);
                  pred[e.v] = u;
                  last = e.v;
               }
             }
          }
          if (last == -1) return null;
       }
       for(int i=0;i<n;i++){ last = pred[last]; }
       int[] p = new int[n]; int cnt = 0;
       for(int u=last; u!=last || cnt==0;u=pred[u]){ p[cnt++]=u;}
       int[] cycle = new int[cnt];
       for(int i=0;i<cycle.length;i++){ cycle[i] = p[--cnt]; }
       return cycle;
   }
}
	
\end{lstlisting}