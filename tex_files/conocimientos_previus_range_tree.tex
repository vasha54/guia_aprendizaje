\subsection{Arbol}
Los árboles son contenedores que permiten organizar un conjunto de objetos en forma 
jerárquica. Ejemplos típicos son los diagramas de organización de las empresas o 
instituciones y la estructura de un sistema de archivos en una computadora. Los  
árboles sirven para representar fórmulas, la descomposición de grandes sistemas en 
sistemas más pequeños en forma recursiva y aparecen en forma sistemática en muchísimas 
aplicaciones de la computación científica. Una de las propiedades más 
llamativas de los árboles es la capacidad de acceder a muchísimos objetos desde un 
punto de partida o raíz en unos pocos pasos.

\subsection{Arbol Binario}
Un árbol binario es una estructura de datos en la cual cada nodo puede tener un hijo izquierdo y un hijo derecho. No pueden tener más de dos hijos (de ahí el nombre "binario"). Si algún hijo tiene como referencia a null, es decir que no almacena ningún dato, entonces este es llamado un nodo externo. En el caso contrario el hijo es llamado un nodo interno. Usos comunes de los árboles binarios son los árboles binarios de búsqueda, los montículos binarios y Codificación de Huffman.

En teoría de grafos, se usa la siguiente definición: «Un árbol binario es un grafo conexo, acíclico y no dirigido tal que el grado de cada vértice no es mayor a 3». De esta forma solo existe un camino entre un par de nodos.

Un árbol binario con enraizado es como un grafo que tiene uno de sus vértices, llamado raíz, de grado no mayor a 2. Con la raíz escogida, cada vértice tendrá un único padre, y nunca más de dos hijos. Si rehusamos el requerimiento de la conectividad, permitiendo múltiples componentes conectados en el grafo, llamaremos a esta última estructura un bosque.