\subsection{C++}

\subsubsection{Programación dinámica con DFS}
\begin{lstlisting}[language=C++]
struct Node {
   vector<int> neighborns;
   int firstMaxPath; // Primer camino mas largo
   int secondMaxPath; // Segundo camino mas largo
   int nodeChildMaxPath; // Nodo hijo o padre que salio el camino mas largo
};

vector<Node> tree;

void dfsLongestPathsChild(int v, int p){
   tree[v].firstMaxPath = tree[v].secondMaxPath = 0;
   for (auto x : tree[v].neighborns) {
      if (x == p) continue;
      dfsLongestPathsChild(x, v);
      if (tree[x].firstMaxPath + 1 > tree[v].firstMaxPath) {
         tree[v].secondMaxPath = tree[v].firstMaxPath;
         tree[v].firstMaxPath = tree[x].firstMaxPath + 1;
         tree[v].nodeChildMaxPath = x;
      } else if (tree[x].firstMaxPath + 1  > tree[v].secondMaxPath) {
         tree[v].secondMaxPath= tree[x].firstMaxPath + 1;
      }
   }
}

void dfsLongestPathsParent(int v, int p){
   for (auto x : tree[v].neighborns) {
      if (x == p) continue;
      if (tree[v].nodeChildMaxPath == x) {
         if (tree[x].firstMaxPath < tree[v].secondMaxPath + 1) {
            tree[x].secondMaxPath = tree[x].firstMaxPath;
            tree[x].firstMaxPath = tree[v].secondMaxPath + 1;
            tree[x].nodeChildMaxPath = v;
         } else {
            tree[x].secondMaxPath = max(tree[x].secondMaxPath,tree[v].secondMaxPath + 1);
         }
      } else {
         tree[x].secondMaxPath = tree[x].firstMaxPath;
         tree[x].firstMaxPath = tree[v].firstMaxPath + 1;
         tree[x].nodeChildMaxPath = v;
      }
      dfsLongestPathsParent(x, v);
   }
}

void allLongestPathsTree(){
   dfsLongestPathsChild(1,-1);
   dfsLongestPathsParent(1,-1);
}
\end{lstlisting}

\subsubsection{Usando BFS}
\begin{lstlisting}[language=C++]
struct Node {
   vector<int> neighborns;
   int maxPath = LONG_MIN;
};

int nnodes;

vector<Node> tree;

int bfs(int src) {
   int top;
   queue<int> q;
   vector<int> d(nnodes+1, -1);
   d[src] = 0;
   tree[src].maxPath = max(tree[src].maxPath, d[src]);
   q.push(src);
   while(!q.empty()) {
      top = q.front(); q.pop();
      for(int v: tree[top].neighborns) {
         if(d[v] == -1) {
            q.push(v);
            d[v] = d[top] + 1;
            tree[v].maxPath = max(tree[v].maxPath, d[v]);
         }
      }
   }
   return top;
}

void allLongestPathsTree(){
   int diam_end_1 = bfs(1);
   int diam_end_2 = bfs(diam_end_1);
   bfs(diam_end_2);
}
\end{lstlisting}

\subsection{Java}

\subsubsection{Programación dinámica con DFS}
\begin{lstlisting}[language=Java]
public class Main {
   private class Node{
      public List<Integer> neighborns ;
      public int firstMaxPath; // Primer camino mas largo
      public int secondMaxPath; // Segundo camino mas largo
	  public int nodeChildMaxPath; // Nodo hijo o padre que salio el camino mas largo
		
      public Node() {
         firstMaxPath =Integer.MIN_VALUE;
         secondMaxPath = Integer.MIN_VALUE;
         nodeChildMaxPath = Integer.MIN_VALUE;
         neighborns =new ArrayList<Integer>();
      }
   }
   
   private Node [] tree ;
   
   public void dfsLongestPathsChild(int v, int p){
      tree[v].firstMaxPath = tree[v].secondMaxPath = 0;
      for (int x : tree[v].neighborns) {
         if (x == p) continue;
         dfsLongestPathsChild(x, v);
         if (tree[x].firstMaxPath + 1 > tree[v].firstMaxPath) {
            tree[v].secondMaxPath = tree[v].firstMaxPath;
            tree[v].firstMaxPath = tree[x].firstMaxPath + 1;
            tree[v].nodeChildMaxPath = x;
         } else if (tree[x].firstMaxPath + 1  > tree[v].secondMaxPath) {
            tree[v].secondMaxPath= tree[x].firstMaxPath + 1;
         }
      }
   }
	
   public void dfsLongestPathsParent(int v, int p){
       for (int x : tree[v].neighborns) {
          if (x == p) continue;
          if (tree[v].nodeChildMaxPath == x) {
             if (tree[x].firstMaxPath < tree[v].secondMaxPath + 1) {
                 tree[x].secondMaxPath = tree[x].firstMaxPath;
                 tree[x].firstMaxPath = tree[v].secondMaxPath + 1;
                 tree[x].nodeChildMaxPath = v;
             } else {
                 tree[x].secondMaxPath = Math.max(tree[x].secondMaxPath,tree[v].secondMaxPath + 1);
             }
          } else {
            tree[x].secondMaxPath = tree[x].firstMaxPath;
			tree[x].firstMaxPath = tree[v].firstMaxPath + 1;
			tree[x].nodeChildMaxPath = v;
		  }
		  dfsLongestPathsParent(x, v);
       }
   }
	
   public void allLongestPathsTree(){
      dfsLongestPathsChild(1,-1);
      dfsLongestPathsParent(1,-1);
   }
}
\end{lstlisting}

\subsubsection{Usando BFS}
\begin{lstlisting}[language=Java]
public class Main {
   private class Node{
      public List<Integer> neighborns ;
      public int maxPath;
			
      public Node() {
         maxPath =Integer.MIN_VALUE;
         neighborns =new ArrayList<Integer>();
      }
   }
		
   private Node [] tree ;
   private int nnodes;
		
   public int bfs(int src) {
      int top=0;
      Queue<Integer> q =new LinkedList<Integer>();
      int [] d = new int[nnodes+2];
      Arrays.fill(d, -1);
      d[src] = 0;
      tree[src].maxPath = Math.max(tree[src].maxPath, d[src]);
      q.add(src);
      while(!q.isEmpty()) {
         top = q.remove();
         for(int v: tree[top].neighborns) {
            if(d[v] == -1) {
               q.add(v);
               d[v] = d[top] + 1;
               tree[v].maxPath = Math.max(tree[v].maxPath, d[v]);
            }
          }
       }
       return top;
    }
		
    public void allLongestPathsTree(){
       int diam_end_1 = bfs(1);
       int diam_end_2 = bfs(diam_end_1);
       bfs(diam_end_2);
    }
}
\end{lstlisting}
