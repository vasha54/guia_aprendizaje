Mientras que el algoritmo euclidiano calcula sólo el máximo común divisor (GCD) de dos números enteros $a$ y $b$ , la versión extendida también encuentra una manera de representar GCD en términos de $a$ y $b$ , es decir, coeficientes $x$ y $y$ para cual:

$$a \cdot x + b \cdot y = \gcd(a, b)$$

Es importante señalar que por la identidad de Bézout siempre podemos encontrar tal representación. Por ejemplo, $\gcd(55, 80) = 5$, por lo tanto podemos representar $5$ como una combinación lineal con los términos $55$ y $80$ : $55 \cdot 3 + 80 \cdot (-2) = 5$

Una forma más general de ese problema se analiza en guía sobre Ecuaciones lineales diofánticas. Se basará en este algoritmo.