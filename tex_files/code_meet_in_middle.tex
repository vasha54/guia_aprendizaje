%Podemos implementar el algoritmo para que su complejidad temporal sea O($2^{\frac{n}{2}}$). Primero, generamos listas ordenadas $S_A$ y $S_B$, que se pueden hacer en O ($2^{\frac{n}{2}}$) tiempo usando una técnica similar a la de fusión. Después de esto, dado que las listas están ordenadas, podemos verificar en O ($2^{\frac{n}{2}}$) tiempo si la suma $x$ se puede crear a partir de $S_A$ y $S_B$.

Para resumir todo el proceso de esta idea:

\begin{enumerate}
	\item Divida el conjunto dado en dos conjuntos de aproximadamente el mismo tamaño.
	\item Calcule las respuestas requeridas para los dos conjuntos formados así y ordene uno del conjunto resultante para prepararlo para la búsqueda binaria.
	\item Itere sobre el conjunto de resultados sin ordenar y busque binariamente el valor requerido en el conjunto ordenado.
\end{enumerate}




