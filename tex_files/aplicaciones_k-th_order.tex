El algoritmo QuickSelect se utiliza principalmente para encontrar el k-ésimo elemento más pequeño en un conjunto de datos no ordenados. Algunas aplicaciones comunes del QuickSelect incluyen:

\begin{enumerate}
	\item Encontrar la mediana de un conjunto de datos.
	\item Encontrar el elemento más pequeño o más grande en un conjunto de datos.
	\item Selección de elementos en orden estadístico en algoritmos de ordenación como QuickSort.
	\item Búsqueda de elementos específicos en bases de datos y sistemas de gestión de información.
	\item Selección de elementos en algoritmos de agrupamiento y clasificación.
\end{enumerate}
 
En general, el QuickSelect es una herramienta útil para la selección eficiente de elementos en grandes conjuntos de datos no ordenados.
