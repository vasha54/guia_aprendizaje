Los árboles también pueden ser recorridos en orden por nivel (de nivel en nivel), donde visitamos cada nodo en un nivel antes de ir a un nivel inferior. Esto también es llamado recorrido en anchura-primero o recorrido en anchura. Se etiquetan los nodos según su profundidad
(nivel). Se recorren ordenados de menor a mayor nivel, a igualdad de nivel
se recorren de izquierda a derecha.

Si asumimos que el procesar la raiz es la impresión del valor almacenado en el nodo el recorrido en anchura-primero en el árbol presentado de muestra anteriormente daría la siguiente secuencia o impresión de salida.

$$ A, B, C, D, E, F, G, H, I, L, M, K $$