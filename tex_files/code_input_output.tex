\subsection{Primitivos de C++}
\begin{lstlisting}[language=C++]
#include <iostream.h>
#include <math.h>
using namespace std ;
int main(){
   cout << "Hola mundo"; // imprimir mensaje (en la pantalla)
   char nombre [80];
   cout << " Entre su nombre : " ;
   cin >> nombre ;
   cout << " Hola , " << nombre ;
   int A,B,C;
   cin >> A >> B >> C ;
   // Ejemplo con una unica linea, se muestra el uso de cout y endl
   cout << "Bienvenido. Soy un programa. Estoy en una linea de
codigo ." <<endl ;
   // Ejemplo con una unica linea de codigo que se puede fraccionar
   // mediante el uso de '<<'
   cout << "Ahora "
   << "estoy fraccionado en el codigo , pero en la consola me
   muestro como una unica frase ."
   << endl ;
   
   // Uso de un codigo largo, que cuesta leer para un programador,
   // y que se ejecutara sin problemas.
   // *** No se recomienda hacer lineas de esta manera,
   // esta forma de programar no es apropiada ***
   cout << " Un gran texto puede ocupar muchas lineas . "
   << endl
   << " Pero eso no frena al programador a que todo se pueda
   poner en una unica linea de codigo y que "
   << endl
   << " el programa , al ejecutarse , lo situe como el
   programador quiso "
   << endl ;
   
   double raiz2 = sqrt( 2.0 ); // calcula la raiz cuadrada de 2
   int posiciones; // precision, varia de 0 a 9
   cout << "Raiz cuadrada de 2 con precisiones de 0 a 9." << endl
   << "Precision establecida mediante la funcion miembro precision "
   << "de ios_base:" << endl;
   cout << fixed; // usa el formato de punto fijo
   // muestra la raiz cuadrada usando la funcion precision de ios_base
   for ( posiciones = 0; posiciones <= 9; posiciones++ ){
   	cout.precision( posiciones );
   	cout << raiz2 << endl;
   } // fin de for


   int valorAnchura = 4;
   char enunciado[ 10 ];
   cout << "Escriba un enunciado:" << endl;
   cin.width( 5 ); // introduce solo 5 caracteres de enunciado
   // establece la anchura de campo y despues muestra los caracteres con base en esa anchura
   while ( cin >> enunciado ){
   	cout.width( valorAnchura++ );
   	cout << enunciado << endl;
   	cin.width( 5 ); // introduce 5 caracteres mas de enunciado
   } // fin de while
   return 0;
}	
\end{lstlisting}


\subsection{Primitivos de C}
\begin{lstlisting}[language=C]
#include <cstdio>
#define PI 3.14159
using namespace std ;
int main (){
   int numero = 5;
   int coste = 50;
   // No olvidar que '\n' es fin de linea
   printf("Los %d jovenes tomaron %d helados .\ n ", numero, numero *2);
   printf("El valor de PI es %f.\n");
   printf("Esta es una linea sin variables.") ;
   printf("%c %d ",'$ ',coste) ;
   
   printf("Modificador'-'\n");
   printf("/%-10d/\n",365);
   printf("/%-6d/\n",365);
   printf("/%-2d/",365);
   printf("en este caso no sirve de nada el modificador\n");
   printf("Modificador 'numero'\n");
   printf("/%2d/",365);
   printf("en este caso tampoco hace efecto el modificador\n");
   printf("/%6d/\n",365);
   printf("/%10d/\n" , 365);
   printf("Modificador'.numero'\n");
   printf("/%.1f/\n",365.897) ;
   printf("/%.6f/\n",365.897) ;
   printf("/%.3s/\n","AsDfGhJkL") ;
   printf("/%.100s/","AsDfGhJkL");
   printf("No se rellena con 0's como en los ejemplos anteriores \n");
   
   int edad ;
   char nombre [30];
   printf("Ingrese su edad y nombre\n") ;
   scanf("%d%s",&edad,nombre);//Note que & no es necesario en nombre
   printf("Nombre :%s , Edad :%d",nombre,edad);
   
   putchar('a');
   
   char c = getchar();
   
   scanf("%[^\n]", nombre); // Leer todo hasta un salto de linea
   
   return 0;
}
\end{lstlisting}

