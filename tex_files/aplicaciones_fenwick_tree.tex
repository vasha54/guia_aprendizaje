La aplicación más común del árbol de Fenwick es calcular la suma de un rango (es decir, usar la suma 
sobre el conjunto de números enteros (i.e. $f(A_1, A_2, \dots, A_k) = A_1 + A_2 + \dots + A_k$) pero esto no significa que sea su único uso. Dicha estructura puede ser utiliza para resolver otros problemas como puede ser el conteo de elementos dentro de un rango y otros para los cuales se se tiene que aplicar un poco de creatividad a partir de lo conocido de la estructura.  