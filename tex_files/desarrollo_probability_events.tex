Un evento en la teoría de la probabilidad se puede representar como un conjunto.

$$A \subset X$$

donde X contiene todos los resultados posibles y A es un subconjunto de
resultados. Por ejemplo, al tirar un dado, los resultados son:

$$X = \{~1,~2,~3,~4,~5,~6\}$$

Ahora, por ejemplo, el evento \textbf{el resultado es par} corresponde al conjunto

$$A = \{~2,~4,~6\}$$

A cada resultado $x$ se le asigna una probabilidad $p(x)$. Entonces, la probabilidad $P(A)$
de un evento $A$ se puede calcular como una suma de probabilidades de resultados usando
la fórmula 

$$P(A) = \sum_{x \in A}^{} p(x)$$

Por ejemplo, al lanzar un dado, $p(x) = \dfrac{1}{6}$ para cada resultado $x$, por lo que la
probabilidad del evento \textbf{el resultado es par} es

$$p(2)+ p(4)+ p(6) = 
\dfrac{1}{2}$$

La probabilidad total de los resultados en $X$ debe ser $1$, es decir, $P(X)$ = $1$.

Dado que los eventos en la teoría de la probabilidad son conjuntos, podemos manipularlos usando operaciones de conjunto

