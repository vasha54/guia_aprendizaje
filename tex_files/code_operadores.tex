\subsection{C++}
\begin{lstlisting}[language=C++]
struct {
   int codigo;
   float precio;
} articulo, *ptrstr; //Operador Indireccion

int main() {
   //Operador asignacion
   int abc = 22;
   //Operador de direccion
   int def = &abc ;
   //Operador Indireccion
   int val = *def;
   //Operador pertenencia directa
   articulo.codigo = 1265;
   //Operador de direccion
   ptrstr = &articulo;
   // Operador pertenencia indirecta,Operador unitario
   ptrstr->codigo = -3451;
   int sum =0;
   //Operador asignacion, operador coma, operarador asigancion,
   // operador relacional,operador logico ,operador relacional, operador asignacion
   for(int chatos=2,ronda=0;ronda<1000 && chatos<=50000;chatos*=2){
      //Operador asignacion, operador aritmetico
      ronda = ronda + chatos;
      //Operador a nivel bits
      sum =sum | chatos;
   }
   //Operador asignacion
   int x=1 ;
   //Operador asignacion
   int y=10;
   //Operador incremental
   y++; 
  //Operador relacional, Operador ternario,Operador aritmetico,Operador aritmetico
  int z = (x<y)?x-3:y/8;
  return 0;
}	
\end{lstlisting}
\subsection{Java}
\begin{lstlisting}[language=Java]
public static void main(String[] args){
   //Operador asignacion
   int abc = 22;
   int sum =0;
   //Operador asignacion, operador coma, operarador asigancion,
   // operador relacional,operador logico ,operador relacional, operador asignacion
   for(int chatos=2,ronda=0;ronda<1000 && chatos<=50000;chatos*=2){
      //Operador asignacion, operador aritmetico
      ronda = ronda + chatos;
      //Operador a nivel bits
      sum =sum | chatos;
   }
   //Operador asignacion
   int x=1 ;
   //Operador asignacion
   int y=10;
   //Operador incremental
   y++; 
   //Operador relacional, Operador ternario,Operador aritmetico,Operador aritmetico
   int z = (x<y)?x-3:y/8;
}
\end{lstlisting}

