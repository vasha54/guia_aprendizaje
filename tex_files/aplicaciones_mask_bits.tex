Es evidente que esta idea algorítmica permite generar todos los posibles subconjuntos de un conjunto de elementos y puede ser utilizados en problemas en donde se debe contar la cantidad de subcojuntos de un conjunto inicial que cumplan con determinadas condiciones. En este caso la máscara serviría para generar todos los posibles subconjuntos y luego bastaría con chequear cada subconjunto generado y contar con aquellos que cumplan con restricciones o condiciones que imponga el problema.

Otro posible uso es generar todos las posibles particiones de un conjunto inicial en dos subcojuntos un subconjunto con todos los bits activos y otro subconjunto con los bits apagados. 
