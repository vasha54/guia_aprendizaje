\begin{lstlisting}[language=C++]
#include <iostream>
#include <map>
#include <vector>
using namespace std ;
bool comvec(vector<int> v1 , vector <int> v2 ){//Esta es el comparador
   return v1.size() < v2.size();
}

int main (){
   //Creacion de un diccionario con 'vector<int>' 
   //como Tipo de Dato de las llaves
   map<vector<int>,int,bool(*)( vector<int> , vector<int>)> aplvec(comvec);
   
   map<char,int> apl;
   apl.insert(make_pair('a',13)); apl.insert(make_pair('b',98));
   cout<<apl['a']<<"\n"; cout<<apl['b']<<"\n";
   //Notese que no existe apl['c] por lo que se creara y pondra como valor 0
   cout<<apl['c']<<"\n";
   
   //Acceso a las llaves y valores mediante iteradores
   for(map<char,int>::iterator it=apl.begin();it!= apl.end();it++){
      cout<<it->first<<" "<<it->second<<"\n";
   }//Resultado: a 13 b 98 c 0
	
   if(apl.empty()){ cout<<" La aplicacion esta vacia\n";}
   else {
      cout<<" La aplicacion lleva ";
      cout<<apl.size()<<" elementos\n";
   }//Salida : La aplicacion lleva 3 elementos
	
   if(apl.count('a')==1) cout<<" 'a ' esta en el diccionario\n";
   else cout<<" 'a ' no esta en el diccionario\n";
   //Resultado: 'a' esta en el diccionario
   
   if(apl.find('d')!=apl.end()) cout<<" 'd ' esta en el diccionario\n";
   else cout<<" 'd ' no esta en el diccionario\n";
   //Resultado: 'd' no esta en el diccionario
	
   if(apl.lower_bound('c')!=apl.upper_bound('c'))
      cout<<" 'c ' esta en el diccionario\n";
   else
      cout<<" 'c ' no esta en el diccionario\n";
   //Resultado: 'c' esta en el diccionario
   
   if(apl.lower_bound('Z')==apl.upper_bound('Z'))
      cout<<" 'Z ' no esta en el diccionario\n" ;
   else cout<<" 'Z ' esta en el diccionario\n";
   //Resultado: 70 no esta en el diccionario
   return 0;
\end{lstlisting}