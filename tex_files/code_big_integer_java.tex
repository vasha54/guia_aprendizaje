Java incorporó una clase destinada a operaciones aritméticas que requieran gran
 precisión con enteros: BigInteger . La forma de operar con objetos de estas clases difiere de las
 operaciones con variables primitivas. En este caso hay que realizar las operaciones utilizando
métodos propios de estas clases.
\begin{itemize}
\item \textbf{abs():} Devuelve un BigInteger cuyo valor es el valor absoluto de este BigInteger.
\item \textbf{add():} Este método devuelve un BigInteger simplemente calculando el valor 'this + val'.
\item \textbf{and():} Este método devuelve un BigInteger calculando el valor 'this \& val'.
\item \textbf{andNot():} Este método devuelve un BigInteger calculando el valor 'this \& $\sim$ val'.
\item \textbf{bitCount():} Este método devuelve el número de bits en la representación de complemento a dos de este BigInteger que difiere de su bit de signo.
\item \textbf{bitLength():} Este método devuelve el número de bits en la representación mínima en complemento a dos de este bit de signo, excluyendo el bit de signo.
\item \textbf{clearBit():} Este método devuelve un BigInteger cuyo valor es igual a este BigInteger cuyo bit designado se borra.
\item \textbf{compareTo():} Este método compara este BigInteger con el BigInteger especificado.
\item \textbf{divide():} Este método devuelve un BigInteger al calcular el valor 'this / val'.
\item \textbf{divideAndRemainder():} Este método devuelve un BigInteger calculando el valor 'this \& $\sim$ val' seguido de 'this\%value'.
\item \textbf{doubleValue():} Este método convierte este BigInteger en double.
\item \textbf{equals():} Este método compara este BigInteger con el Objeto dado para la igualdad. 
\item \textbf{flipBit():} Este método devuelve un BigInteger cuyo valor es igual a este BigInteger con el bit designado invertido.
\item \textbf{floatValue():} Este método convierte este BigInteger en float.
\item \textbf{gcd():} Este método devuelve un BigInteger cuyo valor es el máximo común divisor entre abs(this) y abs(val).
\item \textbf{getLowestSetBit():} Este método devuelve el índice del bit más a la derecha (orden más bajo) en este BigInteger (el número de bits cero a la derecha del bit más a la derecha).
\item \textbf{hashCode():} Este método devuelve el código hash para este BigInteger.
\item \textbf{intValue():} Este método convierte este BigInteger en int.
\item \textbf{isProbablePrime():} Este método devuelve un valor booleano 'verdadero' si y solo si este BigInteger es primo; de lo contrario, para valores compuestos devuelve falso.
\item \textbf{longValue():} Este método convierte este BigInteger en long.
\item \textbf{max():} Este método devuelve el máximo entre este BigInteger y val.
\item \textbf{min():} Este método devuelve el mínimo entre BigInteger y val.
\item \textbf{mod():} Este método devuelve un valor BigInteger para este mod m.
\item \textbf{modInverse():} Este método devuelve un BigInteger cuyo valor es 'this módulo inverso m'.
\item \textbf{modPow():} Este método devuelve un BigInteger cuyo valor es 'this exponente mod m'.
\item \textbf{multiply():} Este método devuelve un BigInteger al calcular el valor 'this *val'.
\item \textbf{negate():} Este método devuelve un BigInteger cuyo valor es '-this'.
\item \textbf{nextProbablePrime():} Este método devuelve el siguiente número primo mayor que este BigInteger.
\item \textbf{not():} Este método devuelve un BigInteger cuyo valor es '$\sim$this'.
\item \textbf{or():} Este método devuelve un BigInteger cuyo valor es 'this | val'
\item \textbf{pow():} Este método devuelve un BigInteger cuyo valor es 'this exponente'.
\item \textbf{probablePrime():} Este método devuelve un BigInteger primo positivo, con el bitLength especificado.
\item \textbf{remainder():} Este método devuelve un BigInteger cuyo valor es 'this \% val'.
\item \textbf{setBit():} Este método devuelve un BigInteger cuyo valor es igual a este BigInteger con el conjunto de bits designado.
\item \textbf{shiftLeft():} Este método devuelve un BigInteger cuyo valor es 'this $\ll$ val'.
\item \textbf{shiftRight():} Este método devuelve un BigInteger cuyo valor es 'this $\gg$ val'.
\item \textbf{signum():} Este método devuelve la función signum de este BigInteger.
\item \textbf{subtract():} Este método devuelve un BigInteger cuyo valor es 'this - val'.
\item \textbf{testbit():} Este método devuelve un valor booleano 'verdadero' si se establece el bit designado.
\item \textbf{toByteArray():} Este método devuelve un arreglo de bytes que contiene la representación en complemento a dos de este BigInteger.
\item \textbf{toString():} Este método devuelve la representación de cadena decimal de este BigInteger.
\item \textbf{valueOf():} Este método devuelve un BigInteger cuyo valor es equivalente al del long especificado.
\item \textbf{xor():} Este método devuelve un BigInteger al calcular el valor 'this \^ val'.
\end{itemize}

\begin{lstlisting}[language=Java]
import java.math.*;

public class Main {
	
   public static void main(String[] args) throws Exception {
      // Inicializar resultado
      BigInteger bigInteger = new BigInteger("1");
      int n = 4;
      for (int i = 2; i <= n; i++) {
          // devuelve un BigInteger calculando ?este *val ? valor.
          bigInteger = bigInteger.multiply(BigInteger.valueOf(i));
      }
      System.out.println("Factorial de 4 : " + bigInteger);
      // devuelve un valor booleano ? verdadero? si y solo 
      // si este BigInteger es primo
      BigInteger bigInteger2 = new BigInteger("197");
      System.out.println("IsProbablePrime method will return : " 
      + bigInteger2.isProbablePrime(2));
      // devuelve el siguiente entero primo que es mayor que este BigInteger.
      BigInteger nextPrimeNumber = bigInteger2.nextProbablePrime();
      System.out.println("Prime Number next to 197 : " + nextPrimeNumber);
      // devuelve el minimo entre este BigInteger y val
      BigInteger min = bigInteger.min(bigInteger2);
      System.out.println("Minimo valor : " + min);
      // devuelve el maximo entre este BigInteger y val
      BigInteger max = bigInteger.max(bigInteger2);
      System.out.println("Maximo valor: " + max);
		
      // Inicializar resultado 
      BigInteger bigInteger3 = new BigInteger("17");  
      //devuelve la funcion signum de este BigInteger  
      BigInteger bigInteger4 = new BigInteger("171");  
      System.out.println("El signo del valor "+bigInteger4+" : "
      + bigInteger4.signum());  
      BigInteger sub=bigInteger4.subtract(bigInteger3);  
      System.out.println(bigInteger4+"-"+bigInteger3+" : "+sub);  
		
      // devuelve el cociente despues de dividir dos valores bigInteger 
      BigInteger quotient=bigInteger4.divide(bigInteger3);  
      System.out.print(bigInteger4+" / "+
      bigInteger3+" :     Cociente : "+quotient);  
		
      //devuelve el resto despues de dividir dos valores bigIntger  
      BigInteger remainder=bigInteger4.remainder(bigInteger3);  
      System.out.println("       Resto : "+remainder);  
		
      //devuelve un BigInteger cuyo valor es ?este << val? 
      BigInteger shiftLeft=bigInteger3.shiftLeft(4);  
      System.out.println("Valor de desplazamiento a la izquierda: "+shiftLeft);  
   }
}
\end{lstlisting}