\subsection{Números coprimos}
En matemáticas, los números coprimos (números primos entre sí o primos relativos) son dos números enteros a y b que no tienen ningún factor primo en común. Dicho de otra manera, si no tienen otro divisor común más que 1 y -1. Equivalentemente son coprimos, si y solo si, su máximo común divisor (GCD) es igual a 1. 

\subsection{Descomposición en factores primos}
Por el teorema fundamental de la aritmética, cada entero positivo tiene una única descomposición en números primos (factores primos). 