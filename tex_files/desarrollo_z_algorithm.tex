 En esta guía, para evitar ambigüedades, asumimos el 0 como índice; es decir: el primer carácter de $s$ tiene índice $0$ y el ultimo tiene índice $n-1$. Realizada dicha aclaración vamos a proceder a resolver el problema planteado en la introducción  de la guía.
 
 Vamos a partir que el algoritmo que obtengamos como resultado le llamaremos \emph{algoritmo Z} o \emph{función Z}. Adicionalmente queremos que el  algoritmo sea capaz de generar un arreglo $z$de enteros con capacidad $n$ donde $z[i]$ es la longitud de la cadena más larga que es, al mismo tiempo, un prefijo de $s$ y un prefijo del sufijo de $s$ a partir de $i$. El primer elemento del algoritmo Z, $z[0]$ , generalmente no está bien definido. En esta guía asumiremos que es cero (aunque no cambia nada en la implementación del algoritmo).

Por ejemplo, aquí están los valores del algoritmo Z calculados para diferentes cadenas:

aaaa - $[0,~4,~3,~2,~1]$

aaabaab - $[0,~2,~1,~0,~2,~1,~0]$

abacaba - $[0,~0,~1,~0,~3,~0,~1]$

\subsection{Algoritmo trivial}

Simplemente iteramos a través de cada posición $i$ y actualizar $z[i]$ para cada uno de ellos, a partir de $z[i] = 0$ e incrementándolo siempre que no
encontremos una discrepancia (y siempre que no lleguemos al final de la línea).
Por supuesto, esta no es una implementación eficiente. Ahora mostraremos la construcción de una implementación eficiente.

\subsection{Algoritmo eficiente}

Para obtener un algoritmo eficiente calcularemos los valores de $z[i]$ a su vez de $i = 1$ a $n-1$ pero al mismo tiempo, al calcular un nuevo valor,
intentaremos hacer el mejor uso posible de los valores calculados previamente.


En aras de la brevedad, llamemos \textbf{coincidencias de segmentos} a aquellas subcadenas que coinciden con un prefijo de $s$. Por ejemplo, el valor del algoritmo Z deseada $z[i]$ es la longitud de la coincidencia del segmento que comienza en la posición $i$ (y eso termina en la posición $i+z[i]-1$ ).

Para hacer esto, mantendremos \textbf{el $[l, r)$ los índices del segmento más a la derecha coinciden}. Es decir, entre todos los segmentos detectados nos
quedaremos con el que termina más a la derecha. En cierto modo, el índice $r$ puede verse como el \emph{límite} a la que nuestra cadena $s$ ha sido
procesado por el algoritmo; todo lo que pase más allá de ese punto aún no se sabe.

Entonces, si el índice actual (para el cual tenemos que calcular el siguiente valor de la función Z) es i , tenemos una de dos opciones:

\begin{itemize}
	\item $i \ge r$: la posición actual está \textbf{fuera} de lo que ya hemos procesado. Luego calcularemos $z[i]$ con el algoritmo trivial (es decir, simplemente comparando valores uno por uno). Tenga en cuenta que al final, si $z[i] > 0$ , tendremos que actualizar los índices del segmento más a la derecha, porque está garantizado que el nuevo $r = i + z[i]$ es mejor que
	el anterior $r$.
	
	\item $i < r$: la posición actual está dentro de la coincidencia del segmento actual $[l, r)$.
	
	Luego podemos usar los valores $z$ ya calculados para inicializar el valor de $z[i]$ a algo (seguro que es mejor que comenzar desde cero), tal vez
	incluso a algún número grande.
	
	Para esto observamos que las subcadenas $s[l \dots r)$ y $s[0 \dots r-l)$ coincidencia. Esto significa que como aproximación inicial para $z[i]$
	podemos tomar el valor ya calculado para el segmento correspondiente $s[0 \dots r-l)$ , y eso es $z[i-l]$.
	
	Sin embargo, el valor $z[i-l]$ podría ser demasiado grande: cuando se aplica a la posición $i$ podría superar el índice $r$. Esto no está permitido porque no sabemos nada sobre los caracteres a la derecha de $r$  pueden diferir de los requeridos.
	
	Aquí hay un ejemplo de un escenario similar:
	
	$$ s = aaaabaa $$
	
	Cuando lleguemos a la última posición ($i=6$), el segmento de coincidencia actual será $[5,7)$. La posición 6 entonces coincidirá con la posición
	$6-5=1$ , para el cual el valor del algoritmo Z es $z[1]=3$. Obviamente, no podemos inicializar $z[6]$ a $3$ , sería completamente incorrecto. El valor máximo al que podríamos inicializarlo es $1$ porque es el valor más grande el que no nos lleva más allá del índice $r$ del segmento de coincidencia $[l, r)$. Por lo tanto, como \textbf{aproximación inicial} para $z[i]$ podemos tomar con seguridad:
	
	$$ z_0[i] = \min(r - i,\; z[i-l]) $$
	
	Después de haber $z[i]$ inicializado a $z_0[i]$ , intentamos incrementar $z[i]$ ejecutando el algoritmo trivial, porque en general, después del extremo $r$ , no podemos saber si el segmento seguirá coincidiendo o no.
	
\end{itemize}

Por lo tanto, todo el algoritmo se divide en dos casos, que difieren sólo en el valor inicial de $z[i]$: en el primer caso se supone que es cero, en el
segundo caso está determinado por los valores calculados previamente (usando la fórmula anterior). Después de eso, ambas opciones de este
algoritmo se pueden reducir a la implementación del el algoritmo trivial, que comienza inmediatamente después de que especificamos el valor
inicial. El algoritmo resulta muy sencillo. A pesar de que en cada iteración se ejecuta el algoritmo trivial, hemos logrado avances significativos al tener un
algoritmo que se ejecuta en tiempo lineal.

El algoritmo mantiene un rango $[l, r-1]$ tal que $s[l \dots x]$ sea un prefijo de $s$ y sea lo más grande posible. Como sabemos que $s[0 \dots y-x]$ y $s[x \dots y]$ son iguales, podemos usar esta información al calcular los valores $z$ para las posiciones $x+1, x+2, \dots, y$.

En cada posición $k$, primero verificamos el valor de $z[k-x]$. Si $k+z[k-x]<y$, sabemos que $z[k] = z[k-x]$. Sin embargo, si $k+z[k-x] \ge y$, $s[0 \dots y-k]$ es igual a $s[k \dots y]$, y para determinar el valor de $z[k]$ necesitamos comparar las subcadenas carácter a carácter. Aún así, el algoritmo funciona en tiempo lineal , porque comenzamos a comparar en las posiciones $y-k+1$ y $y+1$.