El tiempo de ejecución de un algoritmo va a depender de diversos factores
como son: los datos de entrada que le suministremos, la calidad del código
generado por el compilador para crear el programa objeto, la naturaleza y rapidez
de las instrucciones máquina del procesador concreto que ejecute el programa, y la
 complejidad intrínseca del algoritmo. Hay dos estudios posibles sobre el tiempo:

\begin{enumerate}
	\item Uno que proporciona una medida teórica (a priori), que consiste en obtener una
función que acote (por arriba o por abajo) el tiempo de ejecución del algoritmo
para unos valores de entrada dados.
	\item Y otro que ofrece una medida real (a posteriori), consistente en medir el tiempo
de ejecución del algoritmo para unos valores de entrada dados y en un
ordenador concreto.
\end{enumerate}

En el caso nuestro vamos a utilizar la primera variante.

\subsection{Tamaño de la entrada}
El tamaño de la entrada es el número de componentes sobre los que
se va a ejecutar el algoritmo. Por ejemplo, la dimensión del vector a ordenar o el
tamaño de las matrices a multiplicar.