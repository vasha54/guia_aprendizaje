\subsection{C++}

\subsubsection{Declaración y creación de arreglos}
El arreglo se puede declarar según la situación. Ahora veremos cada una de ellas:

\begin{lstlisting}[language=C++]
  /* Sabemos la cantidad de elementos a priori,  
  <tipo_de _dato> <nombre_arreglo> [<cantidad>];*/
  bool isPimes[100];
	
  /*La cantidad de elementos puede variar y depende del 
    valor de una variable.Es la mas usada
    <tipo_de _dato> * <nombre_arreglo>= new <tipo_de _dato> [<variable>];
    <tipo_de _dato> * <nombre_arreglo>= new <tipo_de _dato> [<cantidad>];
  */
  int cantidadMaxima=100;
  int * notas;
  notas=new int [cantidadMaxima]; 
  double * promedio=new double [100];
	
  /*Cuando conocemos los valores que integran el arreglo*/
  string nombres []={ "Luis", "Ernesto", "Susana" };
\end{lstlisting}

\subsubsection{Almacenar valor en el arreglo}
Para almacenar un valor en el arreglo solo debemos inidicar la posición en el arreglo en que se va almacenar dicha posición debe ser un valor o variable entera y debe estar en rango $[0,capacidad-1]$.

\begin{lstlisting}[language=C++]
   /* <nombre arreglo>[<posicion>] = <valor>; donde:
    <posicion> puede ser una variable entera, valor entero literal, expresion cuyo resultado sea entero 
    <valor> puede ser una variable entera , valor entero literal, expresion cuyo resultado sea entero */
   notas[23]=cantidadMaxima;
   notas[cantidadMaxima-50]=34;
   notas[10+5]=2*cantidadMaxima;
\end{lstlisting}

\subsubsection{Obtener valor en el arreglo}

Para obtener un valor en el arreglo solo debemos inidicar la posición en el arreglo en que se va obtener dicha posición debe ser un valor o variable entera y debe estar en rango $[0,capacidad-1]$ .

\begin{lstlisting}[language=C++]
   /* <variable> = <nombre arreglo>[<posicion>]; donde:
    <posicion> puede ser una variable entera, valor entero literal, expresion cuyo resultado sea entero 
    <variable> va ser la el lugar donde se almacenara una copia del valor solicitado al arreglo, debe ser del mismo tipo de dato del arreglo */
   int a =notas[23];
   int b;
   b = notas[cantidadMaxima-50];
\end{lstlisting}

\subsubsection{Recorrer todos los elementos del arreglo}

Para recorrer todos los elementos o parte de los elementos almacenados en el arreglo vamos utilizar la instrucción \textbf{for}

\begin{lstlisting}[language=C++]
   /* for( int i=<posicion_inicial>; i < <posicion_final>; i++ ) 
      i++ si <posicion_inicial> <= <posicion_final>
      i-- si <posicion_inicial> > <posicion_final>
      i <= <posicion_final> si deseo incluir en el rango la ultima posicion pero tiene que ser una posicion valida del arreglo
   */	
	
   for(int i=0;i<cantidadMaxima;i++){
      //notas[i] accedo al valor almacenado en la inesima posicion del arreglo
   }
\end{lstlisting}

Noten que acuerdo como hagan el \textbf{for} pueden recorrer del principio al final o en sentido contrario

\subsubsection{Leer los valores y almacenarlos en el arreglo}

Para leer los valores consola y almacernalos directamente en el arreglo vamos a utilizar una combinación de recorrer todos los elementos y almacenar en arreglo visto anteriormente

\begin{lstlisting}[language=C++]
   /* En cada iteracion del for el cin lee un valor y dicho valor es almacenado en el arreglo en la posicion que indique el valor de la variable i en esa iteracion, en la primera el valor seria 0, en la segunda 1 y asi sucesivamente. Tener en cuenta que los valores de i esten en rango de posiciones validas del arreglo */
   for(int i=0;i<cantidadMaxima;i++){
      cin>>notas[i];
   }
\end{lstlisting}

\subsubsection{Imprimir los valores almacenados en el arreglo}

Para imprimir los valores en consola del arreglo vamos a utilizar una combinación de recorrer todos los elementos y obtener en arreglo visto anteriormente

\begin{lstlisting}[language=C++]
   /* En cada iteracion del for el cin imprimir un valor que se corresponde con el almacenado en el arreglo en la posicion que indique el valor de la variable i en esa iteracion, en la primera el valor seria 0, en la segunda 1 y asi sucesivamente. Tener en cuenta que los valores de i esten en rango de posiciones validas del arreglo */
   for(int i=0;i<cantidadMaxima;i++){
      cout<<notas[i]<<" ";
   }
   cout<<endl;
   for(int i=0;i<cantidadMaxima;i++){
      cout<<notas[i]<<endl;
   } 
\end{lstlisting}

En primer \textbf{for} se imprime todos los valores del arreglo en una sola linea separados por espacios, mientras en el segundo cada valor almacenado en el arreglo se imprime por linea.

\subsection{Java}

\subsubsection{Declaración y creación de arreglos}

El arreglo se puede declarar según la situación. Ahora veremos cada una de ellas:

\begin{lstlisting}[language=C++]
   /*La cantidad de elementos puede variar y depende del 
    valor de una variable.Es la mas usada
   <tipo_de _dato> [] <nombre_arreglo>= new <tipo_de _dato> [<variable>];
   <tipo_de _dato> [] <nombre_arreglo>= new <tipo_de _dato> [<cantidad>];*/
   int cantidadMaxima=100;
   int [] notas;
   notas=new int [cantidadMaxima]; 
   double [] promedio=new double [100];
   
   /*Cuando conocemos los valores que integran el arreglo*/
   String [] nombres ={ "Luis", "Ernesto", "Susana" };
\end{lstlisting}

\subsubsection{Almacenar valor en el arreglo}

Para almacenar un valor en el arreglo solo debemos inidicar la posición en el arreglo en que se va almacenar dicha posición debe ser un valor o variable entera y debe estar en rango $[0,capacidad-1]$.

\begin{lstlisting}[language=Java]
   /* <nombre arreglo>[<posicion>] = <valor>; donde:
    <posicion> puede ser una variable entera, valor entero literal, expresion cuyo resultado sea entero 
    <valor> puede ser una variable entera , valor entero literal, expresion cuyo resultado sea entero*/
   notas[23]=cantidadMaxima;
   notas[cantidadMaxima-50]=34;
   notas[10+5]=2*cantidadMaxima;
\end{lstlisting}


\subsubsection{Obtener valor en el arreglo}

Para obtener un valor en el arreglo solo debemos inidicar la posición en el arreglo en que se va obtener dicha posición debe ser un valor o variable entera y debe estar en rango $[0,capacidad-1]$ .

\begin{lstlisting}[language=Java]
   /* <variable> = <nombre arreglo>[<posicion>]; donde:
   <posicion> puede ser una variable entera, valor entero literal, expresion cuyo resultado sea entero 
   <variable> va ser la el lugar donde se almacenara una copia del valor solicitado al arreglo, debe ser del mismo tipo de dato del arreglo */
   int a =notas[23];
   int b;
   b = notas[cantidadMaxima-50];
\end{lstlisting}

\subsubsection{Recorrer todos los elementos del arreglo}

Para recorrer todos los elementos o parte de los elementos almacenados en el arreglo vamos utilizar la instrucción \textbf{for}

\begin{lstlisting}[language=Java]
   /* for( int i=<posicion_inicial>; i < <posicion_final>; i++ ) 
    i++ si <posicion_inicial> <= <posicion_final>
    i-- si <posicion_inicial> > <posicion_final>
    i <= <posicion_final> si deseo incluir en el rango la ultima posicion pero tiene que ser una posicion valida del arreglo
   */	
	
   for(int i=0;i<cantidadMaxima;i++){
      //notas[i] accedo al valor almacenado en la inesima posicion del arreglo
   }
\end{lstlisting}

Noten que acuerdo como hagan el \textbf{for} pueden recorrer del principio al final o en sentido contrario

\subsubsection{Leer los valores y almacenarlos en el arreglo}
Para leer los valores consola y almacernalos directamente en el arreglo vamos a utilizar una combinación de recorrer todos los elementos y almacenar en arreglo visto anteriormente

\begin{lstlisting}[language=C++]
   /* En cada iteracion del for el cin lee un valor y dicho valor es almacenado en el arreglo en la posicion que indique el valor de la variable i en esa iteracion, en la primera el valor seria 0, en la segunda 1 y asi sucesivamente. Tener en cuenta que los valores de i esten en rango de posiciones validas del arreglo */
   Scanner in =new Scanner(System.in);
   for(int i=0;i<cantidadMaxima;i++){
      notas[i] = in.nextInt();
   }
\end{lstlisting}

\subsubsection{Imprimir los valores almacenados en el arreglo}

Para imprimir los valores en consola del arreglo vamos a utilizar una combinación de recorrer todos los elementos y obtener en arreglo visto anteriormente

\begin{lstlisting}[language=Java]
   /* En cada iteracion del for el cin imprimir un valor que se corresponde con el almacenado en el arreglo en la posicion que indique el valor de la variable i en esa iteracion, en la primera el valor seria 0, en la segunda 1 y asi sucesivamente. Tener en cuenta que los valores de i esten en rango de posiciones validas del arreglo */
   for(int i=0;i<cantidadMaxima;i++){
      System.out.print( notas[i]+" ");
   }
   System.out.println();
   for(int i=0;i<cantidadMaxima;i++){
      System.out.println(notas[i]);
   } 
\end{lstlisting}

En primer \textbf{for} se imprime todos los valores del arreglo en una sola linea separados por espacios, mientras en el segundo cada valor almacenado en el arreglo se imprime por linea.