En varios ejercicios o problemas de concursos se nos puede presentar la situación que tenemos un árbol donde cada nodo o aristas tiene un valor y se nos pide un grupo de consulta de la forma $(a,b)$ donde $a$ y $b$ son nodos del árbol y tenemos que buscar en el único camino que existe entre $a$ y $b$ en al árbol aquellos valores de los nodos o aristas que conforman el camino determinada propiedad. La idea trivial sería realizar una DFS de $a$ a $b$ y buscar la solución pero debido a la cantidad de de consultas y lo grande que puede ser el árbol (en la mayoría de los casos hasta $10^5$ nodos) esta solución puede que su tiempo sea superior al tiempo limite permitido.

En la siguiente guía abordaremos la \textbf{descomposición pesada-ligera (\emph{Heavy-light decomposition})} es una técnica bastante general que nos permite resolver de manera efectiva muchos problemas que se reducen a consultas en un árbol.