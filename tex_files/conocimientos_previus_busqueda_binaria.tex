\paragraph{Arreglos} Es un tipo de datos estructurado que está formado de una colección finita y ordenada de datos del mismo tipo. Es la estructura natural para modelar listas de elementos iguales. Están formados por un conjunto de elementos de un mismo tipo de datos que se almacenan bajo un mismo nombre, y se diferencian por la posición que tiene cada elemento dentro del arreglo de datos. Al declarar un arreglo, se debe inicializar sus elementos antes de utilizarlos. Para declarar un arreglo tiene que indicar su tipo, un nombre único y la cantidad de elementos que va a contener. Accediendo a su indice podemos acceder a cada elemento del arreglo, conociendo que el primer elemento tiene indice 0. En muchas ocasiones para buscar un elemento que desconocemos su indice debemos hacer uso de bucles.

 