\subsection{Autómata (una máquina de estados finitos)}
Un autómata finito (AF) o máquina de estado finito es un modelo computacional que realiza cómputos en forma automática sobre una entrada para producir una salida.

Este modelo está conformado por un alfabeto, un conjunto de estados finito, una función de transición, un estado inicial y un conjunto de estados finales. Su funcionamiento se basa en una función de transición, que recibe a partir de un estado inicial una cadena de caracteres pertenecientes al alfabeto (la entrada), y que va leyendo dicha cadena a medida que el autómata se desplaza de un estado a otro, para finalmente detenerse en un estado final o de aceptación, que representa la salida. 

\subsection{Donald Knuth}
Es un reconocido experto en ciencias de la computación estadounidense y matemático, famoso por su fructífera investigación dentro del análisis de algoritmos y compiladores.

\subsection{Vaughan Ronald Pratt}
Profesor Emeritus en la Universidad Stanford, es un pionero en el campo de informática. Publicando desde 1969, Pratt ha hecho varias contribuciones a áreas fundacionales como algoritmos de búsqueda, algoritmos de ordenación, y tests de primalidad. Más recientemente su búsqueda se ha centrado en el modelado formal de sistemas concurrentes y espacios de Chu. Un patrón de aplicar modelos de áreas diversas de las matemáticas como geometría, álgebra lineal, álgebra abstracta, y especialmente lógica matemática a informática se extiende por su trabajo. 