Levantamiento $a$ a la potencia de $n$ se expresa ingenuamente como multiplicación por $a$ hecho $n-1$
 veces: $a^{n} = a \cdot a \cdot \ldots \cdot a$. Sin embargo, este enfoque no es práctico para grandes $a$ o $n$.

$$a^{2b} = a^b \cdot a^b = (a^b)^2$$


La idea de la exponenciación binaria es que dividimos el trabajo usando la representación binaria del exponente.

Vamos a escribir $n$ en base 2, por ejemplo:

$$3^{13} = 3^{1101_2} = 3^8 \cdot 3^4 \cdot 3^1$$

Desde el número $n$ tiene exactamente $\lfloor \log_2 n \rfloor + 1$ dígitos en base 2, solo necesitamos realizar $O(\log n)$ multiplicaciones, si conocemos las potencias $a^1, a^2, a^4, a^8, \dots, a^{2^{\lfloor \log n \rfloor}}$.

Entonces sólo necesitamos conocer una forma rápida de calcularlos. Por suerte esto es muy fácil, ya que un elemento de la secuencia es sólo el cuadrado del elemento anterior.

\begin{align}
	3^1 &= 3 \\
	3^2 &= \left(3^1\right)^2 = 3^2 = 9 \\
	3^4 &= \left(3^2\right)^2 = 9^2 = 81 \\
	3^8 &= \left(3^4\right)^2 = 81^2 = 6561
\end{align}

Entonces, para obtener la respuesta final para $3^{13}$, sólo necesitamos multiplicar tres de ellos (omitiendo $3^2$ porque el bit correspondiente en $n$ no está configurado): $3^{13} = 6561 \cdot 81 \cdot 3 = 1594323$


