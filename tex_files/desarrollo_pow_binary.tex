La generalización más obvia: los restos de un determinado módulo (obviamente, la asociatividad se conserva). Lo siguiente en la \emph{popularidad} es una generalización del producto matricial (es una asociatividad bien conocida).

Tenga en cuenta que para cualquier número de una identidad obvia factible de un número par (que se deduce de la asociatividad de la multiplicación):

$a^{n}=(a^{n/2})^{2}=a^{n/2} *a^{n/2} $

Es el principal método de exponenciación binaria. De hecho, incluso para n Hemos demostrado cómo, después de realizar solo una operación de multiplicación, podemos reducir el problema a menos de la mitad de la potencia. Queda por entender qué hacer si el grado de n es impar. Aquí lo que hacemos es muy simple: ve en la medida n-1 Eso tendrá incluso:

$a^{n}=a^{n-1}*a$

Entonces, en realidad encontramos una fórmula recursiva: el grado de n que vamos, si es igual a n/2 y de lo contrario, a n-1. Está claro que no habrá más transiciones 2 $\log( n)$ antes de que lleguemos a n = 0 (basado en la fórmula recursiva).