\subsection{Sentencias y expresiones}

Una \textbf{expresión} es un conjunto variables unidos por operadores. Son órdenes que se le dan al
computador para que realice una tarea determinada.

Una \textbf{sentencia} es una \textbf{expresión} que acaba en \textbf{punto y coma} (;). Se permite incluir varias sentencias en una línea, aunque lo habitual es utilizar una línea para cada sentencia.

Las \textbf{expresiones} de C++ y Java son unidades o componentes elementales de unas entidades de rango
superior que son las \textbf{sentencias}. Las \textbf{sentencias} son unidades completas, ejecutables en sí
mismas. Ya se verá que muchos tipos de \textbf{sentencias} incorporan \textbf{expresiones} aritméticas,
lógicas o generales como componentes de dichas \textbf{sentencias}.

\subsection{Expresiones Aritméticas}

Están formadas por variables y/o constantes, y distintos operadores aritméticos e
incrementales (+, -, *, /, \%, ++, - -). Como se ha dicho, también se pueden emplear paréntesis de tantos niveles como se desee, y su interpretación sigue las normas
aritméticas convencionales. Por ejemplo, la solución de la ecuación de segundo grado:

$$ x = \frac{-b+\sqrt{b^2-4ac}}{2a} $$

se escribe, en C++ en la forma:
\begin{lstlisting}[language=C++]
x=(-b+sqrt((b*b)-(4*a*c)))/(2*a);
\end{lstlisting}

donde, estrictamente hablando, sólo lo que está a la derecha del operador de asignación (=) es
una expresión aritmética. El conjunto de la variable que está a la izquierda del signo (=), el
operador de asignación, la expresión aritmética y el carácter (;) constituyen una sentencia. En
la expresión anterior aparece la llamada a la función de librería sqrt(), que tiene como valor
de retorno la raíz cuadrada de su único argumento. En las expresiones se pueden introducir
espacios en blanco entre operandos y operadores; por ejemplo, la expresión anterior se puede
escribir también de la forma:

\begin{lstlisting}[language=C++]
x = (-b + sqrt((b * b) - (4 * a * c)))/(2 * a);
\end{lstlisting}

\subsection{Expresiones Lógicas}
Los elementos con los que se forman estas expresiones son \textbf{valores lógicos}; verdaderos (\textbf{true},
o distintos de 0) y falsos (\textbf{false}, o iguales a 0), y los operadores lógicos ||, \&\& y !. También
se pueden emplear los \textbf{operadores relacionales} (<, >, <=, >=, ==, !=) para producir
estos valores lógicos a partir de valores numéricos. Estas expresiones equivalen siempre a un
valor 1 (\textbf{true}) o a un valor 0 (\textbf{false}). Por ejemplo:

\begin{lstlisting}[language=C++]
a = ((b>c)&&(c>d))||((c==e)||(e==b));
\end{lstlisting}

donde de nuevo la expresión lógica es lo que está entre el operador de asignación (=) y el (;).
La variable \textbf{a} valdrá (1,true) si \textbf{b} es mayor que \textbf{c} y \textbf{c} mayor que \textbf{d}, ó si \textbf{c} es igual a \textbf{e} ó \textbf{e} es igual a \textbf{b}.

\subsection{Expresiones generales}
Una de las características más importantes (y en ocasiones más difíciles de manejar) del C++ o Java es
su flexibilidad para combinar expresiones y operadores de distintos tipos en una expresión que
se podría llamar general, aunque es una expresión absolutamente ordinaria.

Recuérdese que el resultado de una expresión lógica es siempre un valor numérico (un 1
ó un 0); esto permite que cualquier expresión lógica pueda aparecer como sub-expresión en
una expresión aritmética. Recíprocamente, cualquier valor numérico puede ser considerado
como un valor lógico: \textbf{true} si es distinto de 0 y \text{false} si es igual a 0. Esto permite introducir
cualquier expresión aritmética como sub-expresión de una expresión lógica. Por ejemplo:

\begin{lstlisting}[language=C++]
(a - b*2.0) && (c != d)
\end{lstlisting}

A su vez, el operador de asignación (=), además de introducir un nuevo valor en la
variable que figura a su izda, deja también este valor disponible para ser utilizado en una
expresión más general. Por ejemplo, supóngase el siguiente código que inicializa a 1 las tres
variables \textbf{a}, \textbf{b} y \textbf{c}:

\begin{lstlisting}[language=C++]
a = b = c = 1;
\end{lstlisting}

que equivale a:

\begin{lstlisting}[language=C++]
a = (b = (c = 1));
\end{lstlisting}	

En realidad, lo que se ha hecho ha sido lo siguiente. En primer lugar se ha asignado un
valor unidad a \textbf{c}; el resultado de esta asignación es también un valor unidad, que está
disponible para ser asignado a \textbf{b}; a su vez el resultado de esta segunda asignación vuelve a
quedar disponible y se puede asignar a la variable \textbf{a}.

\subsection{Reglas de precedencia y asociatividad}
El resultado de una expresión depende del orden en que se ejecutan las operaciones. El
siguiente ejemplo ilustra claramente la importancia del orden. Considérese la expresión:

\begin{lstlisting}[language=C++]
3 + 4 * 2
\end{lstlisting}

Si se realiza primero la suma (3+4) y después el producto (7*2), el resultado es 14; si se
realiza primero el producto (4*2) y luego la suma (3+8), el resultado es 11. Con objeto de que
el resultado de cada expresión quede claro e inequívoco, es necesario definir las reglas que
definen el orden con el que se ejecutan las expresiones de C++ y Java. Existe dos tipos de reglas para
determinar este orden de evaluación: las reglas de \textbf{precedencia} y de \textbf{asociatividad}. Además, el
orden de evaluación puede modificarse por medio de paréntesis, pues siempre se realizan
primero las operaciones encerradas en los paréntesis más interiores. Los distintos operadores
de C++ y Java se ordenan según su distinta precedencia o prioridad; para operadores de la misma
precedencia o prioridad, en algunos el orden de ejecución es de izquierda a derecha, y otros de
derecha a izquierda (se dice que se asocian de izda a dcha, o de dcha a izda). A este orden se
le llama \textbf{asociatividad}.

\subsection{Sentencias simples}
Una sentencia simple es una expresión de algún tipo terminada con un carácter (;). Un caso
típico son las declaraciones o las sentencias aritméticas. Por ejemplo:

\begin{lstlisting}[language=C++]
float real;
espacio = espacio_inicial + velocidad * tiempo;
\end{lstlisting}

\subsection{Sentencias nulas o vacías}

En algunas ocasiones es necesario introducir en el programa una sentencia que ocupe un
lugar, pero que no realice ninguna tarea. A esta sentencia se le denomina sentencia vacía y
consta de un simple carácter (;). Por ejemplo:

\begin{lstlisting}[language=C++]
;
\end{lstlisting}

\subsection{Sentencias compuestas o bloques}
Muchas veces es necesario poner varias sentencias en un lugar del programa donde debería
haber una sola. Esto se realiza por medio de sentencias compuestas. Una sentencia compuesta
es un conjunto de declaraciones y de sentencias agrupadas dentro de llaves {...}. También
se conocen con el nombre de bloques. Una sentencia compuesta puede incluir otras
sentencias, simples y compuestas. Un ejemplo de sentencia compuesta es el siguiente:

\begin{lstlisting}[language=C++]
{
   int i = 1, j = 3, k;
   double masa;
   masa = 3.0;
   k = y + j;
}
\end{lstlisting}

Las sentencias compuestas se utilizarán con mucha frecuencia en al
introducir las sentencias que permiten modificar el flujo de control del programa o cuando se programa de forma modular o funcional.

\subsection{Comentarios}
Existen dos formas diferentes de introducir comentarios entre el código de Java o C++. Los
comentarios son tremendamente útiles para poder entender el código utilizado, facilitando de ese
modo futuras revisiones y correcciones. Además permite que cualquier persona distinta al
programador original pueda comprender el código escrito de una forma más rápida. Se recomienda
acostumbrarse a comentar el código desarrollado. De esta forma se simplifica también la tarea de
estudio y revisión posteriores.

Java y C++ interpreta que todo lo que aparece a la derecha de dos barras “//” en una línea cualquiera
del código es un comentario del programador y no lo tiene en cuenta. El comentario puede empezar
al comienzo de la línea o a continuación de una instrucción que debe ser ejecutada. La segunda
forma de incluir comentarios consiste en escribir el texto entre los símbolos /*...*/. Este segundo
método es válido para comentar más de una línea de código. Por ejemplo:

\begin{lstlisting}[language=C++]
// Esta linea es un comentario
int a=1; // Comentario a la derecha de una sentencia

// Esta es la forma de comentar mas de una linea utilizando
// las dos barras. Requiere incluir dos barras al comienzo de cada linea

/* Esta segunda forma es mucho mas comoda para comentar un numero elevado de lineas ya que solo requiere modificar el comienzo y el final. */
\end{lstlisting}