El trabajo con cadenas de carácteres es una de las áreas de conocimiento que son abordados en los problemas de concursos por lo que es necesario conocer al menos el funcionamiento del tipo de dato \textbf{string} sus métodos y algoritmos que trabajan con este tipo de dato.

Como se pueden dar cuenta, así de fácil se pueden llegar a manipular las cadenas o el tipo string en C++ o Java, aunque hay una (muy) extensa lista de métodos e implementaciones, estas nos parecen una buena introducción, además de ser las más comunes y conocidas, pueden consultar más \href{http://www.cplusplus.com/reference/string/string/}{página de referencia oficial del lenguaje C++} para el caso de Java puede consultar la siguiente \href{https://www.w3schools.com/java/java_ref_string.asp}{página web}  . 