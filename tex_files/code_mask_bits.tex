\subsection{C++}
\lstset{language=C++, breaklines=true, basicstyle=\footnotesize}
\begin{lstlisting}[language=C++]
	vector<char> s('a','b','c');
	int n= s.size();
	for (int i=0; i< (1<<n); i++)
	{
		int number=i;
		vector<char> k;
		for (int e=0; e< n; e++)
		{
			if( number & (1<< e) )
			{
				k.push_back(s[e]);
			}
		}
		/*Ya  k tiene los elementos que conforma el subconjunto i-nesimo*/
		cout<< "El sub-conjunto "<<e+1<<"es:"<<endl;
		
		cout<<"{ ";
		for(int h=0; h<k.size(); h++)
		    cout<<k[h]<<" ";	
		cout<<"}"<<endl;	
	}
\end{lstlisting}

\subsection{Java}
\begin{lstlisting}[language=C++]
	char [] s= {'a','b','c'};
	int n= s.length;
	for (int i=0; i< (1<<n); i++)
	{
		int number=i;
		ArrayList<Character> k =new ArrayList<Character>() ;
		for (int e=0; e< n; e++)
		{
			if( (number & (1<< e))!=0 )
			{
				k.add(s[e]);
			}
		}
		/*Ya  k tiene los elementos que conforma el subconjunto i-nesimo*/
		System.out.println("El sub-conjunto "+(e+1)+"es:");
		
		System.out.print("{ ");
		for(int h=0; h<k.size(); h++)
			System.out.print(k.get(h)+" ");	
		System.out.println("}");	
	}
\end{lstlisting}
