La teoría de juego es rama de la matemática y la lógica que se ocupa del análisis de juegos (ejemplo: Las situaciones que implica partes con conflictos de intereses). Además de la elegancia matemática y soluciones completas para juegos simples, los principios de la teoría de juego encuentran también aplicación en juegos mas complicados como son los naipes (cartas), damas y ajedrez, así como también los problemas realmente mundiales tan diverso como la economía, la división de la propiedad, la política, y la guerra. En esta guía abordaremos los elementos esenciales de esta teoría a través de determinados juegos. 