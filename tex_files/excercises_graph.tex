Para este tema no existe ejercicios propuestos pero si es importante dominar el contenido abordado en esta sección ya que es la base para solucionar los problemas de teoría de grafos los cuales son algoritmos que parten de trabajar con un grafo el cual debe estar representado en una de las variantes abordadas en esta guía.

Algunos de los problemas más conocidos de grafos son:

\begin{itemize}
	\item Conectividad Simple: Consiste en estudiar si el grafo es conexo, es decir, si existe al menos un camino entre cada par de vértices.
	
	\item Detección de Ciclos: Consiste en estudiar la existencia de al menos un ciclo en el grafo.
	
	\item Camino Simple: Consiste en estudiar la existencia de un camino entre dos vértices cualquiera.
	
	\item Camino de Euler: Consiste en estudiar la existencia de un camino que conecte dos vértices dados usando cada arista del grafo exactamente una sola vez. Si el camino tiene como inicio y final el mismo vértice, entonces se desea encontrar un tour de Euler.
	
	\item Camino de Hamilton: Consiste en estudiar la existencia de un camino que conecte dos vértices dados que visite cada nodo del grafo exactamente una vez. Si el camino tiene como inicio y final el mismo vértice, entonces se desea encontrar un tour de Hamilton.
	
	\item Conectividad Fuerte en Dígrafos: Consiste en estudiar si hay un camino dirigido conectando cada par de vértices del dígrafo. Inclusive se puede estudiar si existe un camino dirigido entre cada par de vértices, en ambas direcciones.
	
	\item Clausura Transitiva: Consiste en tratar de encontrar un conjunto de vértices que pueda ser alcanzado siguiendo aristas dirigidas desde cada vértice del dígrafo.
	
	\item Árbol de Expansión Mínima: Consiste en encontrar, en un grafo pesado, el conjunto de aristas de peso mínimo que conecta a todos los vértices.
	
	\item Caminos cortos a partir de un mismo origen: Consiste en encontrar cuales son los caminos más cortos conectando a un vértice v cualquier con cada uno de los otros vértices de un dígrafo pesado. Este es un problema que por lo general se presenta en redes de computadores, representadas como grafos.
	
	\item Planaridad: Consiste en estudiar si un grafo puede ser dibujado sin que ninguna de las líneas que representan las aristas se intercepten.
	
	\item Pareamiento (Matching): Dado un grafo, consiste en encontrar cual es el subconjunto más largo de sus aristas con las propiedad de que no haya dos conectados al mismo vértice. Se sabe que este problema clásico es resoluble en tiempo proporcional a una función polinomial en el número de vértices y de aristas, pero aun no existe un algoritmo rápido que se ajuste a grandes grafos.
	
	\item Ciclos Pares en Dígrafos: Consiste en encontrar en un dígrafo un camino de longitud par. Este problema puede lucir simple ya que la solución para grafos no dirigidos es sencilla. Sin embargo, aun no se conoce si existe un algoritmo eficiente para resolverlo.
	
	\item Asignación: Este problema se conoce también como pareamiento bipartito pesado (bipartite weigthed matching). Consiste en encontrar un pareamiento perfecto de peso mínimo en un grafo bipartito. Un grafo bipartito es aquel cuyos vértices se pueden separar en dos conjuntos, de tal manera que todas las aristas conecten a un vértice en un conjunto con otro vértice en el otro conjunto.
	
	\item Conectividad General: Consiste en encontrar el número mínimo de aristas que al ser removidas separarán el grafo en dos partes disjuntas (conectividad de aristas). También se puede encontrar el número mínimo de nodos que al ser removidos separarán el grafo en dos partes disjuntas (conectividad de nodos).
	
	\item El camino más largo: Consiste en encontrar cual es el camino más largo que conecte a dos nodos dados en el grafo. Aunque parece sencillo, este problema es una versión del problema del tour de Hamilton y es NP-hard.
	
	\item Colorabilidad: Consiste en estudiar si existe alguna manera de asignar k colores a cada uno de los vértices de un grafo, de tal forma de que ninguna arista conecte dos vértices del mismo color. Este problema clásico es fácil para k=2 pero es NP-hard para k=3.
	
	\item Conjunto Independiente: Consiste en encontrar el tamaño del mayor subconjunto de nodos de un grafo con la propiedad de que no haya ningún par conectado por una arista. Este problema es NP-hard.
	
	\item Clique: Consiste en encontrar el tamaño del clique (subgrafo completo) más grande en un grafo dado.
	
	\item Isomorfismo de grafos: Consiste en estudiar la posibilidad de hacer dos grafos idénticos con solo renombrar sus nodos. Se conocen algoritmos eficientes para solucionar este problema, para varios clases particulares de grafos, pero no se tiene solución para el problema general. Este problema es NP-hard .
\end{itemize}