Con la factorización de un número {\em N} en números primos es posible hallar la cantidad de divisores de {\em N} y la suma de estos. Veamos:

Si descomponemos N en $b_{1}^{a_{1}}$*$b_{2}^{a_{2}}$*$b_{3}^{a_{3}}$* ... * $b_{m}^{a_{m}}$

La cantidad de divisores de {\em N} sería :

$(a_{1}+1)$*$(a_{2}+1)$*$(a_{3}+1)$* ... * $(a_{m}+1)$

Por ejemplo 12=2$^{2}$*3$^{1}$ , (2+1)*(1+1)= 6 el 12 tiene 6 divisores: 1, 2, 3, 4, 6, 12.

La suma de los divisores de N es igual a:

$$ suma = \frac{ \frac{ \frac{{b_1}^{a_1+1}-1}{b_1 - 1}  *{b_2}^{a_2+1}-1}{b_2 - 1} *{b_3}^{a_3+1}-1}{b_3 - 1} * \dots \frac{{b_m}^{a_m+1} - 1}{b_m - 1} $$

Para el caso de 12:

$$ suma= \frac{\frac{2^{2+1}-1}{2-1}*3^{1+1}-1}{3-1} = \frac{\frac{2^3-1}{1}*3^2-1}{2} = \frac{\frac{8-1}{1}*9-1}{2}= 4*7 = 28 $$

Comprobación los divisores de 12 son 1+2+3+4+6+12 = 28.