Los algoritmos greedy normalmente (pero no siempre) no logran encontrar la solución globalmente óptima porque, por lo general, no operan de manera exhaustiva en todos los datos. Pueden comprometerse con ciertas opciones demasiado pronto, lo que les impide encontrar la mejor solución general más adelante. Por ejemplo, todos los algoritmos de coloración greedy conocidos para el problema de coloración de grafos y todos los demás problemas NP-completos no encuentran soluciones óptimas de manera consistente. Sin embargo, son útiles porque son rápidos de pensar y, a menudo, dan buenas aproximaciones al óptimo.

Si se puede demostrar que un algoritmo greedy produce el óptimo global para una clase de problema dada, generalmente se convierte en el método de elección porque es más rápido que otros métodos de optimización como la programación dinámica. Ejemplos de tales algoritmos greedy son el algoritmo de Kruskal y el algoritmo de Prim para encontrar árboles de expansión mínimos y el algoritmo para encontrar árboles de Huffman óptimos.

Los algoritmos greedy también aparecen en el enrutamiento de la red. Mediante el enrutamiento codicioso, se reenvía un mensaje al nodo vecino que está "más cerca" del destino. La noción de ubicación de un nodo (y, por lo tanto, cercanía) puede determinarse por su ubicación física, como en el enrutamiento geográfico utilizado por redes ad hoc. La ubicación también puede ser una construcción completamente artificial como en el enrutamiento de mundo pequeño y la tabla hash distribuida.