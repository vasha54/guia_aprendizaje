\subsection{Leonhard Euler}
Leonhard Euler fue un matemático y físico suizo. Se trata del principal matemático del siglo XVIII y uno de los más grandes y prolíficos de todos los tiempos, muy conocido por el número de Euler (e), número que aparece en muchas fórmulas de cálculo y física.

\subsection{Camino en un grafo}
En teoría de grafos, un camino (en inglés, \emph{walk},y en ocasiones traducido también como recorrido) es una sucesión de vértices y aristas dentro de un grafo, que empieza y termina en vértices, tal que 
cada vértice es incidente con las aristas que le siguen y le preceden en la secuencia. Dos vértices están
conectados o son accesibles si existe un camino que forma una trayectoria para llegar de uno al otro; en
caso contrario, los vértices están desconectados o bien son inaccesibles.

\subsection{Ciclo en un grafo}
La palabra ciclo se emplea en teoría de grafos para indicar un camino cerrado en un grafo, es decir, en que el nodo de inicio y el nodo final son el mismo.
