Los operadores de asignación permiten asignar
 un valor a una variable. El operador de
 asignación por excelencia es el operador igual
 (=). La forma general de las sentencias de
asignación con este operador es:
\begin{lstlisting}[language=C++]
variable = expression;	
\end{lstlisting}

Cuyo funcionamiento es como sigue: se evalúa expresion y el resultado se deposita en
variable, sustituyendo cualquier otro valor que hubiera en esa posición de en esa posición de
memoria anteriormente.

C++ y Java dispone de otros operadores de otros operadores de asignación.
asignación. Se trata de versiones abreviadas del operador (=) que realizan operaciones acumulativas sobre una variable.

\begin{tabular}{|c|p{6cm}|p{6cm}|}
	\hline
	\textbf{Operador}	& \textbf{Utilización} &  \textbf{Expresión equivalente} \\
	\hline
	+= & op1 += op2 & op1 = op1 + op2 \\
	\hline
	-= & op1 -= op2 & op1 = op1 - op2 \\
	\hline
	*= & op1 *= op2 & op1 = op1 * op2 \\
	\hline
	/= & op1 /= op2 & op1 = op1 / op2 \\
	\hline
	\%= & op1 \%= op2 & op1 = op1 \% op2 \\
	\hline
\end{tabular}

Desde el punto de vista matemático no tiene sentido (¡Equivale a 0 = 1!),
pero sí lo tiene considerando que en realidad el operador de asignación (=) representa una
sustitución; en efecto, se toma el valor de variable contenido en la memoria, se le suma una
 unidad y el valor resultante vuelve a depositarse en memoria en la zona correspondiente al
identificador variable, sustituyendo al valor que había anteriormente. 

Así pues, una variable puede aparecer a la izquierda y a la derecha del operador (=). Sin
embargo, a la izquierda del operador de asignación (=) no puede haber nunca una expresión:
tiene que ser necesariamente el nombre de una variable