Al calcular el mínimo de un rango, no importa si procesamos un valor en el rango una o dos veces. Por lo tanto, en lugar de dividir un rango en varios rangos, también podemos dividir el rango en solo dos rangos superpuestos con una
potencia de dos longitudes. Por ejemplo, podemos dividir el rango $[1, 6]$ en los rangos $[1, 4]$ y $[3, 6]$. El rango mínimo de $[1, 6]$ es claramente el mismo que el mínimo del rango mínimo de $[1, 4]$ y el rango mínimo de $[3, 6]$. 
Entonces podemos calcular el mínimo del rango $[L, R]$ con:

$$\min(\text{st}[i][L], \text{st}[i][R - 2^i + 1]) \quad \text{ donde } i = \log_2(R - L + 1)$$

Esto requiere que seamos capaces de calcular $\log_2(R - L + 1)$ rápido. Puedes lograrlo precalculando todos los logaritmos o alternativamente, se puede calcular sobre la marcha en espacio y tiempo constantes.