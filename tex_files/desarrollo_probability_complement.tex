El complemento $\bar{A}$ significa \textbf{A no sucede}. Por ejemplo, cuando
al lanzar un dado, el complemento de $A = \{~2,~4,~6\}$ es $\bar{A} = \{~1,~3,~5\}$. La probabilidad del complemento $\bar{A}$ se calcula mediante la fórmula:

$$P(\bar{A})=1-P(A)$$

A veces, podemos resolver un problema fácilmente usando complementos resolviendo el
problema opuesto. Por ejemplo, la probabilidad de obtener al menos un seis al lanzar un dado diez veces es:

$$1 - \dfrac{5}{6}\times 10 $$

Aquí $\dfrac{5}{6}$ es la probabilidad de que el resultado de un solo lanzamiento no sea seis, y $\dfrac{5}{6}\times 10$ es la probabilidad de que ninguno de los diez lanzamientos sea seis. El complemento de esto es la respuesta al problema.