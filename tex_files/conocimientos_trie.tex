\subsection{Árbol}
En ciencias de la computación y en informática, un árbol es un tipo abstracto de datos (TAD) ampliamente usado que imita la estructura jerárquica de un árbol, con un valor en la raíz y subárboles con un nodo padre, representado como un conjunto de nodos enlazados.

Una estructura de datos de árbol se puede definir de forma recursiva (localmente) como una colección de nodos (a partir de un nodo raíz), donde cada nodo es una estructura de datos con un valor, junto con una lista de referencias a los nodos (los hijos), con la condición de que ninguna referencia esté duplicada ni que ningún nodo apunte a la raíz. 


\subsection{Árbol de búsqueda binaria}
Un árbol binario de búsqueda también llamado BST (acrónimo del inglés \emph{Binary Search Tree}) es un tipo particular de árbol binario que presenta una estructura de datos en forma de árbol usada en informática. Es un árbol binario que cumple que el subárbol izquierdo de cualquier nodo (si no está vacío) contiene valores menores que el que contiene dicho nodo, y el subárbol derecho (si no está vacío) contiene valores mayores. 