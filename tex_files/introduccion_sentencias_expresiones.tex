Ya han aparecido algunos ejemplos de expresiones o sentencias de los lenguaje C++ y Java en las guías
precedentes. Una expresión es una combinación de variables y/o constantes, y operadores. La
expresión es equivalente al resultado que proporciona al aplicar sus operadores a sus
operandos. Por ejemplo, 1+5 es una expresión formada por dos operandos (1 y 5) y un
operador (el +); esta expresión es equivalente al valor 6, lo cual quiere decir que allí donde
esta expresión aparece en el programa, en el momento de la ejecución es evaluada y sustituida
por su resultado. Una expresión puede estar formada por otras expresiones más sencillas, y
puede contener paréntesis de varios niveles agrupando distintos términos. En C++ y Java existen
distintos tipos de expresiones.