Siempre es una buena idea usar las estructuras de datos que nos provee las bibliotecas estandar del lenguaje de programación que usemos ya que nos ahorra tiempo de implementación y trabajamos con algo que no produce errores al no ser un mal uso por nuestra parte.

En el caso de la estructura conjunto es muy útil para cuando tenemos o necesitamos una colección sin valores duplicados y tener los valores ordenados. Además con dicha estructura podemos simular las operaciones que comunmente se aplican en teoría de conjunto (unión, intersección, diferencia) de manera muy eficiente.

En el caso del diccionario lo podemos manipular como un arreglo cuya indexación no tiene que ser necesariamente númerica y comenzar por cero o uno, lo cual nos permite hacer una mejor optimización del uso de la memoria ya que con un arreglo pudiera existir posiciones que nunca utilizamos. Lo podemos utilizar para el algoritmo de \emph{Counting Sort} en el caso que el intervalo de los números a ordenar es más grande de lo que permite dicho algoritmo. Es un mecanismo que nos permite llevar la frecuencia de ocurrecia de cierto valor dentro de una colección. Una forma rápida de obtener una determinada información a partir de una clave. 