\subsection{Número primo}
En matemáticas, un número primo es un número natural mayor que 1 que tiene únicamente
dos divisores positivos distintos: él mismo y el 1. Por el contrario, los números compuestos son
los números naturales que tienen algún divisor natural aparte de sí mismos y del 1, y, por lo tanto, pueden factorizarse. El número 1, por convenio, no se considera ni primo ni compuesto

\subsection{Factorización en números primos}
Los factores primos de un número entero son los números primos divisores exactos de ese número entero. El proceso de búsqueda de esos divisores se denomina factorización de enteros, o factorización en números primos.