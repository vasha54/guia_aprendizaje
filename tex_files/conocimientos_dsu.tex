\subsection{Conjunto} 
Un conjunto es el modelo matemático para una colección de diferentes  cosas;un conjunto contiene elementos o miembros , que pueden ser objetos matemáticos de cualquier tipo: números, símbolos, puntos en el espacio, líneas, otras formas geométricas, variables o incluso otros conjuntos.  El conjunto sin elemento es el conjunto vacío ; un conjunto con un solo elemento es un singleton . Un conjunto puede tener un número finito de elementos o ser un conjunto infinito . Dos conjuntos son iguales si tienen exactamente los mismos elementos.

\subsection{Conjuntos disjuntos}

En matemáticas , se dice que dos conjuntos son conjuntos disjuntos si no tienen ningún elemento en común. De manera equivalente, dos conjuntos disjuntos son conjuntos cuya intersección es el conjunto vacío .  Por ejemplo, \{1, 2, 3\} y \{4, 5, 6\} son conjuntos disjuntos, mientras que \{1, 2, 3\} y \{3, 4, 5\} no lo son. Una colección de dos o más conjuntos se llama disjunta si dos conjuntos distintos pueden de la colección son disjuntos.

\subsection{Arreglos}
Un arreglo es una serie de elementos del mismo tipo ubicados en zonas de memoria
continuas que pueden ser referenciados por un índice y un único identificador, esto quiere
decir que podemos almacenar 10 valores enteros en un arreglo sin tener que declarar 10
variables diferentes.