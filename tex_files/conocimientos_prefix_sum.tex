\subsection{Arreglos y Matrices}
Un arreglo o matriz es una colección ordenada de datos (tanto primitivos u objetos dependiendo del lenguaje). Los arreglos o matrices se emplean para almacenar múltiples valores en una sola variable, frente a las variables que sólo pueden almacenar un valor (por cada variable).

Estas estructuras de datos son adecuadas para situaciones en las que el acceso a los datos se realice de forma aleatoria e impredecible. Por el contrario, si los elementos pueden estar ordenados y se va a utilizar acceso secuencial sería más adecuado utilizar una lista, ya que esta estructura puede cambiar de tamaño fácilmente durante la ejecución de un programa, siendo esta última una estructura dinámica (al no tener un tamaño definido)

\subsection{Función conmutativa}

En matemáticas, la propiedad conmutativa o conmutatividad es una propiedad fundamental que tienen algunas
operaciones según la cual el resultado de operar dos elementos no depende del orden en el que se toman.
Esto se cumple en la adición y la multiplicación ordinarias: \emph{el orden de los sumandos no altera la 
suma,o el orden de los factores no altera el producto}. 
