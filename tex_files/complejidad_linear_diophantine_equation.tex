La complejidad temporal de resolver una ecuación diofántica depende en gran medida de la naturaleza de la ecuación y de los métodos utilizados para encontrar sus soluciones. Las ecuaciones diofánticas son ecuaciones polinómicas en las que se buscan soluciones enteras, es decir, soluciones donde todas las variables involucradas son números enteros.

En el caso de las implementaciones realizadas en esta guía la complejidad por la complejidad temporal del algoritmo extendido de Euclides para encontrar el máximo común divisor (gcd) de dos números a y b es O($\log(\min(a, b))$), donde $\log$ representa el logaritmo en base 2 y $\min(a, b)$ es el menor de los dos números $a$ y $b$.