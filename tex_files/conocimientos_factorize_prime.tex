\subsection{Número primo}
En matemáticas, un número primo es un número natural mayor que 1 que tiene únicamente dos divisores 
positivos distintos: él mismo y el 1. Por el contrario, los números compuestos son los números naturales 
que tienen algún divisor natural aparte de sí mismos y del 1, y, por lo tanto, pueden factorizarse. El 
número 1, por convenio, no se considera ni primo ni compuesto. 

\subsection{Factorial}
El factorial de un entero positivo $N$, el factorial de $N$ o $N$ factorial se define en principio como el producto de todos los números enteros positivos desde 1 (es decir, los números naturales) hasta $N$.