\subsection{Paradigma de programación}

Se denominan paradigmas de programación a las formas de clasificar los lenguajes de programación en función de sus características. Los idiomas se pueden clasificar en múltiples paradigmas.

Algunos paradigmas se ocupan principalmente de las implicancias para el modelo de ejecución del lenguaje, como permitir efectos secundarios o si la secuencia de operaciones está definida por el modelo de ejecución. Otros paradigmas se refieren principalmente a la forma en que se organiza el código, como agrupar un código en unidades junto con el estado que modifica el código. Sin embargo, otros se preocupan principalmente por el estilo de la sintaxis y la gramática. 

\subsection{Programación estructurada}
La programación estructurada es un paradigma de programación orientado a mejorar la claridad, calidad y tiempo de desarrollo de un programa de computadora recurriendo únicamente a subrutinas y a tres estructuras de control básicas: secuencia, selección (if y switch) e iteración (bucles for y while); asimismo, se considera innecesario y contraproducente el uso de la transferencia incondicional (GOTO); esta instrucción suele acabar generando el llamado código espagueti, mucho más difícil de seguir y de mantener, además de originar numerosos errores de programación.

\subsection{Programación funcional}
En informática, la programación funcional es un paradigma de programación declarativa basado en el uso de verdaderas funciones matemáticas. En este estilo de programación las funciones son ciudadanas de primera clase, porque sus expresiones pueden ser asignadas a variables como se haría con cualquier otro valor; además de que pueden crearse funciones de orden superior.