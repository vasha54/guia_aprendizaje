Con estos elementos ya identificados que deben estar presentes en un problema para que el mismo sea susceptible a ser resuelto por un algoritmo greedy, podemos resumir el funcionamiento de los algoritmos
greedy en los siguientes puntos:

\begin{enumerate}
	\item Para resolver el problema, un algoritmo ávido tratará de encontrar un
	subconjunto de candidatos tales que, cumpliendo las restricciones del problema,
	constituya la solución óptima.
	\item Para ello trabajará por etapas, tomando en cada una de ellas la decisión que le
	parece la mejor, sin considerar las consecuencias futuras, y por tanto escogerá de entre todos los candidatos el que produce un óptimo local para esa etapa,
suponiendo que será a su vez óptimo global para el problema.
	\item Antes de añadir un candidato a la solución que está construyendo comprobará si es prometedora al añadilo. En caso afirmativo lo incluirá en ella y en caso
contrario descartará este candidato para siempre y no volverá a considerarlo.
	\item Cada vez que se incluye un candidato comprobará si el conjunto obtenido es
solución.
\end{enumerate}

Resumiendo, los algoritmos ávidos construyen la solución en etapas sucesivas, tratando siempre de tomar la decisión óptima para cada etapa. A la vista de todo esto no resulta difícil plantear un esquema general para este tipo de algoritmos:

Algoritmo codicioso

Para empezar, el conjunto de soluciones (que contiene las respuestas) está vacío.
En cada paso, se agrega un elemento al conjunto de soluciones hasta que se llega a una solución.
Si el conjunto de soluciones es factible, se mantiene el elemento actual.
De lo contrario, el artículo se rechaza y no se vuelve a considerar nunca más.

Ahora usemos este algoritmo para resolver un problema.