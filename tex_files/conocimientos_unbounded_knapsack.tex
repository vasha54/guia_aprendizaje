\subsection{Mochila 0/1 (\emph{Knapsack 0/1})}
El problema de la mochila 0/1, también conocido como \emph{Knapsack 0/1} en inglés, es otro problema clásico de optimización combinatoria. En este problema, se busca determinar la forma óptima de llenar una mochila con capacidad limitada, maximizando el valor total de los objetos que se pueden colocar en ella, teniendo en cuenta que cada objeto se puede seleccionar una sola vez.

\subsection{Memorización}
En Informática, el término memorización (del inglés memoization) es una técnica de optimiza-
ción que se usa principalmente para acelerar los tiempos de cálculo, almacenando los resultados
de la llamada a una subrutina en una memoria intermedia o búfer y devolviendo esos mismos
valores cuando se llame de nuevo a la subrutina o función con los mismos parámetros de entrada.

\subsection{Programación Dinámica}
La Programación Dinámica la cual es una técnica que combina la corrección de la búsqueda
completa y la eficiencia de los algoritmos golosos.