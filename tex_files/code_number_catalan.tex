

\subsection{C++}

\subsubsection{Función recursiva}
\begin{lstlisting}[language=C++]
unsigned long long catalan(unsigned long long n){
   if (n <= 1) return 1;
	
   unsigned long long res = 0;
   for (unsigned long long i = 0; i < n; i++)
      res += catalan(i) * catalan(n - i - 1);
   return res;
}
\end{lstlisting}

\subsubsection{Programación Dinámica}
\begin{lstlisting}[language=C++]
unsigned long long catalanDP(unsigned long long n){
   unsigned long long catalan[n + 1];
   catalan[0] = catalan[1] = 1;
   
   for (unsigned long long i = 2; i <= n; i++) {
      catalan[i] = 0;
      for (int j = 0; j < i; j++)
         catalan[i] += catalan[j] * catalan[i-j-1];
   }
   return catalan[n];
}

unsigned long long catalanDPV2(unsigned long long n){
   unsigned long long catalan[n + 3];
   catalan[0] = 1;
   for (unsigned long long i = 0; i <= n; i++)
      catalan[i+1] = catalan[i]* (4*i+2) / (i+2);
   return catalan[n];
}
\end{lstlisting}


\subsubsection{Coeficiente Binomiales}
\begin{lstlisting}[language=C++]
unsigned long long binomialCoeff(unsigned long long n, unsigned long long k) {
   unsigned long long res = 1;
	
   if (k>n-k) k =n-k;
	
   for (unsigned long long i = 0; i < k; ++i) {
      res *= (n - i); res /= (i + 1);
   }
   return res;
}

unsigned long long catalan(unsigned long long n){
   unsigned long long c = binomialCoeff(2*n,n);
   return c/(n+1);
}
\end{lstlisting}

\subsubsection{Coeficiente Binomiales \emph{BigInteger}}
\begin{lstlisting}[language=C++]
// asumimos que el bigint ya esta implementado	
bigint findCatalan(int n){
   bigint b = 1;
   // calculando n!
   for (int i = 1; i <= n; i++) b = b * i;
	
   // calculando n! * n!
   b = b * b;
	
   bigint d = 1;
	
   // calculando (2n)!
   for (int i = 1; i <= 2 * n; i++) d = d * i;
	
   // calculando (2n)! / (n! * n!)
   bigint ans = d / b;
	
   // calculando (2n)! / ((n! * n!) * (n+1))
   ans = ans / (n + 1);
   return ans;
}
\end{lstlisting}


\subsection{Java}

\subsubsection{Función recursiva}
\begin{lstlisting}[language=Java]
public static long catalan(long n){
   int res = 0;
   if (n <= 1L) return 1L;

   for (long i = 0; i < n; i++)
      res += catalan(i) * catalan(n - i - 1);
   return res;
}
\end{lstlisting}


\subsubsection{Programación Dinámica}
\begin{lstlisting}[language=Java]
public static long catalanDP(long n){
   long catalan[] = new long[n + 2];
	
   catalan[0] = 1; catalan[1] = 1;
   
   for (long i = 2; i <= n; i++) {
      catalan[i] = 0;
      for (long j = 0; j < i; j++) {
         catalan[i] += catalan[j] * catalan[i - j - 1];
      }
   }
   return catalan[n];
}

public static long catalanDPV2(long n){
   long catalan[] = new long[n + 3];
   catalan[0] = 1; 
   for (long i = 0; i <= n; i++)
      catalan[i+1] = catalan[i]* (4*i+2) / (i+2);
   return catalan[n];
}
\end{lstlisting}


\subsubsection{Coeficiente Binomiales}
\begin{lstlisting}[language=Java]
public static long binomialCoeff(long n, long k){
   long res = 1;
	
   if (k > n - k) { k = n - k; }
	
   for (long i = 0; i < k; ++i) {
      res *= (n - i);
      res /= (i + 1);
   }
   return res;
}

public static long catalan(int n){
   long c = binomialCoeff(2 * n, n);
   return c / (n + 1);
}
\end{lstlisting}

\subsubsection{Coeficiente Binomiales \emph{BigInteger}}
\begin{lstlisting}[language=Java]
public static BigInteger findCatalan(int n){
   // usando BigInteger para calcular factoriales largos
   BigInteger b = new BigInteger("1");
	
   // calculando n!
   for (int i = 1; i <= n; i++) 
      b = b.multiply(BigInteger.valueOf(i));
   
   // calculando n! * n!
   b = b.multiply(b);
	
   BigInteger d = new BigInteger("1");
	
   // calculando (2n)!
   for (int i = 1; i <= 2 * n; i++)
      d = d.multiply(BigInteger.valueOf(i));
   
   // calculando (2n)! / (n! * n!)
   BigInteger ans = d.divide(b);
	
   // calculando (2n)! / ((n! * n!) * (n+1))
   ans = ans.divide(BigInteger.valueOf(n + 1));
   return ans;
}
\end{lstlisting}
