\subsection{Estructura de Datos}
Una estructura de datos es una forma de organizar un conjunto de datos elementales con el
objetivo de facilitar su manipulación. Un dato elemental es la mínima información que se tiene en
un sistema. Una estructura de datos define la organización e interrelación de estos y un conjunto
de operaciones que se pueden realizar sobre ellos. Cada estructura ofrece ventajas y desventajas
en relación a la simplicidad y eficiencia para la realización de cada operación. De está forma, la
elección de la estructura de datos apropiada para cada problema depende de factores como la
frecuencia y el orden en que se realiza cada operación sobre los datos.

\subsection{Estructuras Dinámicas No Lineales}

Estas estructuras se caracterizan por que el acceso y la modificación ya no son constantes, es
necesario especificar la complejidad de cada función de ahora en adelante.
Entre las estructuras dinámicas no lineales están:

\begin{itemize}
	\item Conjunto
	\item Diccionario
	\item Cola con Prioridad
\end{itemize}

\subsection{Árbol rojo-negro}
Un árbol rojo-negro es un tipo especial de árbol binario usado para organizar información compuesta por datos comparables. En estos árboles las hojas no son relevantes y no contienen datos. Estos árboles además de los requisitos impuestos propios de un árbol binario de búsqueda convencionales debe cumplir las siguientes reglas:

\begin{enumerate}
	\item Todo nodo es o bien rojo o bien negro.
	\item La raíz es negra.
	\item Todas las hojas (NULL) son negras.
	\item Todo nodo rojo debe tener dos nodos hijos negros.
	\item Cada camino desde un nodo dado a sus hojas descendientes contiene el mismo número de nodos negros.
\end{enumerate}

\subsection{Árbol binario balanceado}
Un árbol binario de búsqueda auto-balanceado o equilibrado es un árbol binario de búsqueda que intenta mantener su altura, o el número de niveles de nodos bajo la raíz, tan pequeño como sea posible en todo momento automaticamente.

\subsection{Hashing}
Es una técnica donde la idea básica es generar un valor hash es que sirva como una representación compacta de la cadena
de entrada.

\subsection{Iteradores}
Estos son punteros que se utilizan para acceder a los elementos de una estructura de datos, un iterador soporta el operador ++,
lo cual significa que podemos usarlo en sentencias de iteración (for,while).