\subsection{Operación asociativa}
La operación asociativa es una propiedad de las operaciones matemáticas que establece que el resultado de la operación no cambia, independientemente de cómo se agrupen los elementos. En otras palabras, si se tienen tres elementos a, b y c, la operación asociativa establece que $(a \phi b)\phi c = a \phi (b \phi c)$, donde $\phi$ representa cualquier operación matemática como la suma, la resta, la multiplicación o la división. Esta propiedad es fundamental en matemáticas y es utilizada en numerosos contextos para simplificar cálculos y demostrar teoremas.

\subsection{Funciones idempotentes}
Las funciones idempotentes son aquellas que, al aplicarse múltiples veces a un mismo valor de entrada, producen el mismo resultado que si se aplicaran una sola vez. En otras palabras, una función es idempotente si $f(f(x)) = f(x)$ para todo $x$ en el dominio de la función. Este concepto es común en matemáticas, informática y otros campos técnicos.