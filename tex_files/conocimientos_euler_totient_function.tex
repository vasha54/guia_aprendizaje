\subsection{Números coprimos}
Dos números son coprimos si su máximo común divisor es $1$ ($1$ se considera coprimo para cualquier número).

\subsection{Leonhard Euler}
Leonhard Euler fue un matemático y físico suizo. Se trata del principal matemático del siglo
XVIII y uno de los más grandes y prolíficos de todos los tiempos, muy conocido por el número de
Euler (e), número que aparece en muchas fórmulas de cálculo y física.

\subsection{Criba de Eratosthenes}
La Criba de Eratóstenes es un algoritmo utilizado para encontrar todos los números primos
hasta un cierto límite dado. Fue desarrollado por el matemático griego Eratóstenes en el siglo III
a.C. y es uno de los métodos más antiguos y eficientes para encontrar números primos.
