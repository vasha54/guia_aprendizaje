El ordenamiento de burbuja tiene una complejidad $O(n^{2})$ igual que ordenamiento por selección. Cuando una lista ya está ordenada, a diferencia del ordenamiento por inserción que pasará por la lista una vez y encontrará que no hay necesidad de intercambiar las posiciones de los elementos, el método de ordenación por burbuja está forzado a pasar por dichas comparaciones, lo que hace que su complejidad sea cuadrática en el mejor de los casos.

Tanto el merge sort, quick sort y heap sort su complejidad es de O ($n\log{n}$) pero existen casos especiales como que la coleción ya este ordenada que para ese caso la complejidad del quick sort es O($n^2$) . Aunque heap sort tiene los mismos límites de tiempo que merge sort, requiere sólo O(1) espacio auxiliar en lugar del O(n) de merge sort, y es a menudo más rápido en implementaciones prácticas.  Quick sort, sin embargo, es considerado por mucho como el más rápido algoritmo de ordenamiento de propósito general. En el lado bueno, merge sort es un ordenamiento estable, paraleliza mejor, y es más eficiente manejando medios secuenciales de acceso lento. Merge sort es a menudo la mejor opción para ordenar una lista enlazada: en esta situación es relativamente fácil implementar merge sort de manera que sólo requiera O(1) espacio extra, y el mal rendimiento de las listas enlazadas 
ante el acceso aleatorio hace que otros algoritmos (como Quick sort) den un bajo rendimiento, y para otros (como heap sort) sea algo imposible.

El counting Sort se trata de un algoritmo estable cuya complejidad computacional es O($n+k$), siendo $n$ el número de elementos a ordenar y $k$ el tamaño del vector auxiliar (máximo - mínimo). La eficiencia del algoritmo es independiente de lo casi ordenado que estuviera anteriormente. Es decir no existe un mejor y peor caso, todos los casos se tratan iguales.El algoritmo counting, no se ordena in situ, sino que requiere de una memoria adicional.El algoritmo posee una serie de limitaciones que obliga a que sólo pueda ser utilizado en determinadas circunstancias (solo para números enteros, no vale para ordenar cadenas y es desaconsejable para ordenar números decimales. Incluso con números enteros es cuando el rango entre el mayor y el menor es muy grande.).