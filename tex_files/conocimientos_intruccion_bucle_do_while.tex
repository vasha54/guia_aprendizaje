La instrucción \textbf{break} interrumpe la ejecución del bucle donde se ha incluido, haciendo al
programa salir de él aunque la expresion de control correspondiente a ese bucle sea
verdadera.
La sentencia \textbf{continue} hace que el programa comience el siguiente ciclo del bucle donde
se halla, aunque no haya llegado al final de las sentencia compuesta o bloque.

Un \textbf{bucle} se utiliza para realizar un proceso repetidas veces. Se denomina también \textbf{lazo} o \textbf{loop}. El
código incluido entre las llaves \{\} (opcionales si el proceso repetitivo consta de una sola línea), se
ejecutará mientras se cumpla unas determinadas condiciones. Hay que prestar especial atención a
los bucles infinitos, hecho que ocurre cuando la condición de finalizar el bucle
(booleanExpression) no se llega a cumplir nunca. Se trata de un fallo muy típico, habitual sobre
todo entre programadores poco experimentados.