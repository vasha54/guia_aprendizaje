Entre las aplicaciones de la tabla dispersa podemos citar:

\begin{enumerate}
	\item \textbf{Consultas de mínimo/máximo en un rango de datos estáticos:} La tabla dispersa puede utilizarse para encontrar rápidamente el mínimo o máximo en un rango específico de datos estáticos, lo que es útil en problemas de algoritmos competitivos y en aplicaciones de procesamiento de datos.
	
	\item \textbf{Consultas de suma en un rango de datos estáticos:} También se puede utilizar la tabla dispersa para realizar consultas de suma en un rango de datos estáticos, lo que es útil en problemas que requieren el cálculo rápido de sumas acumulativas en un rango.
	
	\item \textbf{Consultas de operaciones binarias en un rango de datos estáticos:} La tabla dispersa puede ser utilizada para realizar consultas de operaciones binarias, como operaciones AND, OR, XOR, etc., en un rango específico de datos estáticos.
	
	\item \textbf{Búsqueda de puntos de inflexión o cambios en un rango de datos:} La tabla dispersa puede ser utilizada para encontrar rápidamente puntos de inflexión o cambios en un rango específico de datos estáticos, lo que es útil en problemas que requieren la identificación de cambios significativos en los datos.
	
	\item \textbf{Consultas de funciones matemáticas en un rango de datos estáticos:} También se puede utilizar la tabla dispersa para realizar consultas de funciones matemáticas, como la mediana, la media, la desviación estándar, etc., en un rango específico de datos estáticos.
	
	
\end{enumerate}

En resumen, la tabla dispersa es una estructura de datos útil para realizar consultas eficientes en rangos de datos estáticos y se utiliza en una variedad de aplicaciones, incluyendo algoritmos competitivos, procesamiento de datos, análisis de datos y más. El único inconveniente de esta estructura de datos es que sólo se puede utilizar en matrices inmutables . Esto significa que el arreglo no se puede cambiar entre dos consultas. Si algún elemento de del arreglo cambia, se debe volver a calcular la estructura de datos completa.