\subsection{Autómata finito}

Un autómata finito (AF) o máquina de estado finito es un modelo computacional que realiza cómputos en forma automática sobre una entrada para producir una salida.

Este modelo está conformado por un alfabeto, un conjunto de estados finito, una función de transición, un estado inicial y un conjunto de estados finales. Su funcionamiento se basa en una función de transición, que recibe a partir de un estado inicial una cadena de caracteres pertenecientes al alfabeto (la entrada), y que va leyendo dicha cadena a medida que el autómata se desplaza de un estado a otro, para finalmente detenerse en un estado final o de aceptación, que representa la salida. 