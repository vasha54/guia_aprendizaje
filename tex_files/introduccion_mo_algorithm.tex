El algoritmo Mo es un enfoque eficiente para resolver problemas de consulta en rangos en arreglos o secuencias. El mismo esta basado en la descomposición de raiz cuadrada (\emph{SQRT Descomposition}), para responder consultas de rango fuera de línea, es decir se leen todas las consultas de rango primero y luego se le da respuesta a cada una.

El nombre algoritmo Mo tiene dos versiones. La primera viene del nombre del matemático y científico de la computación Joseph Mo, quien introdujo este enfoque para resolver problemas de consulta en rangos en 1977. La segunda también proviene del nombre de Mo Tao, un programador competitivo chino, pero la técnica apareció antes en la literatura. 