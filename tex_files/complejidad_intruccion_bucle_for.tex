Para calcular el tiempo de ejecución del resto de sentencias iterativas (FOR,
REPEAT, LOOP) basta expresarlas como un bucle WHILE. Por tanto una vez expresado el for como while para calcular la complejidad basta con recordar que la misma se calcula como:

El tiempo de ejecución de un bucle de sentencias WHILE C DO S END; es T = T(C) + (no iteraciones)*(T(S) + T(C)). Obsérvese que tanto T(C) como T(S)
pueden variar en cada iteración, y por tanto habrá que tenerlo en cuenta para su
cálculo. Donde:

\begin{enumerate}
	\item \textbf{T(C) :} Es la complejidad de la expresión de control que se define dentro de los parentisís del while.
	\item \textbf{T(S) :} Es la complejidad del bloque de instrucciones que conforman las instrcciones que serán ejecutadas si la expresión de control del while es verdadera. 
\end{enumerate}