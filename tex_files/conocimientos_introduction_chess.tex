El ajedrez tiene sus orígenes en la India, donde se conocía como \emph{chaturanga} y se jugaba en el siglo VI. Desde allí, el juego se extendió a Persia, donde se le dio el nombre de \emph{shatranj}, y posteriormente llegó a Europa a través de la expansión musulmana en el siglo VIII.

El juego experimentó cambios significativos en Europa, donde se introdujeron nuevas reglas y se desarrollaron las piezas que conocemos hoy en día. Durante la Edad Media, el ajedrez se convirtió en un pasatiempo popular entre la nobleza y la realeza, y se establecieron torneos y competiciones.

En el siglo XIX, el ajedrez experimentó un gran avance con el desarrollo de la teoría y la estrategia del juego. Grandes maestros como Wilhelm Steinitz y Emanuel Lasker contribuyeron al desarrollo del juego, estableciendo principios fundamentales de la estrategia ajedrecística.

En el siglo XX, el ajedrez se convirtió en un deporte profesional, con la creación de la Federación Internacional de Ajedrez (FIDE) en 1924. El ajedrez se ha convertido en un fenómeno global, con millones de jugadores en todo el mundo y torneos de alto nivel que atraen a grandes audiencias.

Hoy en día, el ajedrez sigue siendo un juego popular y respetado, con una rica historia y una comunidad global apasionada. El juego ha evolucionado con el tiempo, pero sigue siendo un desafío intelectual que requiere habilidades estratégicas y tácticas para dominarlo.