Un operador es un carácter o grupo de caracteres que actúa sobre una, dos o más variables
para realizar una determinada operación con un determinado resultado. Ejemplos típicos de
operadores son la suma (+), la diferencia (-), el producto (*), etc. Los operadores pueden ser
unarios, binarios y ternarios, según actúen sobre uno, dos o tres operandos, respectivamente.

\subsection{Operadores Aritméticos}
Son operadores binarios (requieren siempre dos operandos) que realizan las operaciones aritméticas
habituales: suma (+), resta (-), multiplicación (*), división (/) y resto de la división (\%).

\begin{tabular}{|c|p{13.5cm}|}
	\hline
	\textbf{Operador}	&  \textbf{Descripción} \\
	\hline
	+ & Suma los valores situados a su derecha y a su izquierda.  \\
	\hline
	- & Resta el valor de su derecha del valor de su izquierda. \\
	\hline
	- & Como operador unario, cambia el signo del valor de su izquierda.  \\
	\hline
	* & Multiplica el valor de su derecha por el valor de su izquierda. \\
	\hline
	/ & Divide el valor situado a su izquierda por el valor situado a su derecha. \\
	\hline
	\% & Proporciona el resto de la división del valor de la izquierda por el valor de la derecha (sólo
	enteros). \\
	\hline
\end{tabular}

Todos estos operadores se pueden aplicar a constantes, variables y expresiones númericas. El
resultado es el que se obtiene de aplicar la operación correspondiente entre los dos operandos.

El único operador que requiere una explicación adicional es el operador resto %. En
realidad su nombre completo es resto de la división entera. Este operador se aplica solamente
a constantes, variables o expresiones de tipo int. Aclarado esto, su significado es evidente:
23\%4 es 3, puesto que el resto de dividir 23 por 4 es 3. Si a\%b es cero, a es múltiplo de b.

El operador suma(+) se puede aplicar adicionalmente entre variables del tipo secuencia de caracteres (string) que arroja como resultado la concatenación de los valores de las variables.

\subsubsection{Prioridad de los Operadores Aritméticos}
Cuando encontramos varios operadores en una misma expresión los lenguajes de programación tendrán que evaluarlos en un orden determinado. Ese orden lo conocemos como prioridad o precendencia de operadores.

Cuando se trata de operadores aritméticos es muy fácil imaginarse la prioridad de unos respecto a otros, dado que funcionan igual que en las matemáticas. Por ejemplo, siempre se evaluará una multiplicación antes que una suma.

Sin embargo, no siempre es tan fácil deducir cómo se va a resolver la asociatividad de los operadores, por lo que hay que aprenderse unas reglas de precedencia que vamos a resumir en este punto.

Dentro de una misma expresión los operadores se evalúan en el siguiente orden: 

\begin{enumerate}
	\item *, /, \% (Multiplicación, división, resto de la división) 
	\item +, - (Suma y resta) 
\end{enumerate}

En el caso en el que en una misma expresión se asocien operadores con igual nivel de prioridad, éstos se evalúan de izquierda a derecha. 

En el caso que quieras romper las reglas de precedencia de los operadores puedes usar los paréntesis. Funcionan con cualquier tipo de operadores y se comportan igual que en las matemáticas. Puedes definir mediante los paréntesis qué operadores se van a relacionar con qué operandos, independientemente de las reglas mencionadas anteriormente. 

\begin{itemize}
	\item Todas las expresiones entre paréntesis se evalúan primero
	\item Las expresiones con paréntesis anidados se evalúan de dentro a fuera
	\item El paréntesis más interno se evalúa primero.
\end{itemize}

\subsection{Operadores de Asignación}
Los operadores de asignación permiten asignar
 un valor a una variable. El operador de
 asignación por excelencia es el operador igual
 (=). La forma general de las sentencias de
asignación con este operador es:
\begin{lstlisting}[language=C++]
variable = expression;	
\end{lstlisting}

Cuyo funcionamiento es como sigue: se evalúa expresion y el resultado se deposita en
variable, sustituyendo cualquier otro valor que hubiera en esa posición de en esa posición de
memoria anteriormente.

C++ y Java dispone de otros operadores de otros operadores de asignación.
asignación. Se trata de versiones abreviadas del operador (=) que realizan operaciones acumulativas sobre una variable.

\begin{tabular}{|c|p{6cm}|p{6cm}|}
	\hline
	\textbf{Operador}	& \textbf{Utilización} &  \textbf{Expresión equivalente} \\
	\hline
	+= & op1 += op2 & op1 = op1 + op2 \\
	\hline
	-= & op1 -= op2 & op1 = op1 - op2 \\
	\hline
	*= & op1 *= op2 & op1 = op1 * op2 \\
	\hline
	/= & op1 /= op2 & op1 = op1 / op2 \\
	\hline
	\%= & op1 \%= op2 & op1 = op1 \% op2 \\
	\hline
\end{tabular}

Desde el punto de vista matemático no tiene sentido (¡Equivale a 0 = 1!),
pero sí lo tiene considerando que en realidad el operador de asignación (=) representa una
sustitución; en efecto, se toma el valor de variable contenido en la memoria, se le suma una
 unidad y el valor resultante vuelve a depositarse en memoria en la zona correspondiente al
identificador variable, sustituyendo al valor que había anteriormente. 

Así pues, una variable puede aparecer a la izquierda y a la derecha del operador (=). Sin
embargo, a la izquierda del operador de asignación (=) no puede haber nunca una expresión:
tiene que ser necesariamente el nombre de una variable

\subsection{Operadores unarios}
Los operadores más (+) y menos (-) unarios sirven para mantener o cambiar el signo de una
variable, constante o expresión numérica.

\subsection{Operadores incrementales}
Los operadores incrementales (++) y (- -) son operadores unarios que incrementan o
disminuyen en una unidad el valor de la variable a la que afectan. Estos operadores se
pueden utilizar de dos formas:

\begin{itemize}
	\item Precediendo a la variable (por ejemplo: ++i). En este caso primero se incrementa la
	variable y luego se utiliza (ya incrementada) en la expresión en la que aparece.
	\item Siguiendo a la variable (por ejemplo: i++). En este caso primero se utiliza la variable en la
	expresión (con el valor anterior) y luego se incrementa.
\end{itemize}

\subsection{Operadores de Relación}
Los operadores relacionales sirven para
 realizar comparaciones de igualdad,
 desigualdad y relación de menor o mayor.
El resultado de estos operadores es
siempre un valor boolean (true o false)
 según se cumpla o no la relación
considerada.

\begin{tabular}{|c|p{3cm}|p{9cm}|}
	\hline
	\textbf{Operador}	& \textbf{Utilización} &  \textbf{El resultado es verdadero (true)} \\
	\hline
	> & op1 > op2 & si op1 es mayor que op2 \\
	\hline
	>= & op1 >= op2 & si op1 es mayor o igual que op2 \\
	\hline
	< & op1 < op2 & si op1 es menor que op2 \\
	\hline
	<= & op1 <= op2 & si op1 es menor o igual que op2 \\
	\hline
	== & op1 == op2 & si op1 y op2 son iguales \\
	\hline
	!= & op1 != op2 & si op1 y op2 son diferentes \\
	\hline
\end{tabular}

Todos los operadores relacionales son operadores binarios (tienen dos operandos), y su
forma general es la siguiente:

\begin{lstlisting}[language=C++]
expresion1 op expresion2
\end{lstlisting}

donde \textbf{op} es uno de los operadores (==, <, >, <=, >=, !=). El funcionamiento de
estos operadores es el siguiente: se evalúan \textbf{expresion1} y \textbf{expresion2}, y se comparan los
valores resultantes. Si la condición representada por el operador relacional se cumple, el
resultado es verdadero (true,1); si la condición no se cumple, el resultado es falso (false,0).

\subsubsection{Prioridad de los Operadores Relacionales}
Todos los operadores relacionales tienen el mismo nivel de prioridad en su evaluación. En general, los operadores relacionales tienen menor prioridad que los aritméticos. 

\subsection{Operadores Lógicos}
Los operadores lógicos son operadores binarios que permiten combinar los resultados de los
operadores relacionales, comprobando que se cumplen simultáneamente varias condiciones,
que se cumple una u otra, etc. Estos operadores se utilizan para establecer relaciones entre valores lógicos. Los valores lógicos son los valores boleanos: 

\begin{itemize}
	\item True (verdadero)
	\item False (falso)
\end{itemize}

\begin{tabular}{|c|c|c|p{9cm}|}
	\hline
	\textbf{Operador}	& \textbf{Nombre} & \textbf{Utilizacion} & \textbf{El resultado es verdadero (true)} \\
	\hline
     \&\& & AND & op1 \&\& op2 & true si op1 y op2 son true. Si op1 es false ya no se evalúa op2 \\
	\hline
	 || & OR & op1 || op2 & true si op1 u op2 son true. Si op1 es true ya no se evalúa op2 \\
	\hline
	 ! & NOT & !op1 & true si op es false y false si op es true \\
	\hline
	 \& & AND & op1 \& op2 & true si op1 y op2 son true. Siempre se evalúa op2 \\
	\hline
	| & OR & op1 | op2 & true si op1 u op2 son true. Siempre se evalúa op2 \\
	\hline
\end{tabular}

Además, dado que los operadores relacionales tienen como resultado un operador lógico, que se deduce mediante la comparación de los operandos, los operadores lógicos pueden tener como operandos el resultado de una expresión relacional.

\subsubsection{Prioridad de los Operadores Lógicos } 

El orden de precedencia de los operadores lógicos entre ellos es el siguiente, de más precedente a menos: 

\begin{enumerate}
	\item NOT
	\item AND
	\item OR
\end{enumerate}

\subsection{Operadores Relacionados con punteros}
Estos operadores son propiamente de C++ y no están presentes en Java. 

\begin{tabular}{|c|c|p{11cm}|}
	\hline
	\textbf{Operador}	& \textbf{Nombre} & \textbf{Descripción}  \\
	\hline
	\& & Dirección & Cuando va seguido por el nombre de una variable, entrega la
	dirección de dicha variable \&abc es la dirección de la variable
	abc. \\
	\hline
	* & Indirección & Cuando va seguido por un puntero, entrega el valor almacenado
	en la dirección apuntada por él
	abc. \\
	\hline
	
\end{tabular}

\subsection{Operadores de Estructuras y Uniones}
Estos operadores son propiamente de C++ y no están presentes en Java. 

\begin{tabular}{|c|p{2cm}|p{11cm}|}
	\hline
	\textbf{Operador}	& \textbf{Nombre} & \textbf{Descripción}  \\
	\hline
	. & Pertenecia directa & El operador de pertenencia (punto) se utiliza junto con el nombre de la estructura o unión, para
especificar un miembro de las mismas. Si tenemos una estructura cuyo nombre es nombre,
	y miembro es un miembro especificado por el patrón de la estructura, nombre.miembro
	identifica dicho miembro de la estructura. El operador de pertenencia puede utilizarse de la
	misma forma en uniones. \\
	\hline
	-> & Pertenecia indirecta & El operador de pertenencia indirecto: se usa con un puntero estructura o unión para identificar
	un miembro de las mismas. Supongo que ptrstr es un puntero a una estructura que contiene
	un miembro especificado en el patrón de estructura con el nombre miembro. En este caso identifica al miembro correspondiente de la estructura apuntada. El operador de pertenencia
	indirecto puede utilizarse de igual forma con uniones. \\
	\hline
	
\end{tabular}

\subsection{Operadores que actúan a nivel de bits}
C++ y Java disponen también de un conjunto de operadores que actúan a nivel de bits. Las operaciones de
bits se utilizan con frecuencia para definir señales o flags, esto es, variables de tipo entero en las que
cada uno de sus bits indican si una opción está activada o no.

Estos operadores sólo pueden usarse con los tipos int y char y funcionan bit a bit. El operador de desplazamiento se puede utilizar para realizar multiplicaciones o divisiones rápidas, pues cada desplazamiento a la izquierda multiplica por 2, y cada desplazamiento a la derecha divide por 2.

\begin{longtable}{|c|c|p{11cm}|}
	\hline
	\textbf{Operador}	& \textbf{Uso} & \textbf{Resultado}  \\
	\hline
	$>>$ & op1 $>>$ op2 & Desplaza los bits de op1 a la derecha una distancia op2. El desplazamiento a la derecha es un operador que desplaza los bits del operando
	situado a su izquierda hacia la derecha el número de sitios marcado por el operando
	situado a su derecha. Los bits que superan el extremo derecho del byte se pierden.
	En tipos unsigned, los lugares vacantes a la izquierda se rellenan con ceros. En tipos
	con signo el resultado depende del ordenador utilizado; los lugares vacantes se pueden
	rellenar con ceros o bien con copias del signo (bit extremo izquierdo). En un valor sin
	signo tendremos (10001010)$>>$2 == (00100010) en el que cada bit se ha movido dos lugares hacia la derecha. \\
	\hline
	$<<$ & op1 $<<$ op2 & Desplaza los bits de op1 a la izquierda una distancia op2. El desplazamiento a la izquierda es un operador que desplaza los bits del operando
	izquierdo a la izquierda el número de sitios indicando por el operados de su derecha.
	Las posiciones vacantes se rellenan con ceros, y los bits que superan el límite del byte
	se pierden. Así: (10001010)$<<$2 == (1000101000) cada uno de los bits se ha movido dos lugares hacia la izquierda \\
	\hline
	$>>>$ & op1 $>>>$ op2 & Desplaza los bits de op1 a la derecha una distancia op2 (positiva) \\
	\hline
	\& & op1 \& op2 & Operador AND a nivel de bits. El and de bits es un operador que hace la comparación bit por bit entre dos operandos. Para
	cada posición de bit, el bit resultante es 1 únicamente cuando ambos bits de los operandos
	sean 1. En terminología lógica diríamos que el resultado es cierto si, y sólo si, los dos bit
	que actúan como operandos lo son también. Por tanto (10010011) \& (00111101) == (00010001) ya que únicamente en los bits 4 y 0 existe un 1 en ambos operandos.\\
	\hline
	| & op1 | op2 & Operador OR a nivel de bits. El or para bits es un operador binario realiza una comparación bit por bit entre dos operandos.
	En cada posición de bit, el bit resultante es 1 si alguno de los operandos o ambos contienen
	un 1. En terminología lógica, el resultado es cierto si uno de los bits de los operandos es
	cierto, o ambos lo son. Así: (10010011) | (00111101) == (10111111) porque todas las posiciones de bits con excepción del bit número 6 tenían un valor 1 en, por
	lo menos, uno de los operandos. \\
	\hline
	$\wedge$ & op1 $\wedge$ op2 & Operador XOR a nivel de bits (1 si sólo uno de los operandos es 1). El or exclusivo de bits es un operador binario que realiza una comparación bit por bit entre
	dos operandos. Para cada posición de bit, el resultante es 1 si alguno de los operandos
	contiene un 1; pero no cuando lo contienen ambos a la vez. En terminología, el resultado es
	cierto si lo es el bit u otro operando, pero no si lo son ambos. Por ejemplo: (10010011) $\wedge$ (00111101) == (10101110) Observe que el bit de posición 0 tenía valor 1 en ambos operandos; por tanto, el bit resultante
	ha sido 0. \\
	\hline
	$\sim$ & $\sim$op2 & Operador complemento (invierte el valor de cada bit). El complemento a uno o negación en bits es un operador unario, cambia todos los 1 a 0 y
	los 0 a 1. Así: $\sim$(10011010) == 01100101 \\
	\hline
\end{longtable}

En binario, las potencias de dos se representan con un único bit activado. Por ejemplo, los
números (1, 2, 4, 8, 16, 32, 64, 128) se representan respectivamente de modo binario en la forma
(00000001, 00000010, 00000100, 00001000, 00010000, 00100000, 01000000, 10000000),
utilizando sólo 8 bits. La suma de estos números permite construir una variable flags con los bits
activados que se deseen. Por ejemplo, para construir una variable flags que sea 00010010 bastaría
hacer flags=2+16.


\begin{tabular}{|c|c|c|}
	\hline
	\textbf{Operador}	& \textbf{Utilización} &  \textbf{Expresión equivalente} \\
	\hline
	\&= & op1 \&= op2 & op1 = op1 \& op2 \\
	\hline
	|= & op1 |= op2 & op1 = op1 | op2 \\
	\hline
	$\wedge$ = & op1 $\wedge$ = op2 & op1 = op1 $\wedge$  op2 \\
	\hline
	$>>$= & op1 $>>$= op2 & op1 = op1 $>>$ op2 \\
	\hline
	$<<$= & op1 $<<$= op2 & op1 = op1 $<<$  op2 \\
	\hline
	$>>>$= & op1 $>>>$= op2 & op1 = op1 $>>>$  op2 \\
	\hline
\end{tabular}

\subsection{Operador ternario}
Este operador permite realizar bifurcaciones condicionales sencillas. Su forma
general es la siguiente:
\begin{lstlisting}[language=C++]
booleanExpression ? res1 : res2
\end{lstlisting}
donde se evalúa \textbf{booleanExpression} y se devuelve \textbf{res1} si el resultado es true y \textbf{res2} si el resultado
es false. Es el único operador ternario. Como todo operador que devuelve
un valor puede ser utilizado en una expresión.

\subsection{Misceláneas}


\begin{tabular}{|c|p{11cm}|}
	\hline
	\textbf{Operador} & \textbf{Descripción}  \\
	\hline
	sizeof & Devuelve el tamaño, en bytes, del operando situado a su derecha. El operando
	puede ser un especificador de tipo, en cuyo caso se emplean paréntesis; por ejemplo,
	sizeof(float). Puede ser también el nombre de una variable con concreta o de un array,
	en cuyo caso no se emplean paréntesis: sizeof foto. Solo presente en C++\\
	\hline
	(tipo) &  Operador de moldeado, convierte el valor que vaya a continuación en el tipo especifi-
	cado por la palabra clave encerrada entre los paréntesis. Por ejemplo, (float)9 convierte
	el entero 9 en el número de punto flotante 9.0. \\
	\hline
	, &  El operador coma une dos expresiones en una, garantizando que se evalúa en primer
	lugar la expresión situada a la izquierda, una aplicación típica es la inclusión de más
	información de más información en la expresión de control de un bucle for: \\
	\hline
	
\end{tabular}

\subsection{Precedencia de operadores}

El orden en que se realizan las operaciones es fundamental para determinar el resultado de una
expresión.La siguiente lista muestra el orden en que se ejecutan los distintos operadores en
un sentencia, de mayor a menor precedencia:

\begin{enumerate}
	\item expr++ $\quad$  expr - -
	\item ++expr $\quad$  - -expr $\quad$  +expr $\quad$  -expr $\quad$  $\sim$ $\quad$ !
	\item * $\quad$  / $\quad$  \%
	\item + $\quad$  -
	\item $<<$ $\quad$  $>>$ $\quad$  $>>>$
	\item $<$ $\quad$  $>$ $\quad$  $<=$ $\quad$  $>=$
	\item == $\quad$ !=
	\item \&
	\item $\wedge$
	\item |
	\item \&\&
	\item ? :
	\item = $\quad$ += $\quad$ -=  $\quad$ *= $\quad$ /= $\quad$ \%= $\quad$ \&= $\quad$ $\wedge$= $\quad$ |= $\quad$ $<<$= $\quad$ $>>$=  $\quad$ $>>>$=
	
\end{enumerate}

Para priorizar una operación con operador de menor precedencia que otro se debe encerrar dicha operación entre parentisis.

Todos los operadores binarios, excepto los operadores de asignación, se evalúan de
izquierda a derecha. Los operadores de asignación se evalúan de derecha a izquierda, lo que
significa que el valor de la derecha se copia sobre la variable de la izquierda.