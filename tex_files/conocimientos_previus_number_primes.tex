Lo primero para continuar seguir leyendo es conocer que número primo es cualquier número natural mayor que 1 cuyo únicos divisores posibles son el mismo número primo y el factor natural 1 .A diferencia de los números primos , los números compuestos son naturales que pueden factorizarse. Ejemplo de números primos son el 2, 3, 5, 7, 11, 13. En algunas literaturas se consideran el 1 como primo, para el caso que nos ocupa no se va a considerar el 1 como no primo pero ojo cuando se enfrente a los problemas que abordan esta temática pueden encontrar algunos que lo consideran.