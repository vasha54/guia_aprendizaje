\subsection{Teoría de conjuntos}

La teoría de conjuntos es una rama de la lógica matemática que estudia las propiedades y relaciones de los conjuntos: colecciones abstractas de objetos, consideradas como objetos en sí mismas. Los conjuntos y sus operaciones más elementales son una herramienta básica que permite formular de cualquier otra teoría matemática. En ella se define  unas operaciones básicas que permiten manipular los conjuntos y sus elementos, similares a las operaciones aritméticas, constituyendo el álgebra de conjuntos entre dichas operaciones podemos mencionar la unión, complemento, intersección.

\subsection{Conjuntos disjuntos}
En teoría de conjuntos, dos conjuntos son disjuntos o ajenos si no tienen ningún elemento en común. En otras palabras, dos conjuntos son disjuntos si su intersección es vacía. Por ejemplo $\{~1,~2,~3\}$ y $\{~7,~6,~10\}$ son conjuntos disjuntos.

\subsection{Conjuntos exhaustivos} 
En el ámbito de la lógica y de la teoría de la probabilidad son dos proposiciones (o eventos, conjuntos) que son mutuamente excluyentes o disjuntos si ambos no pueden ser verdaderos (o suceder simultáneamente), Un ejemplo de ello es el resultado de arrojar una vez una moneda, el cual solo puede ser cara o cruz, pero no ambos.  