La complejidad temporal de una estructura de datos Trie para la operaciónes de inserción/eliminación/búsqueda es solo O($N$), dónde $N$ es la longitud de la clave. La complejidad espacial de dichas operaciones la inserción su complejidad es O($N\times M$) donde $N$ es la longitud de la cadena almacenar y $M$ es la cantidad de cadenas almacenadas, mientras para las otras dos operaciones es O($1$).

La complejidad espacial de una estructura de datos Trie es O($N\times M\times C$), dónde $N$ es el número total de strings, $M$ es la longitud máxima de la string, y $C$ es el tamaño del alfabeto. El problema de almacenamiento se puede aliviar si solo asignamos memoria para los alfabetos en uso y no desperdiciamos espacio almacenando punteros nulos.