A continuación, se presentan algunas aplicaciones comunes de la búsqueda ternaria:
\begin{itemize}
	\item \textbf{Optimización en funciones unimodales:} La búsqueda ternaria se utiliza para encontrar el máximo o mínimo de una función unimodal (una función que tiene un solo máximo o mínimo) en un intervalo dado. Dividiendo el intervalo en tres partes y comparando los valores de la función en los puntos intermedios, se puede encontrar el punto óptimo con mayor precisión que con la búsqueda binaria.
	\item \textbf{Búsqueda en funciones cóncavas o convexas:} En funciones cóncavas o convexas, la búsqueda ternaria puede utilizarse para encontrar el punto en el que la función alcanza su máximo o mínimo. Dividiendo el intervalo en tres partes y determinando en qué dirección se encuentra el punto óptimo, se puede reducir el espacio de búsqueda de manera eficiente.
	\item \textbf{Determinación de umbrales:} La búsqueda ternaria también se utiliza para determinar umbrales o puntos críticos en una función. Por ejemplo, si se desea encontrar el punto en el que una función supera cierto umbral, la búsqueda ternaria puede ser útil para acercarse a ese valor de manera eficiente.
	\item \textbf{Problemas de optimización numérica:} En problemas de optimización numérica donde se busca maximizar o minimizar una función en un intervalo, la búsqueda ternaria puede ser una herramienta útil para encontrar soluciones cercanas al óptimo.
\end{itemize}
 
En resumen, la búsqueda ternaria es una técnica de búsqueda eficiente que puede aplicarse en una variedad de situaciones donde se necesita encontrar el máximo o mínimo de una función, determinar umbrales o puntos críticos, o resolver problemas de optimización numérica. Su capacidad para dividir el espacio de búsqueda en tres partes puede conducir a una convergencia más rápida y precisa en comparación con otros métodos de búsqueda.