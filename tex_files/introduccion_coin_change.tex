Cuando no dirigimos hacia un cajero automático y solicitamos un cantidad $N$ el cajero puede que nos de dos variantes de respuesta, la primera es que no nos puede entregar la cantidad $N$ solicitada por que no tiene como dar esa cantidad bien sea porque con la denominaciones de los billetes con que cuenta el cajero no puede conformar la cantidad requerida o $N$ es tan grande que con todo los billetes disponibles en el cajero no alcanza la cifra solicitada. La segunda es una cantidad de billetes que sumados dan la cantidad $N$ solicitada inicialmente.

Ahora vamos a pensar en un cajero ideal el cual no tendrá nunca cola, ni estará fuera de servicio y además contará con un sistema de impresión de billetes al instante que utilizará en caso que necesite generar un billete de un determinado valor por lo que nunca se quedará sin billete de cualquier valor.

Con este nuevo cajero a pesar de todas sus bondades nos surgen dos interrogantes:

\begin{itemize}
	\item Dado un valor de $N$ y conocidas la denominaciones de los billetes el cajero podrá devolver esa cantidad $N$ y poder de cuantas formas podrá hacerlo ?.
	\item Dado un valor de $N$ y conocidas la denominaciones de los billetes cual es la cantidad de mínima de billetes que se necesita para obtener el valor de $N$ ?. 
\end{itemize} 

A como resolver estas interrogantes esta enfocada la siguiente guía de aprendizaje.