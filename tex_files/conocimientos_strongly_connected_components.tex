\subsection{Recorrido en profundidad}
La búsqueda en profundidad (DFS) es un algoritmo para atravesar o buscar estructuras de datos de árboles o grafos. El algoritmo comienza en el nodo raíz (seleccionando algún nodo arbitrario como nodo raíz en el caso de un grafo) y explora lo más lejos posible a lo largo de cada rama antes de retroceder. Se necesita memoria adicional, generalmente una pila, para realizar un seguimiento de los nodos descubiertos hasta el momento a lo largo de una rama específica, lo que ayuda a retroceder en el grafo.

\subsection{Ordenación topológica}
Una ordenación topológica (topological sort, topological ordering, topsort o toposort en inglés) de un grafo acíclico dirigido G es una ordenación lineal de todos los nodos de G que satisface que si G contiene la arista dirigida uv entonces el nodo u aparece antes del nodo v. La condición que el grafo no contenga ciclos es importante, ya que no se puede obtener ordenación topológica de grafos que contengan ciclos. 