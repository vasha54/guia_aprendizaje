\subsection{C++}

\begin{lstlisting}[language=C++]
/*Range Tree implementado para hallar el minimo en el rango*/	
#include <stdio.h>
#include <iostream>
#include <algorithm>
#define MID (left+right)/2
#define MAX_N 1000001
#define MAX_TREE (MAX_N << 2)
#define INF 987654321
using namespace std;
typedef long long lld;

int n;
int niz[MAX_N];
int RT[MAX_TREE];

void InitTree(int idx, int left, int right){
   if(left == right){
     RT[idx] = niz[left];return;
   }
   InitTree(2*idx, left, MID);
   InitTree(2*idx+1, MID+1, right);
   RT[idx] = min(RT[2*idx], RT[2*idx+1]);
}

void Update(int idx, int x, int val, int left, int right){
   if(left == right){
     RT[idx] = val;return;
   }
   if (x <= MID) Update(2*idx, x, val, left, MID);
   else Update(2*idx+1, x, val, MID+1, right);
   RT[idx] = min(RT[2*idx], RT[2*idx+1]);
}

int Query(int idx, int l, int r, int left, int right){
   if (l <= left && right <= r) return RT[idx];
   int ret = INF;
   if (l <= MID) ret = min(ret, Query(2*idx, l, r, left, MID));
   if (r > MID) ret = min(ret, Query(2*idx+1, l, r, MID+1, right));
   return ret;
}

int main(){
  n = 6;
  niz[1] = 4;
  niz[2] = 2;
  niz[3] = 5;
  niz[4] = 1;
  niz[5] = 6;
  niz[6] = 3;
	
  InitTree(1, 1, n);
  printf("%d\n",Query(1, 1, 3, 1, n));
  Update(1, 4, 10, 1, n);
  Update(1, 5, 0, 1, n);
  printf("%d\n",Query(1, 4, 6, 1, n));
	
  return 0;
}	
\end{lstlisting}


\subsection{Java}

\begin{lstlisting}[language=Java]
class RangeTree{
  int [] niz;
  int [] RT;
  final int INF = 987654321;
	
  public RangeTree(int [] _array){
     niz = Arrays.copyOf(_array, _array.length);
     RT = new int [niz.length << 2];
  }
	
  public RangeTree(int _n){
     niz = new int [_n]; 
     RT = new int [_n << 2];
  }
	
  public void initTree(int idx, int left, int right){
     if(left == right){ RT[idx] = niz[left];}
     else{
       int MID =(left+right)/2;
       initTree(2*idx, left, MID);
       initTree(2*idx+1, MID+1, right);
       RT[idx] = Math.min(RT[2*idx], RT[2*idx+1]);
     }
  }
	
  public void update(int idx, int x, int val, int left, int right){
     if(left == right){ RT[idx] = val;}
     else{
       int MID =(left+right)/2;
       if (x <= MID) update(2*idx, x, val, left, MID);
       else update(2*idx+1, x, val, MID+1, right);
       RT[idx] = Math.min(RT[2*idx], RT[2*idx+1]);
     }
  }
	
  public int query(int idx, int l, int r, int left, int right){
     if (l <= left && right <= r) return RT[idx];
	 int ret = INF;
     int MID =(left+right)/2;
     if (l <= MID) ret = Math.min(ret, query(2*idx, l, r, left, MID));
     if (r > MID) ret = Math.min(ret, query(2*idx+1, l, r, MID+1, right));
        return ret;
  }	
}	
\end{lstlisting} 