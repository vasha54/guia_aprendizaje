\subsection{Estructura de datos}
Una estructura de datos es una forma de organizar un conjunto de datos elementales con el objetivo
de facilitar su manipulación. Un dato elemental es la mínima información que se tiene en un sistema.
Una estructura de datos define la organización e interrelación de estos y un conjunto de operaciones
que se pueden realizar sobre ellos. Cada estructura ofrece ventajas y desventajas en relación a la
simplicidad y eficiencia para la realización de cada operación. De está forma, la elección de la
estructura de datos apropiada para cada problema depende de factores como la frecuencia y el orden
en que se realiza cada operación sobre los datos.

\subsection{Estructuras de datos estáticas compuestas}
Las estructuras de datos estáticas son aquellas que el tamaño de las mismas no puede ser modificadas en la
ejecución del algoritmo. Así su tamaño tiene que ser defino en la creación de la misma. Está estructura almacena varios elementos del mismo tipo en forma lineal.

\subsection{Operador \emph{new}}
C++ y Java permite a los programadores controlar la asignación  de memoria en un programa, para cualquier tipo integrado o definido por el usuario. Esto se conoce como administración dinámica de memoria y se lleva a cabo
mediante el operador \emph{new}. Podemos usar el operador \emph{new} para asignar (reservar) en forma dinámica la cantidad exacta de memoria requerida para contener cada nombre en tiempo de ejecución. La
asignación dinámica de memoria de esta forma hace que se cree un arreglo (o cualquier otro tipo integrado o definido
por el usuario) en el almacenamiento libre (algunas veces conocido como el heap o montón): una región de memoria
asignada a cada programa para almacenar los objetos que se asignan en forma dinámica. Una vez que se asigna la memoria para un arreglo en el almacenamiento libre, podemos obtener acceso a éste si apuntamos un apuntador al primer elemento del arreglo.

\subsection{Esctructura de repetición (bucles o ciclos)}

Un \textbf{bucle} se utiliza para realizar un proceso repetidas veces. Se denomina también \textbf{lazo} o \textbf{loop}. El
código incluido entre las llaves \{\} (opcionales si el proceso repetitivo consta de una sola línea), se
ejecutará mientras se cumpla unas determinadas condiciones. Hay que prestar especial atención a
los bucles infinitos, hecho que ocurre cuando la condición de finalizar el bucle
(booleanExpression) no se llega a cumplir nunca. Se trata de un fallo muy típico, habitual sobre
todo entre programadores poco experimentados. En el caso de los lenguajes de programación de C++ y Java se cuenta con varias instrucciones de tipo bucle o ciclo como son el \textbf{for},\textbf{while} y \textbf{do ... while}