La función del totiente de Euler en su implementacion utiliza la factorización por tanto su complejidad será de O$(\sqrt{n})$. El caso de encontrar el totiente del 1 al $n$ usando la propiedad de la suma del divisor su implementación es similar a la criba de Eratosthenes por tanto su complejidad será de O$(n \log\log{n})$.