La criba de Eratóstenes es un algoritmo que permite hallar todos los números primos menores que un número natural dado N. Se forma una tabla con todos los números naturales comprendidos entre 2 y n, y se van tachando los números que no son primos de la siguiente manera: Comenzando por el 2, se tachan todos sus múltiplos; comenzando de nuevo, cuando se encuentra un número entero que no ha sido tachado, ese número es declarado primo, y se procede a tachar todos sus múltiplos, así sucesivamente. El proceso termina cuando el cuadrado del mayor número confirmado como primo es mayor que n.

Un refinamiento de la criba consiste en tachar los múltiplos del k-ésimo número primo pk, comenzando por pk2 pues en los anteriores pasos se habían tachado los múltiplos de pk correspondientes a todos los anteriores números primos, esto es, 2pk, 3pk, 5pk,..., hasta (pk-1)pk. El algoritmo acabaría cuando $p^{2}k$>n ya que no habría nada que tachar.