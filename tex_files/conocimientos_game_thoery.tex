\subsection{Operador xor}
En lógica proposicional, la disyunción exclusiva (también llamado bidisyuntor lógico, disyuntor excluyente, or fuerte, or exclusivo, o desigualdad material) es un operador lógico simbolizado como XOR. Una disyunción exclusiva solamente es verdadera cuando ambas frases tienen valores diferentes y es falsa si las dos frases son ambas verdaderas o ambas falsas. 

\subsection{Juego de información perfecta}
Un juego de información perfecta se refiere al hecho de que cada jugador tiene la misma información que estaría disponible al final del juego. Es decir, cada jugador sabe o puede ver los movimientos de otros jugadores. Un buen ejemplo sería el ajedrez, donde cada jugador ve las piezas del otro jugador en el tablero.

\subsection{Juego de información imperfecta}
La información imperfecta aparece cuando las decisiones tienen que hacerse simultáneamente, y los jugadores necesitan analizar todos los posibles resultados a la hora de tomar una decisión. Un buen ejemplo de juegos de información imperfecta es un juego de cartas donde las cartas de cada jugador están escondidas del resto de los jugadores.