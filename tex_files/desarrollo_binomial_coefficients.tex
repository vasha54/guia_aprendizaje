Los coeficientes binomiales también son los coeficientes en la expansión de $ (a + b) ^ n $ (el llamado teorema binomial):

$$(a+b)^n = \binom n 0 a^n + \binom n 1 a^{n-1} b + \binom n 2 a^{n-2} b^2 + \cdots + \binom n k a^{n-k} b^k + \cdots + \binom n n b^n$$

Se cree que esta fórmula, así como el triángulo que permite un cálculo eficiente de los coeficientes, fue descubierta por Blaise Pascal en el siglo XVII. Sin embargo, era conocido por el matemático chino Yang Hui, que vivía en el siglo XIII. Quizás fue descubierto por un erudito persa Omar Khayyam. Además, el matemático indio Pingala, que vivió antes en el 3er. AC, obtuvo resultados similares. El mérito del Newton es que generalizó esta fórmula para los exponentes que no son naturales.

Hay dos fórmulas para los números catalanes: \textbf{recursiva} y \textbf{analítica}. Dado que creemos que todos los problemas donde se utilizan son equivalentes (tienen la misma solución), para la prueba de las fórmulas siguientes elegiremos la tarea que sea más fácil de realizar.



\subsection{Fórmula analítica}
La fórmula analítica para el cálculo es:

$$\binom n k = \frac {n!} {k!(n-k)!}$$

Esta fórmula se puede deducir fácilmente del problema del acuerdo ordenado (número de formas de seleccionar $k$ diferentes elementos de $n$ diferentes elementos). Primero, cuentemos el número de selecciones ordenadas de $k$ elementos. Hay $n$ formas de seleccionar el primer elemento, $ N-1 $ formas de seleccionar el segundo elemento, $n-2$ formas de seleccionar el tercer elemento, y así sucesivamente. Como resultado, obtenemos la fórmula del número de arreglos ordenados:
$n \times(n-1)\times (n-2)\times \ldots \times (n-k +1) = \frac{n!}{(n-k)!}$. Podemos pasar fácilmente a los arreglos desordenados, señalando que cada acuerdo desordenado corresponde a los arreglos ordenados exactamente por $k!$ ($k!$ Es el número de permutaciones posibles de los elementos de $k$). Obtenemos la fórmula final dividiendo
$\frac{n!}{(n-k)!}$ por $k!$.

Hay $n!$ Permutaciones de $n$ elementos. Revisamos todas las permutaciones y siempre incluimos los primeros $k$ elementos de la permutación en el subconjunto. Dado que el orden de los elementos en el subconjunto y fuera del subconjunto no importa, el resultado se divide por $k!$ y $(n-k)!$




\subsection{Fórmula recursiva}
Los coeficientes binomiales se pueden calcular de manera recursiva de la siguiente manera:

$$\binom n k = \binom {n-1} {k-1} + \binom {n-1} k$$

que está asociado con el famoso Triángulo de Pascal. Es fácil deducir esto utilizando la fórmula analítica. La idea es agregar un elemento $x$ en el conjunto. Si se incluye $x$ en el subconjunto, tenemos que elegir elementos $k-1$ de los elementos $n-1$, y si $x$ no está incluido en el subconjunto, tenemos que elegir $k$ elementos de los elementos $n-1$.

Tenga en cuenta que para $n < k$ se supone que el valor de $\binom n k$ es cero.

La idea es fijar un elemento $x$ en el conjunto. Si $x$ está incluido en el subconjunto, tenemos que elegir $k-1$ elementos de $n-1$ elementos, y si $x$ no está incluido en el subconjunto, tenemos que elegir $k$ elementos de $n-1$ elementos.

Los casos base para la recursividad son:
$$\binom n 0 = \binom n n = 1$$

porque siempre hay exactamente una manera de construir un subconjunto vacío y un subconjunto que contenga todos los elementos.



\subsection{Propiedades}
Los coeficientes binomiales tienen muchas propiedades diferentes. Aquí están las más simples:

\begin{enumerate}
	\item Regla de simetría: $\binom n k = \binom n {n-k}$
	\item Teniendo en cuenta: $\binom n k = \frac n k \binom {n-1} {k-1}$
	\item Suma sobre $k$ : $\sum_{k = 0}^n \binom n k = 2 ^ n$
	\item Suma de $n$ y $k$: $\sum_{k = 0}^m \binom {n + k} k = \binom {n + m + 1} m$
	\item Suma de los cuadrados: ${\binom n 0}^2 + {\binom n 1}^2 + \cdots + {\binom n n}^2 = \binom {2n} n$
	\item Suma ponderada: $1 \binom n 1 + 2 \binom n 2 + \cdots + n \binom n n = n 2^{n-1}$
	\item Conexión con los números de Fibonacci: $\binom n 0 + \binom {n-1} 1 + \cdots + \binom {n-k} k + \cdots + \binom 0 n = F_{n+1}$
\end{enumerate}

\subsection{Cálculo}
\subsubsection{Cálculo sencillo mediante fórmula analítica}
La primera fórmula sencilla es muy fácil de codificar, pero es probable que este método se desborde incluso para valores relativamente pequeños de $n$ y $k$ (incluso si la respuesta encaja completamente en algún tipo de datos, el cálculo de los factoriales intermedios puede provocar un desbordamiento). Por lo tanto, este método a menudo sólo se puede utilizar con aritmética larga.

\subsubsection{Implementación mejorada}

Tenga en cuenta que en la idea anterior el numerador y el denominador tienen el mismo número de factores ($k$), cada uno de los cuales es mayor o igual a 1. Por lo tanto, podemos reemplazar nuestra fracción con un producto $k$ fracciones, cada una de las cuales tiene un valor real. Sin embargo, en cada paso después de multiplicar la respuesta actual por cada una de las siguientes fracciones, la respuesta seguirá siendo un número entero (esto se desprende de la propiedad de factorizar).

Aquí convertimos cuidadosamente el número de coma flotante a un número entero, teniendo en cuenta que debido a los errores acumulados, puede ser ligeramente menor que el valor real (por ejemplo, $2.99999$ en lugar de $3$).

\subsubsection{Triángulo de Pascal}

Utilizando la relación de recurrencia podemos construir una tabla de  coeficientes binomiales (triángulo de Pascal) y obtener el resultado de  ella. La ventaja de este método es que los 
resultados intermedios nunca exceden la respuesta y el cálculo de cada nuevo elemento de la tabla requiere solo una suma. El defecto es la ejecución lenta para grandes $n$ y $k$ si solo necesita 
un valor único y no toda la tabla (porque para calcular $\binom n k$ necesitarás construir una tabla de todos $\binom i j, 1 \le i \le n, 1 \le j \le n$, o al menos a $1 \le j \le \min (i,2k)$). Se puede considerar que la complejidad del tiempo es O $(n^2)$.

\subsubsection{Cálculo en O($1$) }

Finalmente, en algunas situaciones es beneficioso precalcular todos los factoriales para producir cualquier coeficiente binomial necesario con sólo dos divisiones después. Esto puede resultar ventajoso cuando se utiliza aritmética larga , cuando la memoria no permite el cálculo previo de todo el triángulo de Pascal.

\subsection{Calcular coeficientes binomiales con módulo $m$}
Muy a menudo te encuentras con el problema de calcular coeficientes binomiales módulo algunos $m$.

\subsubsection{Coeficiente de binomial para pequeños $n$}

El enfoque previamente discutido del triángulo de Pascal se puede utilizar para calcular todos los valores de $\binom n  k  \bmod m$ por un tamaño razonablemente pequeño $n$, ya que requiere complejidad de tiempo O$(n^2)$. Este enfoque puede manejar cualquier módulo, ya que solo se utilizan operaciones de suma.

\subsubsection{Coeficiente binomial módulo primo grande}

La fórmula para los coeficientes binomiales es:

$$\binom n k = \frac {n!} {k!(n-k)!},$$

entonces si queremos calcularlo módulo algún primo $m > n$ obtenemos:

$$\binom n k \equiv n! \cdot (k!)^{-1} \cdot ((n-k)!)^{-1} \mod m$$

Primero calculamos previamente todos los módulos factoriales $m$ hasta $\text{MAXN}!$ en O($\text{MAXN}$) tiempo. Y luego podemos calcular el coeficiente binomial en O$(\log m)$ tiempo. 
Incluso podemos calcular el coeficiente binomial en $O(1)$ tiempo si precalculamos los inversos multiplicativos de todos los factoriales en O$(\text{MAXN} \log m)$ utilizando el método habitual 
para calcular el inverso multiplicativo, o incluso en O$(\text{MAXN})$ tiempo usando la congruencia $(x!)^{-1} \equiv ((x-1)!)^{-1} \cdot x^{-1}$ y el método para calcular todas todos los 
inversos en O$(n)$.

\subsubsection{Coeficiente binomial módulo potencia prima}

Aquí queremos calcular el coeficiente binomial módulo de alguna potencia prima, es decir $m = p^b$ por alguna prima $p$. Si $p > \max(k,nk)$, entonces podemos usar el mismo método descrito en la sección anterior. Pero si $p \le \max(k, nk)$, entonces al menos uno de $k!$ y $(nk)!$ no son coprimos con $m$, y por lo tanto no podemos calcular las inversas: no existen. Sin embargo, podemos calcular el coeficiente binomial.

La idea es la siguiente: Calculamos para cada $x!$ el mayor exponente $c$ tal que $p^c$ divide $x!$, es decir $p^c ~|~ x!$. Dejar $c(x)$ sea es número. Y deja $g(x) := \frac{x!}{p^{c(x)}}$. Entonces podemos escribir el coeficiente binomial como:

$$\binom n k = \frac {g(n) p^{c(n)}} {g(k) p^{c(k)} g(n-k) p^{c(n-k)}} = \frac {g(n)} {g(k) g(n-k)}p^{c(n) - c(k) - c(n-k)}$$

Lo interesante es que $g(x)$ ahora está libre del divisor primo $p$. Por lo tanto $g(x)$ es coprimo de m, y podemos calcular los inversos modulares de $g(k)$ y $g(nk)$.

Después de precalcular todos los valores para $g$ y $c$, que se puede hacer de manera eficiente usando programación dinámica en O($n$), podemos calcular el coeficiente binomial en $O(\log m)$
tiempo. O precalcular todas las inversas y todas las potencias de $p$ y luego calcular el coeficiente binomial en $O(1)$.

Aviso, si $c(n) - c(k) - c(nk) \ge b$, que $p^b ~|~ p^{c(n) - c(k) - c(nk)}$, y el coeficiente binomial es $0$

\subsubsection{Módulo de coeficiente binomial un número arbitrario}

Ahora calculamos el módulo del coeficiente binomial algún módulo arbitrario $m$.

Sea la factorización prima de $m$ ser $m=p_1^{e_1} p_2^{e_2} \cdots p_h^{e_h}$. Podemos calcular el módulo del coeficiente binomial $p_i^{e_i}$ para cada $i$. esto nos da $h$ diferentes 
congruencias. Dado que todos los módulos $p_i^{e_i}$ son coprimos, podemos aplicar el teorema chino del resto para calcular el módulo del coeficiente binomial, el producto de los módulos, que es 
el módulo del coeficiente binomial deseado $m$.

\subsubsection{Coeficiente binomial para grandes $n$ y módulo pequeño}

Cuando $n$ es demasiado grande, el O($n$) los algoritmos discutidos anteriormente se vuelven poco prácticos. Sin embargo, si el módulo $m$ es pequeño todavía hay maneras de calcular $\binom{n}{k} \bmod m$ cuando el módulo $m$ es primo, hay 2 opciones:

\begin{itemize}
	\item Se puede aplicar el teorema de Lucas, lo que resuelve el problema de la informática $\binom{n}{k} \bmod m$ en $\log_mn$ problemas de la forma $\binom{x_i}{y_i} \bmod m$
	dónde $x_i, y_i < m$. Si cada coeficiente reducido se calcula utilizando factoriales precalculados y factoriales inversos, la complejidad es O($m + \log_m n$).
	\item El método de calcular el módulo factorial $P$ se puede utilizar para obtener el valor requerido $g$ y $c$ valores y utilícelos como se describe en la sección de módulo de potencia 
	primaria esto toma O$(m \log_m n)$.
\end{itemize}

Cuando $m$ no está libre de cuadrados, se puede aplicar una generalización del teorema de Lucas para potencias primas en lugar del teorema de Lucas.

Cuando $m$ no es primo sino libre de cuadrados, los factores primos de $m$ se puede obtener y el módulo de coeficiente de cada factor primo se puede calcular utilizando cualquiera de los métodos anteriores, y la respuesta general se puede obtener mediante el teorema del resto chino.




