\subsection{Función unimodal}
Una función unimodal es una función matemática que tiene un único máximo o mínimo absoluto en un intervalo determinado. Esto significa que la función aumenta o disminuye hasta alcanzar un punto máximo o mínimo y luego vuelve a cambiar de dirección. En otras palabras, la función tiene una forma de \emph{campana} con un solo pico o valle. Por función unimodal, nos referimos a uno de dos comportamientos de la función:

\begin{enumerate}
	\item La función primero aumenta estrictamente, alcanza un máximo (en un solo punto o en un intervalo) y luego disminuye estrictamente.
	\item La función primero disminuye estrictamente, alcanza un mínimo y luego aumenta estrictamente.
\end{enumerate}

\subsection{Teorema de Maestro}
En el análisis de algoritmos , el teorema maestro de las recurrencias de divide y vencerás proporciona un análisis asintótico para muchas relaciones de recurrencia que ocurren en el análisis de los algoritmos de divide y vencerás . El enfoque fue presentado por primera vez por Jon Bentley , Dorothea Blostein (de soltera Haken) y James B. Saxe en 1980, donde se describió como un método unificador para resolver tales recurrencias. El nombre \emph{teorema maestro} fue popularizado por el libro de texto de algoritmos ampliamente utilizado Introducción a los algoritmos de Cormen , Leiserson , Rivest y Stein .