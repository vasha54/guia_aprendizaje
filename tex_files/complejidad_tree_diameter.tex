El algoritmo que utiliza dos dfs tiene una complejidad de O($2E$) ya que realizamos dos dfs pero la constante $2$ no influye por tanto su complejidad es O($E$)  siendo $E$ la cantidad de aristas del árbol que por definición de un árbol va ser igual a la cantidad de nodos del árbol menos uno.

El algoritmo que utiliza la programación dinámica tiene una complejidad temporal O($N$) siendo $N$ la cantidad de nodos del árbol 