\subsection{C++}

\subsubsection{Implementación sencilla}

\begin{lstlisting}[language=C++]
void make_set(int v) {
   parent[v] = v;
}

int find_set(int v) {
   if (v == parent[v]) return v;
   return find_set(parent[v]);
}

void union_sets(int a, int b) {
   a = find_set(a); b = find_set(b);
   if (a != b) parent[b] = a;
}
\end{lstlisting}

\subsubsection{Optimización de la compresión de rutas}

\begin{lstlisting}[language=C++]
int find_set(int v) {
   if (v == parent[v]) return v;
   return parent[v] = find_set(parent[v]);
}
\end{lstlisting}

\subsubsection{Unión por tamaño / rango}

Aquí está la implementación de la unión por tamaño:

\begin{lstlisting}[language=C++]
void make_set(int v) {
   parent[v] = v; size[v] = 1;
}

void union_sets(int a, int b) {
   a = find_set(a); b = find_set(b);
   if (a != b) {
     if (size[a] < size[b]) swap(a, b);
     parent[b] = a;
     size[a] += size[b];
   }
}
\end{lstlisting}

Y aquí está la implementación de la unión por rango basada en la profundidad de los árboles:

\begin{lstlisting}[language=C++]
void make_set(int v) {
   parent[v] = v; rank[v] = 0;
}

void union_sets(int a, int b) {
   a = find_set(a); b = find_set(b);
   if (a != b) {
     if (rank[a] < rank[b]) swap(a, b);
     parent[b] = a;
     if (rank[a] == rank[b]) rank[a]++;
   }
}
\end{lstlisting}

Aqui una implementación completa con estructura

\begin{lstlisting}[language=C++]
struct DisjoinSet{
   vector<int> parent,sizes;
   int ncomponents; // cantidad de conjuntos disjuntos
   
   DisjoinSet(int n) : parent(n),sizes(n){
      for(int i = 0; i < n; i++) sizes[parent[i] = i] = 1;
      ncomponents = n;
   }
	
   int root(int x){
      if( x == parent[x] ) return x;
      else{
         parent[x] = root(parent[x])
         return parent[x]
      }
   }
	
   void join(int a,int b){
      int x = root(a); int y = root(b);
      if(x == y) return;
      if(sizes[x] < sizes[y]) swap(x,y);
      parent[y] = x;
      sizes[x] += sizes[y];
      ncomponents--;
   }
};
\end{lstlisting}

\subsection{Java}

\subsubsection{Implementación sencilla}

\begin{lstlisting}[language=C++]
public void make_set(int v) {
    parent[v] = v;
}
	
public int find_set(int v) {
   if (v == parent[v]) return v;
   return find_set(parent[v]);
}
	
public void union_sets(int a, int b) {
   a = find_set(a); b = find_set(b);
   if (a != b) parent[b] = a;
}
\end{lstlisting}

\subsubsection{Optimización de la compresión de rutas}

\begin{lstlisting}[language=C++]
public int find_set(int v) {
   if (v == parent[v]) return v;
   return parent[v] = find_set(parent[v]);
}
\end{lstlisting}

\subsubsection{Unión por tamaño / rango}

Aquí está la implementación de la unión por tamaño:

\begin{lstlisting}[language=C++]
public void make_set(int v) {
   parent[v] = v; size[v] = 1;
}
	
public void union_sets(int a, int b) {
   a = find_set(a); b = find_set(b);
   if (a != b) {
     if (size[a] < size[b]) swap(a, b);
     parent[b] = a;
     size[a] += size[b];
   }
}
\end{lstlisting}

Y aquí está la implementación de la unión por rango basada en la profundidad de los árboles:

\begin{lstlisting}[language=C++]
public void make_set(int v) {
   parent[v] = v; rank[v] = 0;
}
	
public void union_sets(int a, int b) {
   a = find_set(a); b = find_set(b);
   if (a != b) {
     if (rank[a] < rank[b]) swap(a, b);
     parent[b] = a;
     if (rank[a] == rank[b]) rank[a]++;
   }
}
\end{lstlisting}

Aqui una implementación completa con clases

\begin{lstlisting}[language=Java]
private class DisJoinSet{
   public int [] parent;
   public int [] sizes ;
   public int ncomponents;
	
   public DisJoinSet(int n) {
      this.ncomponents = n;
      parent = new int [this.ncomponents+1];
      sizes = new int [this.ncomponents+1];
      for(int i=0;i<this.ncomponents+1;i++) sizes[parent[i]=i]=1;
   }
   
   public int root(int x){
      if (x == parent[x]) return x;
      else{
         parent[x] = root(parent[x]);
         return parent[x];
      }
   }
	
   void join(int a,int b){
      int x = root(a); int y = root(b);
      if(x != y) {
         if(sizes[x] < sizes[y]) {
            int tmp = x; x = y; y =tmp;
         }
         parent[y] = x;
         sizes[x] += sizes[y];
         this.ncomponents--;
      }
   }
}
\end{lstlisting}