\subsection{Operador módulo}

El operador módulo representado en la mayoría de los lenguajes de programación con símbolo (\%) da como resultado el resto de la división entera. Por ejemplo 20\%7 da como resultado 6 que es el resto de la división entre 20 y 7.

\subsection{El teorema del cociente y del residuo}

Dado cualquier entero A y un entero positivo B, existen dos enteros únicos Q y R tales que:
A= B * Q + R donde 0 $\leq$ R < B
Podemos ver que esto viene directamente de la división larga. Cuando dividimos A entre B en la división larga, Q es el cociente y R es el residuo. Si podemos escribir un número en esta forma, entonces A mod B = R.

\subsection{Función Totient de Euler}
Función totient de Euler, también conocida como función $\phi(n)$, cuenta el 
número de enteros entre 1 y $n$ inclusive, que son coprimos de $n$. Dos números 
son coprimos si su máximo común divisor es igual $1$ ($1$ se considera coprimo 
de cualquier número).

\subsection{División Euclidiana}
En aritmética , la división euclidiana , o división con resto , es el proceso de dividir un número entero (el dividendo) por otro (el divisor), de manera que se produce un cociente entero y un resto de número natural estrictamente menor que el valor absoluto del número . divisor. Una propiedad fundamental es que el cociente y el resto existen y son únicos, bajo ciertas condiciones. Debido a esta singularidad, la división euclidiana a menudo se considera sin hacer referencia a ningún método de cálculo y sin calcular explícitamente el cociente y el resto. Los métodos de cálculo se denominanalgoritmos de división de enteros , el más conocido de los cuales es la división larga .

