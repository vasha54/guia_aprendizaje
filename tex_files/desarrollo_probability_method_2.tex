Otra forma de calcular la probabilidad es simular el proceso que genera el evento. En este ejemplo, robamos tres cartas, por lo que el proceso consta de tres pasos. Requerimos que cada paso del proceso sea exitoso.

Sin duda, sacar la primera carta es un éxito, porque no hay restricciones. El segundo paso tiene éxito con probabilidad 3/51, porque quedan 51 cartas. y 3 de ellas tienen el mismo valor que la primera carta. De manera similar, el tercero el paso tiene éxito con una probabilidad de 2/50.

La probabilidad de que todo el proceso tenga éxito es:

$$1 \times \dfrac{3}{51} \times \dfrac{2}{50} = \dfrac{1}{425} $$