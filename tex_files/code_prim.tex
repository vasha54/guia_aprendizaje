\subsection{C++}

\subsubsection{Grafos densos}
\begin{lstlisting}[language=C++]
int n; //cantidad de nodos
vector<vector<int> > adj; //matrix de adyacencia
const int INF = 1000000000; //el peso INF significa que no hay arista
struct Edge { int w = INF, to = -1; };

void prim() {
   int total_weight = 0;
   vector<bool> selected(n, false);
   vector<Edge> min_e(n);
   min_e[0].w = 0;
	
   for(int i=0;i<n;++i){
      int v = -1;
      for(int j = 0; j < n; ++j) {
         if(!selected[j] && (v==-1 || min_e[j].w<min_e[v].w))
            v = j;
      }
      if(min_e[v].w == INF) {
         cout << "No existe MST!" << endl;
         i=n+1;
         continue;
      }
      
      selected[v] = true;
      total_weight += min_e[v].w;
      if(min_e[v].to!=-1)
         cout <<v<<" "<<min_e[v].to<<endl;
      for(int to=0;to<n;++to){
         if(adj[v][to] < min_e[to].w)
            min_e[to] = {adj[v][to], v};
      }
   }
   cout << total_weight << endl;
}
\end{lstlisting}

La matriz de adyacencia adj[][] de tamaño $n\times n$ almacena los pesos de las aristas, y usa el peso INF si no existe una arista entre dos vértices. El algoritmo utiliza dos matrices: el vector selected[], que indica qué vértices ya hemos seleccionado, y la matriz min\_e[]que almacena la arista con peso mínimo en un vértice seleccionado para cada vértice aún no seleccionado (almacena el peso y el vértice final). El algoritmo hace $n$ pasos, en cada iteración se selecciona el vértice con el peso de borde más pequeño y min\_e[]se actualiza el de todos los demás vértices.

\subsubsection{Grafos dispersos}
\begin{lstlisting}[language=C++]
const int INF = 1000000000;

struct Edge {
   int w = INF, to = -1;
   bool operator<(Edge const& other) const {
      return make_pair(w, to) < make_pair(other.w, other.to);
   }
};

int n;
vector<vector<Edge>> adj;

void prim() {
   int total_weight = 0;
   vector<Edge> min_e(n);
   min_e[0].w = 0;
   set<Edge> q;
   q.insert({0, 0});
   vector<bool> selected(n, false);
   for(int i = 0; i<n; ++i){
      if(q.empty()) {
         cout << "No existe MST!" << endl;
         i=n+1;
         continue;
      }
      int v = q.begin()->to;
      selected[v] = true;
      total_weight += q.begin()->w;
      q.erase(q.begin());
      if(min_e[v].to != -1)
         cout << v << " " << min_e[v].to << endl;
      for(Edge e : adj[v]) {
         if (!selected[e.to] && e.w < min_e[e.to].w) {
            q.erase({min_e[e.to].w, e.to});
            min_e[e.to] = {e.w, v};
            q.insert({e.w, e.to});
         }
      }
   }
   cout << total_weight << endl;
}
\end{lstlisting}

Aquí, el grafo se representa a través de una lista de adyacencia adj[], donde adj[v] contiene todas las aristas (en forma de pares de peso y vértice) para el vértice v. min\_e[v] almacenará el peso del borde más pequeño desde el vértice v hasta un vértice ya seleccionado (nuevamente en forma de un par de peso y vértice). Además, la cola q se llena con todos los vértices aún no seleccionados en orden de peso creciente min\_e. El algoritmo realiza $n$ pasos, en cada uno de los cuales selecciona el vértice v con el peso más pequeño min\_e (extrayéndolo del principio de la cola), y luego examina todas las aristas este vértice y actualiza los valores en min\_e (durante una actualización también también es necesario eliminar la antigua arista de la cola $q$ y coloque la nueva arista).

\subsection{Java}

\subsubsection{Grafos densos}
\begin{lstlisting}[language=Java]
public long mstPrim(int[][] d) {
   int n = d.length;
   int[] prev = new int[n];
   int[] dist = new int[n];
   Arrays.fill(dist, Integer.MAX_VALUE);
   dist[0] = 0;
   boolean[] visited = new boolean[n];
   long res = 0;
   for(int i=0;i<n;i++){
      int u = -1;
      for(int j=0;j<n;j++){
         if(!visited[j] && (u==-1 || dist[u]>dist[j])) u = j;
      }
      res += dist[u]; visited[u] = true;
	  for(int j=0;j<n;j++){
         if(!visited[j] && dist[j]>d[u][j]){
            dist[j] = d[u][j]; prev[j] = u;
         }
      }
   }
   return res;
}
\end{lstlisting}

\subsubsection{Grafos dispersos}
\begin{lstlisting}[language=Java]
public long mstPrim(List<Edge>[] edges, int[] pred) {
   int n = edges.length;
   Arrays.fill(pred, -1);
   boolean[] used = new boolean[n];
   int[] prio = new int[n];
   Arrays.fill(prio, Integer.MAX_VALUE);
   prio[0] = 0;
   PriorityQueue<Long> q = new PriorityQueue<>();
   q.add(0L);
   long res = 0;
   
   while (!q.isEmpty()) {
      long cur = q.poll();
      int u = (int) cur;
      if (used[u]) continue;
      used[u] = true;
      res += cur >>> 32;
      for (Edge e : edges[u]) {
         int v = e.t;
         if(!used[v] && prio[v] > e.cost) {
            prio[v] = e.cost; pred[v] = u;
            q.add(((long) prio[v] << 32) + v);
         }
      }
   }
   return res;
}

private class Edge {
   int t, cost;
   public Edge(int t, int cost) {
      this.t = t; this.cost = cost;
   }
}
\end{lstlisting}