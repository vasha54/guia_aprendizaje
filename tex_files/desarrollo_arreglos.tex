En programación, un arreglo (llamados en inglés array) es una zona de almacenamiento continuo, que contiene una serie de elementos del mismo tipo. 

La forma de acceder a los elementos de un arreglo es directa; esto significa que el elemento deseado es obtenido a partir de su índice y no hay que ir buscándolo elemento por elemento (en contraposición, en el caso de una lista, para llegar, por ejemplo, al tercer elemento hay que acceder a los dos anteriores o almacenar un apuntador o puntero que permita acceder de manera rápida a ese elemento).

Para hacer referencia
a cualquiera de estos elementos en un programa, se proporciona el nombre del arreglo seguido del número de posición
del elemento específico entre corchetes ( [] ). Al número de posición se le conoce más formalmente como el \textbf{índice}
o \textbf{subíndice} (este número específica el número de elementos a partir del inicio del arreglo). El primer elemento en todo
arreglo tiene el subíndice 0 (cero) y se conoce algunas veces como el elemento cero.

Un \textbf{subíndice} debe ser un entero o una expresión entera (usando cualquier tipo integral). Si un programa utiliza una
expresión como un subíndice, entonces el programa evalúa la expresión para determinar el subíndice.