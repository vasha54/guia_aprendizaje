El árbol de Fenwick se describió por primera vez en un artículo titulado "\emph{Una nueva estructura de datos para tablas de frecuencias acumulativas}" (Peter M. Fenwick, 1994). El árbol de Fenwick también se llama Árbol Indexado Binario (\emph{Binary Indexed Tree}) , o simplemente BIT abreviado.

El árbol de Fenwick es una estructura de datos que:

\begin{itemize}
	\item Calcula el valor de la función $f$ en el rango dado $[l;r]$ (i.e. $f(A_l, A_{l+1},\dots,A_r)$) en tiempo $O(\log N)$.
	\item Requiere $O(N)$ memoria, o en otras palabras, exactamente la misma memoria requerida para $A$
	\item Es fácil de usar y codificar, especialmente en el caso de arreglos multidimensionales
\end{itemize}