\subsection{Struct}
Las estructuras (también llamadas \textbf{struct}) son una forma de agrupar varias variables relacionadas en un solo lugar. Cada variable en la estructura se conoce como un miembro de la estructura.

A diferencia de una matriz, una estructura puede contener muchos tipos de datos diferentes (int, string, bool, etc.). 

Para crear una estructura, use la palabra clave \textbf{struct} y declare cada uno de sus miembros entre llaves.

Después de la declaración, especifique el nombre de la variable de estructura (myStructure en el ejemplo siguiente): 

\begin{lstlisting}[language=C++]
struct{             // Declaracion de la estructura
   int myNum;         // Miembro (int variable)
   string myString;   // Miembro (string variable)
} myStructure;       // variable de tipo estructura
\end{lstlisting}

Para acceder a los miembros de una estructura, use la sintaxis de punto (.): 

\begin{lstlisting}[language=C++]
// Crear estructura variable llamada myStructure
struct {
   int myNum;
   string myString;
} myStructure;
	
// Asignar los valores a los miembros de myStructure
myStructure.myNum = 1;
myStructure.myString = "Hello World!";
	
// Imprimir los miembros myStructure
cout << myStructure.myNum << "\n";
cout << myStructure.myString << "\n";
\end{lstlisting}

\subsubsection{Estructuras con nombre}

Al darle un nombre a la estructura, puede tratarla como un tipo de datos. Esto significa que puede crear variables con esta estructura en cualquier parte del programa en cualquier momento.

Para crear una estructura con nombre, coloque el nombre de la estructura justo después de la palabra clave \textbf{struct}: 

\begin{lstlisting}[language=C++]
struct myDataType { // Esta estructura es llamada "myDataType"
   int myNum;
   string myString;
};
\end{lstlisting}

Para declarar una variable que usa la estructura, use el nombre de la estructura como el tipo de datos de la variable:

\begin{lstlisting}[language=C++]
// Declarar una estructura llamada "car"
struct car {
   string brand;
   string model;
   int year;
};
	
int main(){
   //Crear una estructura de tipo carro y almacenar en la variable myCar
   car myCar1;
   myCar1.brand = "BMW";
   myCar1.model = "X5";
   myCar1.year = 1999;
		
   //Crear otra estructura de tipo carro y almacenar en la variable myCar2;
   car myCar2;
   myCar2.brand = "Ford";
   myCar2.model = "Mustang";
   myCar2.year = 1969;
		
   //Imprimir los miembros de una estructuras
   cout << myCar1.brand << " " << myCar1.model << " " << myCar1.year << "\n";
   cout << myCar2.brand << " " << myCar2.model << " " << myCar2.year << "\n";
		
   return 0;
}
\end{lstlisting}

\subsection{Matrices}

Una matriz es un arreglo de areglos fila, o más en concreto un arreglo de referencias a los arreglos fila. Con este esquema, cada fila podría tener un número de elementos diferente.
\subsubsection{C++}
Los arreglos bidimensionales o matrices en C++ se pude declarar similar a como se hace un arreglo unidimensional.
\begin{lstlisting}[language=C++]
/*Se conoce de antemano las dimensiones esta manera es 
estatica se recomienda que sea dinamica*/
int mat [3][4];
	
/*De forma dinamica*/
int ** mat;
mat = new int * [ncolumns];
for(int i=0;i<ncolumns;i++)
   mat[i]=new int [nfilas];
	
/*Con los valores conocidos*/
double carrots[3][4] {{2.5, 3.2, 3.7, 4.1},// primera fila
   {4.1, 3.9, 1.6, 3.5},// segunda fila
   {2.8, 2.3, 0.9, 1.1} // tercera fila
};
	
\end{lstlisting}
\subsubsection{Java}
Los arrays bidimensionales de Java se crean de un modo muy similar al de C++ (con reserva dinámica de memoria). En Java una matriz se puede crear directamente en la forma,
\begin{lstlisting}[language=Java]
int [][] mat = new int[3][4];
\end{lstlisting}

o bien se puede crear de modo dinámico dando los siguientes pasos:

\begin{enumerate}
	\item Crear la referencia indicando con un doble corchete que es una referencia a matriz,
	\begin{lstlisting}[language=Java]
int[][] mat;
	\end{lstlisting}
	
	\item Crear el vector de referencias a las filas,
	
	\begin{lstlisting}[language=Java]
mat = new int[nfilas][];
	\end{lstlisting}
	
	\item Reservar memoria para los vectores correspondientes a las filas,
	
	\begin{lstlisting}[language=Java]
for(int i=0; i<nfilas; i++);
   mat[i] = new int[ncols];
	\end{lstlisting}
\end{enumerate}

A continuación se presentan algunos ejemplos de creación de arrays bidimensionales:
\begin{lstlisting}[language=Java]
// crear una matriz 3x3
// se inicializan a cero
double mat[][] = new double[3][3];
int [][] b = {{1, 2, 3},
   {4, 5, 6}, // esta coma es permitida
};
int c = new[3][]; // se crea el array de referencias a arrays
c[0] = new int[5];
c[1] = new int[4];
c[2] = new int[8];
\end{lstlisting}